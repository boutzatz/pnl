\section{Linear Algebra}


%% vector
\subsection{Vectors}
\subsubsection{Short Description}

Vectors are declared for several basic types : double, int, uint and
fcomplex. In the following declarations, {\tt BASE} must be replaced by one
the previous types and the corresponding vector structures are respectively
named PnlVect, PnlVectInt, PnlVectUint, PnlVectComplex
\begin{verbatim}
typedef struct PnlVect {
  int size; /*!< size of the vector */ 
  double *array; /*!< pointer to store the data */
  int owner; /*!< 1 if the structure owns its array pointer */
} PnlVect;

typedef struct PnlVectUint {
  int size; /*!< size of the vector */ 
  uint *array; /*!< pointer to store the data */
  int owner; /*!< 1 if the structure owns its array pointer */
} PnlVect;

typedef struct PnlVectInt {
  int size; /*!< size of the vector */ 
  int *array; /*!< pointer to store the data */
  int owner; /*!< 1 if the structure owns its array pointer */
} PnlVect;

typedef struct PnlVectComplex {
  int size; /*!< size of the vector */ 
  fcomplex *array; /*!< pointer to store the data */
  int owner; /*!< 1 if the structure owns its array pointer */
} PnlVect;
\end{verbatim}
\var{size} is the size of the vector, \var{array} is a pointer containing the
data and \var{owner} is an integer to know if the vector owns its \var{array}
pointer (\var{owner}=$1$) or shares it with another structure (\var{owner}=$0$).

\subsubsection{Functions}

\paragraph{General functions}
These functions exist for all types of vector no matter what the basic type
is. The following conventions are used to name functions operating on vectors.
Here is the table of prefixes used for the different basic types.

\begin{center}
  \begin{tabular}[t]{lll}
    type & prefix & BASE\\
    \hline
    double & pnl_vect & double \\
    \hline
    int & pnl_vect_int & int \\
    \hline
    uint & pnl_vect_uint & uint\\
    \hline
    fcomplex & pnl_vect_complex & fcomplex
  \end{tabular}
\end{center}

In this paragraph, we present the functions operating on \refstruct{PnlVect}
which exist for all types. To deduce the prototypes of these functions for
other basic types, one must replace {\tt pnl_vect} and {\tt double} according
the above table. 
\subparagraph{Constructors and destructors}
\begin{itemize}
\item \describefun{\refstruct{PnlVect} $\ast$}{pnl_vect_create}{int size}
  \sshortdescribe Creates a new \refstruct{PnlVect} pointer.  
\item \describefun{\refstruct{PnlVect} $\ast$}{pnl_vect_create_from_zero}{int size}
  \sshortdescribe Creates a new \refstruct{PnlVect} pointer and sets it to zero.  
\item \describefun{\refstruct{PnlVect} $\ast$}{pnl_vect_create_from_double}{int size, double x}
  \sshortdescribe Creates a new \refstruct{PnlVect} pointer and sets all
  elements t \var{x}.  
\item \describefun{\refstruct{PnlVect} $\ast$}{pnl_vect_create_from_ptr}{int
    size, const double $\ast$x}
  \sshortdescribe Creates a new \refstruct{PnlVect} pointer and copies \var{x}
  to \var{array}.  
\item \describefun{\refstruct{PnlVect} $\ast$}{pnl_vect_create_from_list}{int
    size, ...}
  \sshortdescribe Creates a new \refstruct{PnlVect} pointer of length
  \var{size} filled with the extra arguments passed to the function. The
  number of extra arguments passed must be equal to \var{size}, be aware that
  this cannot be checked inside the function.
\item \describefun{PnlVect}{pnl_vect_create_wrap_array}{const double $\ast$x, 
    int size}
  \sshortdescribe Creates a \refstruct{PnlVect} pointer which contains
  \var{x}. No copy is made. It is just a container.
  
\item \describefun{\refstruct{PnlVect} $\ast$}{pnl_vect_create_from_file}{const char $\ast$file}
  \sshortdescribe Reads a vector from a file and creates the corresponding \refstruct{PnlVect}.  

\item \describefun{\refstruct{PnlVect} $\ast$}{pnl_vect_copy}{const
    \refstruct{PnlVect} $\ast$\refstruct{v}}
  \sshortdescribe This is a copying constructor. It creates a copy of a \refstruct{PnlVect}.
\item \describefun{void}{pnl_vect_clone}{\refstruct{PnlVect} $\ast$clone, 
    const \refstruct{PnlVect} $\ast$\refstruct{v}} 
  \sshortdescribe Clones a \refstruct{PnlVect}. \var{clone} must be an
  already existing  \refstruct{PnlVect}. It is resized to match the size of
  \var{v} and the data are copied. Future modifications to \var{v} will not
  affect \var{clone}.

\item \describefun{void}{pnl_vect_free}{\refstruct{PnlVect} $\ast\ast$\refstruct{v}}
  \sshortdescribe Free a \refstruct{PnlVect} pointer and set the data pointer to NULL  
\end{itemize}

\subparagraph{Resizing vectors}
\begin{itemize}
\item \describefun{int}{pnl_vect_resize}{\refstruct{PnlVect} $\ast$\refstruct{v}, int size}
  \sshortdescribe Resizes a \refstruct{PnlVect}.  
\item \describefun{int}{pnl_vect_resize_from_double}{\refstruct{PnlVect}
    $\ast$\refstruct{v}, int size, double x} 
  \sshortdescribe Resizes a \refstruct{PnlVect}.  
\item \describefun{int}{pnl_vect_resize_from_ptr}{\refstruct{PnlVect}
    $\ast$\refstruct{v}, int size, double $\ast$t} 
  \sshortdescribe Resizes a \refstruct{PnlVect}.  
\end{itemize}  

\subparagraph{Accessing elements}

If it is supported by the compiler, the following functions are declared inline. You
just need to define the macro \verb!HAVE_INLINE! for by passing
\verb!-DHAVE_INLINE! to gcc to use the inline version of the following
functions.
\begin{itemize}
\item \describefun{void}{pnl_vect_set}{\refstruct{PnlVect} $\ast$\refstruct{v}, int i, double x}
  \sshortdescribe Sets v[i]=x  
\item \describefun{double}{pnl_vect_get}{const \refstruct{PnlVect} $\ast$\refstruct{v}, int i}
  \sshortdescribe Returns the value of v[i].  
\item \describefun{void}{pnl_vect_lget}{\refstruct{PnlVect} $\ast$\refstruct{v}, int i}
  \sshortdescribe Returns the address of v[i].  
\item \describefun{void}{pnl_vect_set_double}{\refstruct{PnlVect} $\ast$\refstruct{v}, double x}
  \sshortdescribe Sets all elements to x.  
\item \describefun{void}{pnl_vect_set_zero}{\refstruct{PnlVect} $\ast$\refstruct{v}}
  \sshortdescribe Sets all elements to zero.  
\end{itemize}
Equivalently to these functions, there exist macros for {\bf \refstruct{PnlVect} only}.
\begin{itemize}
\item \describefun{}{GET}{v, i}
  \sshortdescribe Returns \var{v[i]}.
  
\item \describefun{}{LET}{v, i}
  \sshortdescribe Returns \var{v[i]} as a lvalue.
\end{itemize}


\subparagraph{Printing vector}
\begin{itemize}
\item \describefun{void}{pnl_vect_print}{const \refstruct{PnlVect} $\ast$V}
  \sshortdescribe Prints a \refstruct{PnlVect}.  
\item \describefun{void}{pnl_vect_fprint}{FILE $\ast$fic, const \refstruct{PnlVect} $\ast$V}
  \sshortdescribe Prints a \refstruct{PnlVect} in file \var{fic}.  
\end{itemize}

\subparagraph{Applying external operation to vectors}

\begin{itemize}
\item \describefun{void}{pnl_vect_minus}{\refstruct{PnlVect} $\ast$lhs}
  \sshortdescribe In-place unary minus
\item \describefun{void}{pnl_vect_plus_double}{\refstruct{PnlVect} $\ast$lhs, double x}
  \sshortdescribe In-place vector scalar addition  
\item \describefun{void}{pnl_vect_minus_double}{\refstruct{PnlVect} $\ast$lhs, double x}
  \sshortdescribe In-place vector scalar substraction  
\item \describefun{void}{pnl_vect_mult_double}{\refstruct{PnlVect} $\ast$lhs, double x}
  \sshortdescribe In-place vector scalar multiplication  
\item \describefun{void}{pnl_vect_div_double}{\refstruct{PnlVect} $\ast$lhs, double x}
  \sshortdescribe In-place vector scalar division  
\end{itemize}

\subparagraph{Element wise operations}

\begin{itemize}
\item \describefun{void}{pnl_vect_plus_vect}{\refstruct{PnlVect} $\ast$lhs, 
    const \refstruct{PnlVect} $\ast$rhs} 
  \sshortdescribe In-place vector vector addition  

\item \describefun{void}{pnl_vect_minus_vect}{\refstruct{PnlVect} $\ast$lhs, 
    const \refstruct{PnlVect} $\ast$rhs} 
  \sshortdescribe In-place vector vector substraction  

\item \describefun{void}{pnl_vect_inv_term}{\refstruct{PnlVect} $\ast$lhs}
  \sshortdescribe In-place term by term vector inversion  

\item \describefun{void}{pnl_vect_div_vect_term}{\refstruct{PnlVect}
    $\ast$lhs, const \refstruct{PnlVect} $\ast$rhs} 
  \sshortdescribe In-place term by term vector division

\item \describefun{void}{pnl_vect_mult_vect_term}{\refstruct{PnlVect}
    $\ast$lhs, const \refstruct{PnlVect} $\ast$rhs} 
  \sshortdescribe In-place vector vector term by term multiplication  

\item \describefun{void}{pnl_vect_map}{\refstruct{PnlVect} $\ast$lhs, const
    \refstruct{PnlVect} $\ast$rhs, double($\ast$f)(double)} 
  \sshortdescribe Applies the function \var{f} to each element of \var{rhs} and
  stores the result in \var{lhs}

\item \describefun{void}{pnl_vect_map_inplace}{\refstruct{PnlVect} $\ast$lhs, double($\ast$f)(double)}
  \sshortdescribe Same function as \reffun{pnl_vect_map} but the result is
  stored in \var{lhs} itself.

\item \describefun{void}{pnl_vect_axpby}{double a, const \refstruct{PnlVect} $\ast$x, 
    double b, \refstruct{PnlVect} $\ast$y} 
  \sshortdescribe Computes \var{y : = a x + b y}. When \var{b==0}, the content
  of \var{y} is not used on input and instead \var{y} is resized to match \var{x}.

\item \describefun{double}{pnl_vect_sum}{const \refstruct{PnlVect} $\ast$lhs}
  \sshortdescribe Returns the sum of all the elements of a vector  

\item \describefun{void}{pnl_vect_cumsum}{\refstruct{PnlVect} $\ast$lhs}
  \sshortdescribe Computes the cumulative sum of all the elements of a
  vector. The original vector is modified

\item \describefun{double}{pnl_vect_prod}{const \refstruct{PnlVect} $\ast$V}
  \sshortdescribe Returns the product of all the elements of a vector  

\item \describefun{void}{pnl_vect_cumprod}{\refstruct{PnlVect} $\ast$lhs}
  \sshortdescribe Computes the cumulative product of all the elements of a
  vector. The original vector is modified
\end{itemize}

\subparagraph{Ordering functions}
The following functions are not defined for PnlVectComplex because there is
no total ordering on Complex numbers

\begin{itemize}
\item \describefun{double}{pnl_vect_max}{const \refstruct{PnlVect} $\ast$V}
  \sshortdescribe Returns the maximum of a a vector  

\item \describefun{double}{pnl_vect_min}{const \refstruct{PnlVect} $\ast$V}
  \sshortdescribe Returns the minimum of a vector  

\item \describefun{void}{pnl_vect_minmax}{const \refstruct{PnlVect} $\ast$, 
    double $\ast$m, double $\ast$M}
  \sshortdescribe Computes the minimum and maximum of a vector which are
  returned in  \var{m} and \var{M} respectively.
  
\item \describefun{void}{pnl_vect_min_index}{const \refstruct{PnlVect} $\ast$, 
    double $\ast$m, int $\ast$im}
  \sshortdescribe Computes the minimum of a vector and its index stored in 
  sets \var{m} and \var{im} respectively.

\item \describefun{void}{pnl_vect_max_index}{const \refstruct{PnlVect} $\ast$, 
    double $\ast$M, int $\ast$iM}
  \sshortdescribe Computes the maximum of a vector and its index stored in 
  sets \var{m} and \var{im} respectively.

\item \describefun{void}{pnl_vect_minmax_index}{const \refstruct{PnlVect}
    $\ast$, double $\ast$m, double $\ast$M, int $\ast$im, int $\ast$iM}
  \sshortdescribe Computes the minimum and maximum of a vector and the
  corresponding indices stored respectively in \var{m}, \var{M}, \var{im} and
  \var{iM}.

\item \describefun{void}{pnl_vect_qsort}{\refstruct{PnlVect} $\ast$, char order}
  \sshortdescribe Sorts a vector using a quick sort algorithm according to
  \var{order} which can be \verb!'i'! for an increasing order sort or
  \verb!'d'! for an decreasing order sort.

\item \describefun{void}{pnl_vect_qsort_index}{\refstruct{PnlVect} $\ast$,
    \refstruct{PnlVectInt} *index, char order}
  \sshortdescribe Sorts a vector using a quick sort algorithm according to
  \var{order} which can be \verb!'i'! for an increasing order sort or
  \verb!'d'! for an decreasing order sort. On output, \var{index} contains the
  permutation used to sort the vector.
\end{itemize}

\subparagraph{Scalar products and norms}
\begin{itemize}
\item \describefun{double}{pnl_vect_norm_two}{const \refstruct{PnlVect} $\ast$V}
  \sshortdescribe Returns the two norm of a vector  

\item \describefun{double}{pnl_vect_norm_one}{const \refstruct{PnlVect} $\ast$V}
  \sshortdescribe Returns the one norm of a vector  

\item \describefun{double}{pnl_vect_norm_infty}{const \refstruct{PnlVect} $\ast$V}
  \sshortdescribe Returns the infinity norm of a vector  

\item \describefun{double}{pnl_vect_scalar_prod}{const \refstruct{PnlVect}
    $\ast$rhs1, const \refstruct{PnlVect} $\ast$rhs2} 
  \sshortdescribe Computes the scalar product between 2 vectors  
\end{itemize}

\subparagraph{Misc}

\begin{itemize}
\item \describefun{void}{pnl_vect_swap_elements}{\refstruct{PnlVect} $\ast$v, int i, int j}
\item \describefun{void}{pnl_vect_reverse}{\refstruct{PnlVect} $\ast$v}
\end{itemize}


\paragraph{Complex vector functions}

\begin{itemize}
\item \describefun{void}{pnl_vect_complex_mult_double}{\refstruct{PnlVectComplex} $\ast$lhs, double x}
  \sshortdescribe In-place multiplication by a double.

\item \describefun{PnlVectComplex$\ast$}{pnl_vect_complex_create_from_array}{int
    size, const double $\ast$re, const double $\ast$im}
  \sshortdescribe Creates a \refstruct{PnlVectComplex} given the arrays of the
  real parts \var{re} and imaginary parts \var{im}.
\item \describefun{void}{pnl_vect_complex_split_in_array}{\refstruct{PnlVectComplex}
    $\ast$v, double $\ast$re, double $\ast$im}
  \sshortdescribe Splits a complex vector into two arrays : the array of the
  real parts of the elements of \var{v} and the array of the imaginary parts
  of the elements of \var{v}.
\item \describefun{void}{pnl_vect_complex_split_in_vect}{\refstruct{PnlVectComplex}
    $\ast$v, \refstruct{PnlVect} $\ast$re, \refstruct{PnlVect} $\ast$im}
  \sshortdescribe Splits a complex vector into two \refstruct{PnlVect}s : the
  \refstruct{PnlVect} of the real parts of the elements of \var{v} and the
  \refstruct{PnlVect} of the imaginary parts of the elements of \var{v}.
\end{itemize}

There exist functions to directly access the real or imaginary parts of an
element of a complex vector. These functions also have inlined versions that
are used if the variable \var{HAVE_INLINE} was declared at compilation time.

\begin{itemize}
\item \describefun{double}{pnl_vect_complex_get_real}{const \refstruct{PnlVectComplex} $\ast$v, int i}
  \sshortdescribe Returns the real part of \var{v[i]}.
  
\item \describefun{double}{pnl_vect_complex_get_imag}{const \refstruct{PnlVectComplex} $\ast$v, int i}
  \sshortdescribe Returns the imaginary part of \var{v[i]}.

\item \describefun{double$\ast$}{pnl_vect_complex_lget_real}{const \refstruct{PnlVectComplex} $\ast$v, int i}
  \sshortdescribe Returns the real part of \var{v[i]} as a lvalue.

\item \describefun{double$\ast$}{pnl_vect_complex_lget_imag}{const \refstruct{PnlVectComplex} $\ast$v, int i}
  \sshortdescribe Returns the imaginary part of \var{v[i]} as a lvalue.

\item \describefun{void}{pnl_vect_complex_set_real}{const \refstruct{PnlVectComplex} $\ast$v, int i, double re}
  \sshortdescribe Sets the real part of \var{v[i]} to \var{re}.

\item \describefun{void}{pnl_vect_complex_set_imag}{const \refstruct{PnlVectComplex} $\ast$v, int i, double im}
  \sshortdescribe Sets the imaginary part of \var{v[i]} to \var{im}.
\end{itemize}

Equivalently to these functions, there exist macros. When the compiler is able
to handle inline code, there is no gain in using macros instead of inlined
functions at least in principle.
\begin{itemize}
\item \describefun{}{GET_REAL}{v, i}
  \sshortdescribe Returns the real part of \var{v[i]}.
  
\item \describefun{}{GET_IMAG}{v, i}
  \sshortdescribe Returns the imaginary part of \var{v[i]}.
  
\item \describefun{}{LET_REAL}{v, i}
  \sshortdescribe Returns the real part of \var{v[i]} as a lvalue.
  
\item \describefun{}{LET_IMAG}{v, i}
  \sshortdescribe Returns the imaginary part of \var{v[i]} as a lvalue.
\end{itemize}

\subsection{Compact Vectors}
\subsubsection{Short description}

\begin{verbatim}
typedef struct PnlVectCompact {
  int size; /* size of the vector */
  union {
    double val; /* single value */
    double *array; /* Pointer to double values */
  };
  char convert; /* 'a', 'd' : array, double */
} PnlVectCompact;
\end{verbatim}

\subsubsection{Functions}

\begin{itemize}
\item \describefun{\refstruct{PnlVectCompact} $\ast$}{pnl_vect_compact_create}{int n, double x}
  \sshortdescribe Allocates a \refstruct{PnlVectCompact}.  

\item \describefun{int}{pnl_vect_compact_resize}{\refstruct{PnlVectCompact}
    $\ast$\refstruct{v}, int size, double x} 
  \sshortdescribe Resizes a \refstruct{PnlVectCompact}.  

\item \describefun{\refstruct{PnlVectCompact}
    $\ast$}{pnl_vect_compact_copy} {const \refstruct{PnlVectCompact}$\ast$\refstruct{v}}
  \sshortdescribe Copies a \refstruct{PnlVectCompact}  

\item \describefun{void}{pnl_vect_compact_free}{\refstruct{PnlVectCompact} $\ast$$\ast$\refstruct{v}}
  \sshortdescribe Free a \refstruct{PnlVectCompact}  

\item \describefun{\refstruct{PnlVect} $\ast$}{pnl_vect_compact_to_pnl_vect}
  {const \refstruct{PnlVectCompact} $\ast$C} 
  \sshortdescribe Converts a \refstruct{PnlVectCompact} pointer to a \refstruct{PnlVect} pointer.  

\item \describefun{double}{pnl_vect_compact_get}{const \refstruct{PnlVectCompact} $\ast$C, int i}
  \sshortdescribe Access function  
\end{itemize}

%% matrix

\subsection{Matrices}
\subsubsection{Short Description}

\begin{verbatim}
typedef struct PnlMat{
  int m; /*!< nb rows */ 
  int n; /*!< nb columns */ 
  int mn; /*!< product m*n */
  double *array; /*!< pointer to store the data row-wise */
  int owner; /*!< 1 if the structure owns its array pointer */
} PnlMat;

typedef struct PnlMatUint{
  int m; /*!< nb rows */ 
  int n; /*!< nb columns */ 
  int mn; /*!< product m*n */
  uint *array; /*!< pointer to store the data row-wise */
  int owner; /*!< 1 if the structure owns its array pointer */
} PnlMatUint;

typedef struct PnlMatInt{
  int m; /*!< nb rows */ 
  int n; /*!< nb columns */ 
  int mn; /*!< product m*n */
  int *array; /*!< pointer to store the data row-wise */
  int owner; /*!< 1 if the structure owns its array pointer */
} PnlMatInt;

typedef struct PnlMatComplex{
  int m; /*!< nb rows */ 
  int n; /*!< nb columns */ 
  int mn; /*!< product m*n */
  fcomplex *array; /*!< pointer to store the data row-wise */
  int owner; /*!< 1 if the structure owns its array pointer */
} PnlMatComplex;
\end{verbatim}
\var{m} is the number of rows, \var{n} is the number of columns. \var{array}
is a pointer containing the data of the matrix stored linewise, The element
$(i, j)$ of the matrix is \verb!array[i*m+j]!. \var{owner} is an integer to
know if the matrix owns its \var{array} pointer (\var{owner}=$1$) or shares it
with another structure (\var{owner}=$0$).

The following operations are implemented on matrices and vectors. \var{alpha}
and \var{beta} are real numbers, \var{A} and \var{B} are matrices and \var{x}
and \var{y} are vectors.
\begin{tabular}{ll}
  \reffun{pnl_mat_axpy} & \var{B := alpha * A + B} \\
  \reffun{pnl_mat_scalar_prod_A} & \var{y' A x} \\
  \reffun{pnl_mat_dgemm} & \var{C := alpha * op (A) * op (B) + beta * C}\\
  \reffun{pnl_mat_mult_vect_transpose_inplace} & \var{y = A' * x}\\
  \reffun{pnl_mat_mult_vect_inplace} & \var{y = A * x}\\
  \reffun{pnl_mat_lAxpby} & \var{y := alpha * A * x + beta * y}\\
  \reffun{pnl_mat_dgemv} & \var{y := alpha * op (A) * x + beta * y}\\
  \reffun{pnl_mat_dger} & \var{A := alpha x' * y + A}
\end{tabular}


\subsubsection{Generic Functions}
\paragraph{General functions}
These functions exist for all types of matrices no matter what the basic type
is. The following conventions are used to name functions operating on matrices.
Here is the table of prefixes used for the different basic types.

\begin{center}
  \begin{tabular}[t]{lll}
    type & prefix & BASE\\
    \hline
    double & pnl_mat & double \\
    \hline
    int & pnl_mat_int & int \\
    \hline
    uint & pnl_mat_uint & uint\\
    \hline
    fcomplex & pnl_mat_complex & fcomplex
  \end{tabular}
\end{center}

In this paragraph we present the functions operating on \refstruct{PnlMat}
which exist for all types. To deduce the prototypes of these functions for
other basic types, one must replace {\tt pnl_mat} and {\tt double} according
the above table.

\paragraph{Constructors and destructors}
\begin{itemize}
\item \describefun{\refstruct{PnlMat} $\ast$}{pnl_mat_create}{int m, int n}
  \sshortdescribe Creates a \refstruct{PnlMat}  with \var{m} rows and \var{n} columns.

\item \describefun{\refstruct{PnlMat} $\ast$}{pnl_mat_create_from_double}{int m, int n, double x}
  \sshortdescribe Creates a \refstruct{PnlMat} with \var{m} rows and \var{n}
  columns and sets all the elements to \var{x}

\item \describefun{\refstruct{PnlMat} $\ast$}{pnl_mat_create_from_ptr}{int m, int n, const double $\ast$x}
  \sshortdescribe Creates a \refstruct{PnlMat} with \var{m} rows and \var{n}
  columns and copies the array \var{x} to the new vector. Be sure that \var{x}
  is long enough to fill all the vector because it cannot be checked inside the function.

\item \describefun{\refstruct{PnlMat} $\ast$}{pnl_mat_create_from_list}{int
    m, int n, ...}
  \sshortdescribe Creates a new \refstruct{PnlMat} pointer of size $\var{m
    \times n}$ filled with the extra arguments passed to the function. The
  number of extra arguments passed must be equal to $\var{m \times n}$, be
  aware that this cannot be checked inside the function.

\item \describefun{PnlMat}{pnl_mat_create_wrap_array}{const double $\ast$x, 
    int size}
  \sshortdescribe Creates a \refstruct{PnlMat} pointer which contains
  \var{x}. No copy is made. It is just a container.
  
\item \describefun{\refstruct{PnlMat} $\ast$}{pnl_mat_create_diag_from_ptr}{const double $\ast$x, int d}
  \sshortdescribe Creates a new squared \refstruct{PnlMat} by specifying its size and
  diagonal terms as an array.

\item \describefun{\refstruct{PnlMat} $\ast$}{pnl_mat_create_diag}{const \refstruct{PnlVect} $\ast$V}
  \sshortdescribe Creates a new squared \refstruct{PnlMat} by specifying its diagonal
  terms in a \refstruct{PnlVect}.

\item \describefun{\refstruct{PnlMat} $\ast$}{pnl_mat_create_from_file}{const char $\ast$file}
  \sshortdescribe Reads a matrix from a file and creates the corresponding \refstruct{PnlMat}.  

\item \describefun{void}{pnl_mat_free}{\refstruct{PnlMat} $\ast$$\ast$\refstruct{v}}
  \sshortdescribe Frees a \refstruct{PnlMat} and sets \var{$\ast$v} to \var{NULL} 

\item \describefun{\refstruct{PnlMat} $\ast$}{pnl_mat_copy}{const \refstruct{PnlMat} $\ast$\refstruct{v}}
  \sshortdescribe Creates a new \refstruct{PnlMat} which is a copy of
  \var{v}.
  
\item \describefun{void}{pnl_mat_clone}{\refstruct{PnlMat} $\ast$clone, const \refstruct{PnlMat} $\ast$M}
  \sshortdescribe Clones \var{M} into \var{clone}. No no new
  \refstruct{PnlMat} is created.

\item \describefun{int}{pnl_mat_resize}{\refstruct{PnlMat} $\ast$\refstruct{v}, int m, int n}
  \sshortdescribe Resizes a \refstruct{PnlMat}. The new matrix is of size
  \var{m x n}.  
\end{itemize}  


\paragraph{Accessing elements}

\begin{itemize}
\item \describefun{void}{pnl_mat_set}{\refstruct{PnlMat} $\ast$self, int i, int j, double x}
  \sshortdescribe Sets the value of self[i, j]=x  

\item \describefun{double}{pnl_mat_get}{const \refstruct{PnlMat} $\ast$self, int i, int j}
  \sshortdescribe Gets the value of self[i, j]  

\item \describefun{double $\ast$}{pnl_mat_lget}{\refstruct{PnlMat} $\ast$self, int i, int j}
  \sshortdescribe Returns the address of self[i, j] for use as a lvalue.

\item \describefun{void}{pnl_mat_set_double}{\refstruct{PnlMat} $\ast$self, double x}
  \sshortdescribe Sets all elements of \var{self} to \var{x}.
  
\item \describefun{void}{pnl_mat_set_id}{\refstruct{PnlMat} $\ast$self}
  \sshortdescribe Sets the matrix \var{self} to the identity
  matrix. \var{self} must be a square matrix.

\item \describefun{\refstruct{PnlVect}}{pnl_mat_wrap_row}
  {const \refstruct{PnlMat} $\ast$M, int i}
  \sshortdescribe Returns a \refstruct{PnlVect} (not a pointer) whose array is
  the \var{i}-th row of \var{M}. The new vector shares its data with the
  matrix \var{M}, which means that any modification to one of them will affect
  the other.
  
\item \describefun{void}{pnl_mat_get_row}{\refstruct{PnlVect}
    $\ast$V, const \refstruct{PnlMat} $\ast$M, int i}
  \sshortdescribe Extracts and copies the \var{i}-th row of \var{M} into
  \var{V}.

\item \describefun{void}{pnl_mat_get_col}{\refstruct{PnlVect} $\ast$V, 
    const \refstruct{PnlMat} $\ast$M, int j}
  \sshortdescribe Extracts and copies the \var{j}-th column of \var{M} into \var{V}.
  
\item \describefun{void}{pnl_mat_swap_rows}{\refstruct{PnlMat} $\ast$M, int i, int j}
  \sshortdescribe Swaps two rows of a matrix.  

\item \describefun{void}{pnl_mat_set_col}{\refstruct{PnlMat} $\ast$M, 
    const \refstruct{PnlVect} $\ast$V, int j}
  \sshortdescribe Replaces the \var{i}-th column of a matrix M by a vector V 

\item \describefun{void}{pnl_mat_set_row}{\refstruct{PnlMat} $\ast$M, 
    const \refstruct{PnlVect} $\ast$V, int i}
  \sshortdescribe Replaces the \var{i}-th row of a matrix M by a vector V  
\end{itemize}

\paragraph{Printing Matrices}

\begin{itemize}
\item \describefun{void}{pnl_mat_print}{const \refstruct{PnlMat} $\ast$M}
  \sshortdescribe Prints a matrix to the standard output.  

\item \describefun{void}{pnl_mat_fprint}{FILE $\ast$fic, const \refstruct{PnlMat} $\ast$M}
  \sshortdescribe Prints a matrix to a file.
\end{itemize}

\paragraph{Applying external operations}
\begin{itemize}
\item \describefun{void}{pnl_mat_plus_double}{\refstruct{PnlMat} $\ast$lhs, double x}
  \sshortdescribe In-place matrix scalar addition  

\item \describefun{void}{pnl_mat_minus_double}{\refstruct{PnlMat} $\ast$lhs, double x}
  \sshortdescribe In-place matrix scalar substraction  

\item \describefun{void}{pnl_mat_mult_double}{\refstruct{PnlMat} $\ast$lhs, double x}
  \sshortdescribe In-place matrix scalar multiplication  

\item \describefun{void}{pnl_mat_div_double}{\refstruct{PnlMat} $\ast$lhs, double x}
  \sshortdescribe In-place matrix scalar division  

\end{itemize}

\paragraph{Element wise operations}

\begin{itemize}
\item \describefun{void}{pnl_mat_mult_mat_term}{\refstruct{PnlMat} $\ast$lhs, 
    const \refstruct{PnlMat} $\ast$rhs} 
  \sshortdescribe In-place matrix matrix term by term product  

\item \describefun{void}{pnl_mat_div_mat_term}{\refstruct{PnlMat} $\ast$lhs, 
    const \refstruct{PnlMat} $\ast$rhs} 
  \sshortdescribe In-place matrix matrix term by term division

\item \describefun{void}{pnl_mat_map_inplace}{\refstruct{PnlMat} $\ast$lhs, 
    double($\ast$f)(double)} 
  \sshortdescribe Applies function \var{f} to all elements of \var{lhs}.

\item \describefun{void}{pnl_mat_map}{\refstruct{PnlMat} $\ast$lhs, const
    \refstruct{PnlMat} $\ast$rhs, double($\ast$f)(double)} 
  \sshortdescribe Applies function \var{f} to \var{rhs} and stores the result
  into \var{lhs}.

\item \describefun{double}{pnl_mat_sum}{const \refstruct{PnlMat} $\ast$lhs}
  \sshortdescribe Sums matrix component-wise  

\item \describefun{void}{pnl_mat_sum_vect}{\refstruct{PnlVect} $\ast$y, const
    \refstruct{PnlMat} $\ast$A, char c}
  \sshortdescribe Sums matrix column or row wise. Argument \var{c} can be
  either 'r' (to get a row vector) or 'c' (to get a column vector). When
  \var{c='r'}, $y(j) = \sum_i A_{ij}$ and when \var{c='rc}, $y(i) = \sum_j
  A_{ij}$.

\item \describefun{void}{pnl_mat_cumsum}{\refstruct{PnlMat} $\ast$A, char c} 
  \sshortdescribe Cumulative sum over the rows or columns. Argument \var{c}
  can be either 'r' to sum over the rows or 'c' to sum over the columns. When
  \var{c='r'}, $A_{ij} = \sum_{1 \le k \le i} A_{kj}$ and when \var{c='rc}, 
  $A_{ij} = \sum_{1 \le k \le j} A_{ik}$.

\item \describefun{double}{pnl_mat_prod}{const \refstruct{PnlMat} $\ast$lhs}
  \sshortdescribe Products matrix component-wise

\item \describefun{void}{pnl_mat_prod_vect}{\refstruct{PnlVect} $\ast$y, const
    \refstruct{PnlMat} $\ast$A, char c}
  \sshortdescribe Prods matrix column or row wise. Argument \var{c} can be
  either 'r' (to get a row vector) or 'c' (to get a column vector). When
  \var{c='r'}, $y(j) = \prod_i A_{ij}$ and when \var{c='rc}, $y(i) = \prod_j
  A_{ij}$.

\item \describefun{void}{pnl_mat_cumprod}{\refstruct{PnlMat} $\ast$A, char c} 
  \sshortdescribe Cumulative prod over the rows or columns. Argument \var{c}
  can be either 'r' to prod over the rows or 'c' to prod over the columns. When
  \var{c='r'}, $A_{ij} = \prod_{1 \le k \le i} A_{kj}$ and when \var{c='rc}, 
  $A_{ij} = \prod_{1 \le k \le j} A_{ik}$.
\end{itemize}

\paragraph{Ordering operations}

\begin{itemize}
\item \describefun{void}{pnl_mat_max}{const \refstruct{PnlMat} $\ast$A,
    \refstruct{PnlVect} $\ast$M, char d}
  \sshortdescribe On exit, $\var{M}(i) = \max_{j}(\var{A}(i, j))$ when \var{d='c'}
  and $\var{M}(i) = \max_{j}(\var{A}(j, i))$ when \var{d='r'}.

\item \describefun{void}{pnl_mat_min}{const \refstruct{PnlMat} $\ast$A,
    \refstruct{PnlVect} $\ast$m, char d}
  \sshortdescribe On exit, $\var{m}(i) = \min_{j}(\var{A}(i, j))$ when \var{d='c'}
  and $\var{m}(i) = \min_{j}(\var{A}(j, i))$ when \var{d='r'}.

\item \describefun{void}{pnl_mat_minmax}{const \refstruct{PnlMat} $\ast$A, 
    \refstruct{PnlVect} $\ast$m, \refstruct{PnlVect} $\ast$M, char d}
  \sshortdescribe On exit, $\var{m}(i) = \min_{j}(\var{A}(i, j))$ and $\var{M}(i) =
  \max_{j}(\var{A}(i, j))$ when \var{d='c'} and $\var{m}(i) = \min_{j}(\var{A}(j, i))$
  and $\var{M}(i) = \min_{j}(\var{A}(j, i))$ when \var{d='r'}.
  
\item \describefun{void}{pnl_mat_min_index}{const \refstruct{PnlMat} $\ast$ A, 
    \refstruct{PnlVect} $\ast$m, \refstruct{PnlVectInt} $\ast$im, char d}
  \sshortdescribe Idem as \reffun{pnl_mat_min} and \var{index} contains the
  indices of the minima. If \var{index==NULL}, the indices are not computed.

\item \describefun{void}{pnl_mat_max_index}{const \refstruct{PnlMat} $\ast$ A, 
    \refstruct{PnlVect} $\ast$M, \refstruct{PnlVectInt} $\ast$iM, char d}
  \sshortdescribe Idem as \reffun{pnl_mat_max} and \var{index} contains the
  indices of the maxima. If \var{index==NULL}, the indices are not computed.

\item \describefun{void}{pnl_mat_minmax_index}{const \refstruct{PnlMat} $\ast$
    A, \refstruct{PnlVect} $\ast$m, \refstruct{PnlVect} $\ast$M,
    \refstruct{PnlVectInt} $\ast$im, \refstruct{PnlVectInt} $\ast$iM, char d}
  \sshortdescribe Idem as \reffun{pnl_mat_minmax} and \var{im} contains the
  indices of the minima and \var{iM} contains the indices of the minima. If
  \var{im==NULL} (resp. \var{iM==NULL}, the indices of the minima
  (resp. maxima) are not computed.

\item \describefun{void}{pnl_mat_qsort}{\refstruct{PnlMat} $\ast$, char dir, char order}
  \sshortdescribe Sorts a matrix using a quick sort algorithm according to
  \var{order} which can be \verb!'i'! for an increasing order sort or
  \verb!'d'! for an decreasing order sort. The parameter \var{dir} determines
  whether the matrix is sorted by rows or columns. If \var{dir='c'}, each row
  is sorted independtly of the others whereas if \var{dir='r'}, each column
  is sorted independtly of the others.

\item \describefun{void}{pnl_qsort_index}{\refstruct{PnlMat} $\ast$,
    \refstruct{PnlMatInt} *index, char dir, char order}
  \sshortdescribe Sorts a matrix using a quick sort algorithm according to
  \var{order} which can be \verb!'i'! for an increasing order sort or
  \verb!'d'! for an decreasing order sort. The parameter \var{dir} determines
  whether the matrix is sorted by rows or columns. If \var{dir='c'}, each row
  is sorted independtly of the others whereas if \var{dir='r'}, each column
  is sorted independtly of the others. In addition to the function
  \reffun{pnl_mat_qsort}, the permutation index is computed and stored into 
  \var{index}.
\end{itemize}


\paragraph{Standard matrix operations}
\begin{itemize}
\item \describefun{void}{pnl_mat_plus_mat}{\refstruct{PnlMat} $\ast$lhs, const
    \refstruct{PnlMat} $\ast$rhs} 
  \sshortdescribe In-place matrix matrix addition  

\item \describefun{void}{pnl_mat_minus_mat}{\refstruct{PnlMat} $\ast$lhs, 
    const \refstruct{PnlMat} $\ast$rhs} 
  \sshortdescribe In-place matrix matrix substraction  
  
\item \describefun{void}{pnl_mat_sq_transpose}{\refstruct{PnlMat} $\ast$M}
  \sshortdescribe In-place transposition of square matrices  

\item \describefun{\refstruct{PnlMat} $\ast$}{pnl_mat_transpose}{const
    \refstruct{PnlMat} $\ast$M} 
  \sshortdescribe Transposition of matrices
\end{itemize}

\subsubsection{Functions specific to base type {\tt double}}

\paragraph{Printing Matrices}
\begin{itemize}  
\item \describefun{void}{pnl_mat_print_nsp}{const \refstruct{PnlMat} $\ast$M}
  \sshortdescribe Prints a matrix to the standard output in a format
  compatible with Nsp.  
  
\item \describefun{void}{pnl_mat_fprint_nsp}{FILE $\ast$fic, const \refstruct{PnlMat} $\ast$M}
  \sshortdescribe Prints a matrix to a file in a format
  compatible with Nsp.  
\end{itemize}

\paragraph{Standard matrix operations}
\begin{itemize}

\item \describefun{void}{pnl_mat_axpy}{double alpha, const \refstruct{PnlMat}
    $\ast$A, const \refstruct{PnlMat} $\ast$B}
  \sshortdescribe Computes \var{B := alpha * A + B}

\item \describefun{void}{pnl_mat_scalar_prod_A}{const \refstruct{PnlMat}
    $\ast$A, const \refstruct{PnlVect} $\ast$x, const \refstruct{PnlVect} $\ast$y}
  \sshortdescribe Computes \var{y' * A * y}

\item \describefun{void}{pnl_mat_dger}{double alpha, const \refstruct{PnlVect}
    $\ast$x, const \refstruct{PnlVect} $\ast$y, \refstruct{PnlMat} $\ast$A}
  \sshortdescribe Computes \var{A := alpha x' * y + A}
  
\item \describefun{void}{pnl_mat_dgemm}{char transA, char transB, double
    alpha, const \refstruct{PnlMat} $\ast$A, const \refstruct{PnlMat} $\ast$B, 
    double beta, \refstruct{PnlMat} $\ast$C}
  \sshortdescribe Computes \var{C := alpha * op(A) * op (B) + beta *
    C}. When beta=0, the content of \var{C} is unused and instead \var{C}
  is resized to store \var{alpha A $\ast$B}. If \var{transA='N'} or
  \var{transA='n'}, \var{op (A) = A}, whereas If \var{transA='T'} or
  \var{transA='t'}, \var{op (A) = A'}. The same holds for \var{transB}.
  
\item \describefun{\refstruct{PnlMat} $\ast$}{pnl_mat_mult_mat}{const
    \refstruct{PnlMat} $\ast$rhs1, const \refstruct{PnlMat} $\ast$rhs2} 
  \sshortdescribe Matrix multiplication  \var{rhs1 * rhs2}

\item \describefun{void}{pnl_mat_mult_mat_inplace}{\refstruct{PnlMat}
    $\ast$lhs, const \refstruct{PnlMat} $\ast$rhs1, const \refstruct{PnlMat}
    $\ast$rhs2} 
  \sshortdescribe In-place matrix multiplication  \var{lhs = rhs1 * rhs2}

\item \describefun{\refstruct{PnlVect} $\ast$}{pnl_mat_mult_vect}{const
    \refstruct{PnlMat} $\ast$A, const \refstruct{PnlVect} $\ast$x} 
  \sshortdescribe Matrix vector multiplication  \var{A * x}

\item \describefun{void}{pnl_mat_mult_vect_inplace}{\refstruct{PnlVect}
    $\ast$y, const \refstruct{PnlMat} $\ast$A, const \refstruct{PnlVect}
    $\ast$x} 
  \sshortdescribe In place matrix vector multiplication  \var{y = A * x}

\item \describefun{\refstruct{PnlVect} $\ast$}{pnl_mat_mult_vect_transpose}{const
    \refstruct{PnlMat} $\ast$A, const \refstruct{PnlVect} $\ast$x} 
  \sshortdescribe Matrix vector multiplication  \var{A' * x}

\item \describefun{void}{pnl_mat_mult_vect_transpose_inplace}{\refstruct{PnlVect}
    $\ast$y, const \refstruct{PnlMat} $\ast$A, const \refstruct{PnlVect}
    $\ast$x} 
  \sshortdescribe In place matrix vector multiplication  \var{y = A' * x}
  
\item \describefun{void}{pnl_mat_lAxpby}{double lambda, const \refstruct{PnlMat}
    $\ast$A, const \refstruct{PnlVect} $\ast$x, double mu, \refstruct{PnlVect} $\ast$b} 
  \sshortdescribe Computes \var{b := lambda A x + mu b}. When \var{mu==0}, the
  content of \var{b} is not used on input and instead \var{b} is resized to
  match \var{A*x}

\item \describefun{void}{pnl_mat_dgemv}{char trans, double lambda, const
    \refstruct{PnlMat} $\ast$A, const \refstruct{PnlVect} $\ast$x, double mu, 
    \refstruct{PnlVect} $\ast$b} \sshortdescribe Computes \var{b := lambda
    op(A) x + mu b}, where \var{op (X) = X} or \var{op (X) = X'}. When
  \var{mu==0}, the content of \var{b} is not used and instead \var{b} is resized
  to match \var{op(A)*x}
  
\item \describefun{void}{pnl_mat_exp}{\refstruct{PnlMat} $\ast$B, 
    const \refstruct{PnlMat} $\ast$A}
  \sshortdescribe Computes the matrix exponential \var{exp(A)} and stores the
  result in \var{B}.
\end{itemize}

\paragraph{Linear systems and matrix decompositions}

The following functions are designed to solve linear system of the from \var{A
  x = b} where \var{A} is a matrix and \var{b} is a vector except in the
functions \reffun{pnl_mat_syslin_mat} and \reffun{pnl_mat_chol_syslin_mat}
which expect the right hand side member to be a matrix too. Whenever the
vector \var{b} is not needed once the system is solved, you should consider
using ``inplace'' functions.

\begin{itemize}
\item \describefun{void}{pnl_mat_chol}{\refstruct{PnlMat} $\ast$M}
  \sshortdescribe Computes the Cholesky decomposition of \var{M}. \var{M} must
  be symmetric, the positivity is tested in the algorithm. On exist, the lower
  part of \var{M} contains the Cholesky decomposition and the upper part is
  set to zero.

\item \describefun{void}{pnl_mat_chol_robust}{\refstruct{PnlMat} $\ast$M}
  \sshortdescribe Same function as \reffun{pnl_mat_chol} except that if a negative
  eigen value greater than a given threshold is found, this eigenvalue is
  considered to be that positive.

\item \describefun{void}{pnl_mat_lu}{\refstruct{PnlMat} $\ast$A, 
    \refstruct{PnlPermutation} $\ast$p} 
  \sshortdescribe Computes a P A = LU factorization. \var{P} must be an
  already allocated  \refstruct{PnlPermutation}. On exit the decomposition is
  stored in \var{A}, the lower part of \var{A} contains L while the upper part
  (including the diagonal terms) contains U. Remember that the diagonal
  elements of $L$ are all $1$.


\item \describefun{void}{pnl_mat_upper_syslin}{\refstruct{PnlVect}
    $\ast$x, const \refstruct{PnlMat} $\ast$U, const \refstruct{PnlVect}$\ast$b}
  \sshortdescribe Solves an upper triangular linear system \var{U x = b}

\item \describefun{void}{pnl_mat_lower_syslin}{\refstruct{PnlVect}
    $\ast$x, const \refstruct{PnlMat} $\ast$L, const \refstruct{PnlVect}$\ast$b}
  \sshortdescribe Solves a lower triangular linear system  \var{L x = b}
  

\item \describefun{void}{pnl_mat_chol_syslin}{\refstruct{PnlVect} $\ast$x, 
    const \refstruct{PnlMat} $\ast$chol, const \refstruct{PnlVect} $\ast$b} 
  \sshortdescribe Solves a symmetric definite positive linear system A x = b, 
  in which \var{chol} is assumed to be the Cholesky decomposition of A
  computed by \reffun{pnl_mat_chol}


\item \describefun{void}{pnl_mat_chol_syslin_inplace}{
    const \refstruct{PnlMat} $\ast$chol, \refstruct{PnlVect} $\ast$b} 
  \sshortdescribe Solves a symmetric definite positive linear system A x = b, 
  in which \var{chol} is assumed to be the Cholesky decomposition of A
  computed by \reffun{pnl_mat_chol}. The solution of the system is stored in
  \var{b} on exit.

\item \describefun{void}{pnl_mat_lu_syslin}{\refstruct{PnlVect} $\ast$x, const
    \refstruct{PnlMat} $\ast$LU, const \refstruct{PnlPermutation} $\ast$p, 
    const \refstruct{PnlVect} $\ast$b} 
  \sshortdescribe Solves a linear system A x = b using a LU decomposition.
  \var{LU} and \var{P} are assumed to be the PA = LU decomposition as computed
  by \reffun{pnl_mat_lu}. In particular, the structure of the matrix \var{LU}
  is the following : the lower part of \var{A} contains L while the upper part
  (including the diagonal terms) contains U. Remember that the diagonal
  elements of $L$ are all $1$.

\item \describefun{void}{pnl_mat_lu_syslin_inplace}{const
    \refstruct{PnlMat} $\ast$LU, const \refstruct{PnlPermutation} $\ast$p, 
    \refstruct{PnlVect} $\ast$b} 
  \sshortdescribe Solves a linear system A x = b using a LU decomposition.
  \var{LU} and \var{P} are assumed to be the PA = LU decomposition as computed
  by \reffun{pnl_mat_lu}. In particular, the structure of the matrix \var{LU}
  is the following : the lower part of \var{A} contains L while the upper part
  (including the diagonal terms) contains U. Remember that the diagonal
  elements of $L$ are all $1$. The solution of the system is stored in \var{b}
  on exit.
  
\item \describefun{void}{pnl_mat_syslin}{\refstruct{PnlVect} $\ast$x, const
    \refstruct{PnlMat} $\ast$A, const \refstruct{PnlVect} $\ast$b} 
  \sshortdescribe Solves a linear system A x = b using a LU factorization
  which is computed inside this function.

\item \describefun{void}{pnl_mat_syslin_inplace}{\refstruct{PnlMat} $\ast$A, 
    \refstruct{PnlVect} $\ast$b} 
  \sshortdescribe Solves a linear system A x = b using a LU factorization
  which is computed inside this function. The solution of the system is stored
  in \var{b} and \var{A} is overwritten by its LU decomposition.

\item \describefun{void}{pnl_mat_syslin_mat}{\refstruct{PnlMat}$\ast$A, 
    \refstruct{PnlMat} $\ast$B} 
  \sshortdescribe Solves a linear system A X = B using a LU factorization
  which is computed inside this function. \var{A} and  \var{B} are
  matrices. \var{A} must be square. The solution of the system is stored in
  \var{B} on exit. On exit, \var{A} contains the LU decomposition of the input
  matrix which is lost.

\item \describefun{void}{pnl_mat_chol_syslin_mat}{\refstruct{PnlMat}$\ast$A, 
    \refstruct{PnlMat} $\ast$B}
  \sshortdescribe Solves a linear system A X = B
  using a Cholesky factorization which is computed inside this
  function. \var{A} and \var{B} are matrices. \var{A} must be symmetric
  positive definite. The solution of the system is stored in \var{B} on
  exit. On exit, \var{A} contains the Cholesky decomposition of the input
  matrix which is lost.
\end{itemize}


The following functions are designed to invert matrices. The authors provide
these functions although they cannot find good reasons to use them. Note that
to solve a linear system, one must used the \var{syslin} functions and not
invert the system matrix because it is much longer.
\begin{itemize}
\item \describefun{void}{pnl_mat_upper_inverse}{\refstruct{PnlMat} $\ast$A, 
    const \refstruct{PnlMat} $\ast$B}
  \sshortdescribe Inversion of an upper triangular matrix  

\item \describefun{void}{pnl_mat_lower_inverse}{\refstruct{PnlMat} $\ast$A, 
    const \refstruct{PnlMat} $\ast$B}
  \sshortdescribe Inversion of a lower triangular matrix  

\item \describefun{void}{pnl_mat_lu_inverse}{\refstruct{PnlMat}
    $\ast$inverse, const \refstruct{PnlMat}
    $\ast$LU, const \refstruct{PnlPermutation} $\ast$p} 
  \sshortdescribe Computes the inverse of a matrix A given its LU
  decomposition and the associated permutation   
\end{itemize}

\subsubsection{Permutations}

\begin{verbatim}
typedef  struct {
  int size;
  int *array;
} PnlPermutation;
\end{verbatim}

\begin{itemize}
\item \describefun{\refstruct{PnlPermutation} $\ast$}{pnl_permutation_create}{int n}
  \sshortdescribe Creates of a \refstruct{PnlPermutation} of size \var{n}.  

\item \describefun{void}{pnl_permutation_init}{\refstruct{PnlPermutation} $\ast$p}
  \sshortdescribe Initializes an existing permutation to the identity permutation.  

\item \describefun{void}{pnl_permutation_swap}{\refstruct{PnlPermutation} $\ast$p, int i, int j}
  \sshortdescribe Swaps two elements of a permutation.  

\item \describefun{void}{pnl_permutation_free}{\refstruct{PnlPermutation} $\ast$$\ast$p}
  \sshortdescribe Frees a \refstruct{PnlPermutation}.

\item \describefun{void}{pnl_vect_permute}{\refstruct{PnlVect} $\ast$px, const
    \refstruct{PnlVect} $\ast$x, const \refstruct{PnlPermutation} $\ast$p} 
  \sshortdescribe Applies a \refstruct{PnlPermutation} to a \refstruct{PnlVect}.  

\item \describefun{void}{pnl_vect_permute_inplace}{\refstruct{PnlVect} $\ast$x, 
    const \refstruct{PnlPermutation} $\ast$p} 
  \sshortdescribe Applies a \refstruct{PnlPermutation} to a
  \refstruct{PnlVect} in-place.  
  
\item \describefun{void}{pnl_permutation_fprint}{FILE $\ast$fic, const \refstruct{PnlPermutation} $\ast$p}
  \sshortdescribe Prints a permutation to a file.  

\item \describefun{void}{pnl_permutation_print}{const \refstruct{PnlPermutation} $\ast$p}
  \sshortdescribe Prints a permutation to the standard output.  
\end{itemize}

\subsection{Hyper matrices}
\subsubsection{Short description}

\begin{verbatim}
typedef struct PnlHMat{
  int ndim; /*!< nb dimensions */ 
  int *dims; /*!< pointer to store the value of the ndim dimensions */ 
  int mn; /*!< product dim_1 *...*dim_ndim */
  double *array; /*!< pointer to store */
} PnlHMat;

typedef struct PnlHMatUint{
  int ndim; /*!< nb dimensions */ 
  int *dims; /*!< pointer to store the value of the ndim dimensions */ 
  int mn; /*!< product dim_1 *...*dim_ndim */
  uint *array; /*!< pointer to store */
} PnlHMatUint;

typedef struct PnlHMatInt{
  int ndim; /*!< nb dimensions */ 
  int *dims; /*!< pointer to store the value of the ndim dimensions */ 
  int mn; /*!< product dim_1 *...*dim_ndim */
  int *array; /*!< pointer to store */
} PnlHMatInt;

typedef struct PnlHMatComplex{
  int ndim; /*!< nb dimensions */ 
  int *dims; /*!< pointer to store the value of the ndim dimensions */ 
  int mn; /*!< product dim_1 *...*dim_ndim */
  fcomplex *array; /*!< pointer to store */
} PnlHMatComplex;
\end{verbatim}
\var{ndim} is the number of dimensions, \var{dim} is an array to store the
size of each dimension and \var{nm} contains the product of the sizes of each
dimension. \var{array} is an array of size \var{mn} containing the data of the
matrix stored linewise.


\subsubsection{Generic Functions}
\paragraph{General functions}
These functions exist for all types of hypermatrices no matter what the basic type
is. The following conventions are used to name functions operating on hypermatrices.
Here is the table of prefixes used for the different basic types.

\begin{center}
  \begin{tabular}[t]{lll}
    type & prefix & BASE\\
    \hline
    double & pnl_hmat & double \\
    \hline
    int & pnl_hmat_int & int \\
    \hline
    uint & pnl_hmat_uint & uint\\
    \hline
    fcomplex & pnl_hmat_complex & fcomplex
  \end{tabular}
\end{center}

In this paragraph we present the functions operating on \refstruct{PnlMat}
which exist for all types. To deduce the prototypes of these functions for
other basic types, one must replace {\tt pnl_mat} and {\tt double} according
the above table.


\subsubsection{Functions}

\paragraph{Constructors and destructors}
\begin{itemize}
\item \describefun{\refstruct{PnlHMat} $\ast$}{pnl_hmat_create}{int ndim, const int $\ast$dims}
  
\item 
  \describefun{\refstruct{PnlHMat} $\ast$}{pnl_hmat_create_from_double}{int ndim, const int $\ast$dims, double x}
  
\item 
  \describefun{\refstruct{PnlHMat} $\ast$}{pnl_hmat_create_from_ptr}{int ndim, const int $\ast$dims, const double $\ast$x}
  
\item 
  \describefun{void}{pnl_hmat_free}{\refstruct{PnlHMat} $\ast$$\ast$H}
  
\item \describefun{\refstruct{PnlHMat} $\ast$}{pnl_hmat_copy}{const \refstruct{PnlHMat} $\ast$H}
  \sshortdescribe Copies a \refstruct{PnlHMat}.
  
\item \describefun{void}{pnl_hmat_clone}{\refstruct{PnlHMat} $\ast$clone, const \refstruct{PnlHMat} $\ast$H}
  \sshortdescribe Clones a \refstruct{PnlHMat}.
  
\item \describefun{int}{pnl_hmat_resize}{\refstruct{PnlHMat} $\ast$H, int ndim, const int $\ast$dims}
  \sshortdescribe Resizes a \refstruct{PnlHMat}.
\end{itemize}  

\paragraph{Accessing elements}

\begin{itemize}
\item   \describefun{void}{pnl_hmat_set}{\refstruct{PnlHMat} $\ast$self, int $\ast$tab, double x}
  \sshortdescribe Sets the element of index \var{tab} to \var{x}.
  
\item \describefun{double}{pnl_hmat_get}{const \refstruct{PnlHMat} $\ast$self, int $\ast$tab}
  \sshortdescribe Returns the value of the element of index \var{tab} 
  
\item \describefun{double$\ast$}{pnl_hmat_lget}{\refstruct{PnlHMat} $\ast$self, int $\ast$tab}
  \sshortdescribe Returns the address of self[tab] for use as a lvalue.  
\end{itemize}  

\paragraph{Printing hypermatrices}

\begin{itemize}
\item \describefun{void}{pnl_hmat_print}{const \refstruct{PnlHMat} $\ast$H}
  \sshortdescribe Prints an hypermatrix.
\end{itemize}

\paragraph{Accessing elements}

\begin{itemize}
\item \describefun{void}{pnl_hmat_plus_hmat}{\refstruct{PnlHMat} $\ast$lhs, const \refstruct{PnlHMat} $\ast$rhs}
  \sshortdescribe Computes \var{lhs += rhs}.
  
\item \describefun{void}{pnl_hmat_mult_double}{\refstruct{PnlHMat} $\ast$lhs, double x}
  \sshortdescribe Computes \var{lhs *= x} where x is a real number.
\end{itemize}


%% tridiag

\subsection{Tri-diagonal matrix}
\subsubsection{Short Description}

The special case of tridiagonal matrix , that is, has nonzero elements only on
the diagonal plus or minus one column, is one that occurs frequently - specially
for one dimensional PDE or Alternating direction implicit (ADI) method. Also
common matrix that are band diagonal or band matrix, which nonzero elements
only along a few diagonal lines adjacent to the main diagonal (above and
below). This case is discuss in the next sub-section.

For the tridiagonal sets, the solution of a linear system is performed in $O(n)$
operations. Using inline LU decomposition and forward-backward substitution.

Naturally, we store only the nonzero components, stored as three vectors.

\begin{verbatim}
typedef struct PnlTriDiagMat{
  int size;           /*!< diagonal dimension product  */
  int owner;          /*!< 1 if the structure owns its array pointer */
  double *diag;       /*!< pointer to store diagonal*/
  double *diag_up;    /*!< pointer to store up diagonal*/
  double *diag_down;  /*!< pointer to store down diagonal*/
} PnlTriDiagMat;
\end{verbatim}
\var{size} is the size of the matrix, \var{diag}, \var{diag_up}, \var{diag_down}  are pointers containing the
data of diagonal, up diagonal ($M_{i, i+1}$) and  down diagonal ($M_{i-1, i}$).
\var{owner} is an integer to know if the vector owns its three diagonals
pointer (\var{owner}=$1$) or shares it with another structure (\var{owner}=$0$).

\subsubsection{Functions}
\paragraph{Constructors and destructors}
\begin{itemize}
\item 
  \describefun{\refstruct{PnlTriDiagMat} $\ast$}{pnl_tridiagmat_create}{int size}
  \sshortdescribe Creates a \refstruct{PnlTriDiagMat}  
\item \describefun{\refstruct{PnlTriDiagMat} $\ast$}{pnl_tridiagmat_create_from_double}{int size, double x}
  \sshortdescribe Creates a \refstruct{PnlTriDiagMat}  
\item \describefun{\refstruct{PnlTriDiagMat} $\ast$}{pnl_tridiagmat_create_from_two_double}{int size, double x, double y}
  \sshortdescribe Creates a \refstruct{PnlTriDiagMat}  
\item \describefun{\refstruct{PnlTriDiagMat} $\ast$}{pnl_tridiagmat_create_from_ptr}{int size, const double $\ast$diag, const double $\ast$diag-up, const double $\ast$diag-down}
  \sshortdescribe Creates a \refstruct{PnlTriDiagMat}  
\item \describefun{\refstruct{PnlTriDiagMat} $\ast$}{pnl_tridiagmat_create_from_matrix}{const \refstruct{PnlMat} $\ast$mat}
  \sshortdescribe Reads a tri-diagonal matrix from a Matrix and creates
  the corresponding \refstruct{PnlTriDiagMat}.
\item void \refstruct{pnl_tridiagmat_free}{\refstruct{PnlTriDiagMat} $\ast$$\ast$\refstruct{v}}
  \sshortdescribe Frees a \refstruct{PnlTriDiagMat}  
\item \describefun{int}{pnl_tridiagmat_resize}{\refstruct{PnlTriDiagMat} $\ast$\refstruct{v}, int size}
  \sshortdescribe Resizes a \refstruct{PnlTriDiagMat}.  
\end{itemize}
\paragraph{Accessing elements}
\begin{itemize}
\item \describefun{void}{pnl_tridiagmat_set}{\refstruct{PnlTriDiagMat} $\ast$self, int d, int up, double x}
  \sshortdescribe Sets the value of if up= 1self[d, d+1] =x, if up= -1 self[d-1, d] = x if up= 0 self[d, d]  
\item \describefun{double}{pnl_tridiagmat_get}{const \refstruct{PnlTriDiagMat} $\ast$self, int d, int up}
  \sshortdescribe Gets the value of if up= 1self[d, d+1], if up= -1 self[d-1, d] if up= 0 self[d, d]  
\item \describefun{double $\ast$}{pnl_tridiagmat_lget}{\refstruct{PnlTriDiagMat} $\ast$self, int d, int up}
  \sshortdescribe Returns the address of if up= 1self[d, d+1], if up= -1 self[d-1, d] if up= 0 self[d, d] for use as a lvalue  
\end{itemize}

\paragraph{Printing Matrix}
\begin{itemize}
\item \describefun{void}{pnl_tridiagmat_fprint}{FILE $\ast$fic, const \refstruct{PnlTriDiagMat} $\ast$M}
  \sshortdescribe Prints a tri-diagonal matrix to a file.  
\item \describefun{void}{pnl_tridiagmat_print}{const \refstruct{PnlTriDiagMat} $\ast$M}
  \sshortdescribe Prints a tri-diagonal matrix to the standard output.  
\end{itemize}

\paragraph{Algebra operations}
\begin{itemize}
\item \describefun{void}{pnl_tridiagmat_plus_tridiagmat}{\refstruct{PnlTriDiagMat} $\ast$lhs, const \refstruct{PnlTriDiagMat} $\ast$rhs}
  \sshortdescribe In-place matrix matrix addition  
\item \describefun{void}{pnl_tridiagmat_minus_tridiagmat}{\refstruct{PnlTriDiagMat} $\ast$lhs, const \refstruct{PnlTriDiagMat} $\ast$rhs}
  \sshortdescribe In-place matrix matrix substraction  
\item \describefun{void}{pnl_tridiagmat_plus_double}{\refstruct{PnlTriDiagMat} $\ast$lhs, double x}
  \sshortdescribe In-place matrix scalar addition  
\item \describefun{void}{pnl_tridiagmat_minus_double}{\refstruct{PnlTriDiagMat} $\ast$lhs, double x}
  \sshortdescribe In-place matrix scalar substraction  
\item \describefun{void}{pnl_tridiagmat_mult_double}{\refstruct{PnlTriDiagMat} $\ast$lhs, double x}
  \sshortdescribe In-place matrix scalar multiplication  
\item \describefun{void}{pnl_tridiagmat_div_double}{\refstruct{PnlTriDiagMat} $\ast$lhs, double x}
  \sshortdescribe In-place matrix scalar division
\end{itemize}

\paragraph{Element-wise operations}
\begin{itemize}
\item \describefun{void}{pnl_tridiagmat_mult_tridiagmat_term}{\refstruct{PnlTriDiagMat} $\ast$lhs, const \refstruct{PnlTriDiagMat} $\ast$rhs}
  \sshortdescribe In-place matrix matrix term by term product  
\item \describefun{void}{pnl_tridiagmat_div_tridiagmat_term}{\refstruct{PnlTriDiagMat} $\ast$lhs, const \refstruct{PnlTriDiagMat} $\ast$rhs}
  \sshortdescribe In-place matrix matrix term by term division  
\end{itemize}

\paragraph{Standard matrix operations \& Linear system}
\begin{itemize}
\item \describefun{void}{pnl_tridiagmat_mult_vect_inplace}{\refstruct{PnlVect} $\ast$lhs, const \refstruct{PnlTriDiagMat} $\ast$mat, const \refstruct{PnlVect} $\ast$rhs}
  \sshortdescribe In place matrix multiplication  
\item \describefun{\refstruct{PnlVect} $\ast$}{pnl_tridiagmat_mult_vect}{const \refstruct{PnlTriDiagMat} $\ast$mat, const \refstruct{PnlVect} $\ast$vec}
  \sshortdescribe Matrix multiplication  
\item \describefun{void}{pnl_tridiagmat_lu_syslin}{\refstruct{PnlVect} $\ast$lhs, const \refstruct{PnlTriDiagMat} $\ast$M, const \refstruct{PnlVect} $\ast$rhs}
  \sshortdescribe Solves the linear system M x = b. Note, in this function, the LU decomposition
  is not explicitly computed.
\end{itemize}



\subsection{Band Matrix structure}
\subsubsection{Short Description}


Band Matrix have $m_{up}>0$ nonzero elements immediately to the left of(below)
the diagonal and $m_{down}$ nonzero elements immediately to the right of(above
it). Of course the next structure is useful only if $m_{up}$ and $m_{down}$
are smaller compare to $N$, the size of the matrix. In the case of Band
Matrix, the solution of the linear system by classical decomposition algorithms
LU, Cholesky, $\dots$ could be performed much faster and in much less storage, 
than for the $N\times N $ case. For simplification, we make the assumption $m_{up}=m_{down}=band$.


This structure is used in practice to solve multi-dimensional PDE in a Galerkin
framework. It is strongly inspired by work of F.Hecht on RMN class. 


\begin{verbatim}
typedef enum {
  FactorizationNO=0, 
  FactorizationCholeski=1, 
  FactorizationCrout=2, 
  FactorizationLU=3
}FactorizationType;

struct PnlBandMatrix{
  int n;      /*!< size of row */ 
  int m;      /*!< size of col */ 
  double *D;  /*!< diagonal vector */  
  double *Up; /*!< Triangular up matrix, only no nul coeffieciens */ 
  double *Lo; /*!< Triangular down matrix, only no nul coeffieciens */
  int    *pU; /*!< Triangular up profil   */ 
  int    *pL; /*!< Triangular down profil */
  FactorizationType typefac;
  int owner;  /*!< 1 if the structure owns its array pointer */
}PnlBandMatrix;
\end{verbatim}

\var{n} is the number of row, \var{m} is the number of column.
\var{D}, \var{Up}, \var{Low}  are pointers containing the
data of diagonal, the up matrix ($M_{i, j}, \ i<j\leq i+band$) and  down
matrix ($M_{j, i}, , \ i-band\leq j<i$).
\var{pU} and \var{pL} is used to store jump in row, see next equation.
\var{typefac} is an integer to store the factorisation type of the band
matrix, if it's done.
pointer (\var{owner}=$1$) or shares it with another structure (\var{owner}=$0$).


\begin{align*}
  \label{eq:1}
  & D(i) = A(i, i), \nonumber \\
  & Lo(k) = A(i, j), \quad j < i \textit{with: } \quad pL(i)<= k < pL(i+1)
  \textit{and } j = pL(i+1)-k, \\
  & Up(k) =A(i, j), \quad i < j \textit{with: } \quad pU(j)<= k < pU(j+1) \textit{and } i
  = pU(i+1)-k.\nonumber
\end{align*}
remark:
\begin{itemize}
\item $ pL = pU $ in most of case, 
\item if $L = U$ then symmetric matrix.
\end{itemize}


\subsubsection{Functions}
\paragraph{Constructors and destructors}
\begin{itemize}
\item 
  \describefun{PnlBandMatrix$\ast$}{pnl_band_matrix_create}{const int n, int band}
  \sshortdescribe Pnl_band_matrix_create create a void sparse matrix of size n with
  \char`\"{}band\char`\"{} extra diagonal terms
\item
  \describefun{PnlBandMatrix$\ast$}{pnl_band_matrix_create_from_full}{const \refstruct{PnlMat} $\ast$PM, int band}
  \sshortdescribe Extracts a band matrix from a \refstruct{PnlMat}.
\item
  \describefun{void}{pnl_band_matrix_free}{\refstruct{PnlMat} $\ast$
    $\ast$PnlBandMatrix}

\item 
  \describefun{\refstruct{PnlBandMatrix}$\ast$}{pnl_band_matrix_clone}{int n, double $\ast$d, double $\ast$u, int $\ast$pu, double $\ast$l, int $\ast$pl}
  \sshortdescribe Create a void sparse matrix with pointer on real sparse Matrix d, u, l is set
  to NULL in some case.
\end{itemize}
\paragraph{Accessing elements}
\begin{itemize}
\item 
  \describefun{double}{pnl_pnl_band_matrix_diag}{\refstruct{PnlBandMatrix} $\ast$M, int i}
  \sshortdescribe Return the diagonal term $M_{i, i}$.
\item 
  \describefun{void}{pnl_pnl_band_matrix_set}{\refstruct{PnlBandMatrix}
    $\ast$M, int i, int j, double x}
  \sshortdescribe Check if $j$ is a valid index (in most of case, $|i-j|\leq
  band $ - not true for first and last lines )
  and do $M_{i, j}=x$.
\item 
  \describefun{void}{pnl_band_matrix_add}{\refstruct{PnlBandMatrix}
    $\ast$M, int i, int j, double x}
  \sshortdescribe Check if $j$ is a valid index (in most of case, $|i-j|\leq
  band $ - not true for first and last lines )
  and do $M_{i, j} \ +=x$.
\end{itemize}

\paragraph{Extraction operations}
These functions should not be used by standard developers.
There do not create a real band matrix, but create an empty structure with owner
set to $0$ and pointer is set to address of some arguments of $M$.
Take an example, the transposition construct an empty band matrix $TM$ with
$TM\rightarrow Up = M\rightarrow Low, \dots $, but not make a copy of array $M\rightarrow Low$. 
\begin{itemize}
\item \describefun{\refstruct{PnlBandMatrix}$\ast$}{pnl_band_matrix_transpose}{const \refstruct{PnlBandMatrix} $\ast$M}
\item \describefun{\refstruct{PnlBandMatrix}$\ast$}{pnl_band_matrix_Low}{const \refstruct{PnlBandMatrix} $\ast$M}
\item \describefun{\refstruct{PnlBandMatrix}$\ast$}{pnl_band_matrix_Up}{const \refstruct{PnlBandMatrix} $\ast$M}
\item \describefun{\refstruct{PnlBandMatrix}$\ast$}{pnl_band_matrix_Tran_Low}{const \refstruct{PnlBandMatrix} $\ast$M}
\item \describefun{\refstruct{PnlBandMatrix}$\ast$}{pnl_band_matrix_Tran_Up}{const \refstruct{PnlBandMatrix} $\ast$M}
\item \describefun{\refstruct{PnlBandMatrix}$\ast$}{pnl_band_matrix_Diag}{const \refstruct{PnlBandMatrix} $\ast$M}
\item \describefun{\refstruct{PnlBandMatrix}$\ast$}{pnl_band_matrix_Low_Diag}{const \refstruct{PnlBandMatrix} $\ast$M}
\item \describefun{\refstruct{PnlBandMatrix}$\ast$}{pnl_band_matrix_Up_Diag}{const \refstruct{PnlBandMatrix} $\ast$M}
\item \describefun{\refstruct{PnlBandMatrix}$\ast$}{pnl_band_matrix_Tran_Low_Diag}{const \refstruct{PnlBandMatrix} $\ast$M}
\item \describefun{\refstruct{PnlBandMatrix}$\ast$}{pnl_band_matrix_Tran_Up_Diag}{const \refstruct{PnlBandMatrix} $\ast$M}
\end{itemize}

\paragraph{Factorization}
\begin{itemize}
\item 
  \describefun{void}{pnl_band_matrix_cholesky}{\refstruct{PnlBandMatrix} $\ast$M, double eps}
  \sshortdescribe Cholesky factorization $ M= U^T U$ of
  a Band Matrix (Symmetric)
\item  \describefun{void}{pnl_band_matrix_crout}{\refstruct{PnlBandMatrix} $\ast$M, double eps}
  \sshortdescribe Crout factorization $
  M= L U$ of a Band Matrix (Symmetric)
  \begin{equation*}
    \label{eq:2}
    L=\begin{pmatrix}
      l_{11}&0&0&0\\ l_{21}&l_{21}&0&0\\
      l_{31}&l_{32}&l_{33}&0\\ l_{41}&l_{42}&l_{43}&l_{44}
    \end{pmatrix} \quad \quad
    U=\begin{pmatrix} 1&u_{12}&u_{13}&u_{14}\\
      0&1&u_{23}&u_{24}\\ 0&0&1&u_{34}\\ 0&0&&1\\
    \end{pmatrix}    
  \end{equation*}
\item  \describefun{void}{pnl_band_matrix_lu}{\refstruct{PnlBandMatrix} $\ast$M, double eps}
  \sshortdescribe LU factorization $ M= L U$ of a Band
  Matrix (No Symmetric)
\end{itemize}

\paragraph{Standard matrix operations \& Linear system}
\begin{itemize}
\item 
  \describefun{void}{pnl_band_matrix_solve_syslin_inplace}{\refstruct{PnlBandMatrix}
    $\ast$M, \refstruct{PnlVect} $\ast$x}
  \sshortdescribe Solves the linear system $M x = x$ with M a \refstruct{PnlBandMatrix}.
  {\bf Note}, M is modified if and only if M has not been yet factorized.
  In most cases (i.e. for non symmetric matrix), on exit \var{M} contains the
  LU decomposition of the input matrix, 
  if $M\rightarrow typefac$ is previously $0$.   
\item  \describefun{void}{pnl_band_matrix_solve}{\refstruct{PnlBandMatrix} $\ast$M, 
    \refstruct{PnlVect} $\ast$x, const \refstruct{PnlVect} $\ast$b} 
  \sshortdescribe Pnl_band_matrix_solve solves the
  linear system $ M x = b $ with M a \refstruct{PnlBandMatrix}.
  {\bf Note}, M is modified if and only if M has not been yet factorize.
  In most of case (non symmetric matrix) M is return as the LU of the input M
  if $M\rightarrow typefac$ was $0$ before calling the function. 
\end{itemize}

\subsection{Morse Matrix}
\subsubsection{Short Description}

A system of linear equation is called sparse if only a relatively small number
of its matrix elements $M_{i, j}$ are nonzero. It is wasteful to use full
structure to solve the linear system because most of the operations
devoted to solve the system use zero elements. Furthermore, for some problems
high dimensional problem, stored the full matrix with zero elements is not
possible.


In the following, We propose two structures for the Sparse Matrix.
Must of algorithms which use sparse matrix could be divided in two steps.
The first step is the construction of the matrix. For this, 
\refstruct{PnlMorseMat} should be always used. The second step is the
resolution of a sparse linear system. We have two way to do that. The first is
to use an direct method based on matrix-decomposition, like LU. The
\refstruct{PnlSparseMat} is implemented to do that. An other way is to use
iterative methods like Conjugate Gradient, BICGstab or GMRES. These methods are
discussed in the next section. If we use iterative methods, we can use 
\refstruct{PnlMorseMat}. 

\begin{verbatim}
typedef struct SpRow{
  int size;
  /* size of a row */
  int Max_size;
  /* max size allocation of a row */
  int    *Index;
  /* pointer to an int array giving the columns or row i */
  double *Value;
  /* Pointer on values */
}SpRow;
\end{verbatim}
\var{size} is the number of elements, 
\var{Max_size} is the size of memory allocation.
\var{Index}, is pointer containing the index of row or column, 
\var{Value}, is pointer containing the value of row or column.
So for a \refstruct{SpRow} which contains row $i$ of $M$.
If $k \leq size $ then
$$M_{i, Index[k]}=Value[k].$$  

\begin{verbatim}
typedef struct PnlMorseMat{
  int m; /*!< nb rows */ 
  int n; /*!< nb columns */ 
  SpRow * array; 
  /*!< pointer in each row or col to store no nul coefficients */
  int RC; /*!> 0 if store row, 1 if store col*/ 
} PnlMorseMat;
\end{verbatim}
\var{m} is the number of rows, \var{n} is the number of columns.
\var{array} is pointer containing on SpRow array of size n or m (depend of
RC).
\var{RC} is an integer to know if the matrix is stored by row or columns.

\subsubsection{Functions}
\paragraph{Constructors and destructors}
\begin{itemize}
\item \describefun{\refstruct{PnlMorseMat}$\ast$}{pnl_morse_mat_create}{int m, 
    int n, int Max_size_row, int RC}
  \sshortdescribe Construction of empty \refstruct{PnlMorseMat} with memory
  allocation for each component of array. 
\item
  \describefun{\refstruct{PnlMorseMat}$\ast$}{pnl_morse_mat_create_fromfull}{\refstruct{PnlMat}
    $\ast$FM, int RC}
  \sshortdescribe Construct a \refstruct{PnlMorseMat} from  a \refstruct{PnlMat} storing only nonzero elements.
\item \describefun{void}{pnl_morse_mat_free}{\refstruct{PnlMorseMat}$\ast$$\ast$ M}
\item \describefun{int}{pnl_morse_mat_freeze}{PnlMorseMat$\ast$ M}
  \sshortdescribe Free the memory impose Max size equal to size for each SpRow.
\item \describefun{\refstruct{PnlMat}
    $\ast$}{pnl_morse_mat_full}{\refstruct{PnlMorseMat}$\ast$ M}
  \sshortdescribe Construct a full matrix from a morse matrix, use only for debug.
\end{itemize}
\paragraph{Accessing elements}
\begin{itemize}
\item \describefun{double}{ pnl_morse_mat_get}{PnlMorseMat$\ast$ M, int i, int
    j}
  \sshortdescribe Return $M_{i, j}$. 
\item \describefun{int}{ pnl_morse_mat_set}{PnlMorseMat$\ast$ M, int i, int
    j, double Val}
  \sshortdescribe Do $M_{i, j} = Val$. For example, if $RC=1$ and $(i, j)$ is a valid index, replace
  $array[i]\rightarrow Value[k]$ with $k$ such that $array[i]\rightarrow Index[k]=j$.
  If $(i, j)$ is not a valid index, add $j$ to $array[i]\rightarrow Index$ and $Val$ to
  $array[i] \rightarrow Value$ with memory allocation if needed. 
\item \describefun{double$\ast$}{pnl_morse_mat_lget}{PnlMorseMat$\ast$ M, int
    i, int j}
  \sshortdescribe Returns the address of $M_{i, j}$. For example, 
  if $RC=1$ and $(i, j)$ is a valid index, replace return address of
  $array[i]\rightarrow Value[k]$ with $k$ such that $array[i]\rightarrow
  Index[k]=j$.  If $(i, j)$ is not a valid index, add $j$ to
  $array[i]\rightarrow Index$ and add element to $array[i] \rightarrow Value$
  (with memory allocation if needed), returns address of this element. In
  practice this function is used to do $M_{i, j} += a$.
\end{itemize}

\paragraph{Printing Matrix}
\begin{itemize}
\item \describefun{void}{pnl_morse_mat_print}{const \refstruct{PnlMorseMat}$\ast$M}
\end{itemize}

\paragraph{Standard matrix operations}
\begin{itemize}
\item \describefun{void}{pnl_morse_mat_mult_vect_inplace}{\refstruct{PnlVect}
    $\ast$lhs, const \refstruct{PnlMorseMat}$\ast$M, const \refstruct{PnlVect}
    $\ast$rhs}
  \sshortdescribe Compute $ lhs=M \ rhs$.
\item \describefun{\refstruct{PnlVect}$\ast$}{pnl_morse_mat_mult_vect}{const
    \refstruct{PnlMorseMat}$\ast$M, const \refstruct{PnlVect} $\ast$vec}
  \sshortdescribe Compute $ vec=M \ vec$.
\end{itemize}


\subsection{Sparse Matrix}

\refstruct{PnlSparseMat} is the cs struct of Csparse library written by
Timothy A.Davis under the LGPL licence.  For the sake of convenience, we have
renamed some functions and structures. We have also reduced the number of
function parameters for non expert people in sparse matrices.  In the following, 
we only use LU factorization for sparse systems. In some cases, the factorization
can computed only one time PIDE-parabolic problem with same operator at each
time discretisation step. In this case, direct methods are faster than
iterative methods. When the PIDE coefficients are time dependent, the answer is not
so clear.

\subsubsection{Functions}
\paragraph{Constructors and destructors}
\begin{itemize}
\item \describefun{\refstruct{PnlSparseMat}
    $\ast$}{pnl_sparse_mat_create_fromfull}{\refstruct{PnlMat} $\ast$M}
  \sshortdescribe Construct a \refstruct{PnlSparseMat} from  a
  \refstruct{PnlMat} storing only nonzero elements.
\item \describefun{\refstruct{PnlSparseMat}
    $\ast$}{pnl_sparse_mat_create_frommorse}{\refstruct{PnlMorseMat}$\ast$ M}
  \sshortdescribe Construct a \refstruct{PnlSparseMat} from  a
  \refstruct{PnlMorseMat} with $M\rightarrow M->RC =1$.
\item \describefun{void}{pnl_sparse_mat_free}{\refstruct{PnlSparseMat} $\ast$$\ast$M}
\end{itemize}

\paragraph{Printing Matrix}
\begin{itemize}
\item \describefun{void}{pnl_sparse_mat_print}{\refstruct{PnlSparseMat} $\ast$A}
\end{itemize}

\paragraph{Standard matrix operations}
\begin{itemize}
\item \describefun{int}{pnl_sparse_mat_gaxpby}{\refstruct{PnlVect} $\ast$lhs, 
    const \refstruct{PnlSparseMat} $\ast$M, const \refstruct{PnlVect}
    $\ast$rhs}
  \sshortdescribe Compute $lhs=lhs+ M \ rhs$.
\item \describefun{int}{pnl_sparse_mat_mult_vect_inplace}{\refstruct{PnlVect} $\ast$lhs, const \refstruct{PnlSparseMat} $\ast$M, const \refstruct{PnlVect} $\ast$rhs}
  \sshortdescribe Compute $lhs= M \ rhs$.
\end{itemize}

\subsubsection{LU structure}
From the sparse matrix, we extract LU decomposition stored in \refstruct{PnlSparseFactorization}.
\paragraph{Constructors and desctructors}
\begin{itemize}
\item \describefun{\refstruct{PnlSparseFactorization}
    $\ast$}{pnl_sparse_factorization_lu_create}{const \refstruct{PnlSparseMat} $\ast$A, double tol}
\item \describefun{void}{pnl_sparse_factorization_free}{\refstruct{PnlSparseFactorization} $\ast$$\ast$ F}
\end{itemize}
\paragraph{Solve linear system}
\begin{itemize}
\item \describefun{void}{pnl_sparse_factorization_lu_syslin}{const \refstruct{PnlSparseFactorization} $\ast$N, PnlVect $\ast$x}
\end{itemize}

%% solver

\subsection{Solver Functions}
\subsubsection{Short Description}
Most of Iterative method are based on results on Krylov subspace. In linear algebra, the Krylov subspace generated by an n-by-n matrix A and a
vector b of dimension n is the linear subspace spanned by the images of $b$
under the first $r$ powers of $M$ (starting from $M^0 = Id$), that is, 
$$  \mathcal{K}_r(M, b) = \operatorname{span} \, \left\{ b, M b, M^2b, \ldots, M^{r-1}b\right\}.$$ 

Main idea of iterative methods or Krylov subspace methods is to form an
orthogonal basis of the sequence of successive matrix powers times the initial
residual (the Krylov sequence). The approximations to the solution are then
formed by minimizing the residual over the subspace formed.
The prototypical method in this class is the conjugate gradient method (CG). Other methods are the generalized minimal residual method (GMRES) and the Bi-conjugate gradient method (BiCG).

The approximating operator that appears in stationary iterative methods can
also be incorporated in Krylov subspace methods such as GMRES where they
become transformations of the original operator
to a presumably better conditioned one.

Left preconditioner solve the problem :
$$ P M x = P b, $$
and right preconditioner solve
$$ M P y  = b, \quad \quad P y = x.$$

With some simplifications, number of algorithms iterations depend ( linearly is
some case ) of conditioning. Conditioning is ratio of maximum eigen value over minimum
eigen value of $M$. For GMRES algorithm is depend of conditioning of $M^{T}
M$. So if we can find $P_L$ and $P_R$ such that $P_L M P_R$ is closed to
identity matrix, then preconditioning problem converge faster than initial
problem. We have also to solve $P_R y = x$ so $P_R$ has to be constructed to
do that fast.

More information is given in {\em Saad, Yousef (2003). Iterative methods for
  sparse linear systems (2nd ed. ed.). SIAM. ISBN 0898715342. OCLC 51266114}.
Reader find in this book some discussion about right or/and left
preconditioner and description of the following algorithms. 

These algorithms, we implemented with left preconditioner. Right preconditioner can be easily computed changing matrix-vector
multiplication operation from $M \ x $ to $ M \ P_R \ x$ and solving
$P_R y = x$ at the end of algorithm.


\subsubsection{Functions}
Three methods are implemented Conjugate Gradient, BICGstab and GMRES with restart. For each
of them a structure is created to store temporary vectors used in algorithm. So
we have constructor and destructor. In some case, we have to apply iterative
methods more than one time : for example to solve at each time step a discrete form of an elliptic
problem come from parabolic problem. So do not use constructor and free at
each time. Use only an initialization and solve procedure.  

Formally we have, 
\begin{verbatim}
Create iterative method
For each time step
  Initialisation of iterative method
  Solve linear system link to elliptic problem
end for
free iterative method
\end{verbatim}

In these functions, we don't use any particular matrix structure. We give the
matrix-vector multiplication as a parameter of the solver. 

\paragraph{Conjugate Gradient method}

Use only for symmetric and positive matrix.
\begin{itemize}
\item 
  \describefun{\refstruct{PnlCGSolver} $\ast$}{pnl_cg_solver_create}{int Size, int max-iter, double tolerance}
  \sshortdescribe Creates a new \refstruct{PnlCGSolver} pointer.  
\item \describefun{void}{pnl_cg_solver_initialisation}{\refstruct{PnlCGSolver} $\ast$Solver, const \refstruct{PnlVect} $\ast$b}
  \sshortdescribe Initialisation of the solver at the beginning of iterative method.  
\item \describefun{void}{pnl_cg_solver_free}{\refstruct{PnlCGSolver} $\ast$$\ast$Solver}
  \sshortdescribe Destructor of iterative solver  
\item \describefun{int}{pnl_cg_solver_solve}{void($\ast$matrix-vector-product)(const void $\ast$, const \refstruct{PnlVect} $\ast$, const double, const double, \refstruct{PnlVect} $\ast$), const void $\ast$Matrix-Data, void($\ast$matrix-vector-product-PC)(const void $\ast$, const \refstruct{PnlVect} $\ast$, const double, const double, \refstruct{PnlVect} $\ast$), const void $\ast$PC-Data, \refstruct{PnlVect} $\ast$x, const \refstruct{PnlVect} $\ast$b, \refstruct{PnlCGSolver} $\ast$Solver}
  \sshortdescribe Solve the linear system matrix-vector-product is the matrix-vector multiplication function matrix-vector-product-PC is the preconditionner function Matrix-Data \& PC-Data is data to compute matrix-vector multiplication.  
\end{itemize}
\paragraph{BICG stab}
\begin{itemize}
\item \describefun{\refstruct{PnlBICGSolver} $\ast$}{pnl_bicg_solver_create}{int Size, int max-iter, double tolerance}
  \sshortdescribe Creates a new \refstruct{PnlBICGSolver} pointer.  
\item \describefun{void}{pnl_bicg_solver_initialisation}{\refstruct{PnlBICGSolver} $\ast$Solver, const \refstruct{PnlVect} $\ast$b}
  \sshortdescribe Initialisation of the solver at the beginning of iterative method.  
\item \describefun{void}{pnl_bicg_solver_free}{\refstruct{PnlBICGSolver} $\ast$$\ast$Solver}
  \sshortdescribe Destructor of iterative solver  
\item \describefun{int}{pnl_bicg_solver_solve}{void($\ast$matrix-vector-product)(const void $\ast$, const \refstruct{PnlVect} $\ast$, const double, const double, \refstruct{PnlVect} $\ast$), const void $\ast$Matrix-Data, void($\ast$matrix-vector-product-PC)(const void $\ast$, const \refstruct{PnlVect} $\ast$, const double, const double, \refstruct{PnlVect} $\ast$), const void $\ast$PC-Data, \refstruct{PnlVect} $\ast$x, const \refstruct{PnlVect} $\ast$b, \refstruct{PnlBICGSolver} $\ast$Solver}
  \sshortdescribe Solve the linear system matrix-vector-product is the matrix-vector multiplication function matrix-vector-product-PC is the preconditioner function Matrix-Data \& PC-Data is data to compute matrix-vector multiplication.  
\end{itemize}

\paragraph{GMRES with restart}
See {\em Saad, Yousef (2003)} for discussion about restart parameter. In some
words, for GMRES we need to store at the p-th iteration $p$ vectors of the same size of the right
and side. It could be very expensive in term of memory allocation. So GMRES
with restart algorithm stop if $p=restart$ and restart algorithm with the
previous computed solution as initial guess.

Note that with restart equal to $m$, we have classical GMRES algorithm.

\begin{itemize}
\item \describefun{\refstruct{PnlGMRESSolver} $\ast$}{pnl_gmres_solver_create}{int Size, int max-iter, int restart, double tolerance}
  \sshortdescribe Creates a new \refstruct{PnlGMRESSolver} pointer.  
\item \describefun{void}{pnl_gmres_solver_initialisation}{\refstruct{PnlGMRESSolver} $\ast$Solver, const \refstruct{PnlVect} $\ast$b}
  \sshortdescribe Initialisation of the solver at the beginning of iterative method.  
\item \describefun{void}{pnl_gmres_solver_free}{\refstruct{PnlGMRESSolver} $\ast$$\ast$Solver}
  \sshortdescribe Destructor of iterative solver  
\item \describefun{int}{pnl_gmres_solver_solve}{void($\ast$matrix-vector-product)(const void $\ast$, const \refstruct{PnlVect} $\ast$, const double, const double, \refstruct{PnlVect} $\ast$), const void $\ast$Matrix-Data, void($\ast$matrix-vector-product-PC)(const void $\ast$, const \refstruct{PnlVect} $\ast$, const double, const double, \refstruct{PnlVect} $\ast$), const void $\ast$PC-Data, \refstruct{PnlVect} $\ast$x, const \refstruct{PnlVect} $\ast$b, \refstruct{PnlGMRESSolver} $\ast$Solver}
  \sshortdescribe Solve the linear system matrix-vector-product is the matrix-vector multiplication function matrix-vector-product-PC is the preconditionner function Matrix-Data \& PC-Data is data to compute matrix-vector multiplication.  
\end{itemize}


In the next paragraph, we write all the solvers for \refstruct{PnlMat}. This will be done as
follow: construct an application matrix-vector.
\begin{verbatim}
static void pnl_mat_mult_vect_applied(const void *mat, const PnlVect *vec, 
                                      const double a , const double b, 
                                      PnlVect *lhs)
{pnl_mat_lAxpby(a, (PnlMat*)mat, vec, b, lhs);}
\end{verbatim}
and give this as parameter of iterative method
\begin{verbatim}
int pnl_mat_cg_solver_solve(const PnlMat * Matrix, const PnlMat * PC, 
                            PnlVect * x, const PnlVect *b, PnlCGSolver * Solver)
{ return pnl_cg_solver_solve(pnl_mat_mult_vect_applied, 
                             Matrix, pnl_mat_mult_vect_applied, 
                             PC, x, b, Solver);}
\end{verbatim}

In practice, we cannot define all iterative methods for all structure.
With this implementation, user can easily :
\begin{itemize}
\item implemented right precondioner, 
\item implemented method with sparse matrix and diagonal preconditioner, or
  special combination of this form $\dots$
\end{itemize}


\paragraph{Iterative algorithms for \refstruct{PnlMat}}


\begin{itemize}
\item \describefun{int}{pnl_mat_cg_solver_solve}{const \refstruct{PnlMat} $\ast$Matrix, const \refstruct{PnlMat} $\ast$PC, \refstruct{PnlVect} $\ast$x, const \refstruct{PnlVect} $\ast$b, \refstruct{PnlCGSolver} $\ast$Solver}
  \sshortdescribe Solve the linear system Matrix x = b with preconditionner PC.  
\item \describefun{int}{pnl_mat_bicg_solver_solve}{const \refstruct{PnlMat} $\ast$Matrix, const \refstruct{PnlMat} $\ast$PC, \refstruct{PnlVect} $\ast$x, const \refstruct{PnlVect} $\ast$b, \refstruct{PnlBICGSolver} $\ast$Solver}
  \sshortdescribe Solve the linear system Matrix x = b with preconditionner PC.  
\item \describefun{int}{pnl_mat_gmres_solver_solve}{const \refstruct{
      PnlMat} $\ast$Matrix, const \refstruct{PnlMat} $\ast$PC, 
    \refstruct{PnlVect} $\ast$x, \refstruct{PnlVect} $\ast$b, \refstruct{PnlGMRESSolver} $\ast$Solver}
  \sshortdescribe Solve the linear system Matrix x = b with preconditionner PC.
\end{itemize}



%%% Local Variables: 
%%% mode: latex
%%% TeX-master: "pnl-premia-manual"
%%% End: 
