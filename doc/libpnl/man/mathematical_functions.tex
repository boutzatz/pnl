\section{Mathematical framework}

%% --------------------------------------------------------------------- %%
%% Mathtools
\subsection{General tools}
\subsubsection{Short Description}

\subsubsection{Constants} A few mathematical constants are provided by the
library. Most of them are actually already defined in {\tt math.h}, {\tt
  values.h} or {\tt limits.h} and a few others have been added.
\begin{describeconst}
  \constentry{M_E}{$e^1$}
  \constentry{M_LOG2E}{$\log_2 e$}
  \constentry{M_LOG10E}{$\log_{10} e$}
  \constentry{M_LN2}{$\log_e 2$}
  \constentry{M_LN10}{$\log_e 10$}
  \constentry{M_PI}{$\pi$}
  \constentry{M_2PI}{$2 \pi$}
  \constentry{M_PI_2}{$\pi/2$}
  \constentry{M_PI_4}{$\pi/4$}
  \constentry{M_1_PI}{$1/\pi$}
  \constentry{M_2_PI}{$2/\pi$}
  \constentry{M_2_SQRTPI}{$2/\sqrt{\pi}$}
  \constentry{M_SQRT2}{$\sqrt{2}$}
  \constentry{M_EULER}{$\gamma = \lim_{n \rightarrow \infty } \left( \sum_{k=1}^{n} \frac {1}{k} - \ln(n) \right)$}
  \constentry{M_SQRT1_2}{$1/\sqrt{2}$}
  \constentry{M_1_SQRT2PI}{$1/\sqrt{2 \pi}$}
  \constentry{M_SQRT2_PI}{$\sqrt{2/\pi}$}
  \constentry{INT_MAX}{$2147483647$}
  \constentry{MAX_INT}{INT_MAX}
  \constentry{DBL_MAX}{$1.79769313486231470e+308$}
  \constentry{DOUBLE_MAX}{DBL_MAX}
  \constentry{DBL_EPSILON}{$2.2204460492503131e-16$}
  \constentry{PNL_NEGINF}{$-\infty$}
  \constentry{PNL_POSINF}{$+\infty$}
  \constentry{NAN}{Not a Number}
\end{describeconst}

\subsubsection{A few macros}
\begin{itemize}
\item \describemacro{PNL_IS_ODD}{int n}
  \sshortdescribe Returns $1$ if \var{n} is odd and $0$ otherwise.
\item \describemacro{PNL_IS_EVEN}{int n}
  \sshortdescribe Returns $1$ if \var{n} is even and $0$ otherwise.
\end{itemize}

\subsubsection{Functions}
\begin{itemize}
\item \describefun{int}{intapprox}{double s}
  \sshortdescribe Returns the nearest integer with the convention ({\tt
    intapprox(1.5)=1}). This function is similar to the round function
  (provided by the C library) but the result is typed as an integer and not a
  double.

\item \describefun{double}{trunc}{double s}
  \sshortdescribe Returns the nearest integer not greater that the absolute
  value of \var{s}. This function should part of C99.

\item \describefun{double}{Cnp}{int n, int p}
  \sshortdescribe Computes the binomial coefficient $\binom{n}{p}$.

\item \describefun{double}{pnl_fact}{int n}
  \sshortdescribe Computes $n! = \Gamma(n+1)$ in double precision.

\item \describefun{double}{lgamma}{double x}
  \sshortdescribe Computes $\log(\Gamma(x))$. This function is part of C99 and is
  redefined for non C99 compliant compilers.

\item \describefun{double}{tgamma}{double x}
  \sshortdescribe Computes $\Gamma(x)$. This function is part of C99 and is
  redefined for non C99 compliant compilers.

\end{itemize}

%% --------------------------------------------------------------------- %%
%% Complex
\subsection{Complex numbers}
\subsubsection{Short Description}

The first native implementation of complex numbers in the C language has been
provided by the C99 implementation, which is not available on all
platforms. For this reason, we provide here an implementation of complex
numbers and its associated calculus.

\begin{verbatim}
typedef struct {
    double r; /*!< real part */
    double i; /*!< imaginary part */
} fcomplex;
\end{verbatim}


\subsubsection{Constants}

\begin{describeconst}
  \constentry{CZERO}{$0$ as a complex number}
  \constentry{CONE}{$1$ as a complex number}
  \constentry{CI}{$I$ the unit complex number}
\end{describeconst}

\subsubsection{Functions}
\begin{itemize}
\item \describefun{double}{Creal}{\refstruct{fcomplex} z}
  \sshortdescribe $ \mathrm{R}(z) $ 

\item \describefun{double}{Cimag}{\refstruct{fcomplex} z}
  \sshortdescribe  $ \mathrm{Im}(z) $  

\item \describefun{fcomplex}{Cadd}{\refstruct{fcomplex} z, \refstruct{fcomplex} b}
  \sshortdescribe $ z+b $  

\item \describefun{fcomplex}{CRadd}{\refstruct{fcomplex} z, double b}
  \sshortdescribe $ z+b $  

\item \describefun{fcomplex}{RCadd}{double b, \refstruct{fcomplex} z}
  \sshortdescribe $ b+z $  

\item \describefun{fcomplex}{Csub}{\refstruct{fcomplex} z, \refstruct{fcomplex} b}
  \sshortdescribe $ z-b $

\item \describefun{fcomplex}{CRsub}{\refstruct{fcomplex} z, double b}
  \sshortdescribe $ z-b $

\item \describefun{fcomplex}{RCsub}{double b, \refstruct{fcomplex} z}
  \sshortdescribe $ b-z $

\item \describefun{fcomplex}{Cminus}{\refstruct{fcomplex} z}
  \sshortdescribe $ -z $  

\item \describefun{fcomplex}{Cmul}{\refstruct{fcomplex} z, \refstruct{fcomplex} b}
  \sshortdescribe $ z*b $  

\item \describefun{fcomplex}{RCmul}{double x, \refstruct{fcomplex} z}
  \sshortdescribe $ x*z $

\item \describefun{fcomplex}{CRmul}{\refstruct{fcomplex} z, double x}
  \sshortdescribe $ z * x $

\item \describefun{fcomplex}{CRdiv}{\refstruct{fcomplex} z, double x}
  \sshortdescribe $ z/x $

\item \describefun{fcomplex}{Complex}{double x, double y}
  \sshortdescribe $x + \imath \; y$  

\item \describefun{fcomplex}{Complex_polar}{double r, double theta}
  \sshortdescribe  $ r e^{\imath \theta} $  

\item \describefun{fcomplex}{Conj}{\refstruct{fcomplex} z}
  \sshortdescribe $\overline{z}$  

\item \describefun{fcomplex}{Cinv}{\refstruct{fcomplex} z}
  \sshortdescribe $ \frac{1}{z}$  

\item \describefun{fcomplex}{Cdiv}{\refstruct{fcomplex} z, \refstruct{fcomplex} w}
  \sshortdescribe $ \frac{z}{w}$  

\item \describefun{double}{Csqr_norm}{\refstruct{fcomplex} z}
  \sshortdescribe $ \mathrm{Re}(z)^2 + \mathrm{im}(z)^2 $  

\item \describefun{double}{Cabs}{\refstruct{fcomplex} z}
  \sshortdescribe $ \vert z \vert $  

\item \describefun{fcomplex}{Csqrt}{\refstruct{fcomplex} z}
  \sshortdescribe $ \sqrt{z} $,  square root (with positive real part)  

\item \describefun{fcomplex}{Clog}{\refstruct{fcomplex} z}
  \sshortdescribe $\log(z)$  

\item \describefun{fcomplex}{Cexp}{\refstruct{fcomplex} z}
  \sshortdescribe $ e^z $  

\item \describefun{fcomplex}{CIexp}{double t}
  \sshortdescribe $ e^{\imath t} $

\item \describefun{fcomplex}{Cpow}{\refstruct{fcomplex} z, \refstruct{fcomplex} w}
  \sshortdescribe $ z^w$, power function  

\item \describefun{fcomplex}{Cpow_real}{\refstruct{fcomplex} z, double x}
  \sshortdescribe $ z^x$, power function  

\item \describefun{fcomplex}{Ccos}{\refstruct{fcomplex} z}
  \sshortdescribe $ \cos(g)$  

\item \describefun{fcomplex}{Csin}{\refstruct{fcomplex} z}
  \sshortdescribe $\sin(g)$  

\item \describefun{fcomplex}{Ctan}{\refstruct{fcomplex} z}
  \sshortdescribe $\tan(z)$

\item \describefun{fcomplex}{Ccosh}{\refstruct{fcomplex} z}
  \sshortdescribe $ \cosh(g)$  

\item \describefun{fcomplex}{Csinh}{\refstruct{fcomplex} z}
  \sshortdescribe $\sinh(g)$  

\item \describefun{fcomplex}{Ctanh}{\refstruct{fcomplex} z}
  \sshortdescribe $\tanh(z) = \frac{1 - \expp{-2z} }{1 + \expp{-2z} }$  

\item \describefun{double}{Carg}{\refstruct{fcomplex} z}
  \sshortdescribe $\arg(z) $

\item \describefun{fcomplex}{Ctgamma}{\refstruct{fcomplex} z}
  \sshortdescribe $ \Gamma(z)$, the Gamma function  

\item \describefun{fcomplex}{Clgamma}{\refstruct{fcomplex} z}
  \sshortdescribe $ \ln(\Gamma (z)) $, the logarithm of the Gamma function
\end{itemize}

Most algebraic operations on complex numbers are implemented using the
following naming for the functions
\begin{itemize}
\item we prefix all these functions with {\tt C_op_}, 
\item we use two complex numbers {\tt a, b} and a real number {\tt d}, 
\item we note {\tt i} the multiplication by $\imath$, 
\item {\tt c} indicates that the second argument is conjugated first.
\item we note {\tt p, m} the two standard operations {\it plus} and {\it minus}
\end{itemize}
For example C_op_idamcb is $\imath d \left( a - \overline{b} \right)$. So
functions are :
\begin{itemize}
\item \describefun{fcomplex}{C_op_apib}{\refstruct{fcomplex} a, \refstruct{fcomplex} b}
  \sshortdescribe $ a+\imath b  $.
\item \describefun{fcomplex}{C_op_apcb}{\refstruct{fcomplex} a, \refstruct{fcomplex} b}
  \sshortdescribe $ a+\overline{ b}  $.
\item \describefun{fcomplex}{C_op_amcb}{\refstruct{fcomplex} a, \refstruct{fcomplex} b}
  \sshortdescribe $ a-\overline{b}  $.
\item \describefun{fcomplex}{C_op_dapb}{double d, \refstruct{fcomplex} a, 
    \refstruct{fcomplex} b}
  \sshortdescribe $ d(a+ b)  $.
\item \describefun{fcomplex}{C_op_damb}{double d, \refstruct{fcomplex} a, 
    \refstruct{fcomplex} b}
  \sshortdescribe $ d(a-b)  $.
\item \describefun{fcomplex}{C_op_dapib}{double d, \refstruct{fcomplex} a, 
    \refstruct{fcomplex} b}
  \sshortdescribe $ d(a+\imath b)  $.
\item \describefun{fcomplex}{C_op_damib}{double d, \refstruct{fcomplex} a, 
    \refstruct{fcomplex} b}
  \sshortdescribe $ d(a-\imath b)  $.
\item \describefun{fcomplex}{C_op_dapcb}{double d, \refstruct{fcomplex} a, 
    \refstruct{fcomplex} b}
  \sshortdescribe $ d\left(a+\overline{b}\right)  $.
\item \describefun{fcomplex}{C_op_damcb}{double d, \refstruct{fcomplex} a, 
    \refstruct{fcomplex} b}
  \sshortdescribe $ d\left(a-\overline{b}\right)  $.
\item \describefun{fcomplex}{C_op_idapb}{double d, \refstruct{fcomplex} a, 
    \refstruct{fcomplex} b}
  \sshortdescribe $\imath d\left(a+b\right) $.
\item \describefun{fcomplex}{C_op_idamb}{double d, \refstruct{fcomplex} a, 
    \refstruct{fcomplex} b}
  \sshortdescribe $\imath  d\left(a-b\right) $.
\item \describefun{fcomplex}{C_op_idapcb}{double d, \refstruct{fcomplex} a, 
    \refstruct{fcomplex} b}
  \sshortdescribe $ \imath d\left(a+\overline{b}\right) $.
\item \describefun{fcomplex}{C_op_idamcb}{double d, \refstruct{fcomplex} a, 
    \refstruct{fcomplex} b}
  \sshortdescribe $ \imath  d\left(a-\overline{b}\right) $.
\end{itemize}
%% --------------------------------------------------------------------- %%




\subsection{Special Functions}

We provide approximations of the different Bessel (and Hankel) functions both
for real and complex arguments. These approximations were originally written
in Fortran in the Amos library and have been translated into C by
Jean-Philippe Chancelier, to whom we are very greatful, partly by using {\em
  f2c} partly by hand. The definitions of these functions have been extended
to negative orders by using the relations given in the {\em Amos} library

We also provide approximations of the different Incomplete Gamma Functions
(and Exponential integrals) for real arguments. These approximations come from
``Numerical Recipes''.

\subsubsection{Real Bessel Functions}

\begin{itemize}
\item \describefun{double}{pnl_bessel_i}{double v, double x}
  \sshortdescribe   Modified Bessel function of the first
  kind of order \var{v}.
\item \describefun{double}{pnl_bessel_i_scaled}{double v, double x}
  \sshortdescribe   Modified Bessel function of the first
  kind of order \var{v} divided by $e^{|x|}$.
\item \describefun{double}{pnl_bessel_j}{double v, double x}
  \sshortdescribe    Bessel function of the first
  kind of order \var{v}.
\item \describefun{double}{pnl_bessel_j_scaled}{double v, double x}
  \sshortdescribe    Bessel function of the first
  kind of order \var{v}. Same function as \reffun{pnl_bessel_j}.
\item \describefun{double}{pnl_bessel_y}{double v, double x}
  \sshortdescribe   Modified Bessel function of the second
  kind of order \var{v}.
\item \describefun{double}{pnl_bessel_y_scaled}{double v, double x}
  \sshortdescribe   Modified Bessel function of the second
  kind of order \var{v}. Same function as \reffun{pnl_bessel_y}.
\item \describefun{double}{pnl_bessel_k}{double v, double x}
  \sshortdescribe   Bessel function of the third
  kind of order \var{v}.
\item \describefun{double}{pnl_bessel_k_scaled}{double v, double x}
  \sshortdescribe   Bessel function of the third
  kind of order \var{v} multiplied by $e^{x}$.
\item \describefun{fcomplex}{pnl_bessel_h1}{double v, double x}
  \sshortdescribe   Hankel function of the first kind of
  order \var{v}. 
\item \describefun{fcomplex}{pnl_bessel_h1_scaled}{double v, double x}
  \sshortdescribe  Hankel function of the first kind of order
  \var{v}  and divided by $e^{I x}$.
\item \describefun{fcomplex}{pnl_bessel_h2}{double v, double x}
  \sshortdescribe  Hankel function of the second kind of
  order \var{v}. 
\item \describefun{fcomplex}{pnl_bessel_h2_scaled}{double v, double x}
  \sshortdescribe  Hankel function of the second kind of
  order \var{v}  and multiplied by $e^{I x}$.
\end{itemize}

\subsubsection{Complex Bessel Functions}

\begin{itemize}
\item \describefun{fcomplex}{pnl_complex_bessel_i}{double v, fcomplex z}
  \sshortdescribe  Complex Modified Bessel function of the first
  kind of order \var{v}.
\item \describefun{fcomplex}{pnl_complex_bessel_i_scaled}{double v, fcomplex z}
  \sshortdescribe  Complex Modified Bessel function of the first
  kind of order \var{v} divided by $e^{|Creal(z)|}$.
\item \describefun{fcomplex}{pnl_complex_bessel_j}{double v, fcomplex z}
  \sshortdescribe  Complex  Bessel function of the first
  kind of order \var{v}.
\item \describefun{fcomplex}{pnl_complex_bessel_j_scaled}{double v, fcomplex z}
  \sshortdescribe  Complex  Bessel function of the first
  kind of order \var{v} divided by $e^{|Cimag(z)|}$.
\item \describefun{fcomplex}{pnl_complex_bessel_y}{double v, fcomplex z}
  \sshortdescribe  Complex Modified Bessel function of the second
  kind of order \var{v}.
\item \describefun{fcomplex}{pnl_complex_bessel_y_scaled}{double v, fcomplex z}
  \sshortdescribe  Complex Modified Bessel function of the second
  kind of order \var{v} divided by $e^{|Cimag(z)|}$.
\item \describefun{fcomplex}{pnl_complex_bessel_k}{double v, fcomplex z}
  \sshortdescribe  Complex Bessel function of the third
  kind of order \var{v}.
\item \describefun{fcomplex}{pnl_complex_bessel_k_scaled}{double v, fcomplex z}
  \sshortdescribe  Complex Bessel function of the third
  kind of order \var{v} multiplied by $e^{z}$.
\item \describefun{fcomplex}{pnl_complex_bessel_h1}{double v, fcomplex z}
  \sshortdescribe  Complex Hankel function of the first kind of
  order \var{v}. 
\item \describefun{fcomplex}{pnl_complex_bessel_h1_scaled}{double v, fcomplex z}
  \sshortdescribe  Complex  Hankel function of the first kind of order
  \var{v}  and divided by $e^{I z}$.
\item \describefun{fcomplex}{pnl_complex_bessel_h2}{double v, fcomplex z}
  \sshortdescribe  Complex  Hankel function of the second kind of
  order \var{v}. 
\item \describefun{fcomplex}{pnl_complex_bessel_h2_scaled}{double v, fcomplex z}
  \sshortdescribe  Complex  Hankel function of the second kind of
  order \var{v}  and multiplied by $e^{I z}$.
\end{itemize}



\subsubsection{Incomplete Gamma Function }

For $a \in \R$ and $x>0$, the Incomplete Gamma Function is defined by
\begin{equation*}
  \Gamma(a, x)=\int_x^{\infty} \expp{-u} u^{a-1} du.
\end{equation*}
A relation similar to the one existing for the standard Gamma function holds
\begin{equation*}
  \Gamma\paren{a, x}= \frac{- x^{a} \expp{-x} + \Gamma (a+1, x)}{a}.
\end{equation*}

\begin{itemize}
\item \describefun{double}{pnl_gamma_inc_func}{double a, double x}
  \sshortdescribe   computes $\Gamma(a, x), \quad a \in \R , x \geq 0$
\item \describefun{void}{pnl_gamma_inc}{double a, double x, double $\ast$Ga, 
    double $\ast$P, double $\ast$Q}
  \sshortdescribe   computes for $a >0, x \geq 0$
  \begin{align*}
    \Gamma(a)&=\int_0^{\infty} u^{a-1} \expp{-u}du\\ 
    P(a, x) &= \frac{\Gamma(a) - \Gamma(a, x)}{\Gamma(a)} =
    \frac{1}{\Gamma(a)} \int_0^x u^{a-1} \expp{-u}  du\\ 
    Q(a, x) &= 1-P(a, x) =\frac{\Gamma(a, x)}{\Gamma(a)} =
    \frac{1}{\Gamma(a)} \int_x^{\infty} \expp{-u} u^{a-1} du. 
  \end{align*}
\end{itemize}

\subsubsection{Exponential integrals}
For $x>0$ and $n \in \N$, the  function $E_n$ is defined by
\begin{equation*}
  E_n\paren{x}=\int_{1}^{\infty} \expp{-x u} u^{-n} du
\end{equation*}

This function is linked to the Incomplete Gamma function by the relation
\begin{equation*}
  E_n\paren{x}=\int_{x}^{\infty}
  \expp{-xu} (xu)^{-n} x^{n-1} d(xu)=x^{n-1} \int_{x}^{\infty}
  \expp{-t} t^{-n} dt =  x^{n-1}  \Gamma\paren{1-n, x}, 
\end{equation*}
from which we can deduce
\begin{equation*}
  n E_{n+1}(x)  =   \expp{-x} - x E_n(x).
\end{equation*}
For $n>1$, the series expansion is given by
\begin{equation*}
  E_n(x)=x^{n-1}
  \Gamma(1-n)+\recaco{-\frac{1}{1-n}+\frac{x}{2-n}-\frac{x^2}{2(3-n)}
    +\frac{x^3}{6(4-n)}-\dots}.     
\end{equation*}
The asymptotic behaviour is given by
\begin{equation*}
  E_n(x)=\frac{\expp{-x}}{x}\recaco{1-\frac{n}{x}+\frac{n(n+1)}{x^2}+\dots}. 
\end{equation*}
The special case $n=1$ gives 
\begin{equation*}
  E_1(x) = \int_x^\infty \frac{e^{-u}}{u}\, du, \quad |\mathrm{Arg}(x)| \ge \pi. 
\end{equation*}
For any complex number $x$ with positive real part, this can be written
\begin{equation*}
  E_1(x) = \int_1^\infty \frac{e^{-ux}}{u}\, du, \quad \Re(x) \ge 0. 
\end{equation*}
By integrating the Taylor expansion of $\expp{-t}/t$, and extracting the
logarithmic singularity, we can derive the following series representation for
$E_1(x)$, 
\begin{equation*}
  {E_1}(x) =-\gamma-\ln x-\sum_{k=1}^{\infty}\frac{(-1)^k x^k}{k\; k!}
  \qquad |\mathrm{Arg}(x)| < \pi. 
\end{equation*}
The function $E_1$ is linked to the exponential integral $Ei$
\begin{equation*}
  Ei(x)=\int_{-\infty}^x\frac{e^u}u\, du=-\int_{-x}^{\infty}
  \frac{e^{-u}}{u}\, du \quad \forall x \neq 0. 
\end{equation*}
The above definition  can be used for positive values of $x$, but the integral
has to be understood in terms of its Cauchy principal value, due to the
singularity of the integrand at zero.
\begin{equation*}
  {Ei}(-x) = -{E}_1(x) , \quad \Re(x) \ge 0.
\end{equation*}
We deduce, 
\begin{equation*}
  Ei(x) = \gamma + \ln x+ \sum_{k=1}^{\infty} \frac{x^k}{k\; k!}, \quad x>0.
\end{equation*}
For $x \in \R$ 
\begin{equation*}
  \Gamma(0, x)=\left\{
    \begin{array}{l}
      -Ei(-x)-\imath \pi  \quad  x<0, \\
      -Ei(-x) \quad x>0. 
    \end{array}\right.
\end{equation*}

\begin{itemize}
\item \describefun{double}{pnl_expint_En}{int n, double x}
  \sshortdescribe   computes for  $ n\geq 0, x \geq 0$ and $x>0$ if $n=0$ or $1$.
  \begin{equation*}
    E_n(x)= \int_1^{\infty} u^{-n} \expp{-x u} du.
  \end{equation*}
\item \describefun{double}{pnl_expint_Ei}{double x}
  \sshortdescribe   computes the principal value of integral, for $x>0$ 
  \begin{equation*}
    Ei(x) = - \int_{-x}^{\infty} \frac{\expp{-u}}{u} du = \int_{-\infty}^x \frac{\expp{u}}{u} du 
  \end{equation*}
\end{itemize}



%% --------------------------------------------------------------------- %%
%% financial functions

\subsection{Financial functions}
\subsubsection{Short Description}
\subsubsection{Functions}

We use some practitioner convention. Here function depend of Forward price
and bond price. This can be useful for time dependant interest rate model.

We note $F_T$ the forward price at maturity $T$, the expected value of $S_T$.
In Black \& Scholes case, 
$$ F_T = S_0 \expp{(r-q)t}, $$
where $r$ is the interest rate and $q$ then dividend yield.

Expression of Bond price, so contract paid $1$ at maturity, in the BS case, 
is given by
$$ B_0 = \expp{-r T}.$$

Note that:
$$ \Delta_{Forward}  \expp{(r-q)T} = \Delta_{S}, \quad \quad
\Gamma_{Forward}  \expp{2 (r-q)T} = \Gamma_{S}.$$

\begin{itemize}
\item 
  \describefun{double}{pnl_bs_call}{double Vol, double Bond, double Forward, 
    double Strike, double Maturity}
\item 
  \describefun{double}{pnl_bs_put}{double Vol, double Bond, double Forward, 
    double Strike, double Maturity}
\item 
  \describefun{double}{pnl_bs_call_delta_forward}{double Vol, double Bond, 
    double Forward, double Strike, double Maturity}
\item 
  \describefun{double}{pnl_bs_put_delta_forward}{double Vol, double Bond, 
    double Forward, double Strike, double Maturity}
\item 
  \describefun{double}{pnl_bs_call_put}{int Is_call, double Vol, double Bond, 
    double Forward, double Strike, double Maturity}
\item 
  \describefun{double}{pnl_bs_call_put_delta_forward}{int Is_call, double Vol, 
    double Bond, double Forward, double Strike, double Maturity}
\item 
  \describefun{double}{pnl_bs_vega}{double Vol, double Bond, double Forward, 
    double Strike, double Maturity}
\item 
  \describefun{double}{pnl_bs_gamma}{double Vol, double Bond, double Forward, 
    double Strike, double Maturity}
\item 
  \describefun{double}{pnl_bs_s_square_gamma}{double Vol, double Bond, 
    double Forward, double Strike, double Maturity}

  Practitioners don't speak in terms of option price, but rather compare
  prices in terms of implied Black \& Scholes volatility. So this parameter is
  very useful in practice. Here we propose two functions to compute
  $\sigma_{impl}$: the first function is for one up-let, maturity, 
  strike, option price.  the second function is for a list of strikes and
  maturities, a matrix of prices (with strikes varying on the rows).
\item 
  \describefun{double}{pnl_bs_implicit_vol}{int Is_call, double Price, 
    double Bond, double Forward, double Strike, double Maturity}
\item 
  \describefun{int}{pnl_bs_matrix_implicit_vol}{const \refstruct{PnlMatUint}
    $\ast$Is_Call, const \refstruct{PnlMat} $\ast$Price, double spot, 
    double rate, double divid, const \refstruct{PnlVect} $\ast$Strike, 
    const \refstruct{PnlVect} $\ast$Maturity, \refstruct{PnlMat} $\ast$Vol}
\end{itemize}

Conversion to standard Premia functions and classic parameters.
\begin{itemize}
\item
  \describefun{int}{pnl_cf_call_bs}{double s, double k, double t, double r, 
    double divid, double sigma, double $\ast$ptprice, double $\ast$ptdelta}
\item
  \describefun{int}{pnl_cf_put_bs}{double s, double k, double t, double r, 
    double divid, double sigma, double $\ast$ptprice, double $\ast$ptdelta}
\item
  \describefun{double}{pnl_cf_bs_gamma}{double s, double k, double t, double r, 
    double divid, double sigma}

\end{itemize}



%%% Local Variables: 
%%% mode: latex
%%% TeX-master: "pnl-premia-manual"
%%% End: 
