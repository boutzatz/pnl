\section{Polynomial bases and regression}
\subsection{Overview}

To use these functionalities, you should include \verb!pnl/pnl_basis.h!.

\begin{verbatim}
struct PnlBasis_t {
  int         id;
  const char *label; /*!< string to label the basis *
  int         nb_variates;  /*!< number of variates *
  int         nb_func; /*!< number of elements in the basis *
  PnlMatInt  *T; /*!< Tensor matrix *
  double    (*f)(double    *x, int i); /*!< Computes the i-th element
                                        of the one dimensional basis *
  double    (*Df)(double   *x, int i); /*!< Computes the first derivative
                   of the i-th element of the one dimensional basis *
  double    (*D2f)(double  *x, int i); /*!< Computes the second derivative
                      of the i-th element of the one dimensional basis *
  int         isreduced; /* TRUE if the basis is reduced */
  double     *center; /*!< center of the domain */
  double     *scale; /*<! inverse of the scaling factor to map the domain 
                          to [-1, 1]^nb_variates */
};
\end{verbatim}

\begin{table}[h!]
  \begin{describeconst}
    \constentry{PNL_BASIS_CANONICAL}{for the Canonical polynomials}
    \constentry{PNL_BASIS_HERMITIAN}{for the Hermite polynomials}
    \constentry{PNL_BASIS_TCHEBYCHEV}{for the Tchebychev polynomials}
  \end{describeconst}
  \caption{Names of the bases}
  \label{basis_index}
\end{table}

In this section, we provide functions to solve regression problems on
polynomial functions. Let $(x_i, i=1 \dots n)$ be $n$ points in $\R^d$ and a
function $g$ defined by the data $(y_i = g(x_i), i=1 \dots n)$. Assume you
want to approximate the function $g$ by its decomposition on a family of $N$
polynomial functions $(f_j, j=1\dots N)$. Then, we want to compute the vector
$\alpha^\star \in \R^N$ which solves
\begin{equation*} \alpha^\star = \arg\min_\alpha \sum_{i=1}^{n}
  \left(\sum_{j=0}^N \alpha_j f_j(x_i) - y_i\right)^2
\end{equation*}

\subsection{Functions}

\begin{itemize}
\item \describefun{\refstruct{PnlBasis} *}{pnl_basis_new}{}
  \sshortdescribe Creates an empty \refstruct{PnlBasis}.

\item \describefun{void}{pnl_basis_print}{const \refstruct{PnlBasis} \ptr B}
  \sshortdescribe Prints a basis
  
  
\item \describefun{\refstruct{PnlBasis} *}{pnl_basis_create}{int index, int
    nb_func, int nb_variates}
  \sshortdescribe Creates a \refstruct{PnlBasis} for the polynomial family
  defined by \var{index} (see Table~\ref{basis_index}) with \var{nb_variates}
  variates. The basis will contain \var{nb_func}.

\item \describefun{\refstruct{PnlBasis} *}{pnl_basis_create_from_degree}{int
    index, int degree, int nb_variates}
  \sshortdescribe Creates a \refstruct{PnlBasis} for the polynomial family
  defined by \var{index} (see Table~\ref{basis_index}) with total degree less
  or equal than \var{degree} and \var{nb_variates} variates.\\
  For instance, calling \verb!pnl_basis_create_from_degree (index, 2, 4)! is
  equivalent to calling \verb!pnl_basis_create_from_tensor (index, T)! where
  \var{T} is given by
  \[ \left(
    \begin{array}{cccc}
      0 & 0 & 0 & 0\\
      1 & 0 & 0 & 0\\
      0 & 1 & 0 & 0\\
      0 & 0 & 1 & 0\\
      0 & 0 & 0 & 1\\
      1 & 1 & 0 & 0\\
      1 & 0 & 1 & 0\\
      1 & 0 & 0 & 1\\
      0 & 1 & 1 & 0\\
      0 & 1 & 0 & 1\\
      0 & 0 & 1 & 1\\
      2 & 0 & 0 & 0\\
      0 & 2 & 0 & 0\\
      0 & 0 & 2 & 0\\
      0 & 0 & 0 & 2\\
    \end{array}
  \right) \]


\item \describefun{\refstruct{PnlBasis} *}{pnl_basis_create_from_tensor}{int
    index, PnlMatInt \ptr T}
  \sshortdescribe Creates a \refstruct{PnlBasis} for the polynomial family
  defined by \var{index} (see Table~\ref{basis_index}) using the basis
  described by the tensor matrix \var{T}. The number of lines of \var{T} is
  the number of functions of the basis whereas the numbers of columns of
  \var{T} is the number of variates of the functions.
  Note that \var{T} is not copied inside this function but only its address is
  stored, so {\bf never} free \var{T}. It will be freed when calling
  \reffun{pnl_basis_free} on the returned object. i\\
  Here is an example of a tensor matrix. Assume you are working with three
  variate functions, the basis \verb!{ 1, x, y, z, x^2, xy, yz, z^3}! is
  decomposed in the one dimensional canonical basis using the following tensor
  matrix
  \[ \left(
    \begin{array}{ccc}
      0 & 0 & 0 \\
      1 & 0 & 0 \\
      0 & 1 & 0 \\
      0 & 0 & 1 \\
      2 & 0 & 0 \\
      1 & 1 & 0 \\
      0 & 1 & 1\\
      0 & 0 & 3
    \end{array}
  \right) \]

\item \describefun{void}{pnl_basis_clone}{\refstruct{PnlBasis} \ptr dest, const \refstruct{PnlBasis} \ptr src}
  \sshortdescribe Clones \var{src} into \var{dest}. The basis \var{dest} must
  already exist before calling this function. On exit, \var{dest} and \var{src}
  are identical and independent.
\item \describefun{\refstruct{PnlBasis}\ptr }{pnl_basis_copy}{const \refstruct{PnlBasis} \ptr B}
  \sshortdescribe Creates a copy of \var{B}.
\item \describefun{void }{pnl_basis_set_from_tensor}{\refstruct{PnlBasis} \ptr
    b, int index, const \refstruct{PnlMatInt} \ptr T}
  \sshortdescribe Sets an alredy existing basis \var{b} to a polynomial family
  defined by \var{index} (see Table~\ref{basis_index}) using the basis
  described by the tensor matrix \var{T}. The number of lines of \var{T} is
  the number of functions of the basis whereas the numbers of columns of
  \var{T} is the number of variates of the functions. \\
  Same function as \reffun{pnl_basis_create_from_tensor} except that it
  operates on an already existing basis.

\item  \describefun{\refstruct{PnlBasis}\ptr}{pnl_basis_create_from_hyperbolic_degree}
  {int index, double degree, double q, int n}
  \sshortdescribe Creates a sparse basis of polynomial with \var{n}
  variates. We give the example of the Canonical basis. A canonical polynomial
  with \var{n} variates writes $X_1^{\alpha_1} X_2^{\alpha_2} \dots
  X_n^{\alpha_n}$. To be a member of the basis, it must satisfy $\left(\sum_{i=1}^n
    {\alpha_i}^q \right)^{1/q} \leq degree$. This kind of basis based on an
  hyperbolic set of indices gives priority to polynomials associated to low
  order interaction.

\item  \describefun{void}{pnl_basis_free}{\refstruct{PnlBasis} \ptr\ptr basis}
  \sshortdescribe Frees a \refstruct{PnlBasis} created by
  \reffun{pnl_basis_create}. Beware that \var{basis} is the address of a
  \refstruct{PnlBasis}\ptr.

\end{itemize}


Polynomial regression based on a least square approach often leads to ill
conditionned linear systems. One way of improving the stability of the system is to
use centered and renormalised polynomials so that the original domain of interest
$\cD$ (a subset of $\R^d$) is mapped to $[-1,1]^d$. If the domain $\cD$ is
rectangular and writes $[a, b]$ where $a,b \in \R^d$, the mapping is done by 
\begin{equation}
  \label{basis_reduced}
  x \in \cD \longmapsto \left(\frac{x_i - (b_i+a_i)/2}{(b_i - a_i)/2}
  \right)_{i=1,\cdots,d}
\end{equation}
\begin{itemize}
\item \describefun{void}{pnl_basis_set_domain}{\refstruct{PnlBasis} \ptr B, 
  const \refstruct{PnlVect} \ptr a, const \refstruct{PnlVect} \ptr b}
  \sshortdescribe This function declares \var{B} as a centered and normalised basis
  as defined by Equation~\ref{basis_reduced}. Calling this function is equivalent to
  calling \reffun{pnl_basis_set_reduced} with \var{center=(b+a)/2} and
  \var{scale=(b-a)/2}.
\item \describefun{void}{pnl_basis_set_reduced}{\refstruct{PnlBasis} \ptr B,
  const \refstruct{PnlVect} \ptr center, const \refstruct{PnlVect} \ptr scale}
  \sshortdescribe This function declares \var{B} as a centered and normalised basis
  using the mapping
  \begin{equation*}
    x \in \cD \longmapsto \left(\frac{x_i - \var{center}_i }{\var{scale}_i}
    \right)_{i=1,\cdots,d}
  \end{equation*}
\end{itemize}


\begin{itemize}
\item \describefun{double}{pnl_basis_ik}{const PnlBasis \ptr b, const double
  \ptr x, int i, int k}
  \sshortdescribe An element of a basis writes $\prod_{l=0}^{\var{nb_variates}}
  \phi_l(x_l)$ where the $\phi$'s are one dimensional polynomials. This
  functions computes the therm $\phi_k$ of the \var{i-th} basis function at the
  point \var{x}.
\item  \describefun{double}{pnl_basis_i}{\refstruct{PnlBasis }\ptr b,
    double \ptr x, int i}
  \sshortdescribe If \var{b} is composed of $f_0, \dots, f_{\var{nb\_func}-1}$,
  then this function returns $f_i(x)$. \var{x} must be a C array of length
  \var{nb_variates}.

\item \describefun{double}{pnl_basis_i_D}{ const \refstruct{PnlBasis} \ptr b,
    const double \ptr x, int i, int j }
  \sshortdescribe If \var{b} is composed of $f_0, \dots, f_{\var{nb\_func}-1}$,
  then this function returns $\partial_{x_{\var{j}}} f_i(x)$. \var{x} must be a C array
  of length \var{nb_variates}.

  
\item \describefun{double}{pnl_basis_i_D2}{const \refstruct{PnlBasis} \ptr b,
    const double \ptr x, int i, int j1, int j2}
  \sshortdescribe If \var{b} is composed of $f_0, \dots, f_{\var{nb\_func}-1}$,
  then this function returns $\partial^2_{x_{\var{j1}}, x_{\var{j2}}}
  f_i(x)$. \var{x} must be a C array of length \var{nb_variates}.


\item \describefun{int}{pnl_basis_fit_ls}{\refstruct{PnlBasis} \ptr
    P, \refstruct{PnlVect} \ptr  coef, \refstruct{PnlMat} \ptr  x,
    \refstruct{PnlVect} \ptr  y}
  \sshortdescribe Computes the coefficients \var{coef} defined by
  \begin{equation*}
    \var{coef} = \arg\min_\alpha \sum_{i=1}^n
    \left( \sum_{j=0}^{\var{N}} \alpha_j  P_j(x_i) - y_i\right)^2
  \end{equation*}
  where \var{N} is the number of functions to regress upon and $n$ is the
  number of points at which we know the value of the original function. $P_j$
  is the $j-th$ basis function. Each row of the matrix \var{x} defines the
  coordinates of one point $x_i$. The function to be approximated is defined
  by the data \var{y} which is the vector of the values taken by the function
  at the points \var{x}.

\item \describefun{double}{pnl_basis_eval_derivs}{\refstruct{PnlBasis} \ptr
    P, \refstruct{PnlVect}\ptr coef, double \ptr x, double \ptr fx,
    \refstruct{PnlVect} \ptr Dfx, \refstruct{PnlMat} \ptr D2fx}
  \sshortdescribe Computes the function, the gradient and the Hessian matrix
  of $\sum_{k=0}^n \var{coef}_k  P_k(\cdot)$ at the point \var{x}.
  On output, \var{fx} contains the value of the function, \var{Dfx} its
  gradient and \var{D2fx} its Hessian matrix. This function is optimized and
  performs much better than calling \reffun{pnl_basis_eval},
  \reffun{pnl_basis_eval_D} and \reffun{pnl_basis_eval_D2} sequentially.

\item \describefun{double}{pnl_basis_eval}{\refstruct{PnlBasis} \ptr
    P, \refstruct{PnlVect}\ptr  coef, double \ptr x}
  \sshortdescribe Computes the linear combination of \var{P_k(x)} defined by
  \var{coef}. Given the coefficients computed by the function
  \reffun{pnl_basis_fit_ls}, this function returns $\sum_{k=0}^n
  \var{coef}_k  P_k(\var{x})$ where \var{x} is a C array of length
  \var{P->nb_variates}.

\item \describefun{double}{pnl_basis_eval_D}{\refstruct{PnlBasis} \ptr
    P, \refstruct{PnlVect} \ptr  coef, double \ptr x, int i}
  \sshortdescribe Computes the derivative with respect to \var{x_i} of the
  linear combination of \var{P_k(x)} defined by \var{coef}. Given the
  coefficients computed by the function \reffun{pnl_basis_fit_ls}, this
  function returns $\partial_{x_i} \sum_{k=0}^n \var{coef}_k  P_k(\var{x})$
  where \var{x} is a C array of length \var{nb_variates}. The index \var{i}
  may vary between \var{0} and \var{P->nb_variates - 1}.


\item \describefun{double}{pnl_basis_eval_D2}{\refstruct{PnlBasis} \ptr  P,
    \refstruct{PnlVect} \ptr  coef, double \ptr x,  int i, int j}
  \sshortdescribe Computes the derivative with respect to \var{x_i} of the
  linear combination of \var{P_k(x)} defined by \var{coef}. Given the
  coefficients computed by the function \reffun{pnl_basis_fit_ls}, this
  function returns $\partial_{x_i} \partial_{x_j} \sum_{k=0}^n \var{coef}_k  P_k(\var{x})$
  where \var{x} is a C array of length \var{nb_variates}. The indices \var{i}
  and \var{j} may vary between \var{0} and \var{P->nb_variates - 1}.
\end{itemize}


% vim:spelllang=en:spell:

