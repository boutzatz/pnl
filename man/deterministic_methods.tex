\section{Deterministic methods}

%% --------------------------------------------------------------------- %%
%% Roots
\subsection{Root finding}
\subsubsection{Short Description}

To provide a uniformed framework to root finding functions, we use two
structures to store functions either returning a single value or computing the
function and its derivative. The pointer \var{params} is used to store the
extra parameters. To evaluate such functions, one can use the two macros
\var{PNL_EVAL_FUNC} and \var{PNL_EVAL_FDF_FUNC}.
\begin{verbatim}
typedef struct {
  double (*function) (double x, void *params);
  void *params;
} PnlFunc ;

#define PNL_EVAL_FUNC(F, x) (*((F)->function))(x, (F)->params)

typedef struct {
  void (*function) (double x, double *f, double *df, void *params);
  void *params;
} PnlFuncDFunc ;

#define PNL_EVAL_FDF_FUNC(F, x, f, df) (*((F)->function))(x, f, df, (F)->params)
\end{verbatim}

\subsubsection{Functions}

\begin{itemize}

  \item \describefun{double}{pnl_root_brent}{\refstruct{PnlFunc}\ptr  F, double
    x1, double  x2, double \ptr tol}
    \sshortdescribe Finds the root of \var{F} between \var{x1} and \var{x2} with
    an accuracy of order \var{tol}. On exit \var{tol} is an upper bound of the
    error.

  \item \describefun{int}{pnl_find_root}{\refstruct{PnlFuncDFunc}\ptr  Func, 
    double x_min, double x_max, double tol, int N_Max, double\ptr  res}
    \sshortdescribe Finds the root of \var{F} between \var{x1} and \var{x2} with
    an accuracy of order \var{tol} and a maximum of \var{N_max} iterations. On
    exit, the root is stored in \var{res}. Note that the function \var{F} must
    also compute the first derivative of the function.


  \item \describefun{int}{pnl_root_newton}{\refstruct{PnlFuncDFunc} \ptr Func, 
    double x0, double epsrel, double epsabs, int N_max, double \ptr res}
    \sshortdescribe Finds the root of \var{F} starting from \var{x0} with an
    accuracy given both by \var{epsrel} and \var{epsabs} and a maximum number of
    iterations \var{N_max}. On exit, the root is stored in \var{res}.Note that
    the function \var{F} must also compute the first derivative of the function.

  \item \describefun{int}{pnl_root_bisection}{\refstruct{PnlFunc } \ptr Func, 
    double xmin, double xmax, double epsrel, double espabs, int N_max, double
    \ptr res}
    \sshortdescribe Finds the root of \var{F} between \var{x1} and \var{x2} with
    an accuracy given both by \var{epsrel} and \var{epsabs} and a maximum number
    of iterations \var{N_max}. On exit, the root is stored in \var{res}
\end{itemize}


%% --------------------------------------------------------------------- %%
%% Roots
\subsection{Polynomial bases and regression}
\subsubsection{Short Description}

\begin{verbatim}
struct PnlBasis_t {
  int         id;
  const char *label; /*!< string to label the basis */
  int         nb_variates;  /*!< number of variates */
  int         nb_func; /*!< number of elements in the basis */
  PnlMatInt  *T; /*!< Tensor matrix */
  double    (*f)(double    *x, int i); /*!< Computes the i-th element 
                                        of the one dimensional basis */
  double    (*Df)(double   *x, int i); /*!< Computes the first derivative 
                   of the i-th element of the one dimensional basis */
  double    (*D2f)(double  *x, int i); /*!< Computes the second derivative
                      of the i-th element of the one dimensional basis */
};
\end{verbatim}

\begin{table}[h!]
  \begin{describeconst}
    \constentry{CANONICAL}{for the Canonical polynomials}
    \constentry{HERMITIAN}{for the Hermite polynomials}
    \constentry{TCHEBYCHEV}{for the Tchebychev polynomials}
  \end{describeconst}
  \caption{Names of the bases}
  \label{basis_index}
\end{table}

In this section, we provide functions to solve regression problems on
polynomial functions. Let $(x_i, i=1 \dots n)$ be $n$ points in $\R^d$ and a
function $g$ defined by the data $(y_i = g(x_i), i=1 \dots n)$. Assume you
want to approximate the function $g$ by its decomposition on a family of $N$
polynomial functions $(f_j, j=1\dots N)$. Then, we want to compute the vector
$\alpha^\star \in \R^N$ which solves
\begin{equation*} \alpha^\star = \arg\min_\alpha \sum_{i=1}^{n}
  \left(\sum_{j=0}^N \alpha_j f_j(x_i) - y_i\right)^2
\end{equation*}

\subsubsection{Functions}

\begin{itemize}
  \item \describefun{\refstruct{PnlBasis} *}{pnl_basis_create}{int index, int N, int d}
    \sshortdescribe Creates a \refstruct{PnlBasis} for the polynomial family
    defined by \var{index} (see Table~\ref{basis_index}) with at most \var{N}
    elements. \var{d} is the dimension of the space on which the functions are
    defined.

  \item \describefun{\refstruct{PnlBasis} *}{pnl_basis_create_from_tensor}{int
    index, PnlMatInt \ptr T}
    \sshortdescribe Creates a \refstruct{PnlBasis} for the polynomial family
    defined by \var{index} (see Table~\ref{basis_index}) using the basis
    described by the tensor matrix \var{T}. The number of lines of \var{T} is
    the number of functions of the basis whereas the numbers of columns of
    \var{T} is the number of variates of the functions.
    Note that \var{T} is not copied inside this function but only its address is
    stored, so {\bf never} free \var{T}. It will be freed when calling
    \reffun{pnl_basis_free} on the returned object. i\\
    Here is an example of a tensor matrix. Assume you are working with three
    variate functions, the basis \verb!{ 1, x, y, z, x^2, xy, yz, z^3}! is
    decomposed in the one dimensional canonical basis using the following tensor
    matrix
    \[ \left(
    \begin{array}{ccc}
      0 & 0 & 0 \\
      1 & 0 & 0 \\
      0 & 1 & 0 \\
      0 & 0 & 1 \\
      2 & 0 & 0 \\
      1 & 1 & 0 \\
      0 & 1 & 1\\
      0 & 0 & 3
    \end{array}
    \right) \]
    
  \item  \describefun{void}{pnl_basis_free}{\refstruct{PnlBasis} \ptr\ptr basis}
    \sshortdescribe Frees a \refstruct{PnlBasis} created by
    \reffun{pnl_basis_create}. Beware that \var{basis} is the address of a
    \refstruct{PnlBasis}\ptr.

  \item  \describefun{void}{pnl_basis_i}{\refstruct{PnlBasis }\ptr basis,
    double \ptr x, int i}
    \sshortdescribe If \var{basis} is composed of $f_0, \dots, f_{\var{nb\_func}}$,
    then this function returns $f_i(x)$. \var{x} must be a C array of length
    \var{nb_variates}.

  \item \describefun{int}{pnl_basis_fit_ls}{\refstruct{PnlBasis} \ptr
    P, \refstruct{PnlVect} \ptr  coef, \refstruct{PnlMat} \ptr  x,
    \refstruct{PnlVect} \ptr  y}
    \sshortdescribe Computes the coefficients \var{coef} defined by
    \begin{equation*}
      \var{coef} = \arg\min_\alpha \sum_{i=1}^n
      \left( \sum_{j=0}^{\var{N}} \alpha_j  P_j(x_i) - y_i\right)^2
    \end{equation*}
    where \var{N} is the number of functions to regress upon and $n$ is the
    number of points at which we know the value of the original function. $P_j$
    is the $j-th$ basis function. Each row of the matrix \var{x} defines the
    coordinates of one point $x_i$. The function to be approximated is defined
    by the data \var{y} which is the vector of the values taken by the function
    at the points \var{x}.


  \item \describefun{double}{pnl_basis_eval}{\refstruct{PnlBasis} \ptr
    P, \refstruct{PnlVect}\ptr  coef, double \ptr x}
    \sshortdescribe Computes the linear combination of \var{P_k(x)} defined by
    \var{coef}. Given the coefficients computed by the function
    \reffun{pnl_basis_fit_ls}, this function returns $\sum_{k=0}^n
    \var{coef}_k  P_k(\var{x})$ where \var{x} is a C array of length
    \var{nb_variates}.

  \item \describefun{double}{pnl_basis_eval_D}{\refstruct{PnlBasis} \ptr
    P, \refstruct{PnlVect} \ptr  coef, double \ptr x, int i}
    \sshortdescribe Computes the derivative with repsect to \var{x_i} of the
    linear combination of \var{P_k(x)} defined by \var{coef}. Given the
    coefficients computed by the function \reffun{pnl_basis_fit_ls}, this
    function returns $\partial_{x_i} \sum_{k=0}^n \var{coef}_k  P_k(\var{x})$
    where \var{x} is a C array of length \var{nb_variates}.

  \item \describefun{double}{pnl_basis_eval_D2}{\refstruct{PnlBasis} \ptr  P,
    \refstruct{PnlVect} \ptr  coef, double \ptr x,  int i, int j}
    \sshortdescribe Computes the derivative with repsect to \var{x_i} of the
    linear combination of \var{P_k(x)} defined by \var{coef}. Given the
    coefficients computed by the function \reffun{pnl_basis_fit_ls}, this
    function returns $\partial_{x_i} \partial_{x_j} \sum_{k=0}^n \var{coef}_k  P_k(\var{x})$
    where \var{x} is a C array of length \var{nb_variates}.
\end{itemize}


\subsection{Numerical integration}
\subsubsection{Short Description}

Numerical integration methods are designed to numerically evaluate the
integral over an interval (resp. a square) of real valued functions defined on
$\R$ (resp. $\R^2$).

\begin{verbatim}
typedef struct {
  double (*function) (double x, void *params);
  void *params;
} pnl_function ;

typedef struct {
  double (*function) (double x, double y, void *params);
  void *params;
} pnl_function_2D ;
\end{verbatim}

We provide the following two macros to evaluate a \refstruct{pnl_function} at
a given point
\begin{verbatim}
#define PNL_EVAL_FUNC(F, x) (*((F)->function))(x, (F)->params)
#define PNL_EVAL_FUNC2D(F, x, y) (*((F)->function))(x, y, (F)->params)
\end{verbatim}



\subsubsection{Functions}

\begin{itemize}
\item \describefun{int}{ pnl_integration_GK}{const \refstruct{pnl_function} \ptr F, 
    double x0, double x1, double epsabs, double epsrel, double \ptr result, 
    double \ptr abserr,  int \ptr neval}
  \sshortdescribe Evaluates $\int_{x_0}^{x_1} F$ with an absolute error less than
  \var{espabs} and a relative error less than
  \var{esprel}. The value of the integral is stored in \var{result}, while the
  variables \var{abserr} and \var{neval} respectively contain the absolute
  error and the number of iterations.

\item \describefun{int}{ pnl_integration_GK2D}{const \refstruct{pnl_function_2D} \ptr F, 
    double x0, double x1, double y0, double y1, double epsabs, double epsrel, 
    double \ptr result, double \ptr abserr, int \ptr neval}
  \sshortdescribe Evaluates $\int_{[x_0, x_1] \times [y_0, y_1]} F$ with an
  absolute error less than \var{espabs} and a relative error less than
  \var{esprel}. The value of the integral is stored in \var{result}, while the
  variables \var{abserr} and \var{neval} respectively contain the absolute
  error and the number of iterations.

\item \describefun{double}{pnl_integration}{const \refstruct{pnl_function} \ptr F, 
    double x0, double x1, int n, char \ptr meth}
  \sshortdescribe Evaluates $\int_{x_0}^{x_1} F$ using \var{n} discretisation
  steps. The method used to discretise the integral is defined by \var{meth}
  which can be \var{"rect"} (rectangle rule), \var{"trap"} (trapezoidal rule),
  \var{"simpson"} (Simpson's rule).

\item \describefun{double}{pnl_integration_2D}{const \refstruct{pnl_function_2D} \ptr F,
    double x0, double x1, double y0, double y1, int nx, int ny, char \ptr meth}
  \sshortdescribe Evaluates $\int_{[x_0, x_1] \times [y_0, y_1]} F$ using
  \var{nx} (resp. \var{ny}) discretisation steps for \var{[x0, x1]}
  (resp. \var{[y0, y1]}). The method used to discretise the integral is
  defined by \var{meth} which can be \var{"rect"} (rectangle rule),
  \var{"trap"} (trapezoidal rule),   \var{"simpson"} (Simpson's rule).
\end{itemize}


%% FFT function
\subsection{Fast Fourier Transform}
\subsubsection{Short Description}

In the case of Real Fourier transform, the Fourier coefficients satisfy the
following relation
\begin{equation}
  \label{eq:fft-sym}
  z_k = \overline{z_{N-k}}, 
\end{equation}
where $N$ is the number of discretisation points.

A few remarks on the FFT of real functions and its inverse transformation :
\begin{itemize}
\item We only need half of the coefficients.
\item When a value is known to be real the imaginary part is not stored.
So the imaginary part of the zero-frequency component is never stored. It is
known to be zero.
\item For a sequence of even length the imaginary part of the frequency
  $n/2$ is not stored either, since the symmetry (\ref{eq:fft-sym}) implies
  that this is purely real too.
\end{itemize}


\paragraph{FFTPack storage}
\label{sec:fftpack-storage}

The functions use the fftpack storage convention for half-complex sequences.
In this convention, the half-complex transform of a real sequence is stored
with frequencies in increasing order, starting from zero, with the real and
imaginary parts of each frequency in neighboring locations.

The storage scheme is best shown by some examples. The table below shows the
output for an odd-length sequence, $n=5$.  The two columns give the
correspondence between the $5$ values in the half-complex sequence (stored in
a PnlVect $V$) and the values (PnlVectComplex $C$) that would be returned if
the same real input sequence were passed to pnl_dft_complex as a complex
sequence (with imaginary parts set to 0), 
\begin{equation}
  \begin{array}{l}
         C(0) =  V(0) + \imath 0, \\ 
         C(1) =  V(1) + \imath V(2), \\
         C(2) =  V(3) + \imath V(4), \\
         C(3) = V(3) - \imath V(4)=  \overline{C(2)} , \\
         C(4) = V(1) + \imath V(2)=  \overline{C(1)} 
  \end{array}   
\end{equation}

The elements of index greater than $N/2$ of the complex array, as $C(3)$
$C(4)$, are filled in using the symmetry condition.

The next table shows the output for an even-length sequence, $n=6$.
In the even case there are two values which are purely real, 
\begin{equation}
  \begin{array}{l}
         C(0) =  V(0) + \imath 0, \\ 
         C(1) =  V(1) + \imath V(2), \\
         C(2) =  V(3) + \imath V(4), \\
         C(3) = V(5) - \imath 0    =  \overline{C(0)} , \\
         C(4) = V(3) - \imath V(4) =  \overline{C(2)} , \\
         C(5) = V(1) + \imath V(2) =  \overline{C(1)} 
  \end{array}   
 \end{equation}


\subsubsection{Functions}


The following functions comes from a C version of the Fortran FFTPack library
available on \url{http://www.netlib.org/fftpack}.
\begin{itemize}
\item \describefun{int}{pnl_fft_inplace}{\refstruct{PnlVectComplex} \ptr data}
  \sshortdescribe Computes the FFT of \var{data} in place. The original content
  of \var{data} is lost.

\item \describefun{int}{pnl_ifft_inplace}{\refstruct{PnlVectComplex} \ptr data}
  \sshortdescribe Computes the inverse FFT of \var{data} in place. The
  original content of \var{data} is lost.

\item \describefun{int}{pnl_fft}{const \refstruct{PnlVectComplex} \ptr in, 
    \refstruct{PnlVectComplex} \ptr out}
  \sshortdescribe Computes the FFT of \var{in} and stores it into \var{out}.

\item \describefun{int}{pnl_ifft}{const \refstruct{PnlVectComplex} \ptr in, 
    \refstruct{PnlVectComplex} \ptr out}
  \sshortdescribe Computes the inverse FFT of \var{in} and stores it into \var{out}.

\item \describefun{int}{pnl_fft2}{double \ptr re, double \ptr im, int n}
  \sshortdescribe Computes the FFT of the vector of length \var{n} whose real
  (resp. imaginary) parts are given by the arrays \var{re}
  (resp. \var{im}). The real and imaginary parts of the FFT are respectively
  stored in \var{re} and \var{im} on output.

\item \describefun{int}{pnl_ifft2}{double \ptr re, double \ptr im, int n}
  \sshortdescribe Computes the inverse FFT of the vector of length \var{n}
  whose real (resp. imaginary) parts are given by the arrays \var{re}
  (resp. \var{im}). The real and imaginary parts of the inverse FFT are
  respectively stored in \var{re} and \var{im} on output.

\item \describefun{int}{pnl_real_fft}{const \refstruct{PnlVect} \ptr in, 
    \refstruct{PnlVectComplex} \ptr out}
  \sshortdescribe Computes the FFT of the real valued sequence \var{in} and
  stores it into \var{out}.

\item \describefun{int}{pnl_real_ifft}{const \refstruct{PnlVect} \ptr in, 
    \refstruct{PnlVectComplex} \ptr out}
  \sshortdescribe Computes the inverse FFT of \var{in} and stores it into \var{out}.

\item \describefun{int}{pnl_real_fft_inplace}{double \ptr data, int n}
  \sshortdescribe Computes the FFT of the real valued vector \var{data} of
  length \var{n}. The result is stored in \var{data} using the FFTPack storage
  described above, see~\ref{sec:fftpack-storage}.

\item \describefun{int}{pnl_real_ifft_inplace}{double \ptr data, int n}
  \sshortdescribe Computes the inverse FFT of the vector \var{data} of length
  \var{n}. \var{data} is supposed to be the FFT coefficients a real valued
  sequence stored using the FFTPack storage. On output, \var{data} contains
  the inverse FFT.

\item \describefun{int}{pnl_real_fft2}{double \ptr re, double \ptr im, int n}
  \sshortdescribe Computes the FFT of the real vector \var{re} of length \var{n}.
  \var{im} is only used on output to store the imaginary part the FFT. The
  real part is stored into \var{re}
  
\item \describefun{int}{pnl_real_ifft2}{double \ptr re, double \ptr im, int n}
  \sshortdescribe Computes the inverse FFT of the vector \var{re + i * im} of
  length \var{n}, which is supposed to be the FFT of a real valued
  sequence. On exit, \var{im} is unused. 
\end{itemize}

%% Laplace transform
\subsection{Inverse Laplace transform}
\subsubsection{Short Description}

For a real valued function $f$ such that $t \longmapsto f(t) \expp{- \sigma_c
  t}$ is integrable over $\R^+$, we can define its Laplace transform
\begin{equation*}
  \hat{f}(\lambda) = \int_0^\infty f(t) \expp{- \lambda t} dt \qquad
  \mbox{for $\lambda \in \C$ with $\real{\lambda} \ge \sigma_c$}.
\end{equation*}

\subsubsection{Functions}
\begin{itemize}
\item \describefun{double}{pnl_ilap_euler}{\refstruct{PnlCmplxFunc}
    \ptr fhat, double t, int N, int M}
  \sshortdescribe Computes $f(\var{t})$ where $f$ is given by its Laplace
  transform \var{fhat} by numerically inverting the Laplace transform using
  Euler's summation. The values \var{N = M = 15} usually give a very good
  accuracy. For more details on the accuracy of the method. 

\item \describefun{double}{pnl_ilap_cdf_euler}{\refstruct{PnlCmplxFunc}
    \ptr fhat, double t, int N, int M}
  \sshortdescribe Computes the cumulative distribution function $F(\var{t})$
  where $F(x) = \int_0^x f(t) dt$ and $f$ is a density function with values on
  the positive real linegiven by its Laplace transform \var{fhat}. The
  computation is carried out by numerical inversion of the Laplace transform
  using Euler's summation. The values \var{N = M = 15} usually give a very
  good accuracy. The parameter \var{h} is the discretisation step, the
  algorithm is very sensitive to the choice of \var{h}.

\item \describefun{double}{pnl_ilap_fft}{\refstruct{PnlVect} \ptr res,
    \refstruct{PnlCmplxFunc} \ptr fhat, double T, double eps}
  \sshortdescribe Computes $f(t)$ for $t \in [h, \var{T}]$ on a regular grid
  and stores the values in \var{res}, where $h = T / {\mathrm size}(res)$. The
  function $f$ is defined by its Laplace transform \var{fhat}, which is
  numerically inverted using a Fast Fourier Transform algorithm. The size of
  \var{res} is related to the choice of the relative precision \var{eps}
  required on the value of $f(t)$ for all $t \le T$.

\item \describefun{double}{pnl_ilap_gs}{\refstruct{PnlFunc} \ptr fhat, double
    t, int n}
  \sshortdescribe Computes $f(\var{t})$ where $f$ is given by its Laplace
  transform \var{fhat} by numerically inverting the Laplace transform using a
  weighted combination of different Gaver Stehfest's algorithms. Note that
  this function does not need the complex valued Laplace transform but only the
  real valued one. \var{n} is the number of terms used in the weighted combination.

\item \describefun{double}{pnl_ilap_gs_basic}{\refstruct{PnlFunc}
    \ptr fhat, double t, int n}
  \sshortdescribe Computes $f(\var{t})$ where $f$ is given by its Laplace
  transform \var{fhat} by numerically inverting the Laplace transform using
  Gaver Stehfest's method. Note that this function does not
  need the comple valued Laplace transform but only the real valued
  one. \var{n} is the number of iterations of the algorithm.
  {\bf Note : }~This function is provided only for test purposes, even though
  the function \reffun{pnl_ilap_gs} gives far more accurate results.
\end{itemize}

% %% pde tools
% \subsection{PDE tools}
% \subsubsection{Short Description}
% \subsubsection{Functions}
% \begin{itemize}
% \item 
% \describefun{\refstruct{PnlPDEBoundary}\ptr }{pnl_pde_boundary_create}{double X0, double X1}
%   \sshortdescribe creates a \refstruct{PnlPDEBoundary}  
% \item \describefun{double}{pnl_pde_boundary_real_variable}{const \refstruct{ PnlPDEBoundary} BP, double X}
% \item 
% \describefun{double}{pnl_pde_boundary_unit_interval}{const \refstruct{ PnlPDEBoundary} BP, double X}
% \item 
% \describefun{\refstruct{PnlPDEDimBoundary} \ptr }{pnl_pde_dim_boundary_create_from_int}{int dim}
%   \sshortdescribe creates a \refstruct{PnlPDEBoundary} with Left down corner is $ (0, \dots, 0)$ and right up corner is $ (1, \dots, 1)$  
% \item \describefun{\refstruct{ PnlPDEDimBoundary} \ptr }{pnl_pde_dim_boundary_create}{const \refstruct{ PnlVect} \ptr X0, const \refstruct{ PnlVect} \ptr X1}
%   \sshortdescribe 
% \item \describefun{void}{pnl_pde_dim_boundary_free}{\refstruct{ PnlPDEDimBoundary} \ptr \ptr \refstruct{ v}}
%   \sshortdescribe frees a \refstruct{PnlPDEDimBoundary}  
% \item \describefun{double}{pnl_pde_dim_boundary_eval_from_unit}{double(\ptr f)(const \refstruct{ PnlVect} \ptr ), const \refstruct{ PnlPDEDimBoundary} \ptr BP, const \refstruct{ PnlVect} \ptr X}
% \item 
% \describefun{void}{pnl_pde_dim_boundary_from_unit_to_real_variable}{const \refstruct{ PnlPDEDimBoundary} \ptr BP, \refstruct{ PnlVect} \ptr X}
% \item 
% \describefun{double}{pnl_pde_dim_boundary_get_step}{const \refstruct{ PnlPDEDimBoundary} \ptr BP, int i}
% \item 
% \describefun{double}{standard_time_repartition}{int i, int N-T}
% \item \describefun{\refstruct{ PnlPDETimeGrid} \ptr }{pnl_pde_time_grid}{const
%   double T, const int $N-T$, double(\ptr  repartition)(int i, int NN)}
%   \sshortdescribe creates a \refstruct{PnlPDETimeGrid}
% \item \describefun{\refstruct{ PnlPDETimeGrid} \ptr }{pnl_pde_time_homogen_grid}{const double T, const int N-T}
%   \sshortdescribe creates a \refstruct{PnlPDETimeGrid}
% \item \describefun{void}{pnl_pde_time_grid_free}{\refstruct{ PnlPDETimeGrid} \ptr \ptr TG}
%   \sshortdescribe frees a \refstruct{PnlPDETimeGrid}
% \item \describefun{void}{pnl_pde_time_start}{\refstruct{ PnlPDETimeGrid} \ptr TG}
%   \sshortdescribe initialise \refstruct{PnlPDETimeGrid}
% \item \describefun{int}{pnl_pde_time_grid_increase}{\refstruct{ PnlPDETimeGrid} \ptr TG}
%   \sshortdescribe go to the next time step  
% \item \describefun{double}{pnl_pde_time_grid_step}{const \refstruct{ PnlPDETimeGrid} \ptr TG}
%   \sshortdescribe GET function on current step.  
% \item \describefun{double}{pnl_pde_time_grid_time}{const \refstruct{ PnlPDETimeGrid} \ptr TG}
%   \sshortdescribe GET function on current time.
% \end{itemize}

