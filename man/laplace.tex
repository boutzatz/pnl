
\section{Inverse Laplace Transform}

For a real valued function $f$ such that $t \longmapsto f(t) \expp{- \sigma_c
  t}$ is integrable over $\R^+$, we can define its Laplace transform
\begin{equation*}
  \hat{f}(\lambda) = \int_0^\infty f(t) \expp{- \lambda t} dt \qquad
  \mbox{for $\lambda \in \C$ with $\real{\lambda} \ge \sigma_c$}.
\end{equation*}

To use the following functions, you should include \verb!pnl/pnl_laplace.h!.

\begin{itemize}
\item \describefun{double}{pnl_ilap_euler}{\refstruct{PnlCmplxFunc}
    \ptr fhat, double t, int N, int M}
  \sshortdescribe Computes $f(\var{t})$ where $f$ is given by its Laplace
  transform \var{fhat} by numerically inverting the Laplace transform using
  Euler's summation. The values \var{N = M = 15} usually give a very good
  accuracy. For more details on the accuracy of the method.

\item \describefun{double}{pnl_ilap_cdf_euler}{\refstruct{PnlCmplxFunc}
    \ptr fhat, double t, double h, int N, int M}
  \sshortdescribe Computes the cumulative distribution function $F(\var{t})$
  where $F(x) = \int_0^x f(t) dt$ and $f$ is a density function with values on
  the positive real linegiven by its Laplace transform \var{fhat}. The
  computation is carried out by numerical inversion of the Laplace transform
  using Euler's summation. The values \var{N = M = 15} usually give a very
  good accuracy. The parameter \var{h} is the discretization step, the
  algorithm is very sensitive to the choice of \var{h}.

\item \describefun{double}{pnl_ilap_fft}{\refstruct{PnlVect} \ptr res,
    \refstruct{PnlCmplxFunc} \ptr fhat, double T, double eps}
  \sshortdescribe Computes $f(t)$ for $t \in [h, \var{T}]$ on a regular grid
  and stores the values in \var{res}, where $h = T / {\mathrm size}(res)$. The
  function $f$ is defined by its Laplace transform \var{fhat}, which is
  numerically inverted using a Fast Fourier Transform algorithm. The size of
  \var{res} is related to the choice of the relative precision \var{eps}
  required on the value of $f(t)$ for all $t \le T$.

\item \describefun{double}{pnl_ilap_gs}{\refstruct{PnlFunc} \ptr fhat, double
    t, int n}
  \sshortdescribe Computes $f(\var{t})$ where $f$ is given by its Laplace
  transform \var{fhat} by numerically inverting the Laplace transform using a
  weighted combination of different Gaver Stehfest's algorithms. Note that
  this function does not need the complex valued Laplace transform but only the
  real valued one. \var{n} is the number of terms used in the weighted combination.

\item \describefun{double}{pnl_ilap_gs_basic}{\refstruct{PnlFunc}
    \ptr fhat, double t, int n}
  \sshortdescribe Computes $f(\var{t})$ where $f$ is given by its Laplace
  transform \var{fhat} by numerically inverting the Laplace transform using
  Gaver Stehfest's method. Note that this function does not
  need the comple valued Laplace transform but only the real valued
  one. \var{n} is the number of iterations of the algorithm.
  {\bf Note : }~This function is provided only for test purposes, even though
  the function \reffun{pnl_ilap_gs} gives far more accurate results.
\end{itemize}

% vim:spelllang=en:spell:
