\section{Linear Algebra}

% vector
\subsection{Vectors}
\subsubsection{Overview}

The structures and functions related to vectors are declared in
\verb!pnl/pnl_vector.h!.


Vectors are declared for several basic types : double, int, and
dcomplex. In the following declarations, {\tt BASE} must be replaced by one
the previous types and the corresponding vector structures are respectively
named PnlVect, PnlVectInt, PnlVectComplex
\describestruct{PnlVect}\describestruct{PnlVectInt}\describestruct{PnlVectComplex}
\begin{verbatim}
typedef struct _PnlVect {
  /**
   * Must be the first element in order for the object mechanism to work
   * properly. This allows any PnlVect pointer to be cast to a PnlObject
   */
  PnlObject object; 
  int size; /*!< size of the vector */
  int mem_size; /*!< size of the memory block allocated for array */
  double *array; /*!< pointer to store the data */
  int owner; /*!< 1 if the object owns its array member, 0 otherwise */
} PnlVect;

typedef struct _PnlVectInt {
  /**
   * Must be the first element in order for the object mechanism to work
   * properly. This allows any PnlVectInt pointer to be cast to a PnlObject
   */
  PnlObject object; 
  int size; /*!< size of the vector */ 
  int mem_size; /*!< size of the memory block allocated for array */
  int *array; /*!< pointer to store the data */
  int owner; /*!< 1 if the object owns its array member, 0 otherwise */
} PnlVectInt;

typedef struct _PnlVectComplex {
  /**
   * Must be the first element in order for the object mechanism to work
   * properly. This allows any PnlVectComplex pointer to be cast 
   * to a PnlObject
   */
  PnlObject object; 
  int size; /*!< size of the vector */ 
  int mem_size; /*!< size of the memory block allocated for array */
  dcomplex *array; /*!< pointer to store the data */
  int owner; /*!< 1 if the object owns its array member, 0 otherwise */
} PnlVectComplex;
\end{verbatim}
\var{size} is the size of the vector, \var{array} is a pointer containing the
data and \var{owner} is an integer to know if the vector owns its \var{array}
pointer (\var{owner=1}) or shares it with another structure (\var{owner=0}).
\var{mem_size} is the number of elements the vector can hold at most.

\subsubsection{Functions}

\paragraph{General functions}
These functions exist for all types of vector no matter what the basic type
is. The following conventions are used to name functions operating on vectors.
Here is the table of prefixes used for the different basic types.

\begin{center}
  \begin{tabular}[t]{lll}
    type & prefix & BASE\\
    \hline
    double & pnl_vect & double \\
    \hline
    int & pnl_vect_int & int \\
    \hline
    dcomplex & pnl_vect_complex & dcomplex
  \end{tabular}
\end{center}

In this paragraph, we present the functions operating on \PnlVect
which exist for all types. To deduce the prototypes of these functions for
other basic types, one must replace {\tt pnl_vect} and {\tt double} according
the above table. 
\subparagraph{Constructors and destructors}

There are no special functions to access the size of a vector, instead the field
\verb!size! should be accessed directly.

\begin{itemize}
  \item \describefun{\PnlVect \ptr }{pnl_vect_new}{}
  \sshortdescribe Create a new \PnlVect of size 0.  
\item \describefun{\PnlVect \ptr }{pnl_vect_create}{int size}
  \sshortdescribe Create a new \PnlVect pointer.  
\item \describefun{\PnlVect \ptr }{pnl_vect_create_from_zero}{int size}
  \sshortdescribe Create a new \PnlVect pointer and sets it to zero.  
\item \describefun{\PnlVect \ptr }{pnl_vect_create_from_scalar}
  {int size, double x}
  \sshortdescribe Create a new \PnlVect pointer and sets all
  elements t \var{x}.  
\item \describefun{\PnlVect \ptr }{pnl_vect_create_from_ptr}{int
    size, const double \ptr x}
  \sshortdescribe Create a new \PnlVect pointer and copies \var{x}
  to \var{array}.  
\item \describefun{\PnlVect \ptr }{pnl_vect_create_from_mat}{
  const PnlMat *M}
  \sshortdescribe Create a new \PnlVect pointer of size \var{M->mn}
  and copy the content of \var{M} row wise.
\item \describefun{\PnlVect \ptr }{pnl_vect_create_from_list}{int
    size, ...}
  \sshortdescribe Create a new \PnlVect pointer of length
  \var{size} filled with the extra arguments passed to the function. The
  number of extra arguments passed must be equal to \var{size} and they must be of the type BASE.
  Example: To create a vector \{1., 2.\}, you should enter pnl_vect_create_from_list(2, 1.0, 2.0) and NOT pnl_vect_create_from_list(2, 1.0, 2) or pnl_vect_create_from_list(2, 1, 2.0).
  Be aware that this cannot be checked inside the function.
\item \describefun{\PnlVect \ptr }{pnl_vect_create_from_file}
  {const char \ptr file}
  \sshortdescribe Read a vector from a file and creates the corresponding
  \PnlVect. The data might be stored as a single blank separated line
  or as a one column file with one element per line.

\item \describefun{\PnlVect \ptr }{pnl_vect_copy}{const
    \PnlVect \ptr v}
  \sshortdescribe This is a copying constructor. It creates a copy of a \PnlVect.
\item \describefun{void}{pnl_vect_clone}{\PnlVect \ptr clone, 
    const \PnlVect \ptr v} 
  \sshortdescribe Clone a \PnlVect. \var{clone} must be an
  already existing  \PnlVect. It is resized to match the size of
  \var{v} and the data are copied. Future modifications to \var{v} will not
  affect \var{clone}.

\item \describefun{\PnlVect \ptr }{pnl_vect_create_subvect_with_ind}{const
  \PnlVect \ptr V, const \PnlVectInt \ptr ind}
  \sshortdescribe Create a new vector containing \var{V(ind(:))}.

\item \describefun{void}{pnl_vect_extract_subvect_with_ind}{\PnlVect \ptr
  V_sub, const \PnlVect \ptr V, const \PnlVectInt \ptr ind}
  \sshortdescribe On exit, \var{V_sub = V(ind(:))}.

\item \describefun{\PnlVect \ptr }{pnl_vect_create_subvect}{const
    \PnlVect \ptr V, int i, int len}
  \sshortdescribe Create a new vector containing \var{V(i:i+len-1)}. The
  elements are copied.
  
\item \describefun{void}{pnl_vect_extract_subvect}{\PnlVect \ptr
  V_sub, const \PnlVect \ptr V, int i, int len}
  \sshortdescribe On exit, \var{V_sub = V(i:i+len-1)}.  The
  elements are copied.

\item \describefun{void}{pnl_vect_free}{\PnlVect \ptr\ptr v}
  \sshortdescribe Free a \PnlVect pointer and set the data pointer to NULL  
\item \describefun{\PnlVect}{pnl_vect_wrap_array}{const double \ptr x, 
    int size}
    \sshortdescribe Create a \PnlVect containing the data 
  \var{x}. No copy is made. It is just a container.
  
\item \describefun{\PnlVect}{pnl_vect_wrap_subvect}{const
  \PnlVect \ptr x, int i, int s}
  \sshortdescribe Create a \PnlVect containing
  \var{x(i:i+s-1)}. No copy is made. It is just a container. The returned
  \PnlVect has \var{size=s} and \var{owner=0}.

\item \describefun{\PnlVect}{pnl_vect_wrap_subvect_with_last}{const \PnlVect \ptr x, int i, int j}
  \sshortdescribe Create a \PnlVect containing \var{x(i:j)}. No
  copy is made. It is just a container.

\item \describefun{\PnlVect}{pnl_vect_wrap_mat}
  {const \PnlMat \ptr M}
  \sshortdescribe Return a \PnlVect (not a pointer) whose array is
  the row wise array of \var{M}. The new vector shares its data with the
  matrix \var{M}, which means that any modification to one of them will affect
  the other.
\end{itemize}

\subparagraph{Resizing vectors}
\begin{itemize}
\item \describefun{int}{pnl_vect_resize}{\PnlVect \ptr v, int size}
  \sshortdescribe Resize a \PnlVect. It copies as much of the old
  data to fit in the resized object.
\item \describefun{int}{pnl_vect_resize_from_ptr}{\PnlVect
    \ptr v, int size, double \ptr t} 
  \sshortdescribe Resize a \PnlVect and uses \var{t} to fill the
  vector. \var{t} must be of size \var{size}.
\end{itemize}  

\subparagraph{Accessing elements}

If it is supported by the compiler, the following functions are declared
inline. To speed up these functions, you can use the macro constant
\texttt{PNL_RANGE_CHECK_OFF}, see Section~\ref{sec:inline} for an explanation. 

\begin{itemize}
\item \describefun{void}{pnl_vect_set}{\PnlVect \ptr v, int i, double x}
  \sshortdescribe Set v[i]=x  
\item \describefun{double}{pnl_vect_get}{const \PnlVect \ptr v, int i}
  \sshortdescribe Return the value of v[i].  
\item \describefun{void}{pnl_vect_lget}{\PnlVect \ptr v, int i}
  \sshortdescribe Return the address of v[i].  
\item \describefun{void}{pnl_vect_set_all}{\PnlVect \ptr v, double x}
  \sshortdescribe Set all elements to x.  
\item \describefun{void}{pnl_vect_set_zero}{\PnlVect \ptr v}
  \sshortdescribe Set all elements to zero.  
\end{itemize}
Equivalently to these functions, there exist macros for {\bf \PnlVect only}.
\begin{itemize}
\item \describefun{}{GET}{v, i}
  \sshortdescribe Return \var{v[i]} for reading, eg. \var{x=LET(v,i)}
  
\item \describefun{}{LET}{v, i}
  \sshortdescribe Return \var{v[i]} as a lvalue for writing, eg.
  \var{LET(v,i)=x}
\end{itemize}


\subparagraph{Printing vector}
\begin{itemize}
\item \describefun{void}{pnl_vect_print}{const \PnlVect \ptr V}
  \sshortdescribe Print a \PnlVect as a column vector
\item \describefun{void}{pnl_vect_fprint}{FILE \ptr fic, const \PnlVect \ptr V}
  \sshortdescribe Print a \PnlVect in file \var{fic} as a column
  vector.
\item \describefun{void}{pnl_vect_print_asrow}{const \PnlVect \ptr V}
  \sshortdescribe Print a \PnlVect as a row vector
\item \describefun{void}{pnl_vect_fprint_asrow}{FILE \ptr fic, const \PnlVect \ptr V}
  \sshortdescribe Print a \PnlVect in file \var{fic} as a row
  vector.
\item \describefun{void}{pnl_vect_print_nsp}{const \PnlVect \ptr V}
  \sshortdescribe Print a vector to the standard output in a format
  compatible with Nsp.  
\item \describefun{void}{pnl_vect_fprint_nsp}{FILE \ptr fic, const
    \PnlVect \ptr V}
  \sshortdescribe Print a vector to a file in a format compatible with Nsp. 
\end{itemize}

\subparagraph{Applying external operation to vectors}

\begin{itemize}
\item \describefun{void}{pnl_vect_minus}{\PnlVect \ptr lhs}
  \sshortdescribe In-place unary minus
\item \describefun{void}{pnl_vect_plus_scalar}{\PnlVect \ptr lhs, double x}
  \sshortdescribe In-place vector scalar addition  
\item \describefun{void}{pnl_vect_minus_scalar}{\PnlVect \ptr lhs, double x}
  \sshortdescribe In-place vector scalar substraction  
\item \describefun{void}{pnl_vect_mult_scalar}{\PnlVect \ptr lhs, double x}
  \sshortdescribe In-place vector scalar multiplication  
\item \describefun{void}{pnl_vect_div_scalar}{\PnlVect \ptr lhs, double x}
  \sshortdescribe In-place vector scalar division  
\end{itemize}

\subparagraph{Element wise operations}

\begin{itemize}
\item \describefun{void}{pnl_vect_plus_vect}{\PnlVect \ptr lhs, 
    const \PnlVect \ptr rhs} 
  \sshortdescribe In-place vector vector addition  

\item \describefun{void}{pnl_vect_minus_vect}{\PnlVect \ptr lhs, 
    const \PnlVect \ptr rhs} 
  \sshortdescribe In-place vector vector substraction  

\item \describefun{void}{pnl_vect_inv_term}{\PnlVect \ptr lhs}
  \sshortdescribe In-place term by term vector inversion  

\item \describefun{void}{pnl_vect_div_vect_term}{\PnlVect
    \ptr lhs, const \PnlVect \ptr rhs} 
  \sshortdescribe In-place term by term vector division

\item \describefun{void}{pnl_vect_mult_vect_term}{\PnlVect
    \ptr lhs, const \PnlVect \ptr rhs} 
  \sshortdescribe In-place vector vector term by term multiplication  

\item \describefun{void}{pnl_vect_map}{\PnlVect \ptr lhs, const
    \PnlVect \ptr rhs, double(\ptr f)(double)} 
    \sshortdescribe \var{lhs = f(rhs)} 

\item \describefun{void}{pnl_vect_map_inplace}{\PnlVect \ptr lhs, double(\ptr f)(double)}
  \sshortdescribe \var{lhs = f(lhs)} 

\item \describefun{void}{pnl_vect_map_vect}{\PnlVect \ptr lhs, const
  \PnlVect \ptr rhs1, const \PnlVect \ptr rhs2,
  double(\ptr f)(double, double)} 
  \sshortdescribe \var{lhs = f(rhs1, rhs2)} 

\item \describefun{void}{pnl_vect_map_vect_inplace}{\PnlVect \ptr
  lhs, \PnlVect \ptr rhs, double(\ptr f)(double,double)}
  \sshortdescribe \var{lhs = f(lhs,rhs)} 

\item \describefun{void}{pnl_vect_axpby}{double a, const \PnlVect \ptr x, 
    double b, \PnlVect \ptr y} 
  \sshortdescribe Compute \var{y : = a x + b y}. When \var{b==0}, the content
  of \var{y} is not used on input and instead \var{y} is resized to match \var{x}.

\item \describefun{double}{pnl_vect_sum}{const \PnlVect \ptr lhs}
  \sshortdescribe Return the sum of all the elements of a vector  

\item \describefun{void}{pnl_vect_cumsum}{\PnlVect \ptr lhs}
  \sshortdescribe Compute the cumulative sum of all the elements of a
  vector. The original vector is modified

\item \describefun{double}{pnl_vect_prod}{const \PnlVect \ptr V}
  \sshortdescribe Return the product of all the elements of a vector  

\item \describefun{void}{pnl_vect_cumprod}{\PnlVect \ptr lhs}
  \sshortdescribe Compute the cumulative product of all the elements of a
  vector. The original vector is modified
\end{itemize}

\subparagraph{Scalar products and norms}
\begin{itemize}
\item \describefun{double}{pnl_vect_norm_two}{const \PnlVect \ptr V}
  \sshortdescribe Return the two norm of a vector  

\item \describefun{double}{pnl_vect_norm_one}{const \PnlVect \ptr V}
  \sshortdescribe Return the one norm of a vector  

\item \describefun{double}{pnl_vect_norm_infty}{const \PnlVect \ptr V}
  \sshortdescribe Return the infinity norm of a vector  

\item \describefun{double}{pnl_vect_scalar_prod}{const \PnlVect
    \ptr rhs1, const \PnlVect \ptr rhs2} 
  \sshortdescribe Compute the scalar product between 2 vectors  
\item \describefun{int}{pnl_vect_cross}{\PnlVect \ptr lhs, 
  const \PnlVect \ptr x, const \PnlVect \ptr y}
  \sshortdescribe Compute the cross product of \var{x} and \var{y} and store the
  result in \var{lhs}. The vectors \var{x} and \var{y} must be of size 3 and
  FAIL is returned otherwise.

\item \describefun{double}{pnl_vect_dist}{const \PnlVect \ptr x, const \PnlVect \ptr y}
  \sshortdescribe Compute the distance between \var{x} and \var{y}, ie
  $\sqrt{\sum_i |x_i - y_i|^2}$.

\end{itemize}

\subparagraph{Test functions}

\begin{itemize}
\item \describefun{int}{pnl_vect_eq}{const \PnlVect \ptr V1,
    const \PnlVect \ptr V2}
  \sshortdescribe Test if two vectors are equal. Returns \var{TRUE} or
  \var{FALSE}.
\item \describefun{int}{pnl_vect_eq_all}{const \PnlVect \ptr v,
    double x}
  \sshortdescribe Test if all the components of \var{v} are equal to
  \var{x}. Returns \var{TRUE} or \var{FALSE}.
\end{itemize}

\subparagraph{Ordering functions}
The following functions are not defined for PnlVectComplex because there is
no total ordering on Complex numbers

\begin{itemize}
\item \describefun{double}{pnl_vect_max}{const \PnlVect \ptr V}
  \sshortdescribe Return the maximum of a a vector  

\item \describefun{double}{pnl_vect_min}{const \PnlVect \ptr V}
  \sshortdescribe Return the minimum of a vector  

\item \describefun{void}{pnl_vect_minmax}{double \ptr m, double \ptr M, const \PnlVect \ptr}
  \sshortdescribe Compute the minimum and maximum of a vector which are
  returned in  \var{m} and \var{M} respectively.
  
\item \describefun{void}{pnl_vect_min_index}{double \ptr m, int \ptr im, const \PnlVect \ptr}
  \sshortdescribe Compute the minimum of a vector and its index stored in 
  sets \var{m} and \var{im} respectively.

\item \describefun{void}{pnl_vect_max_index}{double \ptr M, int \ptr iM, const \PnlVect \ptr}
  \sshortdescribe Compute the maximum of a vector and its index stored in 
  sets \var{m} and \var{im} respectively.

\item \describefun{void}{pnl_vect_minmax_index}{double \ptr m, double \ptr M,
    int \ptr im, int \ptr iM, const \PnlVect \ptr}
  \sshortdescribe Compute the minimum and maximum of a vector and the
  corresponding indices stored respectively in \var{m}, \var{M}, \var{im} and
  \var{iM}.

\item \describefun{void}{pnl_vect_qsort}{\PnlVect \ptr , char order}
  \sshortdescribe Sort a vector using a quick sort algorithm according to
  \var{order} (\verb!'i'! for increasing or \verb!'d'! for decreasing).

\item \describefun{void}{pnl_vect_qsort_index}{\PnlVect \ptr ,
    \PnlVectInt *index, char order}
  \sshortdescribe Sort a vector using a quick sort algorithm according to
  \var{order} (\verb!'i'! for increasing or \verb!'d'! for decreasing ). On
  output, \var{index} contains the permutation used to sort the vector.

\item \describefun{int}{pnl_vect_find}{\PnlVectInt \ptr
  ind, char \ptr type, int(\ptr f)(double \ptr t), \ldots}
  \sshortdescribe \var{f} is a function taking a C array as argument and
  returning an integer. \var{type} is a string composed by the letters 'r' and
  'v' and is used to describe the types of the arguments appearing after \var{f}.
  This function aims at simulating Scilab's \var{find}
  function. Here are a few examples (capital letters are used for vectors and
  small letters for real values)
  \begin{itemize}
    \item \verb!ind = find ( a < X )!
      \begin{verbatim}
      int isless ( double *t ) { return t[0] < t[1]; }
      pnl_vect_find ( ind, "rv", isless, a, X );
      \end{verbatim}
    \item \verb!ind = find (X <= Y)!
      \begin{verbatim}
      int isless ( double *t ) { return t[0] <= t[1]; }
      pnl_vect_find ( ind, "vv", isless, X, Y );
      \end{verbatim}
    \item \verb!ind = find ((a < X) && (X <= Y))!
      \begin{verbatim}
      int cmp ( double *t ) 
      { 
        return (t[0] <= t[1]) && (t[1] <= t[2]); 
      }
      pnl_vect_find ( ind, "rvv", cmp, a, X, Y );
      \end{verbatim}
  \end{itemize}
  \var{ind} contains on exit the indices \var{i} for which the function \var{f}
  returned \var{1}. This function returns \var{OK} or \var{FAIL} when something
  went wrong (size mismatch between matrices, invalid string type).

\end{itemize}


\subparagraph{Misc}

\begin{itemize}
\item \describefun{void}{pnl_vect_swap_elements}{\PnlVect \ptr v,
    int i, int j}
  \sshortdescribe Exchange \var{v[i]} and \var{v[j]}.
\item \describefun{void}{pnl_vect_reverse}{\PnlVect \ptr v}
  \sshortdescribe Perform a mirror operation on v. On output \var{v[i]
    = v[n-1-i]} for \var{i=0,\ldots,n-1} where \var{n} is the length of the vector.
\end{itemize}


\paragraph{Complex vector functions}

\begin{itemize}
\item \describefun{void}{pnl_vect_complex_mult_double}
  {\PnlVectComplex \ptr lhs, double x}
  \sshortdescribe In-place multiplication by a double.

\item \describefun{PnlVectComplex\ptr }{pnl_vect_complex_create_from_array}{int
    size, const double \ptr re, const double \ptr im}
  \sshortdescribe Create a \PnlVectComplex given the arrays of the
  real parts \var{re} and imaginary parts \var{im}.
\item \describefun{void}{pnl_vect_complex_split_in_array}{const \PnlVectComplex
    \ptr v, double \ptr re, double \ptr im}
    \sshortdescribe Split a complex vector into two C arrays : the
    real parts of the elements of \var{v} are stored into \var{re} and the
    imaginary parts into \var{im}.
\item \describefun{void}{pnl_vect_complex_split_in_vect}{const \PnlVectComplex
    \ptr v, \PnlVect \ptr re, \PnlVect \ptr im}
  \sshortdescribe Split a complex vector into two \PnlVect's : the
  real parts of the elements of \var{v} are stored into \var{re} and the
imaginary parts into \var{im}.
\end{itemize}

There exist functions to directly access the real or imaginary parts of an
element of a complex vector. These functions also have inlined versions that
are used if the variable \var{HAVE_INLINE} was declared at compilation time.

\begin{itemize}
\item \describefun{double}{pnl_vect_complex_get_real}
  {const \PnlVectComplex \ptr v, int i}
  \sshortdescribe Return the real part of \var{v[i]}.
  
\item \describefun{double}{pnl_vect_complex_get_imag}
  {const \PnlVectComplex \ptr v, int i}
  \sshortdescribe Return the imaginary part of \var{v[i]}.

\item \describefun{double\ptr }{pnl_vect_complex_lget_real}
  {const \PnlVectComplex \ptr v, int i}
  \sshortdescribe Return the real part of \var{v[i]} as a lvalue.

\item \describefun{double\ptr }{pnl_vect_complex_lget_imag}
  {const \PnlVectComplex \ptr v, int i}
  \sshortdescribe Return the imaginary part of \var{v[i]} as a lvalue.

\item \describefun{void}{pnl_vect_complex_set_real}
  {const \PnlVectComplex \ptr v, int i, double re}
  \sshortdescribe Set the real part of \var{v[i]} to \var{re}.

\item \describefun{void}{pnl_vect_complex_set_imag}
  {const \PnlVectComplex \ptr v, int i, double im}
  \sshortdescribe Set the imaginary part of \var{v[i]} to \var{im}.
\end{itemize}

Equivalently to these functions, there exist macros. When the compiler is able
to handle inline code, there is no gain in using macros instead of inlined
functions at least in principle.
\begin{itemize}
\item \describefun{}{GET_REAL}{v, i}
  \sshortdescribe Return the real part of \var{v[i]}.
  
\item \describefun{}{GET_IMAG}{v, i}
  \sshortdescribe Return the imaginary part of \var{v[i]}.
  
\item \describefun{}{LET_REAL}{v, i}
  \sshortdescribe Return the real part of \var{v[i]} as a lvalue.
  
\item \describefun{}{LET_IMAG}{v, i}
  \sshortdescribe Return the imaginary part of \var{v[i]} as a lvalue.
\end{itemize}

\subsection{Compact Vectors}
\subsubsection{Short description}

\describestruct{PnlVectCompact}
\begin{verbatim}
typedef struct PnlVectCompact {
  /**
   * Must be the first element in order for the object mechanism to work
   * properly. This allows any PnlVectCompact pointer to be cast to a PnlObject
   */
  PnlObject object; 
  int size; /* size of the vector */
  double val; /* single value */
  double *array; /* Pointer to double values */
  char convert; /* 'a', 'd' : array, double */
} PnlVectCompact;
\end{verbatim}

\subsubsection{Functions}

\begin{itemize}
  \item \describefun{\PnlVectCompact \ptr }{pnl_vect_compact_new}{}
  \sshortdescribe Create a \PnlVectCompact of size 0.  

\item \describefun{\PnlVectCompact \ptr }{pnl_vect_compact_create}{int n, double x}
  \sshortdescribe Create a \PnlVectCompact filled in with \var{x}
\item \describefun{\PnlVectCompact \ptr }{pnl_vect_compact_create}{int n, double *x}
  \sshortdescribe Create a \PnlVectCompact filled in with the content of
  \var{x}. Note that \var{x} must have at least \var{n} elements.

\item \describefun{int}{pnl_vect_compact_resize}{\PnlVectCompact
    \ptr v, int size, double x} 
  \sshortdescribe Resize a \PnlVectCompact.  

\item \describefun{\PnlVectCompact
    \ptr }{pnl_vect_compact_copy} {const \PnlVectCompact\ptr v}
  \sshortdescribe Copy a \PnlVectCompact  

\item \describefun{void}{pnl_vect_compact_free}{\PnlVectCompact \ptr \ptr v}
  \sshortdescribe Free a \PnlVectCompact  

\item \describefun{\PnlVect \ptr }{pnl_vect_compact_to_pnl_vect}
  {const \PnlVectCompact \ptr C} 
  \sshortdescribe Convert a \PnlVectCompact pointer to a \PnlVect pointer.  

\item \describefun{double}{pnl_vect_compact_get}{const \PnlVectCompact \ptr C, int i}
  \sshortdescribe Access function

\item \describefun{void}{pnl_vect_compact_set_all}{\PnlVectCompact
    \ptr C, double x}
  \sshortdescribe Set all elements of \var{C} to \var{x}. \var{C} is
  converted to a compact storage.
  
\item \describefun{void}{pnl_vect_compact_set_ptr}{\PnlVectCompact
    \ptr C, double \ptr ptr}
  \sshortdescribe Copy the array \var{ptr} into \var{C}. We assume that the
  sizes match. \var{C} is converted to a non compact storage.
\end{itemize}

%% matrix

\subsection{Matrices}
\subsubsection{Overview}

The structures and functions related to matrices are declared in
\verb!pnl/pnl_matrix.h!.

\describestruct{PnlMat}\describestruct{PnlMatInt}\describestruct{PnlMatComplex}
\begin{verbatim}
typedef struct _PnlMat{
  /**
   * Must be the first element in order for the object mechanism to work
   * properly. This allows any PnlMat pointer to be cast to a PnlObject
   */
  PnlObject object; 
  int m; /*!< nb rows */ 
  int n; /*!< nb columns */ 
  int mn; /*!< product m*n */
  int mem_size; /*!< size of the memory block allocated for array */
  double *array; /*!< pointer to store the data row-wise */
  int owner; /*!< 1 if the object owns its array member, 0 otherwise */
} PnlMat;

typedef struct _PnlMatInt{
  /**
   * Must be the first element in order for the object mechanism to work
   * properly. This allows any PnlMatInt pointer to be cast to a PnlObject
   */
  PnlObject object; 
  int m; /*!< nb rows */ 
  int n; /*!< nb columns */ 
  int mn; /*!< product m*n */
  int mem_size; /*!< size of the memory block allocated for array */
  int *array; /*!< pointer to store the data row-wise */
  int owner; /*!< 1 if the object owns its array member, 0 otherwise */
} PnlMatInt;

typedef struct _PnlMatComplex{
  /**
   * Must be the first element in order for the object mechanism to work
   * properly. This allows any PnlMatComplex pointer to be cast 
   * to a PnlObject
   */
  PnlObject object; 
  int m; /*!< nb rows */ 
  int n; /*!< nb columns */ 
  int mn; /*!< product m*n */
  int mem_size; /*!< size of the memory block allocated for array */
  dcomplex *array; /*!< pointer to store the data row-wise */
  int owner; /*!< 1 if the object owns its array member, 0 otherwise */
} PnlMatComplex;
\end{verbatim}
\var{m} is the number of rows, \var{n} is the number of columns. \var{array}
is a pointer containing the data of the matrix stored line wise, The element
\verb!(i, j)! of the matrix is \verb!array[i*m+j]!. \var{owner} is an integer to
know if the matrix owns its \var{array} pointer (\var{owner=1}) or shares it
with another structure (\var{owner=0}). \var{mem_size} is the number of
elements the matrix can hold at most.

The following operations are implemented on matrices and vectors. \var{alpha}
and \var{beta} are numbers, \var{A} and \var{B} are matrices and \var{x}
and \var{y} are vectors.
\begin{tabular}{ll}
  \reffun{pnl_mat_axpy} & \var{B := alpha * A + B} \\
  \reffun{pnl_mat_scalar_prod} & \var{x' A y} \\
  \reffun{pnl_mat_dgemm} & \var{C := alpha * op (A) * op (B) + beta * C}\\
  \reffun{pnl_mat_mult_vect_transpose_inplace} & \var{y = A' * x}\\
  \reffun{pnl_mat_mult_vect_inplace} & \var{y = A * x}\\
  \reffun{pnl_mat_lAxpby} & \var{y := alpha * A * x + beta * y}\\
  \reffun{pnl_mat_dgemv} & \var{y := alpha * op (A) * x + beta * y}\\
  \reffun{pnl_mat_dger} & \var{A := alpha x * y' + A}
\end{tabular}


\subsubsection{Generic Functions}
These functions exist for all types of matrices no matter what the basic type
is. The following conventions are used to name functions operating on matrices.
Here is the table of prefixes used for the different basic types.

\begin{center}
  \begin{tabular}[t]{lll}
    type & prefix & BASE\\
    \hline
    double & pnl_mat & double \\
    \hline
    int & pnl_mat_int & int \\
    \hline
    dcomplex & pnl_mat_complex & dcomplex
  \end{tabular}
\end{center}

In this paragraph we present the functions operating on \PnlMat
which exist for all types. To deduce the prototypes of these functions for
other basic types, one must replace {\tt pnl_mat} and {\tt double} according
the above table.

\paragraph{Constructors and destructors}


There are no special functions to access the sizes of a matrix, instead the fields
\verb!m!, \verb!n! and \verb!mn! give direct access to the number of rows, columns
and the size of the matrix.

\begin{itemize}
  \item \describefun{\PnlMat \ptr }{pnl_mat_new}{}
  \sshortdescribe Create a \PnlMat of size 0

\item \describefun{\PnlMat \ptr }{pnl_mat_create}{int m, int n}
  \sshortdescribe Create a \PnlMat  with \var{m} rows and \var{n} columns.

\item \describefun{\PnlMat \ptr }{pnl_mat_create_from_scalar}{int m, int n, double x}
  \sshortdescribe Create a \PnlMat with \var{m} rows and \var{n}
  columns and sets all the elements to \var{x}.

\item \describefun{\PnlMat \ptr }{pnl_mat_create_from_zero}{int m, int n}
  \sshortdescribe Create a \PnlMat with \var{m} rows and \var{n}
  columns and sets all elements to 0.

\item \describefun{\PnlMat \ptr }{pnl_mat_create_from_ptr}{int m, int n, const double \ptr x}
  \sshortdescribe Create a \PnlMat with \var{m} rows and \var{n}
  columns and copies the array \var{x} to the new vector. Be sure that \var{x}
  is long enough to fill all the vector because it cannot be checked inside the function.

\item \describefun{\PnlMat \ptr }{pnl_mat_create_from_list}{int
    m, int n, ...}
  \sshortdescribe Create a new \PnlMat pointer of size \var{m
    x n} filled with the extra arguments passed to the function. The
  number of extra arguments passed must be equal to \var{m x n}, be
  aware that this cannot be checked inside the function.

\item \describefun{\PnlMat \ptr }{pnl_mat_copy}{const \PnlMat \ptr M}
  \sshortdescribe Create a new \PnlMat which is a copy of \var{M}.
  
\item \describefun{\PnlMat \ptr }{pnl_mat_create_diag_from_ptr}
  {const double \ptr x, int d}
  \sshortdescribe Create a new squared \PnlMat by specifying its size and
  diagonal terms as an array.

\item \describefun{\PnlMat \ptr }{pnl_mat_create_diag}
  {const \PnlVect \ptr V}
  \sshortdescribe Create a new squared \PnlMat by specifying its diagonal
  terms in a \PnlVect.

\item \describefun{\PnlMat \ptr }{pnl_mat_create_from_file}{const char \ptr file}
  \sshortdescribe Read a matrix from a file and creates the corresponding
  \PnlMat. The following conventions are used for the storage in a
  file:
  \begin{itemize}
  \item one row of the matrix corresponds to one line of the file
  \item the elements of a row should be separated by blanks (spaces or tabs) and
    nothing else (no comma or semi-colon separators are detected).
  \end{itemize}

\item \describefun{void}{pnl_mat_free}{\PnlMat \ptr \ptr M}
  \sshortdescribe Free a \PnlMat and sets \var{\ptr M} to \var{NULL} 
\item \describefun{\PnlMat}{pnl_mat_wrap_array}{const double \ptr x, 
    int m, int n}
    \sshortdescribe Create a \PnlMat of size \var{m x n} 
    which contains \var{x}. No copy is made. It is just a container.
\item \describefun{\PnlMat}{pnl_mat_wrap_vect}
  {const \PnlVect \ptr V}
  \sshortdescribe Return a \PnlMat (not a pointer) whose array is
  the array of \var{V}. The new matrix shares its data with the
  vector \var{V}, which means that any modification to one of them will affect
  the other.


\item \describefun{void}{pnl_mat_clone}{\PnlMat \ptr clone, const
    \PnlMat \ptr M}
  \sshortdescribe Clone \var{M} into \var{clone}. No no new
  \PnlMat is created.

\item \describefun{int}{pnl_mat_resize}{\PnlMat \ptr M, int m, int n}
  \sshortdescribe Resize a \PnlMat. The new matrix is of size
  \var{m x n}. The old data are lost.
\item \describefun{\PnlVect \ptr }{pnl_vect_create_submat}{const
  \PnlMat \ptr M, const \PnlVectInt \ptr indi, const
  \PnlVectInt \ptr indj}
  \sshortdescribe Create a new vector containing the values \var{M(indi(:),
  indj(:))}. \var{indi} and \var{indj} must be of the same size.

\item \describefun{void}{pnl_vect_extract_submat}{\PnlVect \ptr
  V_sub, const \PnlMat \ptr M, const \PnlVectInt \ptr indi,
  const \PnlVectInt \ptr indj}
  \sshortdescribe On exit, \var{V_sub = M(indi(:), indj(:))}. \var{indi} and
  \var{indj} must be of the same size.

\item \describefun{void}{pnl_mat_extract_subblock}{\PnlMat \ptr
    M_sub, const \PnlMat \ptr M, int i, int len_i, int j, int
    len_j}
  \sshortdescribe \var{M_sub = M(i:i+len_i-1, j:j+len_j-1)}. \var{len_i}
  (resp. \var{len_j}) is the number of rows (resp. columns) to be extracted.
  
\item \describefun{void}{pnl_mat_set_subblock}{\PnlMat \ptr
    M, const \PnlMat \ptr block, int i, int j}
    \sshortdescribe If \var{block} is a matrix of size \var{m_block x n_block},
    the dimensions of \var{M} must satisfy that \var{M->m <= i + m_block}
    \var{M->n <= j + n_block}. On output \var{M(i:i+m_block-1, j:j+n_block-1) =
    block}. 
\end{itemize}  


\paragraph{Accessing elements.}

If it is supported by the compiler, the following functions are declared
inline. To speed up these functions, you can use the macro constant
\texttt{PNL_RANGE_CHECK_OFF}, see Section~\ref{sec:inline} for an explanation. 
\begin{itemize}
\item \describefun{void}{pnl_mat_set}{\PnlMat \ptr M, int i, int j, double x}
  \sshortdescribe Set the value of M[i, j]=x  

\item \describefun{double}{pnl_mat_get}{const \PnlMat \ptr M, int i, int j}
  \sshortdescribe Get the value of M[i, j]  

\item \describefun{double \ptr }{pnl_mat_lget}{\PnlMat \ptr M, int i, int j}
  \sshortdescribe Return the address of M[i, j] for use as a lvalue.

\item \describefun{void}{pnl_mat_set_all}{\PnlMat \ptr M, double x}
  \sshortdescribe Set all elements of \var{M} to \var{x}.
\item \describefun{void}{pnl_mat_set_zero}{\PnlMat \ptr M}
  \sshortdescribe Set all elements of \var{M} to \var{0}.
  
\item \describefun{void}{pnl_mat_set_id}{\PnlMat \ptr M}
  \sshortdescribe Set the matrix \var{M} to the identity
  matrix. \var{M} must be a square matrix.

\item \describefun{void}{pnl_mat_set_diag}{\PnlMat \ptr M,
    double x, int d}
  \sshortdescribe Set the $\var{d}^{\text{th}}$ diagonal terms of the matrix
  \var{M} to the value \var{x}. \var{M} must be a square matrix.
\item \describefun{void}{pnl_mat_set_from_ptr}{\PnlMat \ptr M,
  const double \ptr x}
  \sshortdescribe Set \var{M} row--wise with the values given by \var{x}. The
  array \var{x} must be at least M->mn long.
\item \describefun{void}{pnl_mat_get_row}{\PnlVect
    \ptr V, const \PnlMat \ptr M, int i}
  \sshortdescribe Extract and copies the \var{i}-th row of \var{M} into
  \var{V}.

\item \describefun{void}{pnl_mat_get_col}{\PnlVect \ptr V, 
    const \PnlMat \ptr M, int j}
  \sshortdescribe Extract and copies the \var{j}-th column of \var{M} into \var{V}.
  
\item \describefun{\PnlVect}{pnl_vect_wrap_mat_row}
  {const \PnlMat \ptr M, int i}
  \sshortdescribe Return a \PnlVect (not a pointer) whose array is
  the \var{i}-th row of \var{M}. The new vector shares its data with the
  matrix \var{M}, which means that any modification to one of them will affect
  the other.
  
\item \describefun{void}{pnl_mat_swap_rows}{\PnlMat \ptr M, int i, int j}
  \sshortdescribe Swap two rows of a matrix.  

\item \describefun{void}{pnl_mat_set_col}{\PnlMat \ptr M, 
    const \PnlVect \ptr V, int j}
  \sshortdescribe Replace the \var{i}-th column of a matrix M by a vector V 

\item \describefun{void}{pnl_mat_set_row}{\PnlMat \ptr M, 
    const \PnlVect \ptr V, int i}
  \sshortdescribe Replace the \var{i}-th row of a matrix M by a vector V  

\item \describefun{void}{pnl_mat_add_row}{\PnlMat \ptr M, int i, const \PnlVect \ptr r}
  \sshortdescribe Add a row in matrix \var{M} before position \var{i} and fill
  it with the content of \var{r}. If \var{r == NULL}, row \var{i} is left
  uninitialized. The index \var{i} may vary between \var{0} --- add a row at the
  top of the matrix --- and \var{M->m} --- add a row after all rows.

\item \describefun{void}{pnl_mat_del_row}{\PnlMat \ptr M, int i}
  \sshortdescribe Delete the row with index \var{i} (between \var{0} and
  \var{M->m-1}) of the matrix \var{M}.

\end{itemize}

Equivalently to the functions \reffun{pnl_mat_get} and \reffun{pnl_mat_set},
there exist macros for {\bf \PnlMat only}.
\begin{itemize}
\item \describefun{}{MGET}{M, i, j}
  \sshortdescribe Return \var{M[i,j]} for reading, eg. \var{x=M(i,j)}
  
\item \describefun{}{MLET}{M, i, j}
  \sshortdescribe Return \var{M[i,j]} as a lvalue for writing, eg. eg. \var{x=M(i,j)}
\end{itemize}


\paragraph{Printing Matrices}

\begin{itemize}
\item \describefun{void}{pnl_mat_print}{const \PnlMat \ptr M}
  \sshortdescribe Print a matrix to the standard output.  

\item \describefun{void}{pnl_mat_fprint}{FILE \ptr fic, const \PnlMat \ptr M}
  \sshortdescribe Print a matrix to a file.

\item \describefun{void}{pnl_mat_print_nsp}{const \PnlMat \ptr M}
  \sshortdescribe Print a matrix to the standard output in a format
  compatible with Nsp.  

\item \describefun{void}{pnl_mat_fprint_nsp}{FILE \ptr fic, const
    \PnlMat \ptr M}
  \sshortdescribe Print a matrix to a file in a format compatible with Nsp. The
  saved matrix can be reloaded by the function
  \reffun{pnl_mat_create_from_file}.
\end{itemize}

\paragraph{Applying external operations}
\begin{itemize}
  \item \describefun{void}{pnl_mat_plus_scalar}{\PnlMat \ptr lhs, double x}
  \sshortdescribe In-place matrix scalar addition  

\item \describefun{void}{pnl_mat_minus_scalar}{\PnlMat \ptr lhs, double x}
  \sshortdescribe In-place matrix scalar substraction  

\item \describefun{void}{pnl_mat_mult_scalar}{\PnlMat \ptr lhs, double x}
  \sshortdescribe In-place matrix scalar multiplication  

\item \describefun{void}{pnl_mat_div_scalar}{\PnlMat \ptr lhs, double x}
  \sshortdescribe In-place matrix scalar division  

\end{itemize}

\paragraph{Element wise operations}

\begin{itemize}
\item \describefun{void}{pnl_mat_mult_mat_term}{\PnlMat \ptr lhs, 
    const \PnlMat \ptr rhs} 
  \sshortdescribe In-place matrix matrix term by term product  

\item \describefun{void}{pnl_mat_div_mat_term}{\PnlMat \ptr lhs, 
    const \PnlMat \ptr rhs} 
  \sshortdescribe In-place matrix matrix term by term division

\item \describefun{void}{pnl_mat_map_inplace}{\PnlMat \ptr lhs, 
    double(\ptr f)(double)} 
  \sshortdescribe \var{lhs = f(lhs)}.


\item \describefun{void}{pnl_mat_map}{\PnlMat \ptr lhs, const
    \PnlMat \ptr rhs, double(\ptr f)(double)} 
    \sshortdescribe \var{lhs = f(rhs)}.

  \item \describefun{void}{pnl_mat_map_mat_inplace}{\PnlMat \ptr lhs, const
  \PnlMat \ptr rhs, double(\ptr f)(double, double)} 
  \sshortdescribe \var{lhs = f(lhs, rhs)}.

\item \describefun{void}{pnl_mat_map_mat}{\PnlMat \ptr lhs, const
  \PnlMat \ptr rhs1, const
  \PnlMat \ptr rhs2, double(\ptr f)(double, double)} 
  \sshortdescribe \var{lhs = f(rhs1, rhs2)}.


\item \describefun{double}{pnl_mat_sum}{const \PnlMat \ptr lhs}
  \sshortdescribe Sum matrix component-wise  

\item \describefun{void}{pnl_mat_sum_vect}{\PnlVect \ptr y, const
    \PnlMat \ptr A, char c}
  \sshortdescribe Sum matrix column or row wise. Argument \var{c} can be
  either 'r' (to get a row vector) or 'c' (to get a column vector). When
  \var{c='r'}, $y(j) = \sum_i A_{ij}$ and when \var{c='rc}, $y(i) = \sum_j
  A_{ij}$.

\item \describefun{void}{pnl_mat_cumsum}{\PnlMat \ptr A, char c} 
  \sshortdescribe Cumulative sum over the rows or columns. Argument \var{c}
  can be either 'r' to sum over the rows or 'c' to sum over the columns. When
  \var{c='r'}, $A_{ij} = \sum_{1 \le k \le i} A_{kj}$ and when \var{c='rc}, 
  $A_{ij} = \sum_{1 \le k \le j} A_{ik}$.

\item \describefun{double}{pnl_mat_prod}{const \PnlMat \ptr lhs}
  \sshortdescribe Product matrix component-wise

\item \describefun{void}{pnl_mat_prod_vect}{\PnlVect \ptr y, const
    \PnlMat \ptr A, char c}
  \sshortdescribe Prod matrix column or row wise. Argument \var{c} can be
  either 'r' (to get a row vector) or 'c' (to get a column vector). When
  \var{c='r'}, $y(j) = \prod_i A_{ij}$ and when \var{c='rc}, $y(i) = \prod_j
  A_{ij}$.

\item \describefun{void}{pnl_mat_cumprod}{\PnlMat \ptr A, char c} 
  \sshortdescribe Cumulative prod over the rows or columns. Argument \var{c}
  can be either 'r' to prod over the rows or 'c' to prod over the columns. When
  \var{c='r'}, $A_{ij} = \prod_{1 \le k \le i} A_{kj}$ and when \var{c='rc}, 
  $A_{ij} = \prod_{1 \le k \le j} A_{ik}$.
\end{itemize}

\subparagraph{Test functions}

\begin{itemize}
\item \describefun{int}{pnl_mat_eq}{const \PnlMat \ptr M1,
    const \PnlMat \ptr M2}
  \sshortdescribe Test if two matrices are equal. Returns \var{TRUE} or
  \var{FALSE}.
\item \describefun{int}{pnl_mat_eq_all}{const \PnlMat \ptr M,
    double x}
  \sshortdescribe Test if all the components of \var{M} are equal to
  \var{x}. Returns \var{TRUE} or \var{FALSE}.
\end{itemize}


\paragraph{Ordering operations}

\begin{itemize}
\item \describefun{void}{pnl_mat_max}{ \PnlVect \ptr M, const
    \PnlMat \ptr A, char d}
  \sshortdescribe On exit, $\var{M}(i) = \max_{j}(\var{A}(i, j))$ when \var{d='c'}
  and $\var{M}(i) = \max_{j}(\var{A}(j, i))$ when \var{d='r'} and $\var{M(0)} =
  \max_{i,j} = \var{A}(i, j)$ when \var{d='*'}.

\item \describefun{void}{pnl_mat_min}{ \PnlVect \ptr m,const
    \PnlMat \ptr A, char d}
  \sshortdescribe On exit, $\var{m}(i) = \min_{j}(\var{A}(i, j))$ when \var{d='c'}
  and $\var{m}(i) = \min_{j}(\var{A}(j, i))$ when \var{d='r'} and $\var{M(0)} =
  \min_{i,j} = \var{A}(i, j)$ when \var{d='*'}.

\item \describefun{void}{pnl_mat_minmax}{ \PnlVect \ptr m,
    \PnlVect \ptr M, const \PnlMat \ptr A, char d}
  \sshortdescribe On exit, $\var{m}(i) = \min_{j}(\var{A}(i, j))$ and $\var{M}(i) =
  \max_{j}(\var{A}(i, j))$ when \var{d='c'} and $\var{m}(i) = \min_{j}(\var{A}(j, i))$
  and $\var{M}(i) = \min_{j}(\var{A}(j, i))$ when \var{d='r'} and $\var{M(0)} =
  \max_{i,j} = \var{A}(i, j)$ and  $\var{m(0)} =\min_{i,j} = \var{A}(i, j)$ when \var{d='*'}.
  
\item \describefun{void}{pnl_mat_min_index}{ \PnlVect \ptr m,
    \PnlVectInt \ptr im, const \PnlMat \ptr  A, char d}
  \sshortdescribe Idem as \reffun{pnl_mat_min} and \var{index} contains the
  indices of the minima. If \var{index==NULL}, the indices are not computed.

\item \describefun{void}{pnl_mat_max_index}{ \PnlVect \ptr M,
    \PnlVectInt \ptr iM, const \PnlMat \ptr  A, char d}
  \sshortdescribe Idem as \reffun{pnl_mat_max} and \var{index} contains the
  indices of the maxima. If \var{index==NULL}, the indices are not computed.

\item \describefun{void}{pnl_mat_minmax_index}{ \PnlVect \ptr m,
    \PnlVect \ptr M, \PnlVectInt \ptr im,
    \PnlVectInt \ptr iM, const \PnlMat \ptr A, char d}
  \sshortdescribe Idem as \reffun{pnl_mat_minmax} and \var{im} contains the
  indices of the minima and \var{iM} contains the indices of the minima. If
  \var{im==NULL} (resp. \var{iM==NULL}, the indices of the minima
  (resp. maxima) are not computed.

\item \describefun{void}{pnl_mat_qsort}{\PnlMat \ptr , char dir, char order}
  \sshortdescribe Sort a matrix using a quick sort algorithm according to
  \var{order} (\verb!'i'! for increasing or \verb!'d'! for decreasing). The parameter \var{dir} determines
  whether the matrix is sorted by rows or columns. If \var{dir='c'}, each row
  is sorted independently of the others whereas if \var{dir='r'}, each column
  is sorted independently of the others.

\item \describefun{void}{pnl_mat_qsort_index}{\PnlMat \ptr ,
    \PnlMatInt *index, char dir, char order}
  \sshortdescribe Sort a matrix using a quick sort algorithm according to
  \var{order} (\verb!'i'! for increasing or \verb!'d'! for decreasing). The
  parameter \var{dir} determines whether the matrix is sorted by rows or
  columns. If \var{dir='c'}, each row is sorted independently of the others
  whereas if \var{dir='r'}, each column is sorted independently of the
  others. In addition to the function \reffun{pnl_mat_qsort}, the permutation
  index is computed and stored into \var{index}.

\item \describefun{int}{pnl_mat_find}{\PnlVectInt \ptr
  indi, \PnlVectInt indj, char \ptr type, int(\ptr f)(double \ptr t), \ldots}
  \sshortdescribe \var{f} is a function taking a C array as argument and
  returning an integer. \var{type} is a string composed by the letters 'r' and
  'm' and is used to describe the types of the arguments appearing after \var{f}
  : 'r' for real numbers and 'm' for matrices.
  This function aims at simulating Scilab's \var{find}
  function. Here are a few examples (capital letters are used for matrices and
  small letters for real values)
  \begin{itemize}
    \item \verb![indi, indj] = find ( a < X )!
      \begin{verbatim}
      int isless ( double *t ) { return t[0] < t[1]; }
      pnl_mat_find ( indi, indj, "rm", isless, a, X );
      \end{verbatim}
    \item \verb!ind = find (X <= Y)!
      \begin{verbatim}
      int isless ( double *t ) { return t[0] <= t[1]; }
      pnl_mat_find ( ind, "mm", isless, X, Y );
      \end{verbatim}
    \item \verb![indi, indj] = find ((a < X) && (X <= Y))!
      \begin{verbatim}
      int cmp ( double *t ) 
      { 
        return (t[0] <= t[1]) && (t[1] <= t[2]); 
      }
      pnl_mat_find ( indi, indj, "rmm", cmp, a, X, Y );
      \end{verbatim}
  \end{itemize}
  \var{(indi, indj)} contains on exit the indices \var{(i,j)} for which the function \var{f}
  returned \var{1}. Note that if \var{indj == NULL} on entry, a linear indexing
  is used for matrices, which means that matrices are seen as large vectors
  built up be stacking rows. This function returns \var{OK} or \var{FAIL} if
  something went wrong (size mismatch between matrices, invalid string type).
\end{itemize}


\paragraph{Standard matrix operations}

  
\begin{itemize}
\item \describefun{void}{pnl_mat_plus_mat}{\PnlMat \ptr lhs, const
    \PnlMat \ptr rhs} 
  \sshortdescribe In-place matrix matrix addition  

\item \describefun{void}{pnl_mat_minus_mat}{\PnlMat \ptr lhs, 
    const \PnlMat \ptr rhs} 
  \sshortdescribe In-place matrix matrix substraction  
  
\item \describefun{void}{pnl_mat_sq_transpose}{\PnlMat \ptr M}
  \sshortdescribe On exit, \var{M} is transposed

\item \describefun{\PnlMat \ptr}{pnl_mat_transpose}{const
    \PnlMat \ptr M} 
    \sshortdescribe Create a new matrix which is the transposition of \var{M}

\item \describefun{void}{pnl_mat_tr}{
  \PnlMat \ptr tM, const \PnlMat \ptr M} 
  \sshortdescribe On exit, \var{tM = M'}

\item \describefun{double}{pnl_mat_trace}{const \PnlMat \ptr M}
  \sshortdescribe Return the trace of a square matrix.

\item \describefun{void}{pnl_mat_axpy}{double alpha, const \PnlMat
    \ptr A, \PnlMat \ptr B}
  \sshortdescribe Compute \var{B := alpha * A + B}

\item \describefun{void}{pnl_mat_dger}{double alpha, const \PnlVect
    \ptr x, const \PnlVect \ptr y, \PnlMat \ptr A}
  \sshortdescribe Compute \var{A := alpha x * y' + A}

\item \describefun{\PnlVect \ptr }{pnl_mat_mult_vect}{const
    \PnlMat \ptr A, const \PnlVect \ptr x} 
  \sshortdescribe Matrix vector multiplication  \var{A * x}

\item \describefun{void}{pnl_mat_mult_vect_inplace}{\PnlVect
    \ptr y, const \PnlMat \ptr A, const \PnlVect
    \ptr x} 
    \sshortdescribe In place matrix vector multiplication  \var{y = A * x}. You
    cannot use the same vector for \var{x} and \var{y}.

\item \describefun{\PnlVect \ptr }{pnl_mat_mult_vect_transpose}{const
    \PnlMat \ptr A, const \PnlVect \ptr x} 
  \sshortdescribe Matrix vector multiplication  \var{A' * x}

\item \describefun{void}{pnl_mat_mult_vect_transpose_inplace}{\PnlVect
    \ptr y, const \PnlMat \ptr A, const \PnlVect
    \ptr x} 
  \sshortdescribe In place matrix vector multiplication  \var{y = A' * x}.  You
  cannot use the same vector for \var{x} and \var{y}. The vectors \var{x} and \var{y}
  must be different.

\item \describefun{int}{pnl_mat_cross}{\PnlMat \ptr lhs, const
  \PnlMat \ptr A, const \PnlMat \ptr B}
  \sshortdescribe Compute the cross products of the vectors given in matrices
  \var{A} and \var{B} which must have either 3 rows or 3 columns. A row wise
  computation is first tried, then a column wise approach is tested.
  \var{FAIL} is returned in case no dimension equals 3.

  
\item \describefun{void}{pnl_mat_lAxpby}{double lambda, const \PnlMat
    \ptr A, const \PnlVect \ptr x, double mu, \PnlVect \ptr b} 
  \sshortdescribe Compute \var{b := lambda A x + mu b}. When \var{mu==0}, the
  content of \var{b} is not used on input and instead \var{b} is resized to
  match \var{A*x}. The vectors \var{x} and \var{b} must be different.


\item \describefun{void}{pnl_mat_dgemv}{char trans, double lambda, const
    \PnlMat \ptr A, const \PnlVect \ptr x, double mu, 
    \PnlVect \ptr b}
  \sshortdescribe Compute \var{b := lambda op(A) x + mu b}, where \var{op (X) =
    X} or \var{op (X) = X'}. If \var{trans='N'} or \var{trans='n'}, \var{op (A)
    = A}, whereas if \var{trans='T'} or \var{trans='t'}, \var{op (A) = A'}.When
  \var{mu==0}, the content of \var{b} is not used and instead \var{b} is resized
  to match \var{op(A)*x}. The vectors \var{x} and \var{b} must be different.

\item \describefun{void}{pnl_mat_dgemm}{char transA, char transB, double
    alpha, const \PnlMat \ptr A, const \PnlMat \ptr B, 
    double beta, \PnlMat \ptr C}
  \sshortdescribe Compute \var{C := alpha * op(A) * op (B) + beta *
    C}. When beta=0, the content of \var{C} is unused and instead \var{C}
  is resized to store \var{alpha A \ptr B}. If \var{transA='N'} or
  \var{transA='n'}, \var{op (A) = A}, whereas if \var{transA='T'} or
  \var{transA='t'}, \var{op (A) = A'}. The same holds for \var{transB}. The matrix
  \var{C} must be different from \var{A} and \var{B}.
  
\item \describefun{\PnlMat \ptr }{pnl_mat_mult_mat}{const
    \PnlMat \ptr rhs1, const \PnlMat \ptr rhs2} 
  \sshortdescribe Matrix multiplication  \var{rhs1 * rhs2}

\item \describefun{void}{pnl_mat_mult_mat_inplace}{\PnlMat
    \ptr lhs, const \PnlMat \ptr rhs1, const \PnlMat
    \ptr rhs2} 
    \sshortdescribe In-place matrix multiplication  \var{lhs = rhs1 * rhs2}. The
    matrix \var{lhs} must be different from \var{rhs1} and \var{rhs2}.

\item \describefun{double}{pnl_mat_scalar_prod}{const \PnlMat
    \ptr A, const \PnlVect \ptr x, const \PnlVect \ptr y}
  \sshortdescribe Compute \var{x' * A * y}
\item \describefun{void}{pnl_mat_exp}{\PnlMat \ptr B, 
    const \PnlMat \ptr A}
  \sshortdescribe Compute the matrix exponential \var{B = exp(A)}.

\item \describefun{void}{pnl_mat_log}{\PnlMat \ptr B, 
    const \PnlMat \ptr A}
  \sshortdescribe Compute the matrix logarithm \var{B = log(A)}. For the
  moment, this function only works if \var{A} is diagonalizable.

\item \describefun{void}{pnl_mat_eigen}{\PnlVect *v, \PnlMat \ptr P, 
    const \PnlMat \ptr A, int with_eigenvector}
  \sshortdescribe Compute the eigenvalues (stored in \var{v}) and optionally
  the eigenvectors stored column wise in \var{P} when
  \var{with_eigenvector==TRUE}. If \var{A} is symmetric or Hermitian in the
  complex case, \var{P} is orthonormal. When \var{with_eigenvector=FALSE},
  \var{P} can be \var{NULL}.
\end{itemize}


\paragraph{Linear systems and matrix decompositions}

The following functions are designed to solve linear system of the from \var{A x
= b} where \var{A} is a matrix and \var{b} is a vector except in the functions
\reffun{pnl_mat_syslin_mat}, \reffun{pnl_mat_lu_syslin_mat} and
\reffun{pnl_mat_chol_syslin_mat} which expect the right hand side member to be a
matrix too. Whenever the vector \var{b} is not needed once the system is solved,
you should consider using ``inplace'' functions.


All the functions described in this paragraph return \var{OK} if the
computations have been carried out successfully and \var{FAIL} otherwise.

\begin{itemize}
\item \describefun{int}{pnl_mat_chol}{\PnlMat \ptr M}
  \sshortdescribe Compute the Cholesky decomposition of \var{M}. \var{M} must
  be symmetric, the positivity is tested in the algorithm.  \var{M = L * L'}.
  On exit, the lower part of \var{M} contains the Cholesky decomposition L and
  the upper part is set to zero. 

\item \describefun{int}{pnl_mat_pchol}{\PnlMat \ptr M, double tol,
  int \ptr rank, \PnlVectInt \ptr p}
    \sshortdescribe Compute the
    Cholesky decomposition of \var{M} with complete pivoting. 
    \var{P' * A * P = L * L'}.
    \var{M} must be
    symmetric positive semi-definite. On exit, the lower part of \var{M}
    contains the Cholesky decomposition \var{L} and the upper part is set to zero. The
    permutation matrix is stored in an integer vector \var{p} : the only non
    zero elements of \var{P} are \var{P(p(k),k) = 1}

\item \describefun{int}{pnl_mat_lu}{\PnlMat \ptr A, 
    \PnlPermutation \ptr p} 
  \sshortdescribe Compute a P A = LU factorization. \var{P} must be an
  already allocated  \PnlPermutation. On exit the decomposition is
  stored in \var{A}, the lower part of \var{A} contains L while the upper part
  (including the diagonal terms) contains U. Remember that the diagonal
  elements of \var{L} are all 1. Row \var{i} of \var{A} was interchanged with
  row \var{p(i)}.
  
\item \describefun{int}{pnl_mat_upper_syslin}{\PnlVect
    \ptr x, const \PnlMat \ptr U, const \PnlVect\ptr b}
  \sshortdescribe Solve an upper triangular linear system \var{U x = b}

\item \describefun{int}{pnl_mat_lower_syslin}{\PnlVect
    \ptr x, const \PnlMat \ptr L, const \PnlVect\ptr b}
  \sshortdescribe Solve a lower triangular linear system  \var{L x = b}
  
\item \describefun{int}{pnl_mat_chol_syslin}{\PnlVect \ptr x, 
    const \PnlMat \ptr chol, const \PnlVect \ptr b} 
  \sshortdescribe Solve a symmetric definite positive linear system A x = b, 
  in which \var{chol} is assumed to be the Cholesky decomposition of A
  computed by \reffun{pnl_mat_chol}

\item \describefun{int}{pnl_mat_chol_syslin_inplace}{
    const \PnlMat \ptr chol, \PnlVect \ptr b} 
  \sshortdescribe Solve a symmetric definite positive linear system A x = b, 
  in which \var{chol} is assumed to be the Cholesky decomposition of A
  computed by \reffun{pnl_mat_chol}. The solution of the system is stored in
  \var{b} on exit.

\item \describefun{int}{pnl_mat_lu_syslin}{\PnlVect \ptr x, const
    \PnlMat \ptr LU, const \PnlPermutation \ptr p, 
    const \PnlVect \ptr b} 
  \sshortdescribe Solve a linear system A x = b using a LU decomposition.
  \var{LU} and \var{P} are assumed to be the PA = LU decomposition as computed
  by \reffun{pnl_mat_lu}. In particular, the structure of the matrix \var{LU}
  is the following : the lower part of \var{A} contains L while the upper part
  (including the diagonal terms) contains U. Remember that the diagonal
  elements of \var{L} are all 1.

\item \describefun{int}{pnl_mat_lu_syslin_inplace}{const
    \PnlMat \ptr LU, const \PnlPermutation \ptr p, 
    \PnlVect \ptr b} 
  \sshortdescribe Solve a linear system A x = b using a LU decomposition.
  \var{LU} and \var{P} are assumed to be the PA = LU decomposition as computed
  by \reffun{pnl_mat_lu}. In particular, the structure of the matrix \var{LU}
  is the following : the lower part of \var{A} contains L while the upper part
  (including the diagonal terms) contains U. Remember that the diagonal
  elements of \var{L} are all 1. The solution of the system is stored in \var{b}
  on exit.
  
\item \describefun{int}{pnl_mat_syslin}{\PnlVect \ptr x, const
    \PnlMat \ptr A, const \PnlVect \ptr b} 
  \sshortdescribe Solve a linear system A x = b using a LU factorization
  which is computed inside this function.

\item \describefun{int}{pnl_mat_syslin_inplace}{\PnlMat \ptr A, 
    \PnlVect \ptr b} 
  \sshortdescribe Solve a linear system A x = b using a LU factorization
  which is computed inside this function. The solution of the system is stored
  in \var{b} and \var{A} is overwritten by its LU decomposition.

\item \describefun{int}{pnl_mat_syslin_mat}{\PnlMat\ptr A, 
    \PnlMat \ptr B} 
  \sshortdescribe Solve a linear system A X = B using a LU factorization
  which is computed inside this function. \var{A} and  \var{B} are
  matrices. \var{A} must be square. The solution of the system is stored in
  \var{B} on exit. On exit, \var{A} contains the LU decomposition of the input
  matrix which is lost.

\item \describefun{int}{pnl_mat_chol_syslin_mat}{const \PnlMat\ptr A, \PnlMat \ptr B}
  \sshortdescribe Solve a linear system A X = B using a Cholesky factorization
  of the symmetric positive defnite matrix \var{A}.  \var{A} contains the
  Cholesky decomposition as computed by \reffun{pnl_mat_chol}. \var{B} is matrix
  with the same number of rows as \var{A}. The solution of
  the system is stored in \var{B} on exit. 

\item \describefun{int}{pnl_mat_lu_syslin_mat}{const \PnlMat\ptr A,
  const \PnlPermutation \ptr p, \PnlMat \ptr B}
  \sshortdescribe Solve a linear system A X = B using a \var{P A = L U} factorization.
  \var{A} contains the \var{L U} factors and \var{p} the associated permutation.
  \var{A} and \var{p} must have been computed by \reffun{pnl_mat_lu}. \var{B} is matrix
  with the same number of rows as \var{A}.
  The solution of the system is stored in \var{B} on exit. 

\end{itemize}


The following functions are designed to invert matrices. The authors provide
these functions although they cannot find good reasons to use them. Note that
to solve a linear system, one must used the \var{syslin} functions and not
invert the system matrix because it is much longer.
\begin{itemize}
\item \describefun{int}{pnl_mat_upper_inverse}{\PnlMat \ptr A, 
    const \PnlMat \ptr B}
  \sshortdescribe Inversion of an upper triangular matrix  

\item \describefun{int}{pnl_mat_lower_inverse}{\PnlMat \ptr A, 
    const \PnlMat \ptr B}
  \sshortdescribe Inversion of a lower triangular matrix  

\item \describefun{int}{pnl_mat_inverse}{\PnlMat
  \ptr inverse, const \PnlMat \ptr A}
  \sshortdescribe Compute the inverse of a matrix A and stores the result
  into \var{inverse}. A LU factorisation of the matrix \var{A} is computed
  inside this function.
\item \describefun{int}{pnl_mat_inverse_with_chol}{\PnlMat
  \ptr inverse, const \PnlMat \ptr A}
  \sshortdescribe Compute the inverse of a symmetric positive definite matrix
  A and stores the result into \var{inverse}. The Cholesky factorisation of
  the matrix \var{A} is computed inside this function.
\end{itemize}

\subsubsection{Functions specific to base type {\tt double}}


\paragraph{Linear systems and matrix decompositions}

The following functions are designed to solve linear system of the from \var{A x
= b} where \var{A} is a matrix and \var{b} is a vector except in the functions
\reffun{pnl_mat_syslin_mat}, \reffun{pnl_mat_lu_syslin_mat} and
\reffun{pnl_mat_chol_syslin_mat} which expect the right hand side member to be a
matrix too. Whenever the vector \var{b} is not needed once the system is solved,
you should consider using ``inplace'' functions.


All the functions described in this paragraph return \var{OK} if the
computations have been carried out successfully and \var{FAIL} otherwise.

\begin{itemize}
\item \describefun{int}{pnl_mat_qr}{\PnlMat \ptr Q,
  \PnlMat \ptr R, \PnlPermutation \ptr p,
  const \PnlMat \ptr A} 
  \sshortdescribe Compute a \var{A P = QR} decomposition. If on entry
  \var{P=NULL}, then the decomposition is computed without pivoting, i.e
  \var{A = QR}. When $P \ne NULL$, \var{P} must be an already allocated
  \PnlPermutation. \var{Q} is an orthogonal matrix, i.e
  $\var{Q}^{-1} = \var{Q}^{T}$ and \var{R} is an upper triangular matrix. The
  use of pivoting improves the numerical stability when \var{A} is almost rank
  deficient, i.e when the smallest eigenvalue of \var{A} is very close to $0$.

\item \describefun{int}{pnl_mat_qr_syslin}{\PnlVect \ptr x,
    const \PnlMat \ptr Q, const \PnlMat \ptr R,
    const \PnlVectInt \ptr p, const \PnlVect \ptr b}
  \sshortdescribe Solve a linear system \var{A x = b} where \var{A} is given by
  its QR decomposition with column pivoting as computed by the function
  \reffun{pnl_mat_qr}.
\item \describefun{int}{pnl_mat_ls}{const \PnlMat\ptr A, \PnlVect \ptr b}
  \sshortdescribe Solve a linear system A x = b in the least square sense,
  i.e. $\var{x} = \arg\min_U \| A * u - b\|^2$. The solution is stored into
  \var{b} on exit. It internally uses a \var{AP = QR} decomposition.

\item \describefun{int}{pnl_mat_ls_mat}{const \PnlMat\ptr A,
    \PnlMat \ptr B}
  \sshortdescribe Solve a linear system A X = B with \var{A} and \var{B} two
  matrices in the least square sense, i.e. $\var{X} = \arg\min_U \| A * U -
  B\|^2$. The solution is stored into \var{B} on exit. It internally uses a
  \var{AP = QR} decomposition. Same function as \reffun{pnl_mat_ls} but handles
  several r.h.s.

\end{itemize}

\subsubsection{Functions specific to base type {\tt dcomplex}}

\begin{itemize}
  \item \describefun{\PnlMatComplex\ptr }{pnl_mat_complex_create_from_mat}{const \PnlMat \ptr R}
    \sshortdescribe Create a complex matrix using a real one. The complex parts
    of the entries of the returned matrix are all set to zero.
\end{itemize}

\subsubsection{Permutations}

\describestruct{PnlPermutation}
\begin{verbatim}
typedef PnlVectInt PnlPermutation;
\end{verbatim}

The \verb!PnlPermutation! type is actually nothing else than a vector of
integers, i.e. a \verb!PnlVectInt!. It is used to store the partial pivoting
with row interchanges transformation needed in the LU decomposition.  We use the
{\it Blas} convention for storing permutations. Consider a \verb!PnlPermutation p!
generated by a LU decomposition of a matrix \verb!A! : to compute the
decomposition, row \verb!i! of \verb!A! was interchanged with row \verb!p(i)!.


\begin{itemize}
  \item \describefun{\PnlPermutation \ptr }{pnl_permutation_new}{}
    \sshortdescribe Create an empty \PnlPermutation.  

\item \describefun{\PnlPermutation \ptr }{pnl_permutation_create}{int n}
  \sshortdescribe Create a \PnlPermutation of size \var{n}.  

\item \describefun{void}{pnl_permutation_free}{\PnlPermutation \ptr \ptr p}
  \sshortdescribe Free a \PnlPermutation.

\item \describefun{void}{pnl_permutation_inverse}{\PnlPermutation\ptr
    inv, const \PnlPermutation\ptr p}
  \sshortdescribe Compute in \var{inv} the inverse of the permutation \var{p}.
\item \describefun{void}{pnl_vect_permute}{\PnlVect \ptr px, const
    \PnlVect \ptr x, const \PnlPermutation \ptr p} 
  \sshortdescribe Apply a \PnlPermutation to a \PnlVect.  

\item \describefun{void}{pnl_vect_permute_inplace}{\PnlVect \ptr x, 
    const \PnlPermutation \ptr p} 
  \sshortdescribe Apply a \PnlPermutation to a
  \PnlVect in-place.  

\item \describefun{void}{pnl_vect_permute_inverse}{\PnlVect \ptr px, const
    \PnlVect \ptr x, const \PnlPermutation \ptr p} 
  \sshortdescribe Apply the inverse of \PnlPermutation to a \PnlVect.  

\item \describefun{void}{pnl_vect_permute_inverse_inplace}{\PnlVect \ptr x, 
    const \PnlPermutation \ptr p} 
  \sshortdescribe Apply the inverse of a \PnlPermutation to a
  \PnlVect in-place.  

\item \describefun{void}{pnl_mat_col_permute}{\PnlMat \ptr pX, const
    \PnlMat \ptr X, const
    \PnlPermutation \ptr p}
  \sshortdescribe Apply a \PnlPermutation to the columns of a
  matrix. \var{pX} contains the result of the permutation applied to \var{X}. 
\item \describefun{void}{pnl_mat_row_permute}{\PnlMat \ptr pX, const
    \PnlMat \ptr X, const \PnlPermutation \ptr p}
  \sshortdescribe Apply a \PnlPermutation to the rows of a
  matrix. \var{pX} contains the result of the permutation applied to \var{X}. 
  
\item \describefun{void}{pnl_permutation_fprint}{FILE \ptr fic, const \PnlPermutation \ptr p}
  \sshortdescribe Print a permutation to a file.  

\item \describefun{void}{pnl_permutation_print}{const \PnlPermutation \ptr p}
  \sshortdescribe Print a permutation to the standard output.  
\end{itemize}


%% tridiag

\subsection{Tridiagonal Matrices}
\subsubsection{Overview}

The structures and functions related to tridiagonal matrices are declared in
\verb!pnl/pnl_tridiag_matrix.h!. 

We only store the three main diagonals as three vectors.

\describestruct{PnlTridiagMat}
\begin{verbatim}
typedef struct PnlTridiagMat{
  /**
   * Must be the first element in order for the object mechanism to work
   * properly. This allows any PnlTridiagMat pointer to be cast to a PnlObject
   */
  PnlObject object; 
  int size; /*!< number of rows, the matrix must be square */
  double *D; /*!< diagonal elements */
  double *DU; /*!< upper diagonal elements */
  double *DL; /*!< lower diagonal elements */
} PnlTridiagMat;
\end{verbatim}
\var{size} is the size of the matrix, \var{D} is an array of size \var{size}
containing the diagonal terms. \var{DU},
\var{DL} are two arrays of size \var{size-1} containing respectively the upper
diagonal ($M_{i, i+1}$) and the lower diagonal ($M_{i-1, i}$). 

\describestruct{PnlTridiagMatLU}
\begin{verbatim}
typedef struct PnlTridiagMatLU{
  /** 
   * Must be the first element in order for the object mechanism to work
   * properly. This allows any PnlTridiagMatLU pointer to be cast to a PnlObject
   */
  PnlObject object; 
  int size; /*!< number of rows, the matrix must be square */
  double *D; /*!< diagonal elements */
  double *DU; /*!< upper diagonal elements */
  double *DU2; /*!< second upper diagonal elements */
  double *DL; /*!< lower diagonal elements */
  int *ipiv; /*!< Permutation: row i has been interchanged with row ipiv(i) */
};
\end{verbatim}
This type is used to store the LU decomposition of a tridiagonal matrix. 

\subsubsection{Functions}
\paragraph{Constructors and destructors}
\begin{itemize}
  \item \describefun{\PnlTridiagMat \ptr }{pnl_tridiag_mat_new}{}
    \sshortdescribe Create a \PnlTridiagMat with size 0
  \item \describefun{\PnlTridiagMat \ptr }{pnl_tridiag_mat_create}{int size}
    \sshortdescribe Create a \PnlTridiagMat with size \var{size}
  \item \describefun{\PnlTridiagMat \ptr }{pnl_tridiag_mat_create_from_scalar}{int size, double x}
    \sshortdescribe Create a \PnlTridiagMat with the 3 diagonals
    filled with \var{x}
  \item \describefun{\PnlTridiagMat \ptr }{pnl_tridiag_mat_create_from_two_scalar}{int size, double x, double y}
    \sshortdescribe Create a \PnlTridiagMat  with the diagonal
    filled with \var{x} and the upper and lower diagonals filled with \var{y}
  \item \describefun{\PnlTridiagMat
      \ptr }{pnl_tridiag_mat_create_from_ptr}{int size, const double
      \ptr lower_D, const double \ptr D, const double \ptr upper_D}
    \sshortdescribe Create a \PnlTridiagMat  
  \item \describefun{\PnlTridiagMat \ptr }{pnl_tridiag_mat_create_from_mat}
    {const \PnlMat \ptr mat}
    \sshortdescribe Create a tridiagonal matrix from a full matrix (all the
    elements but the 3 diagonal ones are ignored).
  \item \describefun{\PnlMat \ptr }{pnl_tridiag_mat_to_mat}
    {const \PnlTridiagMat \ptr T}
    \sshortdescribe Create a full matrix from a tridiagonal one.
  \item \describefun{\PnlTridiagMat \ptr }{pnl_tridiag_mat_copy}
    {const \PnlTridiagMat \ptr T}
    \sshortdescribe Copy a tridiagonal matrix.
  \item \describefun{void}{pnl_tridiag_mat_clone}
    {\PnlTridiagMat \ptr clone, const \PnlTridiagMat \ptr T}
    \sshortdescribe Copy the content of \var{T} into \var{clone}
  \item \describefun{void }{pnl_tridiag_mat_free}{\PnlTridiagMat \ptr \ptr v}
    \sshortdescribe Free a \PnlTridiagMat  
  \item \describefun{int}{pnl_tridiag_mat_resize}{\PnlTridiagMat \ptr v, int size}
    \sshortdescribe Resize a \PnlTridiagMat.  
\end{itemize}


\paragraph{Accessing elements.}

If it is supported by the compiler, the following functions are declared
inline. To speed up these functions, you can use the macro constant
\texttt{PNL_RANGE_CHECK_OFF}, see Section~\ref{sec:inline} for an explanation. 
\begin{itemize}
  \item \describefun{void}{pnl_tridiag_mat_set}{\PnlTridiagMat \ptr self, int d, int up, double x}
    \sshortdescribe Set \var{self[d, d+up] = x}, \var{up} can be $\{-1, 0, 1\}$.  
  \item \describefun{double}{pnl_tridiag_mat_get}{const \PnlTridiagMat \ptr self, int d, int up}
    \sshortdescribe Get \var{self[d, d+up]}, \var{up} can be $\{-1, 0, 1\}$.  
  \item \describefun{double \ptr }{pnl_tridiag_mat_lget}{\PnlTridiagMat \ptr self, int d, int up}
    \sshortdescribe Return the address \var{self[d, d+up] = x}, \var{up} can be $\{-1, 0, 1\}$.  
\end{itemize}

\paragraph{Printing Matrix}
\begin{itemize}
  \item \describefun{void}{pnl_tridiag_mat_fprint}{FILE \ptr fic, const \PnlTridiagMat \ptr M}
    \sshortdescribe Print a tri-diagonal matrix to a file.  
  \item \describefun{void}{pnl_tridiag_mat_print}{const \PnlTridiagMat \ptr M}
    \sshortdescribe Print a tridiagonal matrix to the standard output.  
\end{itemize}

\paragraph{Algebra operations}
\begin{itemize}
  \item \describefun{void}{pnl_tridiag_mat_plus_tridiag_mat}{\PnlTridiagMat \ptr lhs, const \PnlTridiagMat \ptr rhs}
    \sshortdescribe In-place matrix matrix addition  
  \item \describefun{void}{pnl_tridiag_mat_minus_tridiag_mat}{\PnlTridiagMat \ptr lhs, const \PnlTridiagMat \ptr rhs}
    \sshortdescribe In-place matrix matrix substraction  
  \item \describefun{void}{pnl_tridiag_mat_plus_scalar}{\PnlTridiagMat \ptr lhs, double x}
    \sshortdescribe In-place matrix scalar addition  
  \item \describefun{void}{pnl_tridiag_mat_minus_scalar}{\PnlTridiagMat \ptr lhs, double x}
    \sshortdescribe In-place matrix scalar substraction  
  \item \describefun{void}{pnl_tridiag_mat_mult_scalar}{\PnlTridiagMat \ptr lhs, double x}
    \sshortdescribe In-place matrix scalar multiplication  
  \item \describefun{void}{pnl_tridiag_mat_div_scalar}{\PnlTridiagMat \ptr lhs, double x}
    \sshortdescribe In-place matrix scalar division
\end{itemize}

\paragraph{Element-wise operations}
\begin{itemize}
  \item \describefun{void}{pnl_tridiag_mat_mult_tridiag_mat_term}{\PnlTridiagMat \ptr lhs, const \PnlTridiagMat \ptr rhs}
    \sshortdescribe In-place matrix matrix term by term product  
  \item \describefun{void}{pnl_tridiag_mat_div_tridiag_mat_term}{\PnlTridiagMat \ptr lhs, const \PnlTridiagMat \ptr rhs}
    \sshortdescribe In-place matrix matrix term by term division  
  \item \describefun{void}{pnl_tridiag_mat_map_inplace}{\PnlTridiagMat \ptr lhs, 
    double(\ptr f)(double)} 
  \sshortdescribe \var{lhs = f(lhs)}.


\item \describefun{void}{pnl_tridiag_mat_map_tridiag_mat_inplace}{\PnlTridiagMat \ptr lhs, const
  \PnlTridiagMat \ptr rhs, double(\ptr f)(double, double)} 
  \sshortdescribe \var{lhs = f(lhs, rhs)}.
\end{itemize}

\paragraph{Standard matrix operations \& Linear systems}
\begin{itemize}
  \item \describefun{void}{pnl_tridiag_mat_mult_vect_inplace}{\PnlVect
    \ptr lhs, const \PnlTridiagMat \ptr mat, const \PnlVect
    \ptr rhs}
    \sshortdescribe In place matrix multiplication. The vector \var{lhs} must be
    different from \var{rhs}.
  \item \describefun{\PnlVect \ptr }{pnl_tridiag_mat_mult_vect}{const
    \PnlTridiagMat \ptr mat, const \PnlVect \ptr vec}
    \sshortdescribe Matrix multiplication  
  \item \describefun{void}{pnl_tridiag_mat_lAxpby}{double lambda, const \PnlTridiagMat
      \ptr A, const \PnlVect \ptr x, double mu, \PnlVect \ptr b} 
    \sshortdescribe Compute \var{b := lambda A x + mu b}. When \var{mu==0}, the
    content of \var{b} is not used on input and instead \var{b} is resized to
    match \var{A*x}. Note that the vectors \var{x} and \var{b} must be different.
  \item \describefun{double}{pnl_tridiag_mat_scalar_prod}{const \PnlVect
    \ptr x,const \PnlTridiagMat \ptr A, const \PnlVect \ptr y}
    \sshortdescribe Compute \var{x' * A * y}
  \item \describefun{void}{pnl_tridiag_mat_syslin_inplace}{
      \PnlTridiagMat \ptr M, \PnlVect \ptr b}
    \sshortdescribe Solve the linear system M x = b. The solution is written into
    \var{b} on exit. On exit, \var{M} is modified and becomes unusable.
  \item \describefun{void}{pnl_tridiag_mat_syslin}{\PnlVect
      \ptr x, \PnlTridiagMat \ptr M, const \PnlVect \ptr b}
    \sshortdescribe Solve the linear system M x = b. On exit, \var{M} is modified and becomes unusable.
  \item \describefun{\PnlTridiagMatLU\ptr }{pnl_tridiag_mat_lu_new}{}
    \sshortdescribe Create an empty \PnlTridiagMatLU
  \item \describefun{\PnlTridiagMatLU\ptr }{pnl_tridiag_mat_lu_create}{int size}
    \sshortdescribe Create a \PnlTridiagMatLU with size \var{size}
  \item \describefun{\PnlTridiagMatLU\ptr }{pnl_tridiag_mat_lu_copy}{const \PnlTridiagMatLU  \ptr mat}
    \sshortdescribe Create a new \PnlTridiagMatLU which is a copy of
    \var{mat}.
  \item \describefun{void}{pnl_tridiag_mat_lu_clone}{\PnlTridiagMatLU \ptr clone, const \PnlTridiagMatLU \ptr mat}
    \sshortdescribe Clone a \PnlTridiagMatLU. \var{clone} must
    already exist, no memory is allocated for the envelope. 
  \item \describefun{void}{pnl_tridiag_mat_lu_free}{\PnlTridiagMatLU \ptr \ptr m}
    \sshortdescribe Free a \PnlTridiagMatLU
  \item \describefun{int}{pnl_tridiag_mat_lu_resize}{\PnlTridiagMatLU \ptr v, int size}
    \sshortdescribe Resize a \PnlTridiagMatLU
  \item \describefun{int}{pnl_tridiag_mat_lu_compute}{\PnlTridiagMatLU \ptr LU, const \PnlTridiagMat \ptr A}
    \sshortdescribe Compute the LU factorisation of a tridiagonal matrix
    \var{A}. \var{LU} must have already been created using
    \reffun{pnl_tridiag_mat_lu_new}. On exit, \var{LU} contains the
    decomposition which is suitable for use in \reffun{pnl_tridiag_mat_lu_syslin}.
  \item \describefun{int}{pnl_tridiag_mat_lu_syslin_inplace}{\PnlTridiagMatLU \ptr LU, \PnlVect \ptr b}
    \sshortdescribe Solve a linear system \var{A x = b} where the matrix \var{LU}
    is given the LU decomposition of A previously computed by
    \reffun{pnl_tridiag_mat_lu_compute}. On exit, \var{b} is overwritten by the
    solution \var{x}. 
  \item \describefun{int}{pnl_tridiag_mat_lu_syslin}{\PnlVect \ptr x, \PnlTridiagMatLU \ptr LU, const \PnlVect \ptr b}
    \sshortdescribe Solve a linear system \var{A x = b} where the matrix \var{LU}
    is given the LU decomposition of A previously computed by
    \reffun{pnl_tridiag_mat_lu_compute}. 
\end{itemize}



\subsection{Band Matrices}
\subsubsection{Overview}

\describestruct{PnlBandMat}
\begin{verbatim}
typedef struct
{
  /**
   * Must be the first element in order for the object mechanism to work
   * properly. This allows any PnlBandMat pointer to be cast to a PnlObject
   */
  PnlObject object; 
  int m; /*!< nb rows */ 
  int n; /*!< nb columns */ 
  int nu; /*!< nb of upperdiagonals */
  int nl; /*!< nb of lowerdiagonals */
  int m_band; /*!< nb rows of the band storage */
  int n_band; /*!< nb columns of the band storage */
  double *array;  /*!< a block to store the bands */  
} PnlBandMat;
\end{verbatim}


The structures and functions related to band matrices are declared in
\verb!pnl/pnl_band_matrix.h!. 


\subsubsection{Functions}
\paragraph{Constructors and destructors}
\begin{itemize}
  \item \describefun{\PnlBandMat\ptr }{pnl_band_mat_new}{}
  \sshortdescribe Create a band matrix of size 0.

\item \describefun{\PnlBandMat\ptr }{pnl_band_mat_create}{int m, int n, int
    nl, int nu}
  \sshortdescribe Create a band matrix of size \var{m x n} with \var{nl} lower
  diagonals and \var{nu} upper diagonals.

\item \describefun{\PnlBandMat\ptr }{pnl_band_mat_create_from_mat}{const
    \PnlMat \ptr BM, int nl, int nu}
  \sshortdescribe Extract a band matrix from a \PnlMat.

\item \describefun{void}{pnl_band_mat_free}{\PnlBandMat\ptr\ptr}
  \sshortdescribe Free a band matrix.

\item \describefun{void}{pnl_band_mat_clone}{\PnlBandMat \ptr clone, 
  const \PnlBandMat \ptr M}
  \sshortdescribe Copy the band matrix \var{M} into \var{clone}. No new
  \PnlBandMat is created.

\item \describefun{\PnlBandMat\ptr }{pnl_band_mat_copy}{\PnlBandMat \ptr BM}
  \sshortdescribe Create a new band matrix which is a copy of \var{BM}. Each
  band matrix owns its data array.

\item \describefun{\PnlMat\ptr }{pnl_band_mat_to_mat}{\PnlBandMat \ptr BM}
  \sshortdescribe Create a full matrix from a band matrix.

\item \describefun{int}{pnl_band_mat_resize}{\PnlBandMat \ptr BM, int
  m, int n, int nl, int nu}
  \sshortdescribe Resize \var{BM} to store a \var{m x n} band matrix with
  \var{nu} upper diagonals and \var{nl} lower diagonals.
\end{itemize}

\paragraph{Accessing elements.}

If it is supported by the compiler, the following functions are declared
inline. To speed up these functions, you can use the macro constant
\texttt{PNL_RANGE_CHECK_OFF}, see Section~\ref{sec:inline} for an explanation. 
\begin{itemize}
\item \describefun{void}{pnl_band_mat_set}{\PnlBandMat
    \ptr M, int i, int j, double x}
  \sshortdescribe $M_{i, j}=x$.

\item \describefun{void}{pnl_band_mat_get}{\PnlBandMat
    \ptr M, int i, int j}
    \sshortdescribe Return $M_{i, j}$.

\item \describefun{void}{pnl_band_mat_lget}{\PnlBandMat
    \ptr M, int i, int j}
    \sshortdescribe Return the address $\&(M_{i, j})$.
\item \describefun{void}{pnl_band_mat_set_all}{\PnlBandMat
    \ptr M, double x}
    \sshortdescribe Set all the elements of \var{M} to \var{x}.

  \item \describefun{void}{pnl_band_mat_print_as_full}{\PnlBandMat
    \ptr M}
    \sshortdescribe Print a band matrix in a full format.
\end{itemize}

\subparagraph{Element wise operations}

\begin{itemize}
  \item \describefun{void}{pnl_band_mat_plus_scalar}{\PnlBandMat \ptr lhs, 
  double x} 
    \sshortdescribe In-place addition, \var{lhs += x} 

  \item \describefun{void}{pnl_band_mat_minus_scalar}{\PnlBandMat \ptr lhs, 
  double x} 
  \sshortdescribe In-place substraction \var{lhs -= x} 

\item \describefun{void}{pnl_band_mat_div_scalar}{\PnlBandMat
  \ptr lhs, double x} 
  \sshortdescribe \var{lhs = lhs ./ x}

\item \describefun{void}{pnl_band_mat_mult_scalar}{\PnlBandMat
  \ptr lhs, double x} 
  \sshortdescribe \var{lhs = lhs * x}

\item \describefun{void}{pnl_band_mat_plus_band_mat}{\PnlBandMat \ptr lhs, 
    const \PnlBandMat \ptr rhs} 
    \sshortdescribe In-place addition, \var{lhs += rhs} 

\item \describefun{void}{pnl_band_mat_minus_band_mat}{\PnlBandMat \ptr lhs, 
    const \PnlBandMat \ptr rhs} 
  \sshortdescribe In-place substraction \var{lhs -= rhs} 

\item \describefun{void}{pnl_band_mat_inv_term}{\PnlBandMat \ptr lhs}
  \sshortdescribe In-place term by term  inversion \var{lhs = 1 ./ rhs} 

\item \describefun{void}{pnl_band_mat_div_band_mat_term}{\PnlBandMat
    \ptr lhs, const \PnlBandMat \ptr rhs} 
  \sshortdescribe In-place term by term  division \var{lhs = lhs ./ rhs}

\item \describefun{void}{pnl_band_mat_mult_band_mat_term}{\PnlBandMat
    \ptr lhs, const \PnlBandMat \ptr rhs} 
  \sshortdescribe In-place term by term multiplication  \var{lhs = lhs .* rhs}

\item \describefun{void}{pnl_band_mat_map}{\PnlBandMat \ptr lhs, const
    \PnlBandMat \ptr rhs, double(\ptr f)(double)} 
  \sshortdescribe \var{lhs = f(rhs)}

\item \describefun{void}{pnl_band_mat_map_inplace}{\PnlBandMat \ptr lhs, double(\ptr f)(double)}
  \sshortdescribe  \var{lhs = f(lhs)}

\item \describefun{void}{pnl_band_mat_map_band_mat_inplace}{\PnlBandMat \ptr lhs,
  const \PnlBandMat \ptr rhs, double(\ptr f)(double,double)} 
  \sshortdescribe \var{lhs = f(lhs,rhs)}
\end{itemize}


\paragraph{Standard matrix operations \& Linear system}
\begin{itemize}
\item \describefun{void}{pnl_band_mat_lAxpby}{double lambda, const \PnlBandMat
    \ptr A, const \PnlVect \ptr x, double mu, \PnlVect \ptr b} 
  \sshortdescribe Compute \var{b := lambda A x + mu b}. When \var{mu==0}, the
  content of \var{b} is not used on input and instead \var{b} is resized to
  match the size of \var{A*x}.
\item \describefun{void}{pnl_band_mat_mult_vect_inplace}{\PnlVect \ptr
  y, const \PnlBandMat \ptr BM, const \PnlVect \ptr x}
  \sshortdescribe \var{y = BM * x}
\item 
  \describefun{void}{pnl_band_mat_syslin_inplace}{\PnlBandMat
    \ptr M, \PnlVect \ptr b}
    \sshortdescribe Solve the linear system \var{M x = b} with \var{M} a \PnlBandMat.
  {\bf Note} that M is modified on output and becomes unusable. On exit, the
  solution \var{x} is stored in \var{b}.
\item 
  \describefun{void}{pnl_band_mat_syslin}{\PnlVect \ptr x,\PnlBandMat
    \ptr M, \PnlVect \ptr b}
    \sshortdescribe Solve the linear system \var{M x = b} with \var{M} a \PnlBandMat.
  {\bf Note} that M is modified on output and becomes unusable. 
\item \describefun{void}{pnl_band_mat_lu}{\PnlBandMat \ptr BM,
  \PnlVectInt \ptr p}
  \sshortdescribe Compute the LU decomposition with partial pivoting with row
  interchanges. On exit, \var{BM} is enlarged to store the LU decomposition. On
  exit, \var{p} stores the permutation applied to the rows. Note that the Lapack format
  is used to store \var{p}, this format differs from the one used by
  \PnlPermutation.
\item  \describefun{void}{pnl_band_mat_lu_syslin_inplace}{const \PnlBandMat \ptr M, 
  \PnlVectInt \ptr p, \PnlVect \ptr b} 
  \sshortdescribe Solve the band linear system \var{M x = b} where \var{M} is
  the LU decomposition computed by \reffun{pnl_band_mat_lu}  and \var{p} the
  associated permutation. On exit, the solution \var{x} is stored in \var{b}.
\item  \describefun{void}{pnl_band_mat_lu_syslin}{\PnlVect \ptr x,
  const \PnlBandMat \ptr M, \PnlVectInt \ptr p, const \PnlVect \ptr b} 
  \sshortdescribe Solve the band linear system \var{M x = b} where \var{M} is the LU
  decomposition computed by \reffun{pnl_band_mat_lu} and \var{p} the associated permutation. 
\end{itemize}


\subsection{Sparse Matrices}
\subsubsection{Short description}
The structures and functions related to matrices are declared in
\verb!pnl/pnl_sp_matrix.h!.

\describestruct{PnlSpMat}\describestruct{PnlSpMatInt}\describestruct{PnlSpMatComplex}
\begin{verbatim}
typedef struct _PnlSpMat
{
  /** 
   * Must be the first element in order for the object mechanism to work
   * properly. This allows a PnlSpMat pointer to be cast to a PnlObject
   */
  PnlObject object; 
  int m; /*!< number of rows */
  int n; /*!< number of columns */
  int nz; /*!< number of non-zero elements */
  int *J; /*!< column indices, vector of size nzmax */
  int *I; /*!< row offset integer vector, 
            array[I[i]] is the first element of row i.
            Vector of size (m+1) */ 
  double *array; /*!< pointer to store the data of size nzmax*/
  int nzmax; /*!< size of the memory block allocated for array */
} PnlSpMat;

typedef struct _PnlSpMatInt
{
  /** 
   * Must be the first element in order for the object mechanism to work
   * properly. This allows a PnlSpMat pointer to be cast to a PnlObject
   */
  PnlObject object; 
  int m; /*!< number of rows */
  int n; /*!< number of columns */
  int nz; /*!< number of non-zero elements */
  int *J; /*!< column indices, vector of size nzmax */
  int *I; /*!< row offset integer vector, 
            array[I[i]] is the first element of row i.
            Vector of size (m+1) */ 
  int *array; /*!< pointer to store the data of size nzmax */
  int nzmax; /*!< size of the memory block allocated for array */
} PnlSpMatInt;

typedef struct _PnlSpMatComplex
{
  /** 
   * Must be the first element in order for the object mechanism to work
   * properly. This allows a PnlSpMat pointer to be cast to a PnlObject
   */
  PnlObject object; 
  int m; /*!< number of rows */
  int n; /*!< number of columns */
  int nz; /*!< number of non-zero elements */
  int *J; /*!< column indices, vector of size nzmax */
  int *I; /*!< row offset integer vector, 
            array[I[i]] is the first element of row i.
            Vector of size (m+1) */ 
  dcomplex *array; /*!< pointer to store the data of size nzmax */
  int nzmax; /*!< size of the memory block allocated for array */
} PnlSpMatComplex;
\end{verbatim}

The non zero elements of row \var{i} are stored in \var{array} between the
indices \var{I[i]} and \var{I[i+1]-1}. The array \var{J} contains the column
indices of every element of \var{array}.
\\

Sparse matrices are defined using the internal template approach and can be used
for integer, float or complex base data according to the following table
\begin{center}
  \begin{tabular}[t]{lll}
    base type & prefix & type \\
    \hline
    double & pnl_sp_mat & PnlSpMat \\
    \hline
    int & pnl_sp_mat_int & PnlSpMatInt \\
    \hline
    dcomplex & pnl_sp_mat_complex & PnlSpMatComplex
  \end{tabular}
\end{center}

\subsubsection{Functions}

\paragraph{Constructors and destructors}

\begin{itemize}
  \item \describefun{\PnlSpMat \ptr}{pnl_sp_mat_new}{}
    \sshortdescribe Create an empty sparse matrix.
  \item \describefun{\PnlSpMat \ptr}{pnl_sp_mat_create}{int m, int n, int nzmax}
    \sshortdescribe Create a sparse matrix with size \var{m x n} designed to
    hold at most \var{nzmax} non zero elements.

  \item \describefun{void}{pnl_sp_mat_clone}{\PnlSpMat \ptr dest, const \PnlSpMat \ptr src}
    \sshortdescribe Clone \var{src} into \var{dest}, which is automatically
    resized. On output, \var{dest} and \var{src} are equal but independent.

  \item \describefun{\PnlSpMat \ptr}{pnl_sp_mat_copy}{\PnlSpMat \ptr src}
    \sshortdescribe Create an independent copy of \var{src}.


  \item \describefun{void}{pnl_sp_mat_free}{\PnlSpMat \ptr\ptr}
    \sshortdescribe Delete a sparse matrix.

  \item \describefun{int}{pnl_sp_mat_resize}{\PnlSpMat \ptr M, int m, int n, int nzmax}
    \sshortdescribe Resize an existing \PnlSpMat to become a \var{m x n} sparse
    matrices holding at most \var{nzmax}. Note that no old data are kept except if
    \var{M->m} and \var{M->n} are left unchanged and we only call this function to
    increase \var{M->nzmax}. Return \var{OK} or \var{FAIL}.


  \item \describefun{\PnlMat\ptr }{pnl_mat_create_from_sp_mat}{const \PnlSpMat \ptr M}
    \sshortdescribe Create a dense \PnlMat from a spare one.
\end{itemize}

\paragraph{Accessing elements}

\begin{itemize}
  \item \describefun{void}{pnl_sp_mat_set}{\PnlSpMat \ptr M, int i, int j, double x}
    \sshortdescribe Set \var{M[i,j] = x}. This function increases \var{M->nzmax}
    if necessary.

  \item \describefun{double}{pnl_sp_mat_get}{const \PnlSpMat \ptr M, int i, int j}
    \sshortdescribe Return \var{M[i,j]}. If \var{M} has no entry with such an
    index, zero is returned.
\end{itemize}


\subsection{Hyper Matrices}
\subsubsection{Short description}

The Hyper matrix types and related functions are defined in the header \verb!pnl/pnl_matrix.h!.

\describestruct{PnlHmat}\describestruct{PnlHmatInt}\describestruct{PnlHmatComplex}
\begin{verbatim}
typedef struct PnlHmat{
  /**
   * Must be the first element in order for the object mechanism to work
   * properly. This allows any PnlHmat pointer to be cast to a PnlObject
   */
  PnlObject object; 
  int ndim; /*!< nb dimensions */ 
  int *dims; /*!< pointer to store the values of the ndim dimensions */ 
  int mn; /*!< product dim_1 *...*dim_ndim */
  int *pdims; /*!< array of size ndim, s.t. pdims[i] = dims[ndim-1] x ... dims[i+1]
                with pdims[ndim - 1] = 1 */
  double *array; /*!< pointer to store */
} PnlHmat;

typedef struct PnlHmatInt{
  /**
   * Must be the first element in order for the object mechanism to work
   * properly. This allows any PnlHmatInt pointer to be cast to a PnlObject
   */
  PnlObject object; 
  int ndim; /*!< nb dimensions */ 
  int *dims; /*!< pointer to store the value of the ndim dimensions */ 
  int mn; /*!< product dim_1 *...*dim_ndim */
  int *pdims; /*!< array of size ndim, s.t. pdims[i] = dims[ndim-1] x ... dims[i+1]
                with pdims[ndim - 1] = 1 */
  int *array; /*!< pointer to store */
} PnlHmatInt;

typedef struct PnlHmatComplex{
  /**
   * Must be the first element in order for the object mechanism to work
   * properly. This allows any PnlHmatComplex pointer to be cast to a PnlObject
   */
  PnlObject object; 
  int ndim; /*!< nb dimensions */ 
  int *dims; /*!< pointer to store the value of the ndim dimensions */ 
  int mn; /*!< product dim_1 *...*dim_ndim */
  int *pdims; /*!< array of size ndim, s.t. pdims[i] = dims[ndim-1] x ... dims[i+1]
                with pdims[ndim - 1] = 1 */
  dcomplex *array; /*!< pointer to store */
} PnlHmatComplex;
\end{verbatim}
\var{ndim} is the number of dimensions, \var{dim} is an array to store the
size of each dimension and \var{nm} contains the product of the sizes of each
dimension. \var{array} is an array of size \var{mn} containing the data. The
integer array \var{pdims} is used to create the one--to--one map between the
natural indexing and the linear indexing used in \var{array}.


\subsubsection{Functions}
These functions exist for all types of hypermatrices no matter what the basic type
is. The following conventions are used to name functions operating on hypermatrices.
Here is the table of prefixes used for the different basic types.

\begin{center}
  \begin{tabular}[t]{lll}
    base type & prefix & type\\
    \hline
    double & pnl_hmat & PnlHmat \\
    \hline
    int & pnl_hmat_int & PnlHmatInt \\
    \hline
    dcomplex & pnl_hmat_complex & PnlHmatComplex
  \end{tabular}
\end{center}

In this paragraph, we present the functions operating on \PnlHmat
which exist for all types. To deduce the prototypes of these functions for
other basic types, one must replace {\tt pnl_hmat} and {\tt double} according
the above table.


\paragraph{Constructors and destructors}
\begin{itemize}
  \item \describefun{\PnlHmat \ptr }{pnl_hmat_new}{}
    \sshortdescribe Create an empty \PnlHmat.

\item \describefun{\PnlHmat \ptr }{pnl_hmat_create}{int ndim, const int \ptr dims}
  \sshortdescribe Create a \PnlHmat with \var{ndim} dimensions and
  the size of each dimension is given by the entries of the integer array
  \var{dims}
  
\item 
  \describefun{\PnlHmat \ptr }{pnl_hmat_create_from_scalar}{int ndim, const int \ptr dims, double x}
  \sshortdescribe Create a \PnlHmat with \var{ndim} dimensions given
  by $\prod_i \var{dims[i]}$ filled with \var{x}.
  
\item 
  \describefun{\PnlHmat \ptr }{pnl_hmat_create_from_ptr}{int ndim, const int \ptr dims, const double \ptr x}
  
\item \describefun{void}{pnl_hmat_free}{\PnlHmat \ptr \ptr H}
  \sshortdescribe Free a \PnlHmat
  
\item \describefun{\PnlHmat \ptr }{pnl_hmat_copy}{const \PnlHmat \ptr H}
  \sshortdescribe Copy a \PnlHmat.
  
\item \describefun{void}{pnl_hmat_clone}{\PnlHmat \ptr clone, const \PnlHmat \ptr H}
  \sshortdescribe Clone a \PnlHmat.
  
\item \describefun{int}{pnl_hmat_resize}{\PnlHmat \ptr H, int ndim, const int \ptr dims}
  \sshortdescribe Resize a \PnlHmat.
\end{itemize}  

\paragraph{Accessing elements}

\begin{itemize}
\item   \describefun{void}{pnl_hmat_set}{\PnlHmat \ptr self, int \ptr tab, double x}
  \sshortdescribe Set the element of index \var{tab} to \var{x}.
  
\item \describefun{double}{pnl_hmat_get}{const \PnlHmat \ptr self, int \ptr tab}
  \sshortdescribe Return the value of the element of index \var{tab} 
  
\item \describefun{double\ptr }{pnl_hmat_lget}{\PnlHmat \ptr self, int \ptr tab}
  \sshortdescribe Return the address of self[tab] for use as a lvalue.  

\item \describefun{\PnlMat}{pnl_mat_wrap_hmat}{\PnlHmat \ptr H, int \ptr t}
  \sshortdescribe Return a true \PnlMat not a pointer holding the data
  \var{H(t,:,:)}. Note that \var{t} must be of size \var{ndim-2} and that it
  cannot be checked within the function. The returned matrix shares its data
  with \var{H}, it is only a view not a true copy.

\item \describefun{\PnlVect}{pnl_vect_wrap_hmat}{\PnlHmat \ptr H, int \ptr t}
  \sshortdescribe Return a true \PnlVect not a pointer holding the data
  \var{H(t,:)}. Note that \var{t} must be of size \var{ndim-1} and that it
  cannot be checked within the function. The returned vector shares its data
  with \var{H}, it is only a view not a true copy.


\end{itemize}  

\paragraph{Printing hypermatrices}

\begin{itemize}
\item \describefun{void}{pnl_hmat_print}{const \PnlHmat \ptr H}
  \sshortdescribe Print an hypermatrix.
\end{itemize}

\paragraph{Term by term operations}

\begin{itemize}
\item \describefun{void}{pnl_hmat_plus_hmat}{\PnlHmat \ptr lhs, const \PnlHmat \ptr rhs}
  \sshortdescribe Compute \var{lhs += rhs}.
  
\item \describefun{void}{pnl_hmat_mult_scalar}{\PnlHmat \ptr lhs, double x}
  \sshortdescribe Compute \var{lhs *= x} where x is a real number.
\end{itemize}

% \subsection{Morse Matrix}
% \subsubsection{Overview}

% A system of linear equation is called sparse if only a relatively small number
% of its matrix elements $M_{i, j}$ are nonzero. It is wasteful to use full
% structure to solve the linear system because most of the operations devoted to
% solving the system use elements with values zero. Furthermore, for some 
% high dimensional problems, storing the full matrix with its zero elements is not
% possible because of memory limitations.


% In the following, we propose two structures for Sparse Matrices.  Must of the
% algorithms which use sparse matrices can be divided in two steps.  The first
% step is the construction of the matrix. For this, \PnlMorseMat should
% be used. The second step is the resolution of a sparse linear system. We
% have two ways of doing that. The first one is to use a direct method based on
% matrix-decomposition, like the LU decomposition. The \PnlSparseMat is
% implemented to do that. The second one is to use iterative methods like
% Conjugate Gradient, BICGstab or GMRES. These methods are discussed in the next
% section. If we use iterative methods, we can use \PnlMorseMat. 

% \begin{verbatim}
% typedef struct SpRow{
%   int size;  /*!< size of a row */
%   int Max_size; /*!< max size allocation of a row */
%   int    *Index; /*!< pointer to an int array giving the columns or row i */
%   double *Value; /*!< Pointer on values */
% }SpRow;
% \end{verbatim}
% \var{size} is the number of elements, 
% \var{Max_size} is the size of memory allocation.
% \var{Index}, is the pointer containing the index of row or column, 
% \var{Value}, is the pointer containing the value of row or column.
% So for a \refstruct{SpRow} which contains row $i$ of $M$.
% If $k \leq size $ then
% $$M_{i, Index[k]}=Value[k].$$  

% \begin{verbatim}
% typedef struct PnlMorseMat{
%   int m; /*!< nb rows */ 
%   int n; /*!< nb columns */ 
%   SpRow * array; /*!< pointer in each row or col to store no nul coefficients */
%   int RC; /*!< 0 if we use row-wise storage, 1 if we use column-wise storage */ 
% } PnlMorseMat;
% \end{verbatim}
% \var{m} is the number of rows, \var{n} is the number of columns.
% \var{array} is the pointer containing on SpRow array of size n or m (depend of
% RC).
% \var{RC} is an integer to know if the matrix is stored by row or columns.

% \subsubsection{Functions}
% \paragraph{Constructors and destructors}
% \begin{itemize}
% \item \describefun{\PnlMorseMat\ptr }{pnl_morse_mat_create}{int m, 
%     int n, int Max_size_row, int RC}
%   \sshortdescribe Create an empty \PnlMorseMat with memory
%   allocated for each component of the array. 
% \item
%   \describefun{\PnlMorseMat\ptr }{pnl_morse_mat_create_fromfull}
%   {\PnlMat \ptr FM, int RC}
%   \sshortdescribe Create a \PnlMorseMat from  a \PnlMat
%   storing only its nonzero elements.

% \item \describefun{void}{pnl_morse_mat_free}{\PnlMorseMat\ptr \ptr  M}
%   \sshortdescribe Free a \PnlMorseMat

% \item \describefun{int}{pnl_morse_mat_freeze}{PnlMorseMat\ptr  M}
%   \sshortdescribe Set Max size equal to size for each SpRow and frees the extra
%   memory.

% \item \describefun{\PnlMat \ptr }{pnl_morse_mat_full}
%   {\PnlMorseMat\ptr  M}
%   \sshortdescribe Create a full matrix from a morse matrix.
% \end{itemize}


% \paragraph{Accessing elements}
% \begin{itemize}
% \item \describefun{double}{ pnl_morse_mat_get}{PnlMorseMat\ptr  M, int i, int j}
%   \sshortdescribe Return $M_{i, j}$. 
% \item \describefun{int}{ pnl_morse_mat_set}{PnlMorseMat\ptr  M, int i, int
%     j, double Val}
%   \sshortdescribe Do $M_{i, j} = Val$. For example, if $RC=1$ and $(i, j)$ is a valid index, replace
%   $array[i]\rightarrow Value[k]$ with $k$ such that $array[i]\rightarrow Index[k]=j$.
%   If $(i, j)$ is not a valid index, add $j$ to $array[i]\rightarrow Index$ and $Val$ to
%   $array[i] \rightarrow Value$ with memory allocation if needed. 
% \item \describefun{double\ptr }{pnl_morse_mat_lget}{PnlMorseMat\ptr  M, int
%     i, int j}
%   \sshortdescribe Return the address of $M_{i, j}$. For example, 
%   if $RC=1$ and $(i, j)$ is a valid index, replace return address of
%   $array[i]\rightarrow Value[k]$ with $k$ such that $array[i]\rightarrow
%   Index[k]=j$.  If $(i, j)$ is not a valid index, add $j$ to
%   $array[i]\rightarrow Index$ and add element to $array[i] \rightarrow Value$
%   (with memory allocation if needed), returns address of this element. In
%   practice this function is used to do $M_{i, j} += a$.
% \end{itemize}

% \paragraph{Printing Matrix}
% \begin{itemize}
% \item \describefun{void}{pnl_morse_mat_print}{const \PnlMorseMat\ptr M}
% \end{itemize}

% \paragraph{Standard matrix operations}
% \begin{itemize}
% \item \describefun{void}{pnl_morse_mat_mult_vect_inplace}{\PnlVect
%     \ptr lhs, const \PnlMorseMat\ptr M, const \PnlVect
%     \ptr rhs}
%   \sshortdescribe Compute $ lhs=M \ rhs$.
% \item \describefun{\PnlVect\ptr }{pnl_morse_mat_mult_vect}{const
%     \PnlMorseMat\ptr M, const \PnlVect \ptr vec}
%   \sshortdescribe Compute $ vec=M \ vec$.
% \end{itemize}


% \subsection{Sparse Matrix}

% \PnlSparseMat is the cs structure of the Csparse library written by
% Timothy A.Davis.  For the sake of convenience, we have renamed some functions
% and structures. We have also reduced the number of function parameters for non
% expert users in sparse matrices.  In the following, we only use the LU
% factorisation for sparse systems. If the same operator is used at each time
% step, direct methods relying on factorisations are faster than iterative
% methods. When the PDE coefficients are time dependent, the answer is not so
% clear.

% \subsubsection{Functions}
% \paragraph{Constructors and destructors}
% \begin{itemize}
% \item \describefun{\PnlSparseMat
%     \ptr }{pnl_sparse_mat_create_fromfull}{\PnlMat \ptr M}
%   \sshortdescribe Create a \PnlSparseMat from  a
%   \PnlMat storing only nonzero elements.
% \item \describefun{\PnlSparseMat
%     \ptr }{pnl_sparse_mat_create_frommorse}{\PnlMorseMat\ptr  M}
%   \sshortdescribe Create a \PnlSparseMat from  a
%   \PnlMorseMat with $M\rightarrow M->RC =1$.
% \item \describefun{void}{pnl_sparse_mat_free}{\PnlSparseMat
%     \ptr \ptr M}
%   \sshortdescribe Free a \PnlSparseMat.
% \end{itemize}

% \paragraph{Printing Matrix}
% \begin{itemize}
% \item \describefun{void}{pnl_sparse_mat_print}{\PnlSparseMat
%     \ptr A}
%   \sshortdescribe Print a \PnlSparseMat.
% \end{itemize}

% \subparagraph{Element wise operations}

% \begin{itemize}
% \item \describefun{void}{pnl_sparse_mat_plus_sparse_mat}{\PnlSparseMat \ptr lhs, 
%     const \PnlSparseMat \ptr rhs} 
%   \sshortdescribe In-place addition  

% \item \describefun{void}{pnl_sparse_mat_minus_sparse_mat}{\PnlSparseMat \ptr lhs, 
%     const \PnlSparseMat \ptr rhs} 
%   \sshortdescribe In-place substraction  

% \item \describefun{void}{pnl_sparse_mat_inv_term}{\PnlSparseMat \ptr lhs}
%   \sshortdescribe In-place term by term inversion  

% \item \describefun{void}{pnl_sparse_mat_div_mat_term}{\PnlSparseMat
%     \ptr lhs, const \PnlSparseMat \ptr rhs} 
%   \sshortdescribe In-place term by term division

% \item \describefun{void}{pnl_sparse_mat_mult_mat_term}{\PnlSparseMat
%     \ptr lhs, const \PnlSparseMat \ptr rhs} 
%   \sshortdescribe In-place term by term multiplication  



% \item \describefun{void}{pnl_sparse_mat_map_inplace}{\PnlSparseMat \ptr M, double(\ptr f)(double)}
%   \sshortdescribe Apply function \var{f} to each entry of \var{M}, which
%   is modified on exit.
% \end{itemize}


% \paragraph{Standard matrix operations}
% \begin{itemize}
% \item \describefun{int}{pnl_sparse_mat_gaxpby}{\PnlVect \ptr lhs, 
%     const \PnlSparseMat \ptr M, const \PnlVect
%     \ptr rhs}
%   \sshortdescribe Compute $lhs=lhs+ M * rhs$.
% \item \describefun{int}{pnl_sparse_mat_mult_vect_inplace}{\PnlVect
%     \ptr lhs, const \PnlSparseMat \ptr M, const
%     \PnlVect \ptr rhs}
%   \sshortdescribe Compute $lhs= M * rhs$.
% \end{itemize}

% \subsubsection{LU structure}

% From the sparse matrix, we extract the LU decomposition stored in \PnlSparseFactorization.
% \paragraph{Constructors and desctructors}
% \begin{itemize}
% \item \describefun{\PnlSparseFactorization
%     \ptr }{pnl_sparse_factorization_lu_create}{const \PnlSparseMat \ptr A, double tol}
%   \sshortdescribe Compute the LU factorisation of \var{A}

% \item \describefun{void}{pnl_sparse_factorization_free}{\PnlSparseFactorization \ptr \ptr  F}
%   \sshortdescribe Free a \PnlSparseFactorization.
% \end{itemize}

% \paragraph{Solving linear systems}

% \begin{itemize}
% \item \describefun{void}{pnl_sparse_factorization_lu_syslin}{const
%     \PnlSparseFactorization \ptr N, PnlVect \ptr b}
%   \sshortdescribe Solve the linear system \var{Nx = b} and stores the solution \var{x}
%   into \var{b} which means that the r.h.s member of the system is overwritten
%   during the resolution of the system. \var{N} is the decomposition computed by
%   \reffun{pnl_sparse_factorization_lu_create}.
% \end{itemize}

%% solver

\subsection{Iterative Solvers}
\subsubsection{Overview}

The structures and functions related to solvers are declared in
\verb!pnl/pnl_linalgsolver.h!. 

\describestruct{PnlIterationBase}
\describestruct{PnlCgSolver}
\describestruct{PnlBicgSolver}
\describestruct{PnlGmresSolver}
\begin{verbatim}
typedef struct _PnlIterationBase PnlIterationBase;
typedef struct _PnlCgSolver PnlCgSolver;
typedef struct _PnlBicgSolver PnlBicgSolver;
typedef struct _PnlGmresSolver PnlGmresSolver;

struct _PnlIterationBase
{
  /** 
   * Must be the first element in order for the object mechanism to work
   * properly. This allows any PnlVectXXX pointer to be cast to a PnlObject
   */
  PnlObject object; 
  int iteration;
  int max_iter;
  double normb;
  double tol_;
  double resid;
  int error;
  /* char *  err_msg; */
};

/* When you repeatedly use iterative solvers, do not malloc each time */
struct _PnlCgSolver 
{
  /** 
   * Must be the first element in order for the object mechanism to work
   * properly. This allows any PnlCgSolver  pointer to be cast to a PnlObject
   */
  PnlObject object; 
  PnlVect * r;
  PnlVect * z;
  PnlVect * p;
  PnlVect * q;
  double rho;
  double oldrho;
  double beta;
  double alpha;
  PnlIterationBase * iter;
} ;

struct _PnlBicgSolver 
{ 
  /** 
   * Must be the first element in order for the object mechanism to work
   * properly. This allows any PnlBicgSolver pointer to be cast to a PnlObject
   */
  PnlObject object; 
  double rho_1, rho_2, alpha, beta, omega;
  PnlVect * p;
  PnlVect * phat;
  PnlVect * s;
  PnlVect * shat;
  PnlVect * t;
  PnlVect * v;
  PnlVect * r;
  PnlVect *  rtilde;
  PnlIterationBase * iter;
} ;

struct _PnlGmresSolver
{ 
  /** 
   * Must be the first element in order for the object mechanism to work
   * properly. This allows any PnlGmresSolver pointer to be cast to a PnlObject
   */
  PnlObject object; 
  int restart;
  double beta;
  PnlVect * s;
  PnlVect * cs;
  PnlVect * sn;
  PnlVect * w;
  PnlVect * r;
  PnlMat * H;
  PnlVect * v[MAX_RESTART];
  PnlIterationBase *iter;
  PnlIterationBase *iter_inner;
} ;
\end{verbatim}

A Left preconditioner solves the problem :
$$ P M x = P b, $$
and whereas right preconditioner solves
$$ M P y  = b, \quad \quad P y = x.$$

%% With some simplifications, the number of algorithm iterations depends on
%% conditioning. Conditioning is ratio of maximum eigenvalue over minimum
%% eigenvalue of $M$. For GMRES algorithm is depend of conditioning of $M^{T}
%% M$. So if we can find $P_L$ and $P_R$ such that $P_L M P_R$ is closed to
%% identity matrix, then preconditioning problem converge faster than initial
%% problem. We have also to solve $P_R y = x$ so $P_R$ has to be constructed to
%% do that fast.

More information is given in {\em Saad, Yousef (2003). Iterative methods for
  sparse linear systems (2nd ed. ed.). SIAM. ISBN 0898715342. OCLC 51266114}.
The reader will find in this book some discussion about right or/and left
preconditioner and a description of the following algorithms.

These algorithms, we implemented with a left preconditioner. Right preconditioner
can be easily computed changing matrix vector multiplication operator from $M \
x $ to $ M \ P_R \ x$ and solving $P_R y = x$ at the end of algorithm.


\subsubsection{Functions}

Three methods are implemented : Conjugate Gradient, BICGstab and GMRES with
restart. For each of them a structure is created to store temporary vectors
used in the algorithm. In some cases, we have to apply iterative methods more
than once : for example to solve at each time step a discrete form of an
elliptic problem come from parabolic problem. In the cases, do not call the constructor and
destructor at each time, but instead use the initialization and solve procedures.

Formally we have, 
\begin{verbatim}
Create iterative method
For each time step
  Initialisation of iterative method
  Solve linear system link to elliptic problem
end for
free iterative method
\end{verbatim}

In these functions, we don't use any particular matrix structure. We give the
matrix vector multiplication as a parameter of the solver. 

\paragraph{Conjugate Gradient method}

Only available for symmetric and positive matrices.
\begin{itemize}
  \item \describefun{\refstruct{PnlCgSolver} \ptr }{pnl_cg_solver_new}{}
    \sshortdescribe Create an empty \refstruct{PnlCgSolver}  
\item   \describefun{\refstruct{PnlCgSolver} \ptr }{pnl_cg_solver_create}{int Size, int max-iter, double tolerance}
  \sshortdescribe Create a new \refstruct{PnlCgSolver} pointer.  
\item \describefun{void}{pnl_cg_solver_initialisation}{\refstruct{PnlCgSolver} \ptr Solver, const \PnlVect \ptr b}
  \sshortdescribe Initialisation of the solver at the beginning of iterative method.  
\item \describefun{void}{pnl_cg_solver_free}{\refstruct{PnlCgSolver} \ptr \ptr Solver}
  \sshortdescribe Destructor of iterative solver  
\item \describefun{int}{pnl_cg_solver_solve}{void(\ptr matrix vector-product)(const void \ptr , const \PnlVect \ptr , const double, const double, \PnlVect \ptr ), const void \ptr Matrix-Data, void(\ptr matrix vector-product-PC)(const void \ptr , const \PnlVect \ptr , const double, const double, \PnlVect \ptr ), const void \ptr PC-Data, \PnlVect \ptr x, const \PnlVect \ptr b, \refstruct{PnlCgSolver} \ptr Solver}
  \sshortdescribe Solve the linear system matrix vector-product is the matrix vector multiplication function matrix vector-product-PC is the preconditionner function Matrix-Data \& PC-Data is data to compute matrix vector multiplication.  
\end{itemize}
\paragraph{BICG stab}
\begin{itemize}
  \item \describefun{\refstruct{PnlBicgSolver} \ptr }{pnl_bicg_solver_new}{}
    \sshortdescribe Create an empty \refstruct{PnlBicgSolver}.  
\item \describefun{\refstruct{PnlBicgSolver} \ptr }{pnl_bicg_solver_create}{int Size, int max-iter, double tolerance}
  \sshortdescribe Create a new \refstruct{PnlBicgSolver} pointer.  
\item \describefun{void}{pnl_bicg_solver_initialisation}{\refstruct{PnlBicgSolver} \ptr Solver, const \PnlVect \ptr b}
  \sshortdescribe Initialisation of the solver at the beginning of iterative method.  
\item \describefun{void}{pnl_bicg_solver_free}{\refstruct{PnlBicgSolver} \ptr \ptr Solver}
  \sshortdescribe Destructor of iterative solver  
\item \describefun{int}{pnl_bicg_solver_solve}{void(\ptr matrix vector-product)(const void \ptr , const \PnlVect \ptr , const double, const double, \PnlVect \ptr ), const void \ptr Matrix-Data, void(\ptr matrix vector-product-PC)(const void \ptr , const \PnlVect \ptr , const double, const double, \PnlVect \ptr ), const void \ptr PC-Data, \PnlVect \ptr x, const \PnlVect \ptr b, \refstruct{PnlBicgSolver} \ptr Solver}
  \sshortdescribe Solve the linear system matrix vector-product is the matrix vector multiplication function matrix vector-product-PC is the preconditioner function Matrix-Data \& PC-Data is data to compute matrix vector multiplication.  
\end{itemize}

\paragraph{GMRES with restart} See {\em Saad, Yousef (2003)} for a discussion
about the restart parameter. For GMRES we need to store at the p-th iteration
$p$ vectors of the same size of the right and side. It could be very expensive
in term of memory allocation. So GMRES with restart algorithm stop if
$p=restart$ and restarts the algorithm with the previously computed solution
as initial guess.

Note that if restart equals $m$, we have a classical GMRES algorithm.

\begin{itemize}
  \item \describefun{\refstruct{PnlGmresSolver} \ptr }{pnl_gmres_solver_new}{}
    \sshortdescribe Create an empty \refstruct{PnlGmresSolver}  
\item \describefun{\refstruct{PnlGmresSolver} \ptr }{pnl_gmres_solver_create}{int Size, int max-iter, int restart, double tolerance}
  \sshortdescribe Create a new \refstruct{PnlGmresSolver} pointer.  
\item \describefun{void}{pnl_gmres_solver_initialisation}{\refstruct{PnlGmresSolver} \ptr Solver, const \PnlVect \ptr b}
  \sshortdescribe Initialisation of the solver at the beginning of iterative method.  
\item \describefun{void}{pnl_gmres_solver_free}{\refstruct{PnlGmresSolver} \ptr \ptr Solver}
  \sshortdescribe Destructor of iterative solver  
\item \describefun{int}{pnl_gmres_solver_solve}{void(\ptr matrix vector-product)(const void \ptr , const \PnlVect \ptr , const double, const double, \PnlVect \ptr ), const void \ptr Matrix-Data, void(\ptr matrix vector-product-PC)(const void \ptr , const \PnlVect \ptr , const double, const double, \PnlVect \ptr ), const void \ptr PC-Data, \PnlVect \ptr x, const \PnlVect \ptr b, \refstruct{PnlGmresSolver} \ptr Solver}
  \sshortdescribe Solve the linear system matrix vector-product is the matrix vector multiplication function matrix vector-product-PC is the preconditionner function Matrix-Data \& PC-Data is data to compute matrix vector multiplication.  
\end{itemize}


In the next paragraph, we write all the solvers for \PnlMat. This
will be done as follows: construct an application matrix vector.
\begin{verbatim}
static void pnl_mat_mult_vect_applied(const void *mat, const PnlVect *vec, 
                                      const double a , const double b, 
                                      PnlVect *lhs)
{pnl_mat_lAxpby(a, (PnlMat*)mat, vec, b, lhs);}
\end{verbatim}
and give it as the parameter of the iterative method
\begin{verbatim}
int pnl_mat_cg_solver_solve(const PnlMat * Matrix, const PnlMat * PC, 
                            PnlVect * x, const PnlVect *b, PnlCgSolver * Solver)
{ return pnl_cg_solver_solve(pnl_mat_mult_vect_applied, 
                             Matrix, pnl_mat_mult_vect_applied, 
                             PC, x, b, Solver);}
\end{verbatim}

In practice, we cannot define all iterative methods for all structures.
With this implementation, the user can easily :
\begin{itemize}
\item implement right precondioner, 
\item implement method with sparse matrix and diagonal preconditioner, or
  special combination of this form $\dots$
\end{itemize}


\paragraph{Iterative algorithms for \PnlMat}


\begin{itemize}
\item \describefun{int}{pnl_mat_cg_solver_solve}{const \PnlMat \ptr M, const \PnlMat \ptr PC, \PnlVect \ptr x, const \PnlVect \ptr b, \refstruct{PnlCgSolver} \ptr Solver}
  \sshortdescribe Solve the linear system \var{M x = b} with preconditionner PC.  
\item \describefun{int}{pnl_mat_bicg_solver_solve}{const \PnlMat \ptr M, const \PnlMat \ptr PC, \PnlVect \ptr x, const \PnlVect \ptr b, \refstruct{PnlBicgSolver} \ptr Solver}
  \sshortdescribe Solve the linear system \var{M x = b} with preconditionner PC.  
\item \describefun{int}{pnl_mat_gmres_solver_solve}{const \PnlMat \ptr M, const \PnlMat \ptr PC, \PnlVect \ptr x, \PnlVect \ptr b, \refstruct{PnlGmresSolver} \ptr Solver}
  \sshortdescribe Solve the linear system \var{M x = b} with preconditionner PC.
\end{itemize}

% vim:spelllang=en:spell:
