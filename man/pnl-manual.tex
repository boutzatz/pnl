
\documentclass[a4paper,11pt,twoside]{article}
\usepackage{a4wide}
\usepackage{t1enc,lmodern}
\usepackage{hyperref}
\usepackage{textcomp,color}
\usepackage{verbatim,moreverb,alltt}
\usepackage{makeidx}
\usepackage[latin1]{inputenc}
\usepackage[english]{babel}
\usepackage[strings]{underscore}
\usepackage{amsmath,amsfonts}

\synctex=1

%% --------------------------------------------
\hbadness=10000
\emergencystretch=\hsize
\tolerance=9999
\parindent=0pt
%% --------------------------------------------

\newcommand{\R}{{\mathbb R}}
\newcommand{\C}{{\mathbb C}}
\newcommand{\N}{{\mathbb N}}
\newcommand{\ptr}{\textasteriskcentered}
\newcommand{\cD}{{\mathcal D}}
\newcommand{\ind}[1]{1_{\left\{#1\right\}}}

%%\renewcommand{\exp}[1]{\operatorname{e}^{ #1 } }
\newcommand{\cotan}{\operatorname{cotan}}
\newcommand{\cotanh}{\operatorname{cotanh}}
\newcommand{\expp}[1]{\operatorname{e}^{ #1 } }
\newcommand{\paren}[1]{\left( #1 \right)}
\newcommand{\recaco}[1]{\left [ #1 \right ]}
\newcommand{\crochet}[1]{\left \langle #1 \right \rangle}
\newcommand{\acolade}[1]{\left\{ #1 \right\}}
\newcommand{\real}[1]{\operatorname{Re}(#1)}

\makeatletter

% For pdflatex generation
\ifx\HCode\undefined%

\def\var#1{{\tt #1}}
% Describing functions
\newcommand{\describefunNoStar}[3]{%
  \index{#2}\label{#2} {#1~{\bf #2}~(#3)}}
\newcommand{\describefunStar}[3]{%
  {#1~{\bf #2}~(#3)}}
\newcommand{\describefun}{\@ifstar
  \describefunStar%
  \describefunNoStar%
}

% Describing struct
\newcommand{\describestruct}[1]{\index{Structs!#1}\label{#1}}

% Describing variables
\newcommand{\describevar}[2]{{#1~{\bf #2}}}

\newcommand{\describemacro}[2]{%
  \index{#1}\label{#1} {{\bf #1}~(#2)}}

\newcommand{\constentry}[2]{%
  \index{#1}\label{#1}{\bf #1} &&  #2 \\ }

\definecolor{Red}{rgb}{1,0,0}
\definecolor{Blue}{rgb}{0,0,1}
\def\refstruct#1{\hyperref[#1]{\color{Red} #1}}
\def\reffun#1{\hyperref[#1]{#1}}
\def\refmacro{\reffun}
\newenvironment{describeconst}{%
  \noindent\begin{tabular}{lp{1cm}l}}{\end{tabular}}
\def\shortdescribe{\unskip\vskip1ex{\color{Blue} Description~}}
\def\sshortdescribe{\unskip\newline\hskip1em {\color{Blue} Description~}}
\def\parameters{\unskip\newline\hskip1em {\color{Blue} Parameters~}}
\def\example{\unskip\newline\hskip1em {\color{Blue} Example~}}

% Html related stuff
\else%

\def\var#1{\HCode{<span class='var'>}#1\HCode{</span>}}
% Describing functions
\newcommand{\describefunStar}[3]{%
{\HCode{<span class='ret'>}#1\HCode{</span>}~{\bf \HCode{<span class='fun'>}#2\HCode{</span>}}~(\HCode{<span class='args'>}#3\HCode{</span>})}}
\newcommand{\describefunNoStar}[3]{%
\index{#2}\label{#2} {\HCode{<span class='ret'>}#1\HCode{</span>}~{\bf \HCode{<span class='fun'>}#2\HCode{</span>}}~(\HCode{<span class='args'>}#3\HCode{</span>})}}
\newcommand{\describefun}{\@ifstar
  \describefunStar%
  \describefunNoStar%
}

% Describing variables
\newcommand{\describevar}[2]{{#1~{\bf #2}}}

\newcommand{\describemacro}[2]{%
  \index{#1}\label{#1} {{\bf #1}~(#2)}}

\newcommand{\constentry}[2]{%
  \index{#1}\label{#1}\HCode{<span class='struct'>}#1\HCode{</span>} && #2 \\ }

% Describing struct
\newcommand{\describestruct}[1]{\index{#1}\label{#1}}

\def\refstruct#1{\hyperref[#1]{\HCode{<span class='struct'>}#1\HCode{</span>}}}
\def\reffun#1{\hyperref[#1]{#1}}
\def\refmacro{\reffun}
\newenvironment{describeconst}{%
  \noindent\begin{tabular}{lp{1cm}l}}{\end{tabular}}
\def\shortdescribe{\unskip\vskip1ex{\HCode{<span class='description'>}Description~\HCode{</span>}}}
\def\sshortdescribe{\unskip\newline\hskip1em \HCode{<span class='description'>}Description~\HCode{</span>}}
\def\parameters{\unskip\newline\hskip1em \HCode{<span class='description'>}Parameters~\HCode{</span>}}
\def\example{\unskip\newline\hskip1em \HCode{<span class='description'>}Example~\HCode{</span>}}

\fi
\makeatother

% Macro for Pnl types
\def\PnlSpMat{\refstruct{PnlSpMat}\space}
\def\PnlMat{\refstruct{PnlMat}\space}
\def\PnlMatInt{\refstruct{PnlMatInt}\space}
\def\PnlMatComplex{\refstruct{PnlMatComplex}\space}
\def\PnlHmat{\refstruct{PnlHmat}\space}
\def\PnlArray{\refstruct{PnlArray}\space}
\def\PnlBandMat{\refstruct{PnlBandMat}\space}
\def\PnlBasis{\refstruct{PnlBasis}\space}
\def\PnlCell{\refstruct{PnlCell}\space}
\def\PnlCmplxFunc{\refstruct{PnlCmplxFunc}\space}
\def\PnlHmat{\refstruct{PnlHmat}\space}
\def\PnlList{\refstruct{PnlList}\space}
\def\PnlMat{\refstruct{PnlMat}\space}
\def\PnlMatComplex{\refstruct{PnlMatComplex}\space}
\def\PnlMatInt{\refstruct{PnlMatInt}\space}
\def\PnlMorseMat{\refstruct{PnlMorseMat}\space}
\def\PnlObject{\refstruct{PnlObject}\space}
\def\PnlPermutation{\refstruct{PnlPermutation}\space}
\def\PnlRng{\refstruct{PnlRng}\space}
\def\PnlSpMat{\refstruct{PnlSpMat}\space}
\def\PnlSparseFactorization{\refstruct{PnlSparseFactorization}\space}
\def\PnlSparseMat{\refstruct{PnlSparseMat}\space}
\def\PnlTridiagMat{\refstruct{PnlTridiagMat}\space}
\def\PnlTridiagMatLU{\refstruct{PnlTridiagMatLU}\space}
\def\PnlVect{\refstruct{PnlVect}\space}
\def\PnlVectCompact{\refstruct{PnlVectCompact}\space}
\def\PnlVectComplex{\refstruct{PnlVectComplex}\space}
\def\PnlVectInt{\refstruct{PnlVectInt}\space}


\setcounter{tocdepth}{2}
\title{Pnl Manual}
\date{\today}
\author{}

\makeindex
\begin{document}
\maketitle
\tableofcontents

\section{Introduction}
\subsection{What is Pnl}

Pnl is a scientific library written in C and distributed under
the Gnu Lesser General Public Licence (LGPL). This manual is divided into four
parts.
\begin{itemize}
\item Mathematical functions: complex numbers, special functions, standard
  financial functions for the Black \& Scholes model.
\item Linear algebra : vectors, matrices (dense and sparse), hypermatrices, tridiagonal matrices,
  band matrices and the corresponding routines to manipulate them and solve linear systems.
\item Probabilistic functions: random number generators and  cumulative
  distribution functions.
\item Deterministic toolbox : FFT, Laplace inversion, numerical integration, zero searching,
  multivariate polynomial regression, $\dots$
\end{itemize}

\subsection{A few helpful conventions}

\begin{itemize}
  \item All header file names are prefixed by \verb!pnl_! and are surrounded by
    the preprocessor conditionals
\begin{verbatim}
#ifndef _PNL_MATRIX_H
#define _PNL_MATRIX_H

...

#endif /* _PNL_MATRIX_H
\end{verbatim}
All the header files are protected by an \verb!extern "C"! declaration for
possible use with a C++ compiler. The header files must be include using
\begin{verbatim}
#include "pnl/pnl_xxx.h"
\end{verbatim}

  \item All function names are prefixed by \verb!pnl_! except those implementing
    complex number arithmetic which are named following the \textit{C99}
    complex library but using a capitalised first letter \verb!C!. \\
    For example, the addition of two complex numbers is performed by the
    function \verb!Cadd!.

  \item Function containing \verb!_create! in their names always return a
    pointer to an object created by one or several calls to dynamic
    allocation. Once these objects are not used, they must be freed by calling
    the same function but ending in \verb!_free!.
    A function \verb!pnl_foo_create_yyy! returns a \verb!PnlFoo *! object (note
    the ``\ptr'') and a function \verb!pnl_foo_bar_create_yyy! returns a
    \verb!PnlFooBar *! object (note the ``\ptr'').
    These objects must be freed by calling respectively \verb!pnl_foo_free! or
    \verb!pnl_foo_bar_free!.

  \item Functions ending in \verb!_clone! take two arguments \verb!src! and
    \verb!dest! and modify \verb!dest! to make it identical to \verb!src!, ie.
    they have the same size and data. Note that no new object is allocated,
    \verb!dest! must exist before calling this function.

  \item Functions ending in \verb!_copy! create a new object identical (ie. with
    the same size and content) as its argument but independent (ie. modifying
    one of them does not alter the other). Calling \verb!A = pnl_xxx_copy(B)!
    is equivalent to first calling \verb!A = pnl_xxx_new()! function and then
    \verb!pnl_xxx_clone(A, B)!.
    
  \item Every object must implement a \verb!pnl_xxx_new! function which returns
    a pointer to an empty object with all its elements properly set to $0$. This
    means that the objects returned by the \verb!pnl_xxx_new! functions can be
    used as output arguments for functions ending in \verb!_inplace! for
    instance. They are suitable for being resized. 

  \item Functions containing \verb!_wrap_! in their names always return an
    object, not a pointer to an object, and do not make any use of dynamic
    allocation. The returned object must not be freed. 
    For instance, a function \verb!pnl_foo_wrap_xxx! returns an object
    \verb!PnlFoo! and a function \verb!pnl_foo_bar_wrap_xxx! returns an object
    \verb!PnlFooBar! 
    \begin{verbatim}
    PnlVectComplex *v1;
    PnlVectComplex v2;
    v1 = pnl_vect_complex_create_from_scalar (5, Complex(0., 1.));
    v2 = pnl_vect_complex_wrap_subvect (v1, 1, 2);

    ...

    pnl_vect_complex_free (&v1);
    \end{verbatim}
    The vector \verb!v1! is of size 5 and contains the pure imaginary number
    $i$. The vector \verb!v2! only provides a view to \verb!v1(1:1+2)!, which
    means that modifying \verb!v2! will also modify \verb!v1! and vice-versa
    because \verb!v1! shares part of its data with \verb!v2!. Note that only
    \verb!v1! must be freed and {\bf not} \verb!v2!.
  
  \item Functions ending in \verb!_init! do not create any object but only
    perform some internal initialisation.
  
  \item Hypermatrices, matrices and vectors are stored using a flat block of
    memory obtained by concatenating the matrix rows and C-style
    pointer-to-pointer arrays. Matrices are stored in row-major order, which
    means that the column index moves continuously.
    Note that this convention is not \textit{Blas \& Lapack} compliant since
    Fortran expects 2-dimensional arrays to be stored in a column-major order.

  \item Type names always begin with \verb!Pnl!, they do not contain underscores
    but instead we use capital letters to separate units in type names. \\
    Examples : \verb!PnlMat!, \verb!PnlMatComplex!.

  \item Object and function names are intimately linked : an object
    \verb!PnlFoo! is manipulated by functions starting in \verb!pnl_foo!, an
    object \verb!PnlFooBar! is manipulated by functions starting in
    \verb!pnl_foo_bar!. In table~\ref{pnltypes}, we summarise the types and their
    corresponding prefixes.

    \begin{figure}[h!]
      \centering\begin{tabular}{|l|l|}
      \hline
      Pnl types & Pnl prefix \\
      \hline
      PnlVect & pnl_vect \\
      PnlVectComplex & pnl_vect_complex \\
      PnlVectInt & pnl_vect_int \\
       &\\
      PnlMat & pnl_mat \\
      PnlMatComplex & pnl_mat_complex \\
      PnlMatInt & pnl_mat_int \\
      & \\
      PnlSpMat & pnl_sp_mat \\
      PnlSpMatComplex & pnl_sp_mat_complex \\
      PnlSpMatInt & pnl_sp_mat_int \\
      & \\
      PnlHmat & pnl_hmat \\
      PnlHmatComplex & pnl_hmat_complex \\
      PnlHmatInt & pnl_hmat_int \\
      & \\
      PnlTridiagMat & pnl_tridiag_mat \\
      PnlBandMat & pnl_band_mat \\
      & \\
      PnlList & pnl_list \\
      & \\
      PnlBasis & pnl_basis \\
      & \\
      PnlCgSolver & pnl_cg_solver \\
      PnlBicgSolver & pnl_bicg_solver \\
      PnlGmresSolver & pnl_gmres_solver \\
      \hline
    \end{tabular}
    \caption{Pnl types}
    \label{pnltypes}
  \end{figure}

  \item All macro names begin with \verb!PNL_! and are capitalised.

  \item Differences between \textbf{copy} and \textbf{clone} methods.
    The \verb!copy! methods take a single argument and return a pointer to an object
    of the same type which is an independent copy of its argument. 
    Example:
    \begin{verbatim}
    PnlVect *v1, *v2;
    v1 = pnl_vect_create_from_scalar (5, 2.5);
    v2 = pnl_vect_copy (v1);
    \end{verbatim}
    \verb!v1! and \verb!v2! are two vectors of size 5 with all their elements
    equal to 2.5. Note that \verb!v2! {\bf must not} have been created by a call
    to \verb!pnl_vect_create_xxx! because otherwise it will cause a memory leak.
    \verb!v1! and \verb!v2! are independent in the sense that a modification to
    one of them does not affect the other.

    The \verb!clone! methods take two arguments and fill the first one with the
    second one. 
    Example:
    \begin{verbatim}
    PnlVect *v1, *v2;
    v1 = pnl_vect_create_from_scalar (5, 2.5);
    v2 = pnl_vect_new ();
    pnl_vect_clone (v2, v1);
    \end{verbatim}
    \verb!v1! and \verb!v2! are two vectors of size 5 with all their elements
    equal to 2.5. Note that \verb!v2! {\bf must} have been created by a call to
    \verb!pnl_vect_new! because otherwise the function
    \verb!pnl_vect_clone!  will crash.  \verb!v1! and \verb!v2! are independent
    in the sense that a modification to one of them does not modify the other.


  \item All objects are measured using integers \verb!int! and not
    \verb!size_t!. Hence, iterations over vectors, matrices, \dots should use an
    index of type \verb!int!.

  \item In fonctions ending in \verb!inplace!, the output parameter must be different
    from any of the input parameters.
\end{itemize}


\subsection{Using Pnl}

In this section, we assume that the library is installed in the directory
\verb!$HOME/pnl-xxx!.

Once the library has been installed, the libraries can be found in the
\verb!$HOME/pnl-xxx/lib!  directory and the headers in the
\verb!$HOME/pnl-xxx/include! directory. 

\subsubsection{Compiling and Linking}

The header files of the library are installed in a root \verb!pnl! directory and
should always be included with this \verb!pnl/! prefix. So, for instance to use
random number generators you should include 
\begin{verbatim}
#include <pnl/pnl_random.h>
\end{verbatim}

\paragraph{Compiling and linking by hand}

If \verb!gcc! is used, you should pass the following options
\begin{itemize}
\item \verb!-I$HOME/pnl-xxx/include! for compiling
\item \verb!-L$HOME/pnl-xxx/lib -lpnl! for linking
\end{itemize}
This does not work straight away on all OS especially if the library is not
installed in a standard directory namely \verb+/usr/+ or \verb+/usr/local/+ for
which you need a privileged writing access.
On some systems, you may need to add to the linker flags the dependencies of the
library, which can become very tedious. Therefore, we provide a second automatic
mechanism which takes care of the dependencies on its own.

\paragraph{Compiling and linking using an automatic Makefile}

This mechanism only works under Unix (it has been tested under various Linux
distributions and Mac OS X).

First, you need to create a new directory wherever you want, put in all your
code and create a Makefile as below

To define your target just add the executable name, say \verb!my-exec!, to the
\verb!BINS!  list and create an entry \verb!my_exec_SRC! carrying the list
of source files needed to create your executable.  Note that if dashes '-' may
appear in an executable name, the name of the associated variable holding the
list of source files is obtained by replacing dashes with underscores '_' and
adding the _SRC suffix.

Assume you want to create two binaries : \verb+my-exec+ based on mixed C and C++
code (\verb+file1.c+ and \verb+file2.cpp+) and \verb+mybinary+ based on
\verb+poo1.cxx+ and \verb+poo2.cpp+. You can use the following Makefile.
\begin{verbatim}
## Flags passed to the linker
LDFLAGS=

## Flags passed to the compiler
CFLAGS=

## list of executables to create
BINS=my-exec mybinary

my_exec_SRC=file1.c file2.cpp
# optional flags for compiling and linking
my_exec_CFLAGS=
my_exec_CXXFLAGS=
my_exec_LDFLAGS=

mybinary_SRC=poo1.cxx poo2.cpp
# optional flags for compiling and linking
mybinary_CFLAGS=
mybinary_CXXFLAGS=
mybinary_LDFLAGS=


## This line must be the last one
include full_path_to_pnl_build/CMakeuser.incl
\end{verbatim}
Let us comment a little the different variables
\begin{itemize}
  \item \verb+CFLAGS+: global flags used for creating objects based on C code
  \item \verb+CXXFLAGS+: global flags used for creating objects based on C++ code
  \item \verb+LDFLAGS+: gobal linker flags.
  \item \verb+binaryname_CFLAGS+: flags used when creating the objects based on
    C code and required by \verb+binaryname+
  \item \verb+binaryname_CXXFLAGS+: flags used when creating the objects based on
    C++ code and required by \verb+binaryname+
  \item \verb+binaryname_LDFLAGS+: flags used when linking objects for creating
    \verb+binaryname+
\end{itemize}
An example of such a Makefile can be found in \verb+pnl-xxx/perso+.

\textbf{Warning:} if a file appears in the source list of several binairies, the
flags used to compile this file are determined by the ones of the first binary
involving this file. In the following example \verb+main.cpp+ will always be compiled
with the flag \verb+-O3+ even for generating \verb+bin2+
\begin{verbatim}
BINS=bin1 bin2

bin1_SRC=main.cpp poo1.c
my_exec_CXXFLAGS=-O3

bin2_SRC=main.cpp poo2.c
mybinary_CXXFLAGS=-g -O0

## This line must be the last one
include full_path_to_pnl_build/CMakeuser.incl
\end{verbatim}

\subsubsection{Inline Functions and getters}
\label{sec:inline}

It it is supported by your compiler, getter and setter functions are declared
as inline functions. This is automatically detected when running CMake. By
default, setter and getter functions check that the required access is valid,
basically it boils down to checking whether the index of the access is within an
acceptable range. These extra tests can become very expensive when getter and
setter function are intensively called. 

Thus,  it is possible to alter this default behaviour by defining the macro
\texttt{PNL_RANGE_CHECK_OFF}. This macro is automatically defined when the
library is compiled in Release mode, ie. with \verb!-DCMAKE_BUILD_TYPE=Release!
passed to CMake.


\section{Objects}

\subsection{The top-level object}

The PnlObject structure is used to simulate some inheritance between the
ojbects of Pnl.  It must be the first element of all the objects existing in
Pnl so that casting any object to a PnlObject is legal

\begin{verbatim}
typedef unsigned int PnlType; 

typedef void (destroy_func) (void **);
struct _PnlObject
{
  PnlType type; /*!< a unique integer id */
  const char *label; /*!< a string identifier (for the moment not useful) */
  PnlType parent_type; /*!< the identifier of the parent object is any,
                          otherwise parent_type=id */
  destroy_func *destroy; /*!< frees an object */
};
\end{verbatim}

Here is the list of all the types actually defined
\begin{table}
  \centering
  \begin{tabular}{l|l}
    \hline
    PnlType & Description \\
    \hline
    PNL_TYPE_VECTOR & general vectors  \\
    PNL_TYPE_VECTOR_DOUBLE & real vectors \\
    PNL_TYPE_VECTOR_INT & integer vectors \\
    PNL_TYPE_VECTOR_COMPLEX & complex vectors \\
    PNL_TYPE_MATRIX & general matrices  \\
    PNL_TYPE_MATRIX_DOUBLE & real matrices \\
    PNL_TYPE_MATRIX_INT & integer matrices \\
    PNL_TYPE_MATRIX_COMPLEX & complex matrices \\
    PNL_TYPE_TRIDIAG_MATRIX & general tridiagonal matrices \\
    PNL_TYPE_TRIDIAG_MATRIX_DOUBLE & real  tridiagonal matrices \\
    PNL_TYPE_BAND_MATRIX & general band matrices \\
    PNL_TYPE_BAND_MATRIX_DOUBLE & real band matrices \\
    PNL_TYPE_HMATRIX & general hyper matrices \\
    PNL_TYPE_HMATRIX_DOUBLE & real hyper matrices \\
    PNL_TYPE_HMATRIX_INT & integer hyper matrices \\
    PNL_TYPE_HMATRIX_COMPLEX & complex hyper matrices \\
    PNL_TYPE_BASIS & bases \\
    PNL_TYPE_RNG & random number generators \\
    PNL_TYPE_LIST & doubly linked list
  \end{tabular}
  \caption{PnlTypes}
  \label{types}
\end{table}

We provide several macros for manipulating PnlObejcts.
\begin{itemize}
\item \describemacro{PNL_OBJECT}{o}
  \sshortdescribe Casts any object into a PnlObject

\item \describemacro{PNL_VECT_OBJECT}{o}
  \sshortdescribe Casts any object into a PnlVectObject

\item \describemacro{PNL_MAT_OBJECT}{o}
  \sshortdescribe Casts any object into a PnlMatObject

\item \describemacro{PNL_HMAT_OBJECT}{o}
  \sshortdescribe Casts any object into a PnlHmatObject

\item \describemacro{PNL_BAND_MAT_OBJECT}{o}
  \sshortdescribe Casts any object into a PnlBandMatObject

\item \describemacro{PNL_TRIDIAGMAT_OBJECT}{o}
  \sshortdescribe Casts any object into a PnlTridiagMatObject

\item \describemacro{PNL_BASIS_OBJECT}{o}
  \sshortdescribe Casts any object into a PnlBasis

\item \describemacro{PNL_RNG_OBJECT}{o}
  \sshortdescribe Casts any object into a PnlRng

\item \describemacro{PNL_LIST_OBJECT}{o}
  \sshortdescribe Casts any object into a PnlList

\item \describemacro{PNL_GET_TYPENAME}{o}
  \sshortdescribe Returns the name of the type of any object inheriting from PnlObject

\item \describemacro{PNL_GET_TYPE}{o}
  \sshortdescribe Returns the type of any object inheriting from PnlObject
  
\item \describemacro{PNL_GET_PARENT_TYPE}{o}
  \sshortdescribe Returns the parent type of any object inheriting from PnlObject
\end{itemize}

\begin{itemize}
\item \describefun{\refstruct{PnlObject}\ptr }{pnl_object_create}{PnlType t}
  \sshortdescribe Creates an empty PnlObject of type \var{t} which can any of
  the registered types, see Table~\ref{types}.
\end{itemize}

\subsection{List object}

This section describes functions for creating an manipulating lists. Lists are
internally stored as doubly linked lists.

The structures and functions related to lists are declared in
\verb!pnl/pnl_list.h!.

\begin{verbatim}
typedef struct _PnlCell PnlCell;
struct _PnlCell
{
  struct _PnlCell *prev;  /*!< previous cell or 0 */
  struct _PnlCell *next;  /*!< next cell or 0 */
  PnlObject *self;       /*!< stored object */
};


typedef struct _PnlList PnlList;
struct _PnlList
{
  /**
   * Must be the first element in order for the object mechanism to work
   * properly. This allows any PnlList pointer to be cast to a PnlObject
   */
  PnlObject object; 
  PnlCell *first; /*!< first element of the list */
  PnlCell *last; /*!< last element of the list */
  PnlCell *curcell; /*!< last accessed element,
                         if never accessed is NULL */
  int icurcell; /*!< index of the last accessed element,
                     if never accessed is NULLINT */
  int len; /*!< length of the list */
};
\end{verbatim}

\begin{itemize}
\item \describefun{\refstruct{PnlList} \ptr }{pnl_list_new}{}
  \sshortdescribe Creates an empty list
\item \describefun{\refstruct{PnlCell} \ptr }{pnl_cell_new}{}
  \sshortdescribe Creates an cell list
\item \describefun{void}{pnl_list_free}{\refstruct{PnlList}  \ptr \ptr L}
  \sshortdescribe Frees a list
\item \describefun{void}{pnl_cell_free}{\refstruct{PnlCell}  \ptr \ptr c}
  \sshortdescribe Frees a list
\item \describefun{\refstruct{PnlObject}\ptr}{pnl_list_get}{
    \refstruct{PnlList} \ptr L, int i}
  \sshortdescribe This function returns the content of the \var{i}--th cell of
  the list \var{L}. This function is optimized for linearly accessing all the
  elements, so it can be used inside a for loop for instance.
\item \describefun{void}{pnl_list_insert_first}{\refstruct{PnlList}  \ptr L,
    \refstruct{PnlObject}  \ptr o}
  \sshortdescribe Insert the object \var{o} on top of the list \var{L}. Note that
  \var{o} is not copied in \var{L}, so do  {\bf not} free \var{o} yourself, it
  will be done automatically when calling \reffun{pnl_list_free}
\item \describefun{void}{pnl_list_insert_last}{\refstruct{PnlList}  \ptr L,
    \refstruct{PnlObject}  \ptr o}
  \sshortdescribe Insert the object \var{o} at the bottom of the list \var{L}. Note that
  \var{o} is not copied in \var{L}, so do  {\bf not} free \var{o} yourself, it
  will be done automatically when calling \reffun{pnl_list_free}
\item \describefun{void}{pnl_list_remove_last}{\refstruct{PnlList}  \ptr L}
  \sshortdescribe Removes the last element of the list \var{L} and frees it.
\item \describefun{void}{pnl_list_remove_first}{\refstruct{PnlList}  \ptr L}
  \sshortdescribe Removes the first element of the list \var{L} and frees it.
\item \describefun{void}{pnl_list_remove_i}{\refstruct{PnlList}  \ptr L, int i}
  \sshortdescribe Removes the \var{i-th} element of the list \var{L} and frees it.
\item \describefun{void}{pnl_list_concat}{\refstruct{PnlList}  \ptr L1,
    \refstruct{PnlList}  \ptr L2}
  \sshortdescribe Concatenates the two lists \var{L1} and \var{L2}. The
  resulting list is store in \var{L1} on exit. Do {\bf not} free \var{L2}
  since concatenation does not actually copy objects but only manipulates
  addresses.
\item \describefun{void}{pnl_list_print}{\refstruct{PnlList}  \ptr L}
  \sshortdescribe Not yet implemented because it would require that the
  structure \refstruct{PnlObject} has a field copy.
\end{itemize}

\subsection{Array object}

This section describes functions for creating an manipulating arrays of
PnlObjects.

The structures and functions related to arrays are declared in
\verb!pnl/pnl_array.h!.

\begin{verbatim}
typedef struct _PnlArray PnlArray;
struct _PnlArray
{
  /**
   * Must be the first element in order for the object mechanism to work
   * properly. This allows any PnlArray pointer to be cast to a PnlObject
   */
  PnlObject object; 
  int size;
  PnlObject **array;
  int mem_size;
};
\end{verbatim}

\begin{itemize}
\item \describefun{\refstruct{PnlArray} \ptr }{pnl_array_new}{}
  \sshortdescribe Creates an empty array
\item \describefun{void}{pnl_array_free}{\refstruct{PnlArray}  \ptr \ptr L}
  \sshortdescribe Frees an array
\item \describefun{int}{pnl_array_resize}{\refstruct{PnlArray} \ptr  v, int size}
  \sshortdescribe Resizes \var{v} to be \var{size} long. As much as possible of
  the original data is kept.
\item \describefun{\refstruct{PnlObject}\ptr}{pnl_array_get}{
    \refstruct{PnlArray} \ptr , int i}
  \sshortdescribe This function returns the content of the \var{i}--th cell of
  the array \var{L}. No copy is made
\item \describefun{\refstruct{PnlObject}\ptr}{pnl_array_set}{
    \refstruct{PnlArray} \ptr T, int i, \refstruct{PnlObject}\ptr O}
  \sshortdescribe \var{T[i] = O}. No copy is made, so the object \var{O} must
  not be freed manually.
\item \describefun{void}{pnl_array_print}{\refstruct{PnlArray}  \ptr L}
  \sshortdescribe Not yet implemented because it would require that the
  structure \refstruct{PnlObject} has a field copy.
\end{itemize}


%%% Local Variables: 
%%% mode: latex
%%% TeX-master: "pnl-manual"
%%% End: 

\section{Mathematical framework}

%% --------------------------------------------------------------------- %%
%% Mathtools
\subsection{General tools}
\subsubsection{Short Description}

\subsubsection{Constants} A few mathematical constants are provided by the
library. Most of them are actually already defined in {\tt math.h}, {\tt
  values.h} or {\tt limits.h} and a few others have been added.
\begin{describeconst}
  \constentry{M_E}{$e^1$}
  \constentry{M_LOG2E}{$\log_2 e$}
  \constentry{M_LOG10E}{$\log_{10} e$}
  \constentry{M_LN2}{$\log_e 2$}
  \constentry{M_LN10}{$\log_e 10$}
  \constentry{M_PI}{$\pi$}
  \constentry{M_2PI}{$2 \pi$}
  \constentry{M_PI_2}{$\pi/2$}
  \constentry{M_PI_4}{$\pi/4$}
  \constentry{M_1_PI}{$1/\pi$}
  \constentry{M_2_PI}{$2/\pi$}
  \constentry{M_2_SQRTPI}{$2/\sqrt{\pi}$}
  \constentry{M_SQRT2}{$\sqrt{2}$}
  \constentry{M_EULER}{$\gamma = \lim_{n \rightarrow \infty } \left( \sum_{k=1}^{n} \frac {1}{k} - \ln(n) \right)$}
  \constentry{M_SQRT1_2}{$1/\sqrt{2}$}
  \constentry{M_1_SQRT2PI}{$1/\sqrt{2 \pi}$}
  \constentry{M_SQRT2_PI}{$\sqrt{2/\pi}$}
  \constentry{INT_MAX}{$2147483647$}
  \constentry{MAX_INT}{INT_MAX}
  \constentry{DBL_MAX}{$1.79769313486231470e+308$}
  \constentry{DOUBLE_MAX}{DBL_MAX}
  \constentry{DBL_EPSILON}{$2.2204460492503131e-16$}
  \constentry{PNL_NEGINF}{$-\infty$}
  \constentry{PNL_POSINF}{$+\infty$}
  \constentry{NAN}{Not a Number}
\end{describeconst}

\subsubsection{A few macros}
\begin{itemize}
\item \describemacro{PNL_IS_ODD}{int n}
  \sshortdescribe Returns $1$ if \var{n} is odd and $0$ otherwise.
\item \describemacro{PNL_IS_EVEN}{int n}
  \sshortdescribe Returns $1$ if \var{n} is even and $0$ otherwise.
\end{itemize}

\subsubsection{Functions}
\begin{itemize}
\item \describefun{int}{intapprox}{double s}
  \sshortdescribe Returns the nearest integer with the convention ({\tt
    intapprox(1.5)=1}). This function is similar to the round function
  (provided by the C library) but the result is typed as an integer and not a
  double.

\item \describefun{double}{trunc}{double s}
  \sshortdescribe Returns the nearest integer not greater than the absolute
  value of \var{s}. This function should part of C99.

\item \describefun{double}{Cnp}{int n, int p}
  \sshortdescribe Computes the binomial coefficient $\binom{n}{p}$.

\item \describefun{double}{pnl_fact}{int n}
  \sshortdescribe Computes $n! = \Gamma(n+1)$ in double precision.

\item \describefun{double}{lgamma}{double x}
  \sshortdescribe Computes $\log(\Gamma(x))$. This function is part of C99 and is
  redefined for non C99 compliant compilers.

\item \describefun{double}{tgamma}{double x}
  \sshortdescribe Computes $\Gamma(x)$. This function is part of C99 and is
  redefined for non C99 compliant compilers.

\end{itemize}

%% --------------------------------------------------------------------- %%
%% Complex
\subsection{Complex numbers}
\subsubsection{Short Description}

The first native implementation of complex numbers in the C language has been
provided by the C99 implementation, which is not available on all
platforms. For this reason, we provide here an implementation of complex
numbers and its associated calculus.

\begin{verbatim}
typedef struct {
    double r; /*!< real part */
    double i; /*!< imaginary part */
} fcomplex;
\end{verbatim}


\subsubsection{Constants}

\begin{describeconst}
  \constentry{CZERO}{$0$ as a complex number}
  \constentry{CONE}{$1$ as a complex number}
  \constentry{CI}{$I$ the unit complex number}
\end{describeconst}

\subsubsection{Functions}
\begin{itemize}
\item \describefun{double}{Creal}{\refstruct{fcomplex} z}
  \sshortdescribe $ \mathrm{R}(z) $ 

\item \describefun{double}{Cimag}{\refstruct{fcomplex} z}
  \sshortdescribe  $ \mathrm{Im}(z) $  

\item \describefun{fcomplex}{Cadd}{\refstruct{fcomplex} z, \refstruct{fcomplex} b}
  \sshortdescribe $ z+b $  

\item \describefun{fcomplex}{CRadd}{\refstruct{fcomplex} z, double b}
  \sshortdescribe $ z+b $  

\item \describefun{fcomplex}{RCadd}{double b, \refstruct{fcomplex} z}
  \sshortdescribe $ b+z $  

\item \describefun{fcomplex}{Csub}{\refstruct{fcomplex} z, \refstruct{fcomplex} b}
  \sshortdescribe $ z-b $

\item \describefun{fcomplex}{CRsub}{\refstruct{fcomplex} z, double b}
  \sshortdescribe $ z-b $

\item \describefun{fcomplex}{RCsub}{double b, \refstruct{fcomplex} z}
  \sshortdescribe $ b-z $

\item \describefun{fcomplex}{Cminus}{\refstruct{fcomplex} z}
  \sshortdescribe $ -z $  

\item \describefun{fcomplex}{Cmul}{\refstruct{fcomplex} z, \refstruct{fcomplex} b}
  \sshortdescribe $ z*b $  

\item \describefun{fcomplex}{RCmul}{double x, \refstruct{fcomplex} z}
  \sshortdescribe $ x*z $

\item \describefun{fcomplex}{CRmul}{\refstruct{fcomplex} z, double x}
  \sshortdescribe $ z * x $

\item \describefun{fcomplex}{CRdiv}{\refstruct{fcomplex} z, double x}
  \sshortdescribe $ z/x $

\item \describefun{fcomplex}{Complex}{double x, double y}
  \sshortdescribe $x + \imath \; y$  

\item \describefun{fcomplex}{Complex_polar}{double r, double theta}
  \sshortdescribe  $ r e^{\imath \theta} $  

\item \describefun{fcomplex}{Conj}{\refstruct{fcomplex} z}
  \sshortdescribe $\overline{z}$  

\item \describefun{fcomplex}{Cinv}{\refstruct{fcomplex} z}
  \sshortdescribe $ \frac{1}{z}$  

\item \describefun{fcomplex}{Cdiv}{\refstruct{fcomplex} z, \refstruct{fcomplex} w}
  \sshortdescribe $ \frac{z}{w}$  

\item \describefun{double}{Csqr_norm}{\refstruct{fcomplex} z}
  \sshortdescribe $ \mathrm{Re}(z)^2 + \mathrm{im}(z)^2 $  

\item \describefun{double}{Cabs}{\refstruct{fcomplex} z}
  \sshortdescribe $ \vert z \vert $  

\item \describefun{fcomplex}{Csqrt}{\refstruct{fcomplex} z}
  \sshortdescribe $ \sqrt{z} $,  square root (with positive real part)  

\item \describefun{fcomplex}{Clog}{\refstruct{fcomplex} z}
  \sshortdescribe $\log(z)$  

\item \describefun{fcomplex}{Cexp}{\refstruct{fcomplex} z}
  \sshortdescribe $ e^z $  

\item \describefun{fcomplex}{CIexp}{double t}
  \sshortdescribe $ e^{\imath t} $

\item \describefun{fcomplex}{Cpow}{\refstruct{fcomplex} z, \refstruct{fcomplex} w}
  \sshortdescribe $ z^w$, power function  

\item \describefun{fcomplex}{Cpow_real}{\refstruct{fcomplex} z, double x}
  \sshortdescribe $ z^x$, power function  

\item \describefun{fcomplex}{Ccos}{\refstruct{fcomplex} z}
  \sshortdescribe $ \cos(g)$  

\item \describefun{fcomplex}{Csin}{\refstruct{fcomplex} z}
  \sshortdescribe $\sin(g)$  

\item \describefun{fcomplex}{Ctan}{\refstruct{fcomplex} z}
  \sshortdescribe $\tan(z)$

\item \describefun{fcomplex}{Ccotan}{\refstruct{fcomplex} z}
  \sshortdescribe $\cotan(z)$

\item \describefun{fcomplex}{Ccosh}{\refstruct{fcomplex} z}
  \sshortdescribe $ \cosh(g)$  

\item \describefun{fcomplex}{Csinh}{\refstruct{fcomplex} z}
  \sshortdescribe $\sinh(g)$  

\item \describefun{fcomplex}{Ctanh}{\refstruct{fcomplex} z}
  \sshortdescribe $\tanh(z) = \frac{1 - \expp{-2z} }{1 + \expp{-2z} }$  

\item \describefun{fcomplex}{Ccotanh}{\refstruct{fcomplex} z}
  \sshortdescribe $\cotanh(z) = \frac{1 + \expp{-2z} }{1 - \expp{-2z} }$  

\item \describefun{double}{Carg}{\refstruct{fcomplex} z}
  \sshortdescribe $\arg(z) $

\item \describefun{fcomplex}{Ctgamma}{\refstruct{fcomplex} z}
  \sshortdescribe $ \Gamma(z)$, the Gamma function  

\item \describefun{fcomplex}{Clgamma}{\refstruct{fcomplex} z}
  \sshortdescribe $ \ln(\Gamma (z)) $, the logarithm of the Gamma function
\end{itemize}

Most algebraic operations on complex numbers are implemented using the
following naming for the functions
\begin{itemize}
\item we prefix all these functions with {\tt C_op_}, 
\item we use two complex numbers {\tt a, b} and a real number {\tt d}, 
\item we note {\tt i} the multiplication by $\imath$, 
\item {\tt c} indicates that the second argument is conjugated first.
\item we note {\tt p, m} the two standard operations {\it plus} and {\it minus}
\end{itemize}
For example C_op_idamcb is $\imath d \left( a - \overline{b} \right)$. So
functions are :
\begin{itemize}
\item \describefun{fcomplex}{C_op_apib}{\refstruct{fcomplex} a, \refstruct{fcomplex} b}
  \sshortdescribe $ a+\imath b  $.
\item \describefun{fcomplex}{C_op_apcb}{\refstruct{fcomplex} a, \refstruct{fcomplex} b}
  \sshortdescribe $ a+\overline{ b}  $.
\item \describefun{fcomplex}{C_op_amcb}{\refstruct{fcomplex} a, \refstruct{fcomplex} b}
  \sshortdescribe $ a-\overline{b}  $.
\item \describefun{fcomplex}{C_op_dapb}{double d, \refstruct{fcomplex} a, 
    \refstruct{fcomplex} b}
  \sshortdescribe $ d(a+ b)  $.
\item \describefun{fcomplex}{C_op_damb}{double d, \refstruct{fcomplex} a, 
    \refstruct{fcomplex} b}
  \sshortdescribe $ d(a-b)  $.
\item \describefun{fcomplex}{C_op_dapib}{double d, \refstruct{fcomplex} a, 
    \refstruct{fcomplex} b}
  \sshortdescribe $ d(a+\imath b)  $.
\item \describefun{fcomplex}{C_op_damib}{double d, \refstruct{fcomplex} a, 
    \refstruct{fcomplex} b}
  \sshortdescribe $ d(a-\imath b)  $.
\item \describefun{fcomplex}{C_op_dapcb}{double d, \refstruct{fcomplex} a, 
    \refstruct{fcomplex} b}
  \sshortdescribe $ d\left(a+\overline{b}\right)  $.
\item \describefun{fcomplex}{C_op_damcb}{double d, \refstruct{fcomplex} a, 
    \refstruct{fcomplex} b}
  \sshortdescribe $ d\left(a-\overline{b}\right)  $.
\item \describefun{fcomplex}{C_op_idapb}{double d, \refstruct{fcomplex} a, 
    \refstruct{fcomplex} b}
  \sshortdescribe $\imath d\left(a+b\right) $.
\item \describefun{fcomplex}{C_op_idamb}{double d, \refstruct{fcomplex} a, 
    \refstruct{fcomplex} b}
  \sshortdescribe $\imath  d\left(a-b\right) $.
\item \describefun{fcomplex}{C_op_idapcb}{double d, \refstruct{fcomplex} a, 
    \refstruct{fcomplex} b}
  \sshortdescribe $ \imath d\left(a+\overline{b}\right) $.
\item \describefun{fcomplex}{C_op_idamcb}{double d, \refstruct{fcomplex} a, 
    \refstruct{fcomplex} b}
  \sshortdescribe $ \imath  d\left(a-\overline{b}\right) $.
\end{itemize}
%% --------------------------------------------------------------------- %%




\subsection{Special Functions}

We provide approximations of the different Bessel (and Hankel) functions both
for real and complex arguments. These approximations were originally written
in Fortran in the Amos library and have been translated into C by
Jean-Philippe Chancelier, to whom we are very greatful, partly by using {\em
  f2c} partly by hand. The definitions of these functions have been extended
to negative orders by using the relations given in the {\em Amos} library

We also provide approximations of the different Incomplete Gamma Functions
(and Exponential integrals) for real arguments. These approximations come from
``Numerical Recipes''.

\subsubsection{Real Bessel Functions}

\begin{itemize}
\item \describefun{double}{pnl_bessel_i}{double v, double x}
  \sshortdescribe   Modified Bessel function of the first
  kind of order \var{v}.
\item \describefun{double}{pnl_bessel_i_scaled}{double v, double x}
  \sshortdescribe   Modified Bessel function of the first
  kind of order \var{v} divided by $e^{|x|}$.
\item \describefun{double}{pnl_bessel_j}{double v, double x}
  \sshortdescribe    Bessel function of the first
  kind of order \var{v}.
\item \describefun{double}{pnl_bessel_j_scaled}{double v, double x}
  \sshortdescribe    Bessel function of the first
  kind of order \var{v}. Same function as \reffun{pnl_bessel_j}.
\item \describefun{double}{pnl_bessel_y}{double v, double x}
  \sshortdescribe   Modified Bessel function of the second
  kind of order \var{v}.
\item \describefun{double}{pnl_bessel_y_scaled}{double v, double x}
  \sshortdescribe   Modified Bessel function of the second
  kind of order \var{v}. Same function as \reffun{pnl_bessel_y}.
\item \describefun{double}{pnl_bessel_k}{double v, double x}
  \sshortdescribe   Bessel function of the third
  kind of order \var{v}.
\item \describefun{double}{pnl_bessel_k_scaled}{double v, double x}
  \sshortdescribe   Bessel function of the third
  kind of order \var{v} multiplied by $e^{x}$.
\item \describefun{fcomplex}{pnl_bessel_h1}{double v, double x}
  \sshortdescribe   Hankel function of the first kind of
  order \var{v}. 
\item \describefun{fcomplex}{pnl_bessel_h1_scaled}{double v, double x}
  \sshortdescribe  Hankel function of the first kind of order
  \var{v}  and divided by $e^{I x}$.
\item \describefun{fcomplex}{pnl_bessel_h2}{double v, double x}
  \sshortdescribe  Hankel function of the second kind of
  order \var{v}. 
\item \describefun{fcomplex}{pnl_bessel_h2_scaled}{double v, double x}
  \sshortdescribe  Hankel function of the second kind of
  order \var{v}  and multiplied by $e^{I x}$.
\end{itemize}

\subsubsection{Complex Bessel Functions}

\begin{itemize}
\item \describefun{fcomplex}{pnl_complex_bessel_i}{double v, fcomplex z}
  \sshortdescribe  Complex Modified Bessel function of the first
  kind of order \var{v}.
\item \describefun{fcomplex}{pnl_complex_bessel_i_scaled}{double v, fcomplex z}
  \sshortdescribe  Complex Modified Bessel function of the first
  kind of order \var{v} divided by $e^{|Creal(z)|}$.
\item \describefun{fcomplex}{pnl_complex_bessel_j}{double v, fcomplex z}
  \sshortdescribe  Complex  Bessel function of the first
  kind of order \var{v}.
\item \describefun{fcomplex}{pnl_complex_bessel_j_scaled}{double v, fcomplex z}
  \sshortdescribe  Complex  Bessel function of the first
  kind of order \var{v} divided by $e^{|Cimag(z)|}$.
\item \describefun{fcomplex}{pnl_complex_bessel_y}{double v, fcomplex z}
  \sshortdescribe  Complex Modified Bessel function of the second
  kind of order \var{v}.
\item \describefun{fcomplex}{pnl_complex_bessel_y_scaled}{double v, fcomplex z}
  \sshortdescribe  Complex Modified Bessel function of the second
  kind of order \var{v} divided by $e^{|Cimag(z)|}$.
\item \describefun{fcomplex}{pnl_complex_bessel_k}{double v, fcomplex z}
  \sshortdescribe  Complex Bessel function of the third
  kind of order \var{v}.
\item \describefun{fcomplex}{pnl_complex_bessel_k_scaled}{double v, fcomplex z}
  \sshortdescribe  Complex Bessel function of the third
  kind of order \var{v} multiplied by $e^{z}$.
\item \describefun{fcomplex}{pnl_complex_bessel_h1}{double v, fcomplex z}
  \sshortdescribe  Complex Hankel function of the first kind of
  order \var{v}. 
\item \describefun{fcomplex}{pnl_complex_bessel_h1_scaled}{double v, fcomplex z}
  \sshortdescribe  Complex  Hankel function of the first kind of order
  \var{v}  and divided by $e^{I z}$.
\item \describefun{fcomplex}{pnl_complex_bessel_h2}{double v, fcomplex z}
  \sshortdescribe  Complex  Hankel function of the second kind of
  order \var{v}. 
\item \describefun{fcomplex}{pnl_complex_bessel_h2_scaled}{double v, fcomplex z}
  \sshortdescribe  Complex  Hankel function of the second kind of
  order \var{v}  and multiplied by $e^{I z}$.
\end{itemize}



\subsubsection{Incomplete Gamma Function }

For $a \in \R$ and $x>0$, the Incomplete Gamma Function is defined by
\begin{equation*}
  \Gamma(a, x)=\int_x^{\infty} \expp{-u} u^{a-1} du.
\end{equation*}
A relation similar to the one existing for the standard Gamma function holds
\begin{equation*}
  \Gamma\paren{a, x}= \frac{- x^{a} \expp{-x} + \Gamma (a+1, x)}{a}.
\end{equation*}

\begin{itemize}
\item \describefun{double}{pnl_gamma_inc_func}{double a, double x}
  \sshortdescribe   computes $\Gamma(a, x), \quad a \in \R , x \geq 0$
\item \describefun{void}{pnl_gamma_inc}{double a, double x, double $\ast$Ga, 
    double $\ast$P, double $\ast$Q}
  \sshortdescribe   computes for $a >0, x \geq 0$
  \begin{align*}
    \Gamma(a)&=\int_0^{\infty} u^{a-1} \expp{-u}du\\ 
    P(a, x) &= \frac{\Gamma(a) - \Gamma(a, x)}{\Gamma(a)} =
    \frac{1}{\Gamma(a)} \int_0^x u^{a-1} \expp{-u}  du\\ 
    Q(a, x) &= 1-P(a, x) =\frac{\Gamma(a, x)}{\Gamma(a)} =
    \frac{1}{\Gamma(a)} \int_x^{\infty} \expp{-u} u^{a-1} du. 
  \end{align*}
\end{itemize}

\subsubsection{Exponential integrals}
For $x>0$ and $n \in \N$, the  function $E_n$ is defined by
\begin{equation*}
  E_n\paren{x}=\int_{1}^{\infty} \expp{-x u} u^{-n} du
\end{equation*}

This function is linked to the Incomplete Gamma function by the relation
\begin{equation*}
  E_n\paren{x}=\int_{x}^{\infty}
  \expp{-xu} (xu)^{-n} x^{n-1} d(xu)=x^{n-1} \int_{x}^{\infty}
  \expp{-t} t^{-n} dt =  x^{n-1}  \Gamma\paren{1-n, x}, 
\end{equation*}
from which we can deduce
\begin{equation*}
  n E_{n+1}(x)  =   \expp{-x} - x E_n(x).
\end{equation*}
For $n>1$, the series expansion is given by
\begin{equation*}
  E_n(x)=x^{n-1}
  \Gamma(1-n)+\recaco{-\frac{1}{1-n}+\frac{x}{2-n}-\frac{x^2}{2(3-n)}
    +\frac{x^3}{6(4-n)}-\dots}.     
\end{equation*}
The asymptotic behaviour is given by
\begin{equation*}
  E_n(x)=\frac{\expp{-x}}{x}\recaco{1-\frac{n}{x}+\frac{n(n+1)}{x^2}+\dots}. 
\end{equation*}
The special case $n=1$ gives 
\begin{equation*}
  E_1(x) = \int_x^\infty \frac{e^{-u}}{u}\, du, \quad |\mathrm{Arg}(x)| \ge \pi. 
\end{equation*}
For any complex number $x$ with positive real part, this can be written
\begin{equation*}
  E_1(x) = \int_1^\infty \frac{e^{-ux}}{u}\, du, \quad \Re(x) \ge 0. 
\end{equation*}
By integrating the Taylor expansion of $\expp{-t}/t$, and extracting the
logarithmic singularity, we can derive the following series representation for
$E_1(x)$, 
\begin{equation*}
  {E_1}(x) =-\gamma-\ln x-\sum_{k=1}^{\infty}\frac{(-1)^k x^k}{k\; k!}
  \qquad |\mathrm{Arg}(x)| < \pi. 
\end{equation*}
The function $E_1$ is linked to the exponential integral $Ei$
\begin{equation*}
  Ei(x)=\int_{-\infty}^x\frac{e^u}u\, du=-\int_{-x}^{\infty}
  \frac{e^{-u}}{u}\, du \quad \forall x \neq 0. 
\end{equation*}
The above definition  can be used for positive values of $x$, but the integral
has to be understood in terms of its Cauchy principal value, due to the
singularity of the integrand at zero.
\begin{equation*}
  {Ei}(-x) = -{E}_1(x) , \quad \Re(x) \ge 0.
\end{equation*}
We deduce, 
\begin{equation*}
  Ei(x) = \gamma + \ln x+ \sum_{k=1}^{\infty} \frac{x^k}{k\; k!}, \quad x>0.
\end{equation*}
For $x \in \R$ 
\begin{equation*}
  \Gamma(0, x)=\left\{
    \begin{array}{l}
      -Ei(-x)-\imath \pi  \quad  x<0, \\
      -Ei(-x) \quad x>0. 
    \end{array}\right.
\end{equation*}

\begin{itemize}
\item \describefun{double}{pnl_expint_En}{int n, double x}
  \sshortdescribe   computes for  $ n\geq 0, x \geq 0$ and $x>0$ if $n=0$ or $1$.
  \begin{equation*}
    E_n(x)= \int_1^{\infty} u^{-n} \expp{-x u} du.
  \end{equation*}
\item \describefun{double}{pnl_expint_Ei}{double x}
  \sshortdescribe   computes the principal value of integral, for $x>0$ 
  \begin{equation*}
    Ei(x) = - \int_{-x}^{\infty} \frac{\expp{-u}}{u} du = \int_{-\infty}^x \frac{\expp{u}}{u} du 
  \end{equation*}
\end{itemize}



%% --------------------------------------------------------------------- %%
%% financial functions

\subsection{Financial functions}
\subsubsection{Short Description}
\subsubsection{Functions}


\begin{itemize}
\item
  \describefun{int}{pnl_cf_call_bs}{double s, double k, double T, double r, 
    double divid, double sigma, double $\ast$ptprice, double $\ast$ptdelta}
  \sshortdescribe Computes the price and delta of a call option $(\var{s} -
  \var{k})_+$ in the Black-Scholes model with volatility \var{sigma},
  instantaneous interest rate \var{r}, maturity \var{T} and dividend rate
  \var{divid}. The parameters \var{ptprice} and \var{ptdelta} are respectively
  set to the price and delta on output.
\item
  \describefun{int}{pnl_cf_put_bs}{double s, double k, double T, double r, 
    double divid, double sigma, double $\ast$ptprice, double $\ast$ptdelta}
  \sshortdescribe Computes the price and delta of a put option $(\var{k} - 
  \var{s})_+$ in the Black-Scholes model with volatility \var{sigma},
  instantaneous interest rate \var{r}, maturity \var{T} and dividend rate
  \var{divid}.  The parameters \var{ptprice} and \var{ptdelta} are respectively
  set to the price and delta on output.
\item
  \describefun{double}{pnl_cf_bs_gamma}{double s, double k, double t, double r, 
    double divid, double sigma}
  \sshortdescribe Computes the vega of a put or call option with spot \var{s}
  and strike \var{k} in the Black-Scholes model with volatility \var{sigma},
  instantaneous interest rate \var{r}, maturity \var{T} and dividend rate
  \var{divid}.
\end{itemize}


We use some practitioner's conventions. Here the functions depend on the
forward and bond prices. This can be useful for time dependent interest rate
models.

Let $F_T$ denote the forward price at maturity $T$, the expected value of $S_T$.
In the Black \& Scholes case, 
$$ F_T = S_0 \expp{(r-q)t}, $$
where $r$ is the interest rate and $q$ then dividend yield.

The expression of the Bond price, which pays $1$ at maturity time,  
is given in the BS case by
$$ B_0 = \expp{-r T}.$$

Note that:
$$ \Delta_{Forward}  \expp{(r-q)T} = \Delta_{S}, \quad \quad
\Gamma_{Forward}  \expp{2 (r-q)T} = \Gamma_{S}.$$

\begin{itemize}
\item 
  \describefun{double}{pnl_bs_call}{double Vol, double Bond, double Forward, 
    double Strike, double Maturity}
\item 
  \describefun{double}{pnl_bs_put}{double Vol, double Bond, double Forward, 
    double Strike, double Maturity}
\item 
  \describefun{double}{pnl_bs_call_delta_forward}{double Vol, double Bond, 
    double Forward, double Strike, double Maturity}
\item 
  \describefun{double}{pnl_bs_put_delta_forward}{double Vol, double Bond, 
    double Forward, double Strike, double Maturity}
\item 
  \describefun{double}{pnl_bs_call_put}{int Is_call, double Vol, double Bond, 
    double Forward, double Strike, double Maturity}
\item 
  \describefun{double}{pnl_bs_call_put_delta_forward}{int Is_call, double Vol, 
    double Bond, double Forward, double Strike, double Maturity}
\item 
  \describefun{double}{pnl_bs_vega}{double Vol, double Bond, double Forward, 
    double Strike, double Maturity}
\item 
  \describefun{double}{pnl_bs_gamma}{double Vol, double Bond, double Forward, 
    double Strike, double Maturity}
\item 
  \describefun{double}{pnl_bs_s_square_gamma}{double Vol, double Bond, 
    double Forward, double Strike, double Maturity}

  Practitioners don't speak in terms of option prices, but rather compare
  prices in terms of their implied Black \& Scholes volatilities. So this
  parameter is very useful in practice. Here we propose two functions to
  compute $\sigma_{impl}$: the first one is for one up-let, maturity,
  strike, option price.  the second function is for a list of strikes and
  maturities, a matrix of prices (with strikes varying row-wise).
\item 
  \describefun{double}{pnl_bs_implicit_vol}{int Is_call, double Price, 
    double Bond, double Forward, double Strike, double Maturity}
\item 
  \describefun{int}{pnl_bs_matrix_implicit_vol}{const \refstruct{PnlMatUint}
    $\ast$Is_Call, const \refstruct{PnlMat} $\ast$Price, double spot, 
    double rate, double divid, const \refstruct{PnlVect} $\ast$Strike, 
    const \refstruct{PnlVect} $\ast$Maturity, \refstruct{PnlMat} $\ast$Vol}
\end{itemize}




%%% Local Variables: 
%%% mode: latex
%%% TeX-master: "pnl-premia-manual"
%%% End: 

\section{Linear Algebra}


%% vector
\subsection{Vectors}
\subsubsection{Short Description}

The structures and functions related to vectors are declared in
\verb!pnl_vector.h!.


Vectors are declared for several basic types : double, int, uint and
dcomplex. In the following declarations, {\tt BASE} must be replaced by one
the previous types and the corresponding vector structures are respectively
named PnlVect, PnlVectInt, PnlVectUint, PnlVectComplex
\begin{verbatim}
typedef struct PnlVect {
  int size; /*!< size of the vector */
  int mem_size; /*!< size of the memory block allocated for array */
  double *array; /*!< pointer to store the data */
  int owner; /*!< 1 if the structure owns its array pointer */
} PnlVect;

typedef struct PnlVectUint {
  int size; /*!< size of the vector */ 
  int mem_size; /*!< size of the memory block allocated for array */
  uint *array; /*!< pointer to store the data */
  int owner; /*!< 1 if the structure owns its array pointer */
} PnlVect;

typedef struct PnlVectInt {
  int size; /*!< size of the vector */ 
  int mem_size; /*!< size of the memory block allocated for array */
  int *array; /*!< pointer to store the data */
  int owner; /*!< 1 if the structure owns its array pointer */
} PnlVect;

typedef struct PnlVectComplex {
  int size; /*!< size of the vector */ 
  int mem_size; /*!< size of the memory block allocated for array */
  dcomplex *array; /*!< pointer to store the data */
  int owner; /*!< 1 if the structure owns its array pointer */
} PnlVect;
\end{verbatim}
\var{size} is the size of the vector, \var{array} is a pointer containing the
data and \var{owner} is an integer to know if the vector owns its \var{array}
pointer (\var{owner}=$1$) or shares it with another structure (\var{owner}=$0$).
\var{mem_size} is the number of elements the vector can hold at most.

\subsubsection{Functions}

\paragraph{General functions}
These functions exist for all types of vector no matter what the basic type
is. The following conventions are used to name functions operating on vectors.
Here is the table of prefixes used for the different basic types.

\begin{center}
  \begin{tabular}[t]{lll}
    type & prefix & BASE\\
    \hline
    double & pnl_vect & double \\
    \hline
    int & pnl_vect_int & int \\
    \hline
    uint & pnl_vect_uint & uint\\
    \hline
    dcomplex & pnl_vect_complex & dcomplex
  \end{tabular}
\end{center}

In this paragraph, we present the functions operating on \refstruct{PnlVect}
which exist for all types. To deduce the prototypes of these functions for
other basic types, one must replace {\tt pnl_vect} and {\tt double} according
the above table. 
\subparagraph{Constructors and destructors}
\begin{itemize}
\item \describefun{\refstruct{PnlVect} $\ast$}{pnl_vect_create}{int size}
  \sshortdescribe Creates a new \refstruct{PnlVect} pointer.  
\item \describefun{\refstruct{PnlVect} $\ast$}{pnl_vect_create_from_zero}{int size}
  \sshortdescribe Creates a new \refstruct{PnlVect} pointer and sets it to zero.  
\item \describefun{\refstruct{PnlVect} $\ast$}{pnl_vect_create_from_double}
  {int size, double x}
  \sshortdescribe Creates a new \refstruct{PnlVect} pointer and sets all
  elements t \var{x}.  
\item \describefun{\refstruct{PnlVect} $\ast$}{pnl_vect_create_from_ptr}{int
    size, const double $\ast$x}
  \sshortdescribe Creates a new \refstruct{PnlVect} pointer and copies \var{x}
  to \var{array}.  
\item \describefun{\refstruct{PnlVect} $\ast$}{pnl_vect_create_from_list}{int
    size, ...}
  \sshortdescribe Creates a new \refstruct{PnlVect} pointer of length
  \var{size} filled with the extra arguments passed to the function. The
  number of extra arguments passed must be equal to \var{size}, be aware that
  this cannot be checked inside the function.
\item \describefun{\refstruct{PnlVect}}{pnl_vect_create_wrap_array}{const double $\ast$x, 
    int size}
  \sshortdescribe Creates a \refstruct{PnlVect} pointer which contains
  \var{x}. No copy is made. It is just a container.
  
\item \describefun{\refstruct{PnlVect} $\ast$}{pnl_vect_create_from_file}
  {const char $\ast$file}
  \sshortdescribe Reads a vector from a file and creates the corresponding
  \refstruct{PnlVect}.  

\item \describefun{\refstruct{PnlVect} $\ast$}{pnl_vect_copy}{const
    \refstruct{PnlVect} $\ast$\refstruct{v}}
  \sshortdescribe This is a copying constructor. It creates a copy of a \refstruct{PnlVect}.
\item \describefun{void}{pnl_vect_clone}{\refstruct{PnlVect} $\ast$clone, 
    const \refstruct{PnlVect} $\ast$\refstruct{v}} 
  \sshortdescribe Clones a \refstruct{PnlVect}. \var{clone} must be an
  already existing  \refstruct{PnlVect}. It is resized to match the size of
  \var{v} and the data are copied. Future modifications to \var{v} will not
  affect \var{clone}.

\item \describefun{void}{pnl_vect_free}{\refstruct{PnlVect} $\ast\ast$\refstruct{v}}
  \sshortdescribe Frees a \refstruct{PnlVect} pointer and set the data pointer to NULL  
\end{itemize}

\subparagraph{Resizing vectors}
\begin{itemize}
\item \describefun{int}{pnl_vect_resize}{\refstruct{PnlVect} $\ast$\refstruct{v}, int size}
  \sshortdescribe Resizes a \refstruct{PnlVect}. The old data are kept up to
  the new size.
\item \describefun{int}{pnl_vect_resize_from_double}{\refstruct{PnlVect}
    $\ast$\refstruct{v}, int size, double x} 
  \sshortdescribe Resizes a \refstruct{PnlVect}.  The old data are kept. If
  the new size is larger than the old one, the new cells are set to \var{x}.
\item \describefun{int}{pnl_vect_resize_from_ptr}{\refstruct{PnlVect}
    $\ast$\refstruct{v}, int size, double $\ast$t} 
  \sshortdescribe Resizes a \refstruct{PnlVect} and uses \var{t} to fill the
  vector. \var{t} must be of size \var{size}.
\end{itemize}  

\subparagraph{Accessing elements}

If it is supported by the compiler, the following functions are declared
inline. You just need to define the macro \verb!HAVE_INLINE! for by passing
\verb!-DHAVE_INLINE! to gcc to use the inline versions of the following
functions.
\begin{itemize}
\item \describefun{void}{pnl_vect_set}{\refstruct{PnlVect} $\ast$\refstruct{v}, int i, double x}
  \sshortdescribe Sets v[i]=x  
\item \describefun{double}{pnl_vect_get}{const \refstruct{PnlVect} $\ast$\refstruct{v}, int i}
  \sshortdescribe Returns the value of v[i].  
\item \describefun{void}{pnl_vect_lget}{\refstruct{PnlVect} $\ast$\refstruct{v}, int i}
  \sshortdescribe Returns the address of v[i].  
\item \describefun{void}{pnl_vect_set_double}{\refstruct{PnlVect} $\ast$\refstruct{v}, double x}
  \sshortdescribe Sets all elements to x.  
\item \describefun{void}{pnl_vect_set_zero}{\refstruct{PnlVect} $\ast$\refstruct{v}}
  \sshortdescribe Sets all elements to zero.  
\end{itemize}
Equivalently to these functions, there exist macros for {\bf \refstruct{PnlVect} only}.
\begin{itemize}
\item \describefun{}{GET}{v, i}
  \sshortdescribe Returns \var{v[i]}.
  
\item \describefun{}{LET}{v, i}
  \sshortdescribe Returns \var{v[i]} as a lvalue.
\end{itemize}


\subparagraph{Printing vector}
\begin{itemize}
\item \describefun{void}{pnl_vect_print}{const \refstruct{PnlVect} $\ast$V}
  \sshortdescribe Prints a \refstruct{PnlVect}.  
\item \describefun{void}{pnl_vect_fprint}{FILE $\ast$fic, const \refstruct{PnlVect} $\ast$V}
  \sshortdescribe Prints a \refstruct{PnlVect} in file \var{fic}.  
\item \describefun{void}{pnl_vect_print_nsp}{const \refstruct{PnlVect} $\ast$V}
  \sshortdescribe Prints a vector to the standard output in a format
  compatible with Nsp.  

\item \describefun{void}{pnl_vect_fprint_nsp}{FILE $\ast$fic, const
    \refstruct{PnlVect} $\ast$V}
  \sshortdescribe Prints a vector to a file in a format compatible with Nsp.
\end{itemize}

\subparagraph{Applying external operation to vectors}

\begin{itemize}
\item \describefun{void}{pnl_vect_minus}{\refstruct{PnlVect} $\ast$lhs}
  \sshortdescribe In-place unary minus
\item \describefun{void}{pnl_vect_plus_double}{\refstruct{PnlVect} $\ast$lhs, double x}
  \sshortdescribe In-place vector scalar addition  
\item \describefun{void}{pnl_vect_minus_double}{\refstruct{PnlVect} $\ast$lhs, double x}
  \sshortdescribe In-place vector scalar substraction  
\item \describefun{void}{pnl_vect_mult_double}{\refstruct{PnlVect} $\ast$lhs, double x}
  \sshortdescribe In-place vector scalar multiplication  
\item \describefun{void}{pnl_vect_div_double}{\refstruct{PnlVect} $\ast$lhs, double x}
  \sshortdescribe In-place vector scalar division  
\end{itemize}

\subparagraph{Element wise operations}

\begin{itemize}
\item \describefun{void}{pnl_vect_plus_vect}{\refstruct{PnlVect} $\ast$lhs, 
    const \refstruct{PnlVect} $\ast$rhs} 
  \sshortdescribe In-place vector vector addition  

\item \describefun{void}{pnl_vect_minus_vect}{\refstruct{PnlVect} $\ast$lhs, 
    const \refstruct{PnlVect} $\ast$rhs} 
  \sshortdescribe In-place vector vector substraction  

\item \describefun{void}{pnl_vect_inv_term}{\refstruct{PnlVect} $\ast$lhs}
  \sshortdescribe In-place term by term vector inversion  

\item \describefun{void}{pnl_vect_div_vect_term}{\refstruct{PnlVect}
    $\ast$lhs, const \refstruct{PnlVect} $\ast$rhs} 
  \sshortdescribe In-place term by term vector division

\item \describefun{void}{pnl_vect_mult_vect_term}{\refstruct{PnlVect}
    $\ast$lhs, const \refstruct{PnlVect} $\ast$rhs} 
  \sshortdescribe In-place vector vector term by term multiplication  

\item \describefun{void}{pnl_vect_map}{\refstruct{PnlVect} $\ast$lhs, const
    \refstruct{PnlVect} $\ast$rhs, double($\ast$f)(double)} 
  \sshortdescribe Applies the function \var{f} to each element of \var{rhs} and
  stores the result in \var{lhs}

\item \describefun{void}{pnl_vect_map_inplace}{\refstruct{PnlVect} $\ast$lhs, double($\ast$f)(double)}
  \sshortdescribe Same function as \reffun{pnl_vect_map} but the result is
  stored in \var{lhs} itself.

\item \describefun{void}{pnl_vect_axpby}{double a, const \refstruct{PnlVect} $\ast$x, 
    double b, \refstruct{PnlVect} $\ast$y} 
  \sshortdescribe Computes \var{y : = a x + b y}. When \var{b==0}, the content
  of \var{y} is not used on input and instead \var{y} is resized to match \var{x}.

\item \describefun{double}{pnl_vect_sum}{const \refstruct{PnlVect} $\ast$lhs}
  \sshortdescribe Returns the sum of all the elements of a vector  

\item \describefun{void}{pnl_vect_cumsum}{\refstruct{PnlVect} $\ast$lhs}
  \sshortdescribe Computes the cumulative sum of all the elements of a
  vector. The original vector is modified

\item \describefun{double}{pnl_vect_prod}{const \refstruct{PnlVect} $\ast$V}
  \sshortdescribe Returns the product of all the elements of a vector  

\item \describefun{void}{pnl_vect_cumprod}{\refstruct{PnlVect} $\ast$lhs}
  \sshortdescribe Computes the cumulative product of all the elements of a
  vector. The original vector is modified
\end{itemize}

\subparagraph{Ordering functions}
The following functions are not defined for PnlVectComplex because there is
no total ordering on Complex numbers

\begin{itemize}
\item \describefun{double}{pnl_vect_max}{const \refstruct{PnlVect} $\ast$V}
  \sshortdescribe Returns the maximum of a a vector  

\item \describefun{double}{pnl_vect_min}{const \refstruct{PnlVect} $\ast$V}
  \sshortdescribe Returns the minimum of a vector  

\item \describefun{void}{pnl_vect_minmax}{const \refstruct{PnlVect} $\ast$, 
    double $\ast$m, double $\ast$M}
  \sshortdescribe Computes the minimum and maximum of a vector which are
  returned in  \var{m} and \var{M} respectively.
  
\item \describefun{void}{pnl_vect_min_index}{const \refstruct{PnlVect} $\ast$, 
    double $\ast$m, int $\ast$im}
  \sshortdescribe Computes the minimum of a vector and its index stored in 
  sets \var{m} and \var{im} respectively.

\item \describefun{void}{pnl_vect_max_index}{const \refstruct{PnlVect} $\ast$, 
    double $\ast$M, int $\ast$iM}
  \sshortdescribe Computes the maximum of a vector and its index stored in 
  sets \var{m} and \var{im} respectively.

\item \describefun{void}{pnl_vect_minmax_index}{const \refstruct{PnlVect}
    $\ast$, double $\ast$m, double $\ast$M, int $\ast$im, int $\ast$iM}
  \sshortdescribe Computes the minimum and maximum of a vector and the
  corresponding indices stored respectively in \var{m}, \var{M}, \var{im} and
  \var{iM}.

\item \describefun{void}{pnl_vect_qsort}{\refstruct{PnlVect} $\ast$, char order}
  \sshortdescribe Sorts a vector using a quick sort algorithm according to
  \var{order} (\verb!'i'! for increasing or \verb!'d'! for decreasing).

\item \describefun{void}{pnl_vect_qsort_index}{\refstruct{PnlVect} $\ast$,
    \refstruct{PnlVectInt} *index, char order}
  \sshortdescribe Sorts a vector using a quick sort algorithm according to
  \var{order} (\verb!'i'! for increasing or \verb!'d'! for decreasing ). On
  output, \var{index} contains the permutation used to sort the vector.
\end{itemize}

\subparagraph{Scalar products and norms}
\begin{itemize}
\item \describefun{double}{pnl_vect_norm_two}{const \refstruct{PnlVect} $\ast$V}
  \sshortdescribe Returns the two norm of a vector  

\item \describefun{double}{pnl_vect_norm_one}{const \refstruct{PnlVect} $\ast$V}
  \sshortdescribe Returns the one norm of a vector  

\item \describefun{double}{pnl_vect_norm_infty}{const \refstruct{PnlVect} $\ast$V}
  \sshortdescribe Returns the infinity norm of a vector  

\item \describefun{double}{pnl_vect_scalar_prod}{const \refstruct{PnlVect}
    $\ast$rhs1, const \refstruct{PnlVect} $\ast$rhs2} 
  \sshortdescribe Computes the scalar product between 2 vectors  
\end{itemize}

\subparagraph{Misc}

\begin{itemize}
\item \describefun{void}{pnl_vect_swap_elements}{\refstruct{PnlVect} $\ast$v,
    int i, int j}
  \sshortdescribe Exchanges \var{v[i]} and \var{v[j]}.
\item \describefun{void}{pnl_vect_reverse}{\refstruct{PnlVect} $\ast$v}
  \sshortdescribe Performs a mirror operation on v. On output \var{v[i]
    $\leftarrow$ v[n-i]} where \var{n} is the length of the vector.
\end{itemize}


\paragraph{Complex vector functions}

\begin{itemize}
\item \describefun{void}{pnl_vect_complex_mult_double}
  {\refstruct{PnlVectComplex} $\ast$lhs, double x}
  \sshortdescribe In-place multiplication by a double.

\item \describefun{PnlVectComplex$\ast$}{pnl_vect_complex_create_from_array}{int
    size, const double $\ast$re, const double $\ast$im}
  \sshortdescribe Creates a \refstruct{PnlVectComplex} given the arrays of the
  real parts \var{re} and imaginary parts \var{im}.
\item \describefun{void}{pnl_vect_complex_split_in_array}{\refstruct{PnlVectComplex}
    $\ast$v, double $\ast$re, double $\ast$im}
  \sshortdescribe Splits a complex vector into two arrays : the array of the
  real parts of the elements of \var{v} and the array of the imaginary parts
  of the elements of \var{v}.
\item \describefun{void}{pnl_vect_complex_split_in_vect}{\refstruct{PnlVectComplex}
    $\ast$v, \refstruct{PnlVect} $\ast$re, \refstruct{PnlVect} $\ast$im}
  \sshortdescribe Splits a complex vector into two \refstruct{PnlVect}s : the
  \refstruct{PnlVect} of the real parts of the elements of \var{v} and the
  \refstruct{PnlVect} of the imaginary parts of the elements of \var{v}.
\end{itemize}

There exist functions to directly access the real or imaginary parts of an
element of a complex vector. These functions also have inlined versions that
are used if the variable \var{HAVE_INLINE} was declared at compilation time.

\begin{itemize}
\item \describefun{double}{pnl_vect_complex_get_real}
  {const \refstruct{PnlVectComplex} $\ast$v, int i}
  \sshortdescribe Returns the real part of \var{v[i]}.
  
\item \describefun{double}{pnl_vect_complex_get_imag}
  {const \refstruct{PnlVectComplex} $\ast$v, int i}
  \sshortdescribe Returns the imaginary part of \var{v[i]}.

\item \describefun{double$\ast$}{pnl_vect_complex_lget_real}
  {const \refstruct{PnlVectComplex} $\ast$v, int i}
  \sshortdescribe Returns the real part of \var{v[i]} as a lvalue.

\item \describefun{double$\ast$}{pnl_vect_complex_lget_imag}
  {const \refstruct{PnlVectComplex} $\ast$v, int i}
  \sshortdescribe Returns the imaginary part of \var{v[i]} as a lvalue.

\item \describefun{void}{pnl_vect_complex_set_real}
  {const \refstruct{PnlVectComplex} $\ast$v, int i, double re}
  \sshortdescribe Sets the real part of \var{v[i]} to \var{re}.

\item \describefun{void}{pnl_vect_complex_set_imag}
  {const \refstruct{PnlVectComplex} $\ast$v, int i, double im}
  \sshortdescribe Sets the imaginary part of \var{v[i]} to \var{im}.
\end{itemize}

Equivalently to these functions, there exist macros. When the compiler is able
to handle inline code, there is no gain in using macros instead of inlined
functions at least in principle.
\begin{itemize}
\item \describefun{}{GET_REAL}{v, i}
  \sshortdescribe Returns the real part of \var{v[i]}.
  
\item \describefun{}{GET_IMAG}{v, i}
  \sshortdescribe Returns the imaginary part of \var{v[i]}.
  
\item \describefun{}{LET_REAL}{v, i}
  \sshortdescribe Returns the real part of \var{v[i]} as a lvalue.
  
\item \describefun{}{LET_IMAG}{v, i}
  \sshortdescribe Returns the imaginary part of \var{v[i]} as a lvalue.
\end{itemize}

\subsection{Compact Vectors}
\subsubsection{Short description}

\begin{verbatim}
typedef struct PnlVectCompact {
  int size; /* size of the vector */
  union {
    double val; /* single value */
    double *array; /* Pointer to double values */
  };
  char convert; /* 'a', 'd' : array, double */
} PnlVectCompact;
\end{verbatim}

\subsubsection{Functions}

\begin{itemize}
\item \describefun{\refstruct{PnlVectCompact} $\ast$}{pnl_vect_compact_create}{int n, double x}
  \sshortdescribe Allocates a \refstruct{PnlVectCompact}.  

\item \describefun{int}{pnl_vect_compact_resize}{\refstruct{PnlVectCompact}
    $\ast$\refstruct{v}, int size, double x} 
  \sshortdescribe Resizes a \refstruct{PnlVectCompact}.  

\item \describefun{\refstruct{PnlVectCompact}
    $\ast$}{pnl_vect_compact_copy} {const \refstruct{PnlVectCompact}$\ast$\refstruct{v}}
  \sshortdescribe Copies a \refstruct{PnlVectCompact}  

\item \describefun{void}{pnl_vect_compact_free}{\refstruct{PnlVectCompact} $\ast$$\ast$\refstruct{v}}
  \sshortdescribe Free a \refstruct{PnlVectCompact}  

\item \describefun{\refstruct{PnlVect} $\ast$}{pnl_vect_compact_to_pnl_vect}
  {const \refstruct{PnlVectCompact} $\ast$C} 
  \sshortdescribe Converts a \refstruct{PnlVectCompact} pointer to a \refstruct{PnlVect} pointer.  

\item \describefun{double}{pnl_vect_compact_get}{const \refstruct{PnlVectCompact} $\ast$C, int i}
  \sshortdescribe Access function  
\end{itemize}

%% matrix

\subsection{Matrices}
\subsubsection{Short Description}

The structures and functions related to matrices are declared in
\verb!pnl_matrix.h!.

\begin{verbatim}
typedef struct PnlMat{
  int m; /*!< nb rows */ 
  int n; /*!< nb columns */ 
  int mn; /*!< product m*n */
  int mem_size; /*!< size of the memory block allocated for array */
  double *array; /*!< pointer to store the data row-wise */
  int owner; /*!< 1 if the structure owns its array pointer */
} PnlMat;

typedef struct PnlMatUint{
  int m; /*!< nb rows */ 
  int n; /*!< nb columns */ 
  int mn; /*!< product m*n */
  int mem_size; /*!< size of the memory block allocated for array */
  uint *array; /*!< pointer to store the data row-wise */
  int owner; /*!< 1 if the structure owns its array pointer */
} PnlMatUint;

typedef struct PnlMatInt{
  int m; /*!< nb rows */ 
  int n; /*!< nb columns */ 
  int mn; /*!< product m*n */
  int mem_size; /*!< size of the memory block allocated for array */
  int *array; /*!< pointer to store the data row-wise */
  int owner; /*!< 1 if the structure owns its array pointer */
} PnlMatInt;

typedef struct PnlMatComplex{
  int m; /*!< nb rows */ 
  int n; /*!< nb columns */ 
  int mn; /*!< product m*n */
  int mem_size; /*!< size of the memory block allocated for array */
  dcomplex *array; /*!< pointer to store the data row-wise */
  int owner; /*!< 1 if the structure owns its array pointer */
} PnlMatComplex;
\end{verbatim}
\var{m} is the number of rows, \var{n} is the number of columns. \var{array}
is a pointer containing the data of the matrix stored linewise, The element
$(i, j)$ of the matrix is \verb!array[i*m+j]!. \var{owner} is an integer to
know if the matrix owns its \var{array} pointer (\var{owner}=$1$) or shares it
with another structure (\var{owner}=$0$). \var{mem_size} is the number of
elements the matrix can hold at most.

The following operations are implemented on matrices and vectors. \var{alpha}
and \var{beta} are real numbers, \var{A} and \var{B} are matrices and \var{x}
and \var{y} are vectors.
\begin{tabular}{ll}
  \reffun{pnl_mat_axpy} & \var{B := alpha * A + B} \\
  \reffun{pnl_mat_scalar_prod_A} & \var{y' A x} \\
  \reffun{pnl_mat_dgemm} & \var{C := alpha * op (A) * op (B) + beta * C}\\
  \reffun{pnl_mat_mult_vect_transpose_inplace} & \var{y = A' * x}\\
  \reffun{pnl_mat_mult_vect_inplace} & \var{y = A * x}\\
  \reffun{pnl_mat_lAxpby} & \var{y := alpha * A * x + beta * y}\\
  \reffun{pnl_mat_dgemv} & \var{y := alpha * op (A) * x + beta * y}\\
  \reffun{pnl_mat_dger} & \var{A := alpha x * y' + A}
\end{tabular}


\subsubsection{Generic Functions}
\paragraph{General functions}
These functions exist for all types of matrices no matter what the basic type
is. The following conventions are used to name functions operating on matrices.
Here is the table of prefixes used for the different basic types.

\begin{center}
  \begin{tabular}[t]{lll}
    type & prefix & BASE\\
    \hline
    double & pnl_mat & double \\
    \hline
    int & pnl_mat_int & int \\
    \hline
    uint & pnl_mat_uint & uint\\
    \hline
    dcomplex & pnl_mat_complex & dcomplex
  \end{tabular}
\end{center}

In this paragraph we present the functions operating on \refstruct{PnlMat}
which exist for all types. To deduce the prototypes of these functions for
other basic types, one must replace {\tt pnl_mat} and {\tt double} according
the above table.

\paragraph{Constructors and destructors}
\begin{itemize}
\item \describefun{\refstruct{PnlMat} $\ast$}{pnl_mat_create}{int m, int n}
  \sshortdescribe Creates a \refstruct{PnlMat}  with \var{m} rows and \var{n} columns.

\item \describefun{\refstruct{PnlMat} $\ast$}{pnl_mat_create_from_double}{int m, int n, double x}
  \sshortdescribe Creates a \refstruct{PnlMat} with \var{m} rows and \var{n}
  columns and sets all the elements to \var{x}

\item \describefun{\refstruct{PnlMat} $\ast$}{pnl_mat_create_from_ptr}{int m, int n, const double $\ast$x}
  \sshortdescribe Creates a \refstruct{PnlMat} with \var{m} rows and \var{n}
  columns and copies the array \var{x} to the new vector. Be sure that \var{x}
  is long enough to fill all the vector because it cannot be checked inside the function.

\item \describefun{\refstruct{PnlMat} $\ast$}{pnl_mat_create_from_list}{int
    m, int n, ...}
  \sshortdescribe Creates a new \refstruct{PnlMat} pointer of size $\var{m
    \times n}$ filled with the extra arguments passed to the function. The
  number of extra arguments passed must be equal to $\var{m \times n}$, be
  aware that this cannot be checked inside the function.

\item \describefun{PnlMat}{pnl_mat_create_wrap_array}{const double $\ast$x, 
    int m, int n}
  \sshortdescribe Creates a \refstruct{PnlMat} of size $\var{m} \times
  \var{n}$ which contains \var{x}. No copy is made. It is just a container.
  
\item \describefun{\refstruct{PnlMat} $\ast$}{pnl_mat_create_diag_from_ptr}
  {const double $\ast$x, int d}
  \sshortdescribe Creates a new squared \refstruct{PnlMat} by specifying its size and
  diagonal terms as an array.

\item \describefun{\refstruct{PnlMat} $\ast$}{pnl_mat_create_diag}
  {const \refstruct{PnlVect} $\ast$V}
  \sshortdescribe Creates a new squared \refstruct{PnlMat} by specifying its diagonal
  terms in a \refstruct{PnlVect}.

\item \describefun{\refstruct{PnlMat} $\ast$}{pnl_mat_create_from_file}{const char $\ast$file}
  \sshortdescribe Reads a matrix from a file and creates the corresponding \refstruct{PnlMat}.  

\item \describefun{void}{pnl_mat_free}{\refstruct{PnlMat} $\ast$$\ast$\refstruct{v}}
  \sshortdescribe Frees a \refstruct{PnlMat} and sets \var{$\ast$v} to \var{NULL} 

\item \describefun{\refstruct{PnlMat} $\ast$}{pnl_mat_copy}{const \refstruct{PnlMat} $\ast$\refstruct{v}}
  \sshortdescribe Creates a new \refstruct{PnlMat} which is a copy of
  \var{v}.
  
\item \describefun{void}{pnl_mat_clone}{\refstruct{PnlMat} $\ast$clone, const \refstruct{PnlMat} $\ast$M}
  \sshortdescribe Clones \var{M} into \var{clone}. No no new
  \refstruct{PnlMat} is created.

\item \describefun{int}{pnl_mat_resize}{\refstruct{PnlMat} $\ast$\refstruct{v}, int m, int n}
  \sshortdescribe Resizes a \refstruct{PnlMat}. The new matrix is of size
  \var{m x n}.  
\end{itemize}  


\paragraph{Accessing elements}

\begin{itemize}
\item \describefun{void}{pnl_mat_set}{\refstruct{PnlMat} $\ast$self, int i, int j, double x}
  \sshortdescribe Sets the value of self[i, j]=x  

\item \describefun{double}{pnl_mat_get}{const \refstruct{PnlMat} $\ast$self, int i, int j}
  \sshortdescribe Gets the value of self[i, j]  

\item \describefun{double $\ast$}{pnl_mat_lget}{\refstruct{PnlMat} $\ast$self, int i, int j}
  \sshortdescribe Returns the address of self[i, j] for use as a lvalue.

\item \describefun{void}{pnl_mat_set_double}{\refstruct{PnlMat} $\ast$self, double x}
  \sshortdescribe Sets all elements of \var{self} to \var{x}.
  
\item \describefun{void}{pnl_mat_set_id}{\refstruct{PnlMat} $\ast$self}
  \sshortdescribe Sets the matrix \var{self} to the identity
  matrix. \var{self} must be a square matrix.

\item \describefun{void}{pnl_mat_set_diag}{\refstruct{PnlMat} $\ast$self,
    double x, int d}
  \sshortdescribe Sets the $\var{d}^{\text{th}}$ diagonal terms of the matrix
  \var{self} to the value \var{x}. \var{self} must be a square matrix.

\item \describefun{\refstruct{PnlVect}}{pnl_mat_wrap_row}
  {const \refstruct{PnlMat} $\ast$M, int i}
  \sshortdescribe Returns a \refstruct{PnlVect} (not a pointer) whose array is
  the \var{i}-th row of \var{M}. The new vector shares its data with the
  matrix \var{M}, which means that any modification to one of them will affect
  the other.
  
\item \describefun{\refstruct{PnlVect}}{pnl_mat_wrap_vect}
  {const \refstruct{PnlMat} $\ast$M}
  \sshortdescribe Returns a \refstruct{PnlVect} (not a pointer) whose array is
  the row-wise array of \var{M}. The new vector shares its data with the
  matrix \var{M}, which means that any modification to one of them will affect
  the other.

\item \describefun{void}{pnl_mat_get_row}{\refstruct{PnlVect}
    $\ast$V, const \refstruct{PnlMat} $\ast$M, int i}
  \sshortdescribe Extracts and copies the \var{i}-th row of \var{M} into
  \var{V}.

\item \describefun{void}{pnl_mat_get_col}{\refstruct{PnlVect} $\ast$V, 
    const \refstruct{PnlMat} $\ast$M, int j}
  \sshortdescribe Extracts and copies the \var{j}-th column of \var{M} into \var{V}.
  
\item \describefun{void}{pnl_mat_swap_rows}{\refstruct{PnlMat} $\ast$M, int i, int j}
  \sshortdescribe Swaps two rows of a matrix.  

\item \describefun{void}{pnl_mat_set_col}{\refstruct{PnlMat} $\ast$M, 
    const \refstruct{PnlVect} $\ast$V, int j}
  \sshortdescribe Replaces the \var{i}-th column of a matrix M by a vector V 

\item \describefun{void}{pnl_mat_set_row}{\refstruct{PnlMat} $\ast$M, 
    const \refstruct{PnlVect} $\ast$V, int i}
  \sshortdescribe Replaces the \var{i}-th row of a matrix M by a vector V  
\end{itemize}

Equivalently to the functions \reffun{pnl_mat_get} and \reffun{pnl_mat_set},
there exist macros for {\bf \refstruct{PnlMat} only}.
\begin{itemize}
\item \describefun{}{MGET}{M, i, j}
  \sshortdescribe Returns \var{M[i,j]}.
  
\item \describefun{}{MLET}{M, i, j}
  \sshortdescribe Returns \var{M[i,j]} as a lvalue for assignment.
\end{itemize}


\paragraph{Printing Matrices}

\begin{itemize}
\item \describefun{void}{pnl_mat_print}{const \refstruct{PnlMat} $\ast$M}
  \sshortdescribe Prints a matrix to the standard output.  

\item \describefun{void}{pnl_mat_fprint}{FILE $\ast$fic, const \refstruct{PnlMat} $\ast$M}
  \sshortdescribe Prints a matrix to a file.

\item \describefun{void}{pnl_mat_print_nsp}{const \refstruct{PnlMat} $\ast$M}
  \sshortdescribe Prints a matrix to the standard output in a format
  compatible with Nsp.  

\item \describefun{void}{pnl_mat_fprint_nsp}{FILE $\ast$fic, const
    \refstruct{PnlMat} $\ast$M}
  \sshortdescribe Prints a matrix to a file in a format compatible with Nsp.
\end{itemize}

\paragraph{Applying external operations}
\begin{itemize}
\item \describefun{void}{pnl_mat_plus_double}{\refstruct{PnlMat} $\ast$lhs, double x}
  \sshortdescribe In-place matrix scalar addition  

\item \describefun{void}{pnl_mat_minus_double}{\refstruct{PnlMat} $\ast$lhs, double x}
  \sshortdescribe In-place matrix scalar substraction  

\item \describefun{void}{pnl_mat_mult_double}{\refstruct{PnlMat} $\ast$lhs, double x}
  \sshortdescribe In-place matrix scalar multiplication  

\item \describefun{void}{pnl_mat_div_double}{\refstruct{PnlMat} $\ast$lhs, double x}
  \sshortdescribe In-place matrix scalar division  

\end{itemize}

\paragraph{Element wise operations}

\begin{itemize}
\item \describefun{void}{pnl_mat_mult_mat_term}{\refstruct{PnlMat} $\ast$lhs, 
    const \refstruct{PnlMat} $\ast$rhs} 
  \sshortdescribe In-place matrix matrix term by term product  

\item \describefun{void}{pnl_mat_div_mat_term}{\refstruct{PnlMat} $\ast$lhs, 
    const \refstruct{PnlMat} $\ast$rhs} 
  \sshortdescribe In-place matrix matrix term by term division

\item \describefun{void}{pnl_mat_map_inplace}{\refstruct{PnlMat} $\ast$lhs, 
    double($\ast$f)(double)} 
  \sshortdescribe Applies function \var{f} to all elements of \var{lhs}.

\item \describefun{void}{pnl_mat_map}{\refstruct{PnlMat} $\ast$lhs, const
    \refstruct{PnlMat} $\ast$rhs, double($\ast$f)(double)} 
  \sshortdescribe Applies function \var{f} to \var{rhs} and stores the result
  into \var{lhs}.

\item \describefun{double}{pnl_mat_sum}{const \refstruct{PnlMat} $\ast$lhs}
  \sshortdescribe Sums matrix component-wise  

\item \describefun{void}{pnl_mat_sum_vect}{\refstruct{PnlVect} $\ast$y, const
    \refstruct{PnlMat} $\ast$A, char c}
  \sshortdescribe Sums matrix column or row wise. Argument \var{c} can be
  either 'r' (to get a row vector) or 'c' (to get a column vector). When
  \var{c='r'}, $y(j) = \sum_i A_{ij}$ and when \var{c='rc}, $y(i) = \sum_j
  A_{ij}$.

\item \describefun{void}{pnl_mat_cumsum}{\refstruct{PnlMat} $\ast$A, char c} 
  \sshortdescribe Cumulative sum over the rows or columns. Argument \var{c}
  can be either 'r' to sum over the rows or 'c' to sum over the columns. When
  \var{c='r'}, $A_{ij} = \sum_{1 \le k \le i} A_{kj}$ and when \var{c='rc}, 
  $A_{ij} = \sum_{1 \le k \le j} A_{ik}$.

\item \describefun{double}{pnl_mat_prod}{const \refstruct{PnlMat} $\ast$lhs}
  \sshortdescribe Products matrix component-wise

\item \describefun{void}{pnl_mat_prod_vect}{\refstruct{PnlVect} $\ast$y, const
    \refstruct{PnlMat} $\ast$A, char c}
  \sshortdescribe Prods matrix column or row wise. Argument \var{c} can be
  either 'r' (to get a row vector) or 'c' (to get a column vector). When
  \var{c='r'}, $y(j) = \prod_i A_{ij}$ and when \var{c='rc}, $y(i) = \prod_j
  A_{ij}$.

\item \describefun{void}{pnl_mat_cumprod}{\refstruct{PnlMat} $\ast$A, char c} 
  \sshortdescribe Cumulative prod over the rows or columns. Argument \var{c}
  can be either 'r' to prod over the rows or 'c' to prod over the columns. When
  \var{c='r'}, $A_{ij} = \prod_{1 \le k \le i} A_{kj}$ and when \var{c='rc}, 
  $A_{ij} = \prod_{1 \le k \le j} A_{ik}$.
\end{itemize}

\paragraph{Ordering operations}

\begin{itemize}
\item \describefun{void}{pnl_mat_max}{const \refstruct{PnlMat} $\ast$A,
    \refstruct{PnlVect} $\ast$M, char d}
  \sshortdescribe On exit, $\var{M}(i) = \max_{j}(\var{A}(i, j))$ when \var{d='c'}
  and $\var{M}(i) = \max_{j}(\var{A}(j, i))$ when \var{d='r'}.

\item \describefun{void}{pnl_mat_min}{const \refstruct{PnlMat} $\ast$A,
    \refstruct{PnlVect} $\ast$m, char d}
  \sshortdescribe On exit, $\var{m}(i) = \min_{j}(\var{A}(i, j))$ when \var{d='c'}
  and $\var{m}(i) = \min_{j}(\var{A}(j, i))$ when \var{d='r'}.

\item \describefun{void}{pnl_mat_minmax}{const \refstruct{PnlMat} $\ast$A, 
    \refstruct{PnlVect} $\ast$m, \refstruct{PnlVect} $\ast$M, char d}
  \sshortdescribe On exit, $\var{m}(i) = \min_{j}(\var{A}(i, j))$ and $\var{M}(i) =
  \max_{j}(\var{A}(i, j))$ when \var{d='c'} and $\var{m}(i) = \min_{j}(\var{A}(j, i))$
  and $\var{M}(i) = \min_{j}(\var{A}(j, i))$ when \var{d='r'}.
  
\item \describefun{void}{pnl_mat_min_index}{const \refstruct{PnlMat} $\ast$ A, 
    \refstruct{PnlVect} $\ast$m, \refstruct{PnlVectInt} $\ast$im, char d}
  \sshortdescribe Idem as \reffun{pnl_mat_min} and \var{index} contains the
  indices of the minima. If \var{index==NULL}, the indices are not computed.

\item \describefun{void}{pnl_mat_max_index}{const \refstruct{PnlMat} $\ast$ A, 
    \refstruct{PnlVect} $\ast$M, \refstruct{PnlVectInt} $\ast$iM, char d}
  \sshortdescribe Idem as \reffun{pnl_mat_max} and \var{index} contains the
  indices of the maxima. If \var{index==NULL}, the indices are not computed.

\item \describefun{void}{pnl_mat_minmax_index}{const \refstruct{PnlMat} $\ast$
    A, \refstruct{PnlVect} $\ast$m, \refstruct{PnlVect} $\ast$M,
    \refstruct{PnlVectInt} $\ast$im, \refstruct{PnlVectInt} $\ast$iM, char d}
  \sshortdescribe Idem as \reffun{pnl_mat_minmax} and \var{im} contains the
  indices of the minima and \var{iM} contains the indices of the minima. If
  \var{im==NULL} (resp. \var{iM==NULL}, the indices of the minima
  (resp. maxima) are not computed.

\item \describefun{void}{pnl_mat_qsort}{\refstruct{PnlMat} $\ast$, char dir, char order}
  \sshortdescribe Sorts a matrix using a quick sort algorithm according to
  \var{order} (\verb!'i'! for increasing or \verb!'d'! for decreasing). The parameter \var{dir} determines
  whether the matrix is sorted by rows or columns. If \var{dir='c'}, each row
  is sorted independtly of the others whereas if \var{dir='r'}, each column
  is sorted independtly of the others.

\item \describefun{void}{pnl_qsort_index}{\refstruct{PnlMat} $\ast$,
    \refstruct{PnlMatInt} *index, char dir, char order}
  \sshortdescribe Sorts a matrix using a quick sort algorithm according to
  \var{order} (\verb!'i'! for increasing or \verb!'d'! for decreasing). The
  parameter \var{dir} determines whether the matrix is sorted by rows or
  columns. If \var{dir='c'}, each row is sorted independently of the others
  whereas if \var{dir='r'}, each column is sorted independently of the
  others. In addition to the function \reffun{pnl_mat_qsort}, the permutation
  index is computed and stored into \var{index}.
\end{itemize}


\paragraph{Standard matrix operations}
\begin{itemize}
\item \describefun{void}{pnl_mat_plus_mat}{\refstruct{PnlMat} $\ast$lhs, const
    \refstruct{PnlMat} $\ast$rhs} 
  \sshortdescribe In-place matrix matrix addition  

\item \describefun{void}{pnl_mat_minus_mat}{\refstruct{PnlMat} $\ast$lhs, 
    const \refstruct{PnlMat} $\ast$rhs} 
  \sshortdescribe In-place matrix matrix substraction  
  
\item \describefun{void}{pnl_mat_sq_transpose}{\refstruct{PnlMat} $\ast$M}
  \sshortdescribe In-place transposition of square matrices  

\item \describefun{\refstruct{PnlMat} $\ast$}{pnl_mat_transpose}{const
    \refstruct{PnlMat} $\ast$M} 
  \sshortdescribe Transposition of matrices

\item \describefun{void}{pnl_mat_axpy}{double alpha, const \refstruct{PnlMat}
    $\ast$A, \refstruct{PnlMat} $\ast$B}
  \sshortdescribe Computes \var{B := alpha * A + B}

\item \describefun{void}{pnl_mat_dger}{double alpha, const \refstruct{PnlVect}
    $\ast$x, const \refstruct{PnlVect} $\ast$y, \refstruct{PnlMat} $\ast$A}
  \sshortdescribe Computes \var{A := alpha x * y' + A}

\item \describefun{\refstruct{PnlVect} $\ast$}{pnl_mat_mult_vect}{const
    \refstruct{PnlMat} $\ast$A, const \refstruct{PnlVect} $\ast$x} 
  \sshortdescribe Matrix vector multiplication  \var{A * x}

\item \describefun{void}{pnl_mat_mult_vect_inplace}{\refstruct{PnlVect}
    $\ast$y, const \refstruct{PnlMat} $\ast$A, const \refstruct{PnlVect}
    $\ast$x} 
  \sshortdescribe In place matrix vector multiplication  \var{y = A * x}

\item \describefun{\refstruct{PnlVect} $\ast$}{pnl_mat_mult_vect_transpose}{const
    \refstruct{PnlMat} $\ast$A, const \refstruct{PnlVect} $\ast$x} 
  \sshortdescribe Matrix vector multiplication  \var{A' * x}

\item \describefun{void}{pnl_mat_mult_vect_transpose_inplace}{\refstruct{PnlVect}
    $\ast$y, const \refstruct{PnlMat} $\ast$A, const \refstruct{PnlVect}
    $\ast$x} 
  \sshortdescribe In place matrix vector multiplication  \var{y = A' * x}
  
\item \describefun{void}{pnl_mat_lAxpby}{double lambda, const \refstruct{PnlMat}
    $\ast$A, const \refstruct{PnlVect} $\ast$x, double mu, \refstruct{PnlVect} $\ast$b} 
  \sshortdescribe Computes \var{b := lambda A x + mu b}. When \var{mu==0}, the
  content of \var{b} is not used on input and instead \var{b} is resized to
  match \var{A*x}

\item \describefun{void}{pnl_mat_dgemv}{char trans, double lambda, const
    \refstruct{PnlMat} $\ast$A, const \refstruct{PnlVect} $\ast$x, double mu, 
    \refstruct{PnlVect} $\ast$b} \sshortdescribe Computes \var{b := lambda
    op(A) x + mu b}, where \var{op (X) = X} or \var{op (X) = X'}. When
  \var{mu==0}, the content of \var{b} is not used and instead \var{b} is resized
  to match \var{op(A)*x}

\item \describefun{void}{pnl_mat_dgemm}{char transA, char transB, double
    alpha, const \refstruct{PnlMat} $\ast$A, const \refstruct{PnlMat} $\ast$B, 
    double beta, \refstruct{PnlMat} $\ast$C}
  \sshortdescribe Computes \var{C := alpha * op(A) * op (B) + beta *
    C}. When beta=0, the content of \var{C} is unused and instead \var{C}
  is resized to store \var{alpha A $\ast$B}. If \var{transA='N'} or
  \var{transA='n'}, \var{op (A) = A}, whereas If \var{transA='T'} or
  \var{transA='t'}, \var{op (A) = A'}. The same holds for \var{transB}.
  
\item \describefun{\refstruct{PnlMat} $\ast$}{pnl_mat_mult_mat}{const
    \refstruct{PnlMat} $\ast$rhs1, const \refstruct{PnlMat} $\ast$rhs2} 
  \sshortdescribe Matrix multiplication  \var{rhs1 * rhs2}

\item \describefun{void}{pnl_mat_mult_mat_inplace}{\refstruct{PnlMat}
    $\ast$lhs, const \refstruct{PnlMat} $\ast$rhs1, const \refstruct{PnlMat}
    $\ast$rhs2} 
  \sshortdescribe In-place matrix multiplication  \var{lhs = rhs1 * rhs2}
\end{itemize}

\subsubsection{Functions specific to base type {\tt double}}

\paragraph{Standard matrix operations}
\begin{itemize}

\item \describefun{double}{pnl_mat_scalar_prod_A}{const \refstruct{PnlMat}
    $\ast$A, const \refstruct{PnlVect} $\ast$x, const \refstruct{PnlVect} $\ast$y}
  \sshortdescribe Computes \var{y' * A * y}

  
\item \describefun{void}{pnl_mat_exp}{\refstruct{PnlMat} $\ast$B, 
    const \refstruct{PnlMat} $\ast$A}
  \sshortdescribe Computes the matrix exponential \var{B = exp(A)}.

\item \describefun{void}{pnl_mat_log}{\refstruct{PnlMat} $\ast$B, 
    const \refstruct{PnlMat} $\ast$A}
  \sshortdescribe Computes the matrix logarithm \var{B = log(A)}. For the
  moment, this function only works if \var{A} is diagonalizable.

\item \describefun{void}{pnl_mat_eigen}{\refstruct{PnlVect} *v, \refstruct{PnlMat} $\ast$P, 
    const \refstruct{PnlMat} $\ast$A, int with_eigenvector}
  \sshortdescribe Computes the eigenvalues (storred in \var{v}) and optionally
  the eigenvectors storred columnwise in \var{P} when
  \var{with_eigenvector==TRUE}. If \var{A} is symmetric, \var{P} is orthonormalized.
\end{itemize}

\paragraph{Linear systems and matrix decompositions}

The following functions are designed to solve linear system of the from \var{A
  x = b} where \var{A} is a matrix and \var{b} is a vector except in the
functions \reffun{pnl_mat_syslin_mat} and \reffun{pnl_mat_chol_syslin_mat}
which expect the right hand side member to be a matrix too. Whenever the
vector \var{b} is not needed once the system is solved, you should consider
using ``inplace'' functions.

\begin{itemize}
\item \describefun{void}{pnl_mat_chol}{\refstruct{PnlMat} $\ast$M}
  \sshortdescribe Computes the Cholesky decomposition of \var{M}. \var{M} must
  be symmetric, the positivity is tested in the algorithm. On exist, the lower
  part of \var{M} contains the Cholesky decomposition and the upper part is
  set to zero.

\item \describefun{void}{pnl_mat_chol_robust}{\refstruct{PnlMat} $\ast$M}
  \sshortdescribe Same function as \reffun{pnl_mat_chol} except that if a negative
  eigen value greater than a given threshold is found, this eigenvalue is
  considered to be that positive.

\item \describefun{void}{pnl_mat_lu}{\refstruct{PnlMat} $\ast$A, 
    \refstruct{PnlPermutation} $\ast$p} 
  \sshortdescribe Computes a P A = LU factorization. \var{P} must be an
  already allocated  \refstruct{PnlPermutation}. On exit the decomposition is
  stored in \var{A}, the lower part of \var{A} contains L while the upper part
  (including the diagonal terms) contains U. Remember that the diagonal
  elements of $L$ are all $1$.
  
\item \describefun{void}{pnl_mat_qr}{\refstruct{PnlMat} $\ast$Q,
    \refstruct{PnlMat} $\ast$R, \refstruct{PnlPermutation} $\ast$p,
    \refstruct{PnlMat} $\ast$A} 
  \sshortdescribe Computes a $A P = QR$ decomposition. If on entry $P=NULL$,
  then the decomposition is computed without pivoting, i.e $A = QR$. When $P
  \ne NULL$, \var{P} must be an already allocated
  \refstruct{PnlPermutation}. \var{Q} is an orthogonal matrix, i.e
  $\var{Q}^{-1} = \var{Q}^{T}$ and \var{R} is an upper triangualr matrix. The
  use os pivoting improves the numerical stability when \var{A} is almost rank
  deficient, i.e when the smallest eigenvalue of \var{A} is very close to $0$.

\item \describefun{void}{pnl_mat_upper_syslin}{\refstruct{PnlVect}
    $\ast$x, const \refstruct{PnlMat} $\ast$U, const \refstruct{PnlVect}$\ast$b}
  \sshortdescribe Solves an upper triangular linear system \var{U x = b}

\item \describefun{void}{pnl_mat_lower_syslin}{\refstruct{PnlVect}
    $\ast$x, const \refstruct{PnlMat} $\ast$L, const \refstruct{PnlVect}$\ast$b}
  \sshortdescribe Solves a lower triangular linear system  \var{L x = b}
  
\item \describefun{void}{pnl_mat_chol_syslin}{\refstruct{PnlVect} $\ast$x, 
    const \refstruct{PnlMat} $\ast$chol, const \refstruct{PnlVect} $\ast$b} 
  \sshortdescribe Solves a symmetric definite positive linear system A x = b, 
  in which \var{chol} is assumed to be the Cholesky decomposition of A
  computed by \reffun{pnl_mat_chol}

\item \describefun{void}{pnl_mat_chol_syslin_inplace}{
    const \refstruct{PnlMat} $\ast$chol, \refstruct{PnlVect} $\ast$b} 
  \sshortdescribe Solves a symmetric definite positive linear system A x = b, 
  in which \var{chol} is assumed to be the Cholesky decomposition of A
  computed by \reffun{pnl_mat_chol}. The solution of the system is stored in
  \var{b} on exit.

\item \describefun{void}{pnl_mat_lu_syslin}{\refstruct{PnlVect} $\ast$x, const
    \refstruct{PnlMat} $\ast$LU, const \refstruct{PnlPermutation} $\ast$p, 
    const \refstruct{PnlVect} $\ast$b} 
  \sshortdescribe Solves a linear system A x = b using a LU decomposition.
  \var{LU} and \var{P} are assumed to be the PA = LU decomposition as computed
  by \reffun{pnl_mat_lu}. In particular, the structure of the matrix \var{LU}
  is the following : the lower part of \var{A} contains L while the upper part
  (including the diagonal terms) contains U. Remember that the diagonal
  elements of $L$ are all $1$.

\item \describefun{void}{pnl_mat_lu_syslin_inplace}{const
    \refstruct{PnlMat} $\ast$LU, const \refstruct{PnlPermutation} $\ast$p, 
    \refstruct{PnlVect} $\ast$b} 
  \sshortdescribe Solves a linear system A x = b using a LU decomposition.
  \var{LU} and \var{P} are assumed to be the PA = LU decomposition as computed
  by \reffun{pnl_mat_lu}. In particular, the structure of the matrix \var{LU}
  is the following : the lower part of \var{A} contains L while the upper part
  (including the diagonal terms) contains U. Remember that the diagonal
  elements of $L$ are all $1$. The solution of the system is stored in \var{b}
  on exit.
  
\item \describefun{void}{pnl_mat_syslin}{\refstruct{PnlVect} $\ast$x, const
    \refstruct{PnlMat} $\ast$A, const \refstruct{PnlVect} $\ast$b} 
  \sshortdescribe Solves a linear system A x = b using a LU factorization
  which is computed inside this function.

\item \describefun{void}{pnl_mat_syslin_inplace}{\refstruct{PnlMat} $\ast$A, 
    \refstruct{PnlVect} $\ast$b} 
  \sshortdescribe Solves a linear system A x = b using a LU factorization
  which is computed inside this function. The solution of the system is stored
  in \var{b} and \var{A} is overwritten by its LU decomposition.

\item \describefun{void}{pnl_mat_syslin_mat}{\refstruct{PnlMat}$\ast$A, 
    \refstruct{PnlMat} $\ast$B} 
  \sshortdescribe Solves a linear system A X = B using a LU factorization
  which is computed inside this function. \var{A} and  \var{B} are
  matrices. \var{A} must be square. The solution of the system is stored in
  \var{B} on exit. On exit, \var{A} contains the LU decomposition of the input
  matrix which is lost.

\item \describefun{void}{pnl_mat_chol_syslin_mat}{\refstruct{PnlMat}$\ast$A, 
    \refstruct{PnlMat} $\ast$B}
  \sshortdescribe Solves a linear system A X = B
  using a Cholesky factorization which is computed inside this
  function. \var{A} and \var{B} are matrices. \var{A} must be symmetric
  positive definite. The solution of the system is stored in \var{B} on
  exit. On exit, \var{A} contains the Cholesky decomposition of the input
  matrix which is lost.

\item \describefun{void}{pnl_mat_ls}{\refstruct{PnlMat}$\ast$A, 
    \refstruct{PnlVect} $\ast$b}
  \sshortdescribe Solves a linear system A x = b in the least square sense,
  i.e. $\var{x} = \arg\min_U \| A * u - b\|^2$. The solution is stored into
  \var{b} on exit. It internally uses a $AP = QR$ decomposition.

\item \describefun{void}{pnl_mat_ls_mat}{\refstruct{PnlMat}$\ast$A,
    \refstruct{PnlMat} $\ast$B}
  \sshortdescribe Solves a linear system A X = B with \var{A} and \var{B} two
  matrices in the least square sense, i.e. $\var{X} = \arg\min_U \| A * U -
  B\|^2$. The solution is stored into \var{B} on exit. It internally uses a
  $AP = QR$ decomposition. Same function as \reffun{pnl_mat_ls} but handles
  several r.h.s.

\end{itemize}


The following functions are designed to invert matrices. The authors provide
these functions although they cannot find good reasons to use them. Note that
to solve a linear system, one must used the \var{syslin} functions and not
invert the system matrix because it is much longer.
\begin{itemize}
\item \describefun{void}{pnl_mat_upper_inverse}{\refstruct{PnlMat} $\ast$A, 
    const \refstruct{PnlMat} $\ast$B}
  \sshortdescribe Inversion of an upper triangular matrix  

\item \describefun{void}{pnl_mat_lower_inverse}{\refstruct{PnlMat} $\ast$A, 
    const \refstruct{PnlMat} $\ast$B}
  \sshortdescribe Inversion of a lower triangular matrix  

\item \describefun{void}{pnl_mat_chol_inverse}{\refstruct{PnlMat}
    $\ast$inverse, const \refstruct{PnlMat} $\ast$A}
  \sshortdescribe Computes the inverse of a symmetric definite positive matrix
  A and stores the result into \var{inverse}. The inverse is computed using the
  Cholesky factorization of \var{A}.

\item \describefun{void}{pnl_mat_inverse}{\refstruct{PnlMat}
    $\ast$inverse, const \refstruct{PnlMat} $\ast$A}
  \sshortdescribe Computes the inverse of a matrix A and stores the result
  into \var{inverse}. The inverse is computed using a LU factorization of
  \var{A}.
\end{itemize}

\subsubsection{Permutations}

\begin{verbatim}
typedef  struct {
  int size;
  int *array;
} PnlPermutation;
\end{verbatim}

\begin{itemize}
\item \describefun{\refstruct{PnlPermutation} $\ast$}{pnl_permutation_create}{int n}
  \sshortdescribe Creates of a \refstruct{PnlPermutation} of size \var{n}.  

\item \describefun{void}{pnl_permutation_init}{\refstruct{PnlPermutation} $\ast$p}
  \sshortdescribe Initializes an existing permutation to the identity permutation.  

\item \describefun{void}{pnl_permutation_swap}{\refstruct{PnlPermutation} $\ast$p, int i, int j}
  \sshortdescribe Swaps two elements of a permutation.  

\item \describefun{void}{pnl_permutation_free}{\refstruct{PnlPermutation} $\ast$$\ast$p}
  \sshortdescribe Frees a \refstruct{PnlPermutation}.

\item \describefun{void}{pnl_vect_permute}{\refstruct{PnlVect} $\ast$px, const
    \refstruct{PnlVect} $\ast$x, const \refstruct{PnlPermutation} $\ast$p} 
  \sshortdescribe Applies a \refstruct{PnlPermutation} to a \refstruct{PnlVect}.  

\item \describefun{void}{pnl_vect_permute_inplace}{\refstruct{PnlVect} $\ast$x, 
    const \refstruct{PnlPermutation} $\ast$p} 
  \sshortdescribe Applies a \refstruct{PnlPermutation} to a
  \refstruct{PnlVect} in-place.  
  
\item \describefun{void}{pnl_permutation_fprint}{FILE $\ast$fic, const \refstruct{PnlPermutation} $\ast$p}
  \sshortdescribe Prints a permutation to a file.  

\item \describefun{void}{pnl_permutation_print}{const \refstruct{PnlPermutation} $\ast$p}
  \sshortdescribe Prints a permutation to the standard output.  
\end{itemize}

\subsection{Hyper matrices}
\subsubsection{Short description}

The Hyper matrix types and related functions are defined in the header \verb!pnl_matrix.h!.

\begin{verbatim}
typedef struct PnlHMat{
  int ndim; /*!< nb dimensions */ 
  int *dims; /*!< pointer to store the value of the ndim dimensions */ 
  int mn; /*!< product dim_1 *...*dim_ndim */
  double *array; /*!< pointer to store */
} PnlHMat;

typedef struct PnlHMatUint{
  int ndim; /*!< nb dimensions */ 
  int *dims; /*!< pointer to store the value of the ndim dimensions */ 
  int mn; /*!< product dim_1 *...*dim_ndim */
  uint *array; /*!< pointer to store */
} PnlHMatUint;

typedef struct PnlHMatInt{
  int ndim; /*!< nb dimensions */ 
  int *dims; /*!< pointer to store the value of the ndim dimensions */ 
  int mn; /*!< product dim_1 *...*dim_ndim */
  int *array; /*!< pointer to store */
} PnlHMatInt;

typedef struct PnlHMatComplex{
  int ndim; /*!< nb dimensions */ 
  int *dims; /*!< pointer to store the value of the ndim dimensions */ 
  int mn; /*!< product dim_1 *...*dim_ndim */
  dcomplex *array; /*!< pointer to store */
} PnlHMatComplex;
\end{verbatim}
\var{ndim} is the number of dimensions, \var{dim} is an array to store the
size of each dimension and \var{nm} contains the product of the sizes of each
dimension. \var{array} is an array of size \var{mn} containing the data of the
matrix stored linewise.


\subsubsection{Generic Functions}
\paragraph{General functions}
These functions exist for all types of hypermatrices no matter what the basic type
is. The following conventions are used to name functions operating on hypermatrices.
Here is the table of prefixes used for the different basic types.

\begin{center}
  \begin{tabular}[t]{lll}
    type & prefix & BASE\\
    \hline
    double & pnl_hmat & double \\
    \hline
    int & pnl_hmat_int & int \\
    \hline
    uint & pnl_hmat_uint & uint\\
    \hline
    dcomplex & pnl_hmat_complex & dcomplex
  \end{tabular}
\end{center}

In this paragraph we present the functions operating on \refstruct{PnlMat}
which exist for all types. To deduce the prototypes of these functions for
other basic types, one must replace {\tt pnl_mat} and {\tt double} according
the above table.


\subsubsection{Functions}

\paragraph{Constructors and destructors}
\begin{itemize}
\item \describefun{\refstruct{PnlHMat} $\ast$}{pnl_hmat_create}{int ndim, const int $\ast$dims}
  
\item 
  \describefun{\refstruct{PnlHMat} $\ast$}{pnl_hmat_create_from_double}{int ndim, const int $\ast$dims, double x}
  
\item 
  \describefun{\refstruct{PnlHMat} $\ast$}{pnl_hmat_create_from_ptr}{int ndim, const int $\ast$dims, const double $\ast$x}
  
\item 
  \describefun{void}{pnl_hmat_free}{\refstruct{PnlHMat} $\ast$$\ast$H}
  
\item \describefun{\refstruct{PnlHMat} $\ast$}{pnl_hmat_copy}{const \refstruct{PnlHMat} $\ast$H}
  \sshortdescribe Copies a \refstruct{PnlHMat}.
  
\item \describefun{void}{pnl_hmat_clone}{\refstruct{PnlHMat} $\ast$clone, const \refstruct{PnlHMat} $\ast$H}
  \sshortdescribe Clones a \refstruct{PnlHMat}.
  
\item \describefun{int}{pnl_hmat_resize}{\refstruct{PnlHMat} $\ast$H, int ndim, const int $\ast$dims}
  \sshortdescribe Resizes a \refstruct{PnlHMat}.
\end{itemize}  

\paragraph{Accessing elements}

\begin{itemize}
\item   \describefun{void}{pnl_hmat_set}{\refstruct{PnlHMat} $\ast$self, int $\ast$tab, double x}
  \sshortdescribe Sets the element of index \var{tab} to \var{x}.
  
\item \describefun{double}{pnl_hmat_get}{const \refstruct{PnlHMat} $\ast$self, int $\ast$tab}
  \sshortdescribe Returns the value of the element of index \var{tab} 
  
\item \describefun{double$\ast$}{pnl_hmat_lget}{\refstruct{PnlHMat} $\ast$self, int $\ast$tab}
  \sshortdescribe Returns the address of self[tab] for use as a lvalue.  
\end{itemize}  

\paragraph{Printing hypermatrices}

\begin{itemize}
\item \describefun{void}{pnl_hmat_print}{const \refstruct{PnlHMat} $\ast$H}
  \sshortdescribe Prints an hypermatrix.
\end{itemize}

\paragraph{Accessing elements}

\begin{itemize}
\item \describefun{void}{pnl_hmat_plus_hmat}{\refstruct{PnlHMat} $\ast$lhs, const \refstruct{PnlHMat} $\ast$rhs}
  \sshortdescribe Computes \var{lhs += rhs}.
  
\item \describefun{void}{pnl_hmat_mult_double}{\refstruct{PnlHMat} $\ast$lhs, double x}
  \sshortdescribe Computes \var{lhs *= x} where x is a real number.
\end{itemize}


%% tridiag

\subsection{Tri-diagonal matrix}
\subsubsection{Short Description}

The structures and functions related to tri-diagonal matrices are declared in
\verb!pnl_tridiag_matrix.h!. 

Naturally, we only store the nonzero components, stored as three vectors.

\begin{verbatim}
typedef struct PnlTriDiagMat{
  int size;           /*!< diagonal dimension product  */
  int owner;          /*!< 1 if the structure owns its array pointer */
  double *diag;       /*!< pointer to store the diagonal*/
  double *diag_up;    /*!< pointer to store the upper diagonal*/
  double *diag_down;  /*!< pointer to store the lower diagonal*/
} PnlTriDiagMat;
\end{verbatim}

\var{size} is the size of the matrix, \var{diag}, \var{diag_up},
\var{diag_down} are pointers containing the data of the diagonal, the upper
diagonal ($M_{i, i+1}$) and the lower diagonal ($M_{i-1, i}$).  \var{owner} is
an integer to know if the vector owns its three  pointers
(\var{owner}=$1$) or shares them with another object (\var{owner}=$0$).

\subsubsection{Functions}
\paragraph{Constructors and destructors}
\begin{itemize}
\item 
  \describefun{\refstruct{PnlTriDiagMat} $\ast$}{pnl_tridiagmat_create}{int size}
  \sshortdescribe Creates a \refstruct{PnlTriDiagMat}
\item \describefun{\refstruct{PnlTriDiagMat} $\ast$}{pnl_tridiagmat_create_from_double}{int size, double x}
  \sshortdescribe Creates a \refstruct{PnlTriDiagMat} with the 3 diagonals
  filled with \var{x}
\item \describefun{\refstruct{PnlTriDiagMat} $\ast$}{pnl_tridiagmat_create_from_two_double}{int size, double x, double y}
  \sshortdescribe Creates a \refstruct{PnlTriDiagMat}  with the diagonal
  filled with \var{x} and the upper ans lower diagonals filled with \var{y}
\item \describefun{\refstruct{PnlTriDiagMat} $\ast$}{pnl_tridiagmat_create_from_ptr}{int size, const double $\ast$diag, const double $\ast$upper_diag, const double $\ast$lower_diag}
  \sshortdescribe Creates a \refstruct{PnlTriDiagMat}  
\item \describefun{\refstruct{PnlTriDiagMat} $\ast$}{pnl_tridiagmat_create_from_matrix}{const \refstruct{PnlMat} $\ast$mat}
  \sshortdescribe Reads a tri-diagonal matrix from a full matrix and creates
  the corresponding \refstruct{PnlTriDiagMat}.
\item \describefun{void }{pnl_tridiagmat_free}{\refstruct{PnlTriDiagMat} $\ast$$\ast$\refstruct{v}}
  \sshortdescribe Frees a \refstruct{PnlTriDiagMat}  
\item \describefun{int}{pnl_tridiagmat_resize}{\refstruct{PnlTriDiagMat} $\ast$\refstruct{v}, int size}
  \sshortdescribe Resizes a \refstruct{PnlTriDiagMat}.  
\end{itemize}
\paragraph{Accessing elements}
\begin{itemize}
\item \describefun{void}{pnl_tridiagmat_set}{\refstruct{PnlTriDiagMat} $\ast$self, int d, int up, double x}
  \sshortdescribe Sets \var{self[d, d+up] = x}, \var{up} can be $\{-1, 0, 1\}$.  
\item \describefun{double}{pnl_tridiagmat_get}{const \refstruct{PnlTriDiagMat} $\ast$self, int d, int up}
  \sshortdescribe Gets \var{self[d, d+up]}, \var{up} can be $\{-1, 0, 1\}$.  
\item \describefun{double $\ast$}{pnl_tridiagmat_lget}{\refstruct{PnlTriDiagMat} $\ast$self, int d, int up}
  \sshortdescribe Returns the address \var{self[d, d+up] = x}, \var{up} can be $\{-1, 0, 1\}$.  
\end{itemize}

\paragraph{Printing Matrix}
\begin{itemize}
\item \describefun{void}{pnl_tridiagmat_fprint}{FILE $\ast$fic, const \refstruct{PnlTriDiagMat} $\ast$M}
  \sshortdescribe Prints a tri-diagonal matrix to a file.  
\item \describefun{void}{pnl_tridiagmat_print}{const \refstruct{PnlTriDiagMat} $\ast$M}
  \sshortdescribe Prints a tri-diagonal matrix to the standard output.  
\end{itemize}

\paragraph{Algebra operations}
\begin{itemize}
\item \describefun{void}{pnl_tridiagmat_plus_tridiagmat}{\refstruct{PnlTriDiagMat} $\ast$lhs, const \refstruct{PnlTriDiagMat} $\ast$rhs}
  \sshortdescribe In-place matrix matrix addition  
\item \describefun{void}{pnl_tridiagmat_minus_tridiagmat}{\refstruct{PnlTriDiagMat} $\ast$lhs, const \refstruct{PnlTriDiagMat} $\ast$rhs}
  \sshortdescribe In-place matrix matrix substraction  
\item \describefun{void}{pnl_tridiagmat_plus_double}{\refstruct{PnlTriDiagMat} $\ast$lhs, double x}
  \sshortdescribe In-place matrix scalar addition  
\item \describefun{void}{pnl_tridiagmat_minus_double}{\refstruct{PnlTriDiagMat} $\ast$lhs, double x}
  \sshortdescribe In-place matrix scalar substraction  
\item \describefun{void}{pnl_tridiagmat_mult_double}{\refstruct{PnlTriDiagMat} $\ast$lhs, double x}
  \sshortdescribe In-place matrix scalar multiplication  
\item \describefun{void}{pnl_tridiagmat_div_double}{\refstruct{PnlTriDiagMat} $\ast$lhs, double x}
  \sshortdescribe In-place matrix scalar division
\end{itemize}

\paragraph{Element-wise operations}
\begin{itemize}
\item \describefun{void}{pnl_tridiagmat_mult_tridiagmat_term}{\refstruct{PnlTriDiagMat} $\ast$lhs, const \refstruct{PnlTriDiagMat} $\ast$rhs}
  \sshortdescribe In-place matrix matrix term by term product  
\item \describefun{void}{pnl_tridiagmat_div_tridiagmat_term}{\refstruct{PnlTriDiagMat} $\ast$lhs, const \refstruct{PnlTriDiagMat} $\ast$rhs}
  \sshortdescribe In-place matrix matrix term by term division  
\end{itemize}

\paragraph{Standard matrix operations \& Linear system}
\begin{itemize}
\item \describefun{void}{pnl_tridiagmat_mult_vect_inplace}{\refstruct{PnlVect} $\ast$lhs, const \refstruct{PnlTriDiagMat} $\ast$mat, const \refstruct{PnlVect} $\ast$rhs}
  \sshortdescribe In place matrix multiplication  
\item \describefun{void}{pnl_tridiagmat_mult_vect_inplace_from_lhs}{\refstruct{PnlVect} $\ast$lhs, const \refstruct{PnlTriDiagMat} $\ast$mat}
  \sshortdescribe In place matrix multiplication : lhs = M lhs  
\item \describefun{\refstruct{PnlVect} $\ast$}{pnl_tridiagmat_mult_vect}{const \refstruct{PnlTriDiagMat} $\ast$mat, const \refstruct{PnlVect} $\ast$vec}
  \sshortdescribe Matrix multiplication  
\item \describefun{void}{pnl_tridiagmat_lAxpby}{double lambda, const \refstruct{PnlTriDiagMat}
    $\ast$A, const \refstruct{PnlVect} $\ast$x, double mu, \refstruct{PnlVect} $\ast$b} 
  \sshortdescribe Computes \var{b := lambda A x + mu b}. When \var{mu==0}, the
  content of \var{b} is not used on input and instead \var{b} is resized to
  match \var{A*x}
\item \describefun{double}{pnl_tridiagmat_scalar_prod}{const \refstruct{PnlVect} $\ast$x,const \refstruct{PnlTriDiagMat}
    $\ast$A, const \refstruct{PnlVect} $\ast$y}
  \sshortdescribe Computes \var{x' * A * y}
\item \describefun{void}{pnl_tridiagmat_lu_syslin}{\refstruct{PnlVect} $\ast$lhs, const \refstruct{PnlTriDiagMat} $\ast$M, const \refstruct{PnlVect} $\ast$rhs}
  \sshortdescribe Solves the linear system M x = b. Note, in this function, the LU decomposition
  is not explicitly computed.
\end{itemize}



\subsection{Band Matrix structure}
\subsubsection{Short Description}

The structures and functions related to band matrices are declared in
\verb!pnl_band_matrix.h!. 


Band Matrices have $m_{up}>0$ nonzero upper diagonals and $m_{low}$ nonzero
lower diagonals. Of course the next structure is useful only if $m_{up}$ and
$m_{down}$ are smaller compare to $N$, the size of the matrix. In the case of
Band Matrix, the solution of the linear system by classical decomposition
algorithms LU, Cholesky, $\dots$ could be performed much faster and in much
less storage, than for the $N\times N $ case. For simplification, we make the
assumption $m_{up}=m_{down}=band$.


This structure is used in practice to solve multi-dimensional PDE in a Galerkin
framework. It is strongly inspired by the work of F.Hecht on RMN class. 


\begin{verbatim}
typedef enum {
  FactorizationNO=0, 
  FactorizationCholeski=1, 
  FactorizationCrout=2, 
  FactorizationLU=3
}FactorizationType;

struct PnlBandMatrix{
  int n;      /*!< size of row */ 
  int m;      /*!< size of col */ 
  double *D;  /*!< diagonal vector */  
  double *Up; /*!< upper Triangular matrix, only no nul coeffieciens */ 
  double *Lo; /*!< lower Triangular matrix, only no nul coeffieciens */
  int    *pU; /*!< upper Triangular  profil   */ 
  int    *pL; /*!< lower Triangular profil */
  FactorizationType typefac;
  int owner;  /*!< 1 if the structure owns its array pointer */
}PnlBandMatrix;
\end{verbatim}

\var{n} is the number of row, \var{m} is the number of column.
\var{D}, \var{Up}, \var{Low}  are pointers containing the
data of diagonal, the up matrix ($M_{i, j}, \ i<j\leq i+band$) and  down
matrix ($M_{j, i}, , \ i-band\leq j<i$).
\var{pU} and \var{pL} is used to store jump in row, see next equation.
\var{typefac} is an integer to store the factorisation type of the band
matrix, if it's done.
pointer (\var{owner}=$1$) or shares it with another structure (\var{owner}=$0$).


\begin{align*}
  \label{eq:1}
  & D(i) = A(i, i), \nonumber \\
  & Lo(k) = A(i, j), \quad j < i \textit{with: } \quad pL(i)<= k < pL(i+1)
  \textit{and } j = pL(i+1)-k, \\
  & Up(k) =A(i, j), \quad i < j \textit{with: } \quad pU(j)<= k < pU(j+1) \textit{and } i
  = pU(i+1)-k.\nonumber
\end{align*}
remark:
\begin{itemize}
\item $ pL = pU $ in most of case, 
\item if $L = U$ then symmetric matrix.
\end{itemize}


\subsubsection{Functions}
\paragraph{Constructors and destructors}
\begin{itemize}
\item 
  \describefun{PnlBandMatrix$\ast$}{pnl_band_matrix_create}{const int n, int band}
  \sshortdescribe Pnl_band_matrix_create create a void sparse matrix of size n with
  \char`\"{}band\char`\"{} extra diagonal terms
\item
  \describefun{PnlBandMatrix$\ast$}{pnl_band_matrix_create_from_full}{const \refstruct{PnlMat} $\ast$PM, int band}
  \sshortdescribe Extracts a band matrix from a \refstruct{PnlMat}.
\item
  \describefun{void}{pnl_band_matrix_free}{\refstruct{PnlMat} $\ast$
    $\ast$PnlBandMatrix}
\item \describefun{\refstruct{PnlBandMatrix}$\ast$}{pnl_band_matrix_clone}{\refstruct{PnlBandMatrix} $\ast$clone, const \refstruct{PnlBandMatrix} $\ast$M}
  \sshortdescribe Clones \var{M} into \var{clone}. No no new
  \refstruct{PnlBandMatrix} is created.
\end{itemize}
\paragraph{Accessing elements}
\begin{itemize}
\item 
  \describefun{double}{pnl_pnl_band_matrix_diag}{\refstruct{PnlBandMatrix} $\ast$M, int i}
  \sshortdescribe Return the diagonal term $M_{i, i}$.
\item 
  \describefun{void}{pnl_pnl_band_matrix_set}{\refstruct{PnlBandMatrix}
    $\ast$M, int i, int j, double x}
  \sshortdescribe Check if $j$ is a valid index (in most of case, $|i-j|\leq
  band $ - not true for first and last lines )
  and do $M_{i, j}=x$.
\item 
  \describefun{void}{pnl_band_matrix_add}{\refstruct{PnlBandMatrix}
    $\ast$M, int i, int j, double x}
  \sshortdescribe Check if $j$ is a valid index (in most of case, $|i-j|\leq
  band $ - not true for first and last lines )
  and do $M_{i, j} \ +=x$.
\end{itemize}

\subparagraph{Element wise operations}

\begin{itemize}
\item \describefun{void}{pnl_band_matrix_plus_band_matrix}{\refstruct{PnlBand_Matrix} $\ast$lhs, 
    const \refstruct{PnlBand_Matrix} $\ast$rhs} 
  \sshortdescribe In-place addition  

\item \describefun{void}{pnl_band_matrix_minus_band_matrix}{\refstruct{PnlBand_Matrix} $\ast$lhs, 
    const \refstruct{PnlBand_Matrix} $\ast$rhs} 
  \sshortdescribe In-place substraction  

\item \describefun{void}{pnl_band_matrix_inv_term}{\refstruct{PnlBand_Matrix} $\ast$lhs}
  \sshortdescribe In-place term by term  inversion  

\item \describefun{void}{pnl_band_matrix_div_mat_term}{\refstruct{PnlBand_Matrix}
    $\ast$lhs, const \refstruct{PnlBand_Matrix} $\ast$rhs} 
  \sshortdescribe In-place term by term  division

\item \describefun{void}{pnl_band_matrix_mult_mat_term}{\refstruct{PnlBand_Matrix}
    $\ast$lhs, const \refstruct{PnlBand_Matrix} $\ast$rhs} 
  \sshortdescribe In-place term by term multiplication  

\item \describefun{void}{pnl_band_matrix_map}{\refstruct{PnlBand_Matrix} $\ast$lhs, const
    \refstruct{PnlBand_Matrix} $\ast$rhs, double($\ast$f)(double)} 
  \sshortdescribe Applies the function \var{f} to each element of \var{rhs} and
  stores the result in \var{lhs}

\item \describefun{void}{pnl_band_matrix_map_inplace}{\refstruct{PnlBand_Matrix} $\ast$lhs, double($\ast$f)(double)}
  \sshortdescribe Same function as \reffun{pnl_band_matrix_map} but the result is
  stored in \var{lhs} itself.
\end{itemize}


\paragraph{Extraction operations} These functions should not be used by
standard developers.  They do not create a real band matrix, but create an
empty structure with owner set to $0$ and pointer is set to address of some
elements of $M$, which means that a modifcation to $M$ is automically
propagated to the \refstruct{PnlBandMatrix} created by the functions below and
vice-versa..  Take an example, the transposition construct an empty band
matrix $TM$ with $TM\rightarrow Up = M\rightarrow Low, \dots $, but not make a
copy of array $M\rightarrow Low$.
\begin{itemize}
\item \describefun{\refstruct{PnlBandMatrix}$\ast$}{pnl_band_matrix_transpose}
  {const \refstruct{PnlBandMatrix} $\ast$M}
\item \describefun{\refstruct{PnlBandMatrix}$\ast$}{pnl_band_matrix_Low}
  {const \refstruct{PnlBandMatrix} $\ast$M}
\item \describefun{\refstruct{PnlBandMatrix}$\ast$}{pnl_band_matrix_Up}
  {const \refstruct{PnlBandMatrix} $\ast$M}
\item \describefun{\refstruct{PnlBandMatrix}$\ast$}{pnl_band_matrix_Tran_Low}
  {const \refstruct{PnlBandMatrix} $\ast$M}
\item \describefun{\refstruct{PnlBandMatrix}$\ast$}{pnl_band_matrix_Tran_Up}
  {const \refstruct{PnlBandMatrix} $\ast$M}
\item \describefun{\refstruct{PnlBandMatrix}$\ast$}{pnl_band_matrix_Diag}
  {const \refstruct{PnlBandMatrix} $\ast$M}
\item \describefun{\refstruct{PnlBandMatrix}$\ast$}{pnl_band_matrix_Low_Diag}
  {const \refstruct{PnlBandMatrix} $\ast$M}
\item \describefun{\refstruct{PnlBandMatrix}$\ast$}{pnl_band_matrix_Up_Diag}
  {const \refstruct{PnlBandMatrix} $\ast$M}
\item
  \describefun{\refstruct{PnlBandMatrix}$\ast$}{pnl_band_matrix_Tran_Low_Diag}
  {const \refstruct{PnlBandMatrix} $\ast$M}
\item
  \describefun{\refstruct{PnlBandMatrix}$\ast$}{pnl_band_matrix_Tran_Up_Diag}
  {const \refstruct{PnlBandMatrix} $\ast$M}
\end{itemize}

\paragraph{Factorization}
\begin{itemize}
\item 
  \describefun{void}{pnl_band_matrix_cholesky}{\refstruct{PnlBandMatrix} $\ast$M, double eps}
  \sshortdescribe Cholesky factorization $ M= U^T U$ of
  a Band Matrix (Symmetric)
\item  \describefun{void}{pnl_band_matrix_crout}{\refstruct{PnlBandMatrix} $\ast$M, double eps}
  \sshortdescribe Crout factorization $
  M= L U$ of a Band Matrix (Symmetric)
  \begin{equation*}
    \label{eq:2}
    L=\begin{pmatrix}
      l_{11}&0&0&0\\ l_{21}&l_{21}&0&0\\
      l_{31}&l_{32}&l_{33}&0\\ l_{41}&l_{42}&l_{43}&l_{44}
    \end{pmatrix} \quad \quad
    U=\begin{pmatrix} 1&u_{12}&u_{13}&u_{14}\\
      0&1&u_{23}&u_{24}\\ 0&0&1&u_{34}\\ 0&0&&1\\
    \end{pmatrix}    
  \end{equation*}
\item  \describefun{void}{pnl_band_matrix_lu}{\refstruct{PnlBandMatrix} $\ast$M, double eps}
  \sshortdescribe LU factorization $ M= L U$ of a Band
  Matrix (No Symmetric)
\end{itemize}

\paragraph{Standard matrix operations \& Linear system}
\begin{itemize}
\item \describefun{void}{pnl_band_matrix_lAxpby}{double lambda, const \refstruct{PnlBandMatrix}
    $\ast$A, const \refstruct{PnlVect} $\ast$x, double mu, \refstruct{PnlVect} $\ast$b} 
  \sshortdescribe Computes \var{b := lambda A x + mu b}. When \var{mu==0}, the
  content of \var{b} is not used on input and instead \var{b} is resized to
  match \var{A*x}
\item \describefun{double}{pnl_band_matrix_scalar_prod}{const \refstruct{PnlVect} $\ast$x,const \refstruct{PnlBandMatrix}
    $\ast$A, const \refstruct{PnlVect} $\ast$y}
  \sshortdescribe Computes \var{x' * A * y}
\item 
  \describefun{void}{pnl_band_matrix_solve_syslin_inplace}{\refstruct{PnlBandMatrix}
    $\ast$M, \refstruct{PnlVect} $\ast$x}
  \sshortdescribe Solves the linear system $M x = x$ with M a \refstruct{PnlBandMatrix}.
  {\bf Note}, M is modified if and only if M has not been yet factorized.
  In most cases (i.e. for non symmetric matrix), on exit \var{M} contains the
  LU decomposition of the input matrix, 
  if $M\rightarrow typefac$ is previously $0$.   
\item  \describefun{void}{pnl_band_matrix_solve}{\refstruct{PnlBandMatrix} $\ast$M, 
    \refstruct{PnlVect} $\ast$x, const \refstruct{PnlVect} $\ast$b} 
  \sshortdescribe Pnl_band_matrix_solve solves the
  linear system $ M x = b $ with M a \refstruct{PnlBandMatrix}.
  {\bf Note}, M is modified if and only if M has not been yet factorize.
  In most of case (non symmetric matrix) M is return as the LU of the input M
  if $M\rightarrow typefac$ was $0$ before calling the function. 
\end{itemize}

\subsection{Morse Matrix}
\subsubsection{Short Description}

A system of linear equation is called sparse if only a relatively small number
of its matrix elements $M_{i, j}$ are nonzero. It is wasteful to use full
structure to solve the linear system because most of the operations devoted to
solving the system use elements with values zero. Furthermore, for some 
high dimensional problems, storing the full matrix with its zero elements is not
possible because of memory limitations.


In the following, we propose two structures for Sparse Matrices.  Must of the
algorithms which use sparse matrices can be divided in two steps.  The first
step is the construction of the matrix. For this, \refstruct{PnlMorseMat} should
be used. The second step is the resolution of a sparse linear system. We
have two ways of doing that. The first one is to use a direct method based on
matrix-decomposition, like the LU decomposition. The \refstruct{PnlSparseMat} is
implemented to do that. The second one is to use iterative methods like
Conjugate Gradient, BICGstab or GMRES. These methods are discussed in the next
section. If we use iterative methods, we can use \refstruct{PnlMorseMat}. 

\begin{verbatim}
typedef struct SpRow{
  int size;  /*!< size of a row */
  int Max_size; /*!< max size allocation of a row */
  int    *Index; /*!< pointer to an int array giving the columns or row i */
  double *Value; /*!< Pointer on values */
}SpRow;
\end{verbatim}
\var{size} is the number of elements, 
\var{Max_size} is the size of memory allocation.
\var{Index}, is the pointer containing the index of row or column, 
\var{Value}, is the pointer containing the value of row or column.
So for a \refstruct{SpRow} which contains row $i$ of $M$.
If $k \leq size $ then
$$M_{i, Index[k]}=Value[k].$$  

\begin{verbatim}
typedef struct PnlMorseMat{
  int m; /*!< nb rows */ 
  int n; /*!< nb columns */ 
  SpRow * array; /*!< pointer in each row or col to store no nul coefficients */
  int RC; /*!< 0 if we use row-wise storage, 1 if we use column-wise storage */ 
} PnlMorseMat;
\end{verbatim}
\var{m} is the number of rows, \var{n} is the number of columns.
\var{array} is the pointer containing on SpRow array of size n or m (depend of
RC).
\var{RC} is an integer to know if the matrix is stored by row or columns.

\subsubsection{Functions}
\paragraph{Constructors and destructors}
\begin{itemize}
\item \describefun{\refstruct{PnlMorseMat}$\ast$}{pnl_morse_mat_create}{int m, 
    int n, int Max_size_row, int RC}
  \sshortdescribe Creates an empty \refstruct{PnlMorseMat} with memory
  allocated for each component of the array. 
\item
  \describefun{\refstruct{PnlMorseMat}$\ast$}{pnl_morse_mat_create_fromfull}
  {\refstruct{PnlMat} $\ast$FM, int RC}
  \sshortdescribe Creates a \refstruct{PnlMorseMat} from  a \refstruct{PnlMat}
  storing only its nonzero elements.

\item \describefun{void}{pnl_morse_mat_free}{\refstruct{PnlMorseMat}$\ast$$\ast$ M}
  \sshortdescribe Frees a \refstruct{PnlMorseMat}

\item \describefun{int}{pnl_morse_mat_freeze}{PnlMorseMat$\ast$ M}
  \sshortdescribe Sets Max size equal to size for each SpRow and frees the extra
  memory.

\item \describefun{\refstruct{PnlMat} $\ast$}{pnl_morse_mat_full}
  {\refstruct{PnlMorseMat}$\ast$ M}
  \sshortdescribe Creates a full matrix from a morse matrix.
\end{itemize}


\paragraph{Accessing elements}
\begin{itemize}
\item \describefun{double}{ pnl_morse_mat_get}{PnlMorseMat$\ast$ M, int i, int j}
  \sshortdescribe Return $M_{i, j}$. 
\item \describefun{int}{ pnl_morse_mat_set}{PnlMorseMat$\ast$ M, int i, int
    j, double Val}
  \sshortdescribe Do $M_{i, j} = Val$. For example, if $RC=1$ and $(i, j)$ is a valid index, replace
  $array[i]\rightarrow Value[k]$ with $k$ such that $array[i]\rightarrow Index[k]=j$.
  If $(i, j)$ is not a valid index, add $j$ to $array[i]\rightarrow Index$ and $Val$ to
  $array[i] \rightarrow Value$ with memory allocation if needed. 
\item \describefun{double$\ast$}{pnl_morse_mat_lget}{PnlMorseMat$\ast$ M, int
    i, int j}
  \sshortdescribe Returns the address of $M_{i, j}$. For example, 
  if $RC=1$ and $(i, j)$ is a valid index, replace return address of
  $array[i]\rightarrow Value[k]$ with $k$ such that $array[i]\rightarrow
  Index[k]=j$.  If $(i, j)$ is not a valid index, add $j$ to
  $array[i]\rightarrow Index$ and add element to $array[i] \rightarrow Value$
  (with memory allocation if needed), returns address of this element. In
  practice this function is used to do $M_{i, j} += a$.
\end{itemize}

\paragraph{Printing Matrix}
\begin{itemize}
\item \describefun{void}{pnl_morse_mat_print}{const \refstruct{PnlMorseMat}$\ast$M}
\end{itemize}

\paragraph{Standard matrix operations}
\begin{itemize}
\item \describefun{void}{pnl_morse_mat_mult_vect_inplace}{\refstruct{PnlVect}
    $\ast$lhs, const \refstruct{PnlMorseMat}$\ast$M, const \refstruct{PnlVect}
    $\ast$rhs}
  \sshortdescribe Compute $ lhs=M \ rhs$.
\item \describefun{\refstruct{PnlVect}$\ast$}{pnl_morse_mat_mult_vect}{const
    \refstruct{PnlMorseMat}$\ast$M, const \refstruct{PnlVect} $\ast$vec}
  \sshortdescribe Compute $ vec=M \ vec$.
\end{itemize}


\subsection{Sparse Matrix}

\refstruct{PnlSparseMat} is the cs structure of the Csparse library written by
Timothy A.Davis.  For the sake of convenience, we have renamed some functions
and structures. We have also reduced the number of function parameters for non
expert users in sparse matrices.  In the following, we only use the LU
factorisation for sparse systems. If the same operator is used at each time
step, direct methods relying on factorisations are faster than iterative
methods. When the PDE coefficients are time dependent, the answer is not so
clear.

\subsubsection{Functions}
\paragraph{Constructors and destructors}
\begin{itemize}
\item \describefun{\refstruct{PnlSparseMat}
    $\ast$}{pnl_sparse_mat_create_fromfull}{\refstruct{PnlMat} $\ast$M}
  \sshortdescribe Creates a \refstruct{PnlSparseMat} from  a
  \refstruct{PnlMat} storing only nonzero elements.
\item \describefun{\refstruct{PnlSparseMat}
    $\ast$}{pnl_sparse_mat_create_frommorse}{\refstruct{PnlMorseMat}$\ast$ M}
  \sshortdescribe Creates a \refstruct{PnlSparseMat} from  a
  \refstruct{PnlMorseMat} with $M\rightarrow M->RC =1$.
\item \describefun{void}{pnl_sparse_mat_free}{\refstruct{PnlSparseMat}
    $\ast$$\ast$M}
  \sshortdescribe Frees a \refstruct{PnlSparseMat}.
\end{itemize}

\paragraph{Printing Matrix}
\begin{itemize}
\item \describefun{void}{pnl_sparse_mat_print}{\refstruct{PnlSparseMat}
    $\ast$A}
  \sshortdescribe Prints a \refstruct{PnlSparseMat}.
\end{itemize}

\subparagraph{Element wise operations}

\begin{itemize}
\item \describefun{void}{pnl_sparse_mat_plus_sparse_mat}{\refstruct{PnlSparseMat} $\ast$lhs, 
    const \refstruct{PnlSparseMat} $\ast$rhs} 
  \sshortdescribe In-place addition  

\item \describefun{void}{pnl_sparse_mat_minus_sparse_mat}{\refstruct{PnlSparseMat} $\ast$lhs, 
    const \refstruct{PnlSparseMat} $\ast$rhs} 
  \sshortdescribe In-place substraction  

\item \describefun{void}{pnl_sparse_mat_inv_term}{\refstruct{PnlSparseMat} $\ast$lhs}
  \sshortdescribe In-place term by term inversion  

\item \describefun{void}{pnl_sparse_mat_div_mat_term}{\refstruct{PnlSparseMat}
    $\ast$lhs, const \refstruct{PnlSparseMat} $\ast$rhs} 
  \sshortdescribe In-place term by term division

\item \describefun{void}{pnl_sparse_mat_mult_mat_term}{\refstruct{PnlSparseMat}
    $\ast$lhs, const \refstruct{PnlSparseMat} $\ast$rhs} 
  \sshortdescribe In-place term by term multiplication  



\item \describefun{void}{pnl_sparse_mat_map_inplace}{\refstruct{PnlSparseMat} $\ast$M, double($\ast$f)(double)}
  \sshortdescribe Applies function \var{f} to each entry of \var{M}, which
  is modified on exit.
\end{itemize}


\paragraph{Standard matrix operations}
\begin{itemize}
\item \describefun{int}{pnl_sparse_mat_gaxpby}{\refstruct{PnlVect} $\ast$lhs, 
    const \refstruct{PnlSparseMat} $\ast$M, const \refstruct{PnlVect}
    $\ast$rhs}
  \sshortdescribe Computes $lhs=lhs+ M * rhs$.
\item \describefun{int}{pnl_sparse_mat_mult_vect_inplace}{\refstruct{PnlVect}
    $\ast$lhs, const \refstruct{PnlSparseMat} $\ast$M, const
    \refstruct{PnlVect} $\ast$rhs}
  \sshortdescribe Computes $lhs= M * rhs$.
\end{itemize}

\subsubsection{LU structure}

From the sparse matrix, we extract the LU decomposition stored in \refstruct{PnlSparseFactorization}.
\paragraph{Constructors and desctructors}
\begin{itemize}
\item \describefun{\refstruct{PnlSparseFactorization}
    $\ast$}{pnl_sparse_factorization_lu_create}{const \refstruct{PnlSparseMat} $\ast$A, double tol}
  \sshortdescribe Computes the LU factorisation of \var{A}

\item \describefun{void}{pnl_sparse_factorization_free}{\refstruct{PnlSparseFactorization} $\ast$$\ast$ F}
  \sshortdescribe Frees a \refstruct{PnlSparseFactorization}.
\end{itemize}

\paragraph{Solving linear systems}

\begin{itemize}
\item \describefun{void}{pnl_sparse_factorization_lu_syslin}{const
    \refstruct{PnlSparseFactorization} $\ast$N, PnlVect $\ast$b}
  \sshortdescribe Solves the linear system \var{Nx = b} and stores the solution \var{x}
  into \var{b} which means that the r.h.s member of the system is overwritten
  during the resolution of the system. \var{N} is the decomposition computed by
  \reffun{pnl_sparse_factorization_lu_create}.
\end{itemize}

%% solver

\subsection{Solver Functions}
\subsubsection{Short Description}

The structures and functions related to solvers are declared in
\verb!pnl_linalgsolver.h!. 

A Left preconditioner solves the problem :
$$ P M x = P b, $$
and whereas right preconditioner solves
$$ M P y  = b, \quad \quad P y = x.$$

%% With some simplifications, the number of algorithm iterations depends on
%% conditioning. Conditioning is ratio of maximum eigenvalue over minimum
%% eigenvalue of $M$. For GMRES algorithm is depend of conditioning of $M^{T}
%% M$. So if we can find $P_L$ and $P_R$ such that $P_L M P_R$ is closed to
%% identity matrix, then preconditioning problem converge faster than initial
%% problem. We have also to solve $P_R y = x$ so $P_R$ has to be constructed to
%% do that fast.

More information is given in {\em Saad, Yousef (2003). Iterative methods for
  sparse linear systems (2nd ed. ed.). SIAM. ISBN 0898715342. OCLC 51266114}.
The reader will find in this book some discussion about right or/and left
preconditioner and a description of the following algorithms.

These algorithms, we implemented with a left preconditioner. Right preconditioner
can be easily computed changing matrix vector multiplication operator from $M \
x $ to $ M \ P_R \ x$ and solving $P_R y = x$ at the end of algorithm.


\subsubsection{Functions}

Three methods are implemented : Conjugate Gradient, BICGstab and GMRES with
restart. For each of them a structure is created to store temporary vectors
used in the algorithm. In some cases, we have to apply iterative methods more
than once : for example to solve at each time step a discrete form of an
elliptic problem come from parabolic problem. In the cases, do not call the constructor and
destructor at each time, but instead use the initialization and solve procedures.

Formally we have, 
\begin{verbatim}
Create iterative method
For each time step
  Initialisation of iterative method
  Solve linear system link to elliptic problem
end for
free iterative method
\end{verbatim}

In these functions, we don't use any particular matrix structure. We give the
matrix vector multiplication as a parameter of the solver. 

\paragraph{Conjugate Gradient method}

Only available for symmetric and positive matrices.
\begin{itemize}
\item 
  \describefun{\refstruct{PnlCGSolver} $\ast$}{pnl_cg_solver_create}{int Size, int max-iter, double tolerance}
  \sshortdescribe Creates a new \refstruct{PnlCGSolver} pointer.  
\item \describefun{void}{pnl_cg_solver_initialisation}{\refstruct{PnlCGSolver} $\ast$Solver, const \refstruct{PnlVect} $\ast$b}
  \sshortdescribe Initialisation of the solver at the beginning of iterative method.  
\item \describefun{void}{pnl_cg_solver_free}{\refstruct{PnlCGSolver} $\ast$$\ast$Solver}
  \sshortdescribe Destructor of iterative solver  
\item \describefun{int}{pnl_cg_solver_solve}{void($\ast$matrix vector-product)(const void $\ast$, const \refstruct{PnlVect} $\ast$, const double, const double, \refstruct{PnlVect} $\ast$), const void $\ast$Matrix-Data, void($\ast$matrix vector-product-PC)(const void $\ast$, const \refstruct{PnlVect} $\ast$, const double, const double, \refstruct{PnlVect} $\ast$), const void $\ast$PC-Data, \refstruct{PnlVect} $\ast$x, const \refstruct{PnlVect} $\ast$b, \refstruct{PnlCGSolver} $\ast$Solver}
  \sshortdescribe Solves the linear system matrix vector-product is the matrix vector multiplication function matrix vector-product-PC is the preconditionner function Matrix-Data \& PC-Data is data to compute matrix vector multiplication.  
\end{itemize}
\paragraph{BICG stab}
\begin{itemize}
\item \describefun{\refstruct{PnlBICGSolver} $\ast$}{pnl_bicg_solver_create}{int Size, int max-iter, double tolerance}
  \sshortdescribe Creates a new \refstruct{PnlBICGSolver} pointer.  
\item \describefun{void}{pnl_bicg_solver_initialisation}{\refstruct{PnlBICGSolver} $\ast$Solver, const \refstruct{PnlVect} $\ast$b}
  \sshortdescribe Initialisation of the solver at the beginning of iterative method.  
\item \describefun{void}{pnl_bicg_solver_free}{\refstruct{PnlBICGSolver} $\ast$$\ast$Solver}
  \sshortdescribe Destructor of iterative solver  
\item \describefun{int}{pnl_bicg_solver_solve}{void($\ast$matrix vector-product)(const void $\ast$, const \refstruct{PnlVect} $\ast$, const double, const double, \refstruct{PnlVect} $\ast$), const void $\ast$Matrix-Data, void($\ast$matrix vector-product-PC)(const void $\ast$, const \refstruct{PnlVect} $\ast$, const double, const double, \refstruct{PnlVect} $\ast$), const void $\ast$PC-Data, \refstruct{PnlVect} $\ast$x, const \refstruct{PnlVect} $\ast$b, \refstruct{PnlBICGSolver} $\ast$Solver}
  \sshortdescribe Solves the linear system matrix vector-product is the matrix vector multiplication function matrix vector-product-PC is the preconditioner function Matrix-Data \& PC-Data is data to compute matrix vector multiplication.  
\end{itemize}

\paragraph{GMRES with restart} See {\em Saad, Yousef (2003)} for discussion
about the restart parameter. For GMRES we need to store at the p-th iteration
$p$ vectors of the same size of the right and side. It could be very expensive
in term of memory allocation. So GMRES with restart algorithm stop if
$p=restart$ and restarts the algorithm with the previously computed solution
as initial guess.

Note that if restart equals $m$, we have a classical GMRES algorithm.

\begin{itemize}
\item \describefun{\refstruct{PnlGMRESSolver} $\ast$}{pnl_gmres_solver_create}{int Size, int max-iter, int restart, double tolerance}
  \sshortdescribe Creates a new \refstruct{PnlGMRESSolver} pointer.  
\item \describefun{void}{pnl_gmres_solver_initialisation}{\refstruct{PnlGMRESSolver} $\ast$Solver, const \refstruct{PnlVect} $\ast$b}
  \sshortdescribe Initialisation of the solver at the beginning of iterative method.  
\item \describefun{void}{pnl_gmres_solver_free}{\refstruct{PnlGMRESSolver} $\ast$$\ast$Solver}
  \sshortdescribe Destructor of iterative solver  
\item \describefun{int}{pnl_gmres_solver_solve}{void($\ast$matrix vector-product)(const void $\ast$, const \refstruct{PnlVect} $\ast$, const double, const double, \refstruct{PnlVect} $\ast$), const void $\ast$Matrix-Data, void($\ast$matrix vector-product-PC)(const void $\ast$, const \refstruct{PnlVect} $\ast$, const double, const double, \refstruct{PnlVect} $\ast$), const void $\ast$PC-Data, \refstruct{PnlVect} $\ast$x, const \refstruct{PnlVect} $\ast$b, \refstruct{PnlGMRESSolver} $\ast$Solver}
  \sshortdescribe Solves the linear system matrix vector-product is the matrix vector multiplication function matrix vector-product-PC is the preconditionner function Matrix-Data \& PC-Data is data to compute matrix vector multiplication.  
\end{itemize}


In the next paragraph, we write all the solvers for \refstruct{PnlMat}. This will be done as
follow: construct an application matrix vector.
\begin{verbatim}
static void pnl_mat_mult_vect_applied(const void *mat, const PnlVect *vec, 
                                      const double a , const double b, 
                                      PnlVect *lhs)
{pnl_mat_lAxpby(a, (PnlMat*)mat, vec, b, lhs);}
\end{verbatim}
and give this as parameter of iterative method
\begin{verbatim}
int pnl_mat_cg_solver_solve(const PnlMat * Matrix, const PnlMat * PC, 
                            PnlVect * x, const PnlVect *b, PnlCGSolver * Solver)
{ return pnl_cg_solver_solve(pnl_mat_mult_vect_applied, 
                             Matrix, pnl_mat_mult_vect_applied, 
                             PC, x, b, Solver);}
\end{verbatim}

In practice, we cannot define all iterative methods for all structures.
With this implementation, the user can easily :
\begin{itemize}
\item implement right precondioner, 
\item implement method with sparse matrix and diagonal preconditioner, or
  special combination of this form $\dots$
\end{itemize}


\paragraph{Iterative algorithms for \refstruct{PnlMat}}


\begin{itemize}
\item \describefun{int}{pnl_mat_cg_solver_solve}{const \refstruct{PnlMat} $\ast$M, const \refstruct{PnlMat} $\ast$PC, \refstruct{PnlVect} $\ast$x, const \refstruct{PnlVect} $\ast$b, \refstruct{PnlCGSolver} $\ast$Solver}
  \sshortdescribe Solves the linear system \var{M x = b} with preconditionner PC.  
\item \describefun{int}{pnl_mat_bicg_solver_solve}{const \refstruct{PnlMat} $\ast$M, const \refstruct{PnlMat} $\ast$PC, \refstruct{PnlVect} $\ast$x, const \refstruct{PnlVect} $\ast$b, \refstruct{PnlBICGSolver} $\ast$Solver}
  \sshortdescribe Solves the linear system \var{M x = b} with preconditionner PC.  
\item \describefun{int}{pnl_mat_gmres_solver_solve}{const \refstruct{
      PnlMat} $\ast$M, const \refstruct{PnlMat} $\ast$PC, 
    \refstruct{PnlVect} $\ast$x, \refstruct{PnlVect} $\ast$b, \refstruct{PnlGMRESSolver} $\ast$Solver}
  \sshortdescribe Solve the linear system \var{M x = b} with preconditionner PC.
\end{itemize}



%%% Local Variables: 
%%% mode: latex
%%% TeX-master: "pnl-premia-manual"
%%% End: 



\section{Cumulative distribution Functions}

The functions related to this chapter are declared in \verb!pnl/pnl_cdf.h!.

For various distribution functions, we provide functions named
\var{pnl_cdf_xxx} where \var{xxx} is the abbreviation of the distribution
name. All these functions are based on the same prototype
\begin{equation*}
  p = 1-q; \quad p = \int^x density(u) du
\end{equation*}

\begin{itemize}
\item \var{which} If \var{which=1}, it computes \var{p} and \var{q}. If
  \var{which=2}, it computes \var{x}. For higher values of \var{which} it
  computes one the parameters characterizing the distribution using all the
  others, \var{p, q, x}.
\item \var{p} the probability $\int^x density(u) du $
\item \var{q} $= 1 - p$
\item \var{x} the upper bound of the integral
\item \var{status} an integer which indicates on exit the success of the
  computation. (0) if calculation completed correctly. (-I) if the input
  parameter number I was out of range. (1) if the answer appears to be lower
  than the lowest search bound.  (2) if the answer appears to be higher than
  the greatest search bound.  (3) if $p + q \ne 1$.
\item \var{bound} is undefined if STATUS is 0.  Bound exceeded by parameter
  number I if STATUS is negative. Lower search bound if STATUS is 1.  Upper
  search bound if STATUS is 2.
\end{itemize}

\begin{itemize}
\item \describefun{void}{pnl_cdf_bet}{int \ptr which, double \ptr p, double
    \ptr q, double \ptr x, double \ptr y, double \ptr a, double \ptr b,
    int \ptr status, double \ptr bound}
  \sshortdescribe Cumulative Distribution Function BETA distribution.

\item \describefun{void}{pnl_cdf_bin}{int \ptr which, double \ptr p, double
    \ptr q, double \ptr x, double \ptr xn, double \ptr pr, double
    \ptr ompr, int \ptr status, double \ptr bound}
  \sshortdescribe Cumulative Distribution Function BINa distribution.

\item \describefun{void}{pnl_cdf_chi}{int \ptr which, double \ptr p, double
    \ptr q, double \ptr x, double \ptr df, int \ptr status, double
    \ptr bound}
  \sshortdescribe Cumulative Distribution Function CHI-Square distribution.

\item \describefun{void}{pnl_cdf_chn}{int \ptr which, double \ptr p, double
    \ptr q, double \ptr x, double \ptr df, double \ptr pnonc, int
    \ptr status, double \ptr bound}
  \sshortdescribe Cumulative Distribution Function Non-central Chi-Square distribution.

\item \describefun{void}{pnl_cdf_f}{int \ptr which, double \ptr p, double
    \ptr q, double \ptr x, double \ptr dfn, double \ptr dfd, int
    \ptr status, double \ptr bound}
  \sshortdescribe Cumulative Distribution Function F distribution.

\item \describefun{void}{pnl_cdf_fnc}{int \ptr which, double \ptr p, double
    \ptr q, double \ptr x, double \ptr dfn, double \ptr dfd, double
    \ptr pnonc, int \ptr status, double \ptr bound}
  \sshortdescribe Cumulative Distribution Function Non-central F distribution.

\item \describefun{void}{pnl_cdf_gam}{int \ptr which, double \ptr p, double
    \ptr q, double \ptr x, double \ptr shape, double \ptr scale, int
    \ptr status, double \ptr bound}
  \sshortdescribe Cumulative Distribution Function GAMma distribution.

\item \describefun{void}{pnl_cdf_nbn}{int \ptr which, double \ptr p, double
    \ptr q, double \ptr x, double \ptr xn, double \ptr pr, double
    \ptr ompr, int \ptr status, double \ptr bound}
  \sshortdescribe Cumulative Distribution Function Negative BiNomial distribution.

\item \describefun{void}{pnl_cdf_nor}{int \ptr which, double \ptr p, double
    \ptr q, double \ptr x, double \ptr mean, double \ptr sd, int
    \ptr status, double \ptr bound}
  \sshortdescribe Cumulative Distribution Function NORmal distribution.

\item \describefun{void}{pnl_cdf_poi}{int \ptr which, double \ptr p, double
    \ptr q, double \ptr x, double \ptr xlam, int \ptr status, double
    \ptr bound}
  \sshortdescribe Cumulative Distribution Function POIsson distribution.

\item \describefun{void}{pnl_cdf_t}{int \ptr which, double \ptr p, double
    \ptr q, double \ptr x, double \ptr df, int \ptr status, double
    \ptr bound}
  \sshortdescribe Cumulative Distribution Function T distribution.

\item \describefun{double}{pnl_cdfchi2n}{double x, double df, double ncparam}
  \sshortdescribe Compute the cumulative density function at \var{x} of the
  non central $\chi^2$ distribution with \var{df} degrees of freedom and non
  centrality parameter \var{ncparam}.

\item \describefun{void}{pnl_cdfbchi2n}{double x, double df, double ncparam, double
    beta, double \ptr P}
  \sshortdescribe Store in \var{P} the cumulative density function at \var{x}
  of the random variable \var{beta \ptr X} where \var{X} is non central $\chi^2$
  random variable with \var{df} degrees of freedom and non
  centrality parameter \var{ncparam}.


\item \describefun{double}{pnl_normal_density}{double x}
  \sshortdescribe Normal density function.

\item \describefun{double}{pnl_cdfnor}{double x}
  \sshortdescribe Cumulative normal distribution function.

\item \describefun{double}{pnl_cdf2nor}{double a, double b, double r}
  \sshortdescribe Cumulative bivariate normal distribution function, returns
  $\frac{1} {2\pi \sqrt{1-r^2}} \int_{- \infty}^a\int_{- \infty}^b e^{-
    \frac{x^2 - 2 r xy+y^2} {2(1-r^2)} } dxdy.$

\item \describefun{double}{pnl_inv_cdfnor}{double x}
  \sshortdescribe Inverse of the cumulative normal distribution function.
\end{itemize}


\section{Random Number Generators}

The functionalities described in this chapter are declared in
\verb!pnl/pnl_random.h!.

Random number generators should be called through the new {\em rng} interface
based on the \PnlRng object. This interface uses reentrant functions
and is suitable for multi-threaded applications.

The older {\em rand} interface is kept for compatibility
issues only and hsould not be used in newly written code.
\begin{table}[h!]
  \begin{tabular}{l|l|l|l}
    Random generator & index & Type & Info\\
    \hline
    KNUTH & PNL_RNG_KNUTH & pseudo\\
    MRGK3 & PNL_RNG_MRGK3 & pseudo\\
    MRGK5 & PNL_RNG_MRGK5 & pseudo\\
    SHUFL & PNL_RNG_SHUFL & pseudo\\
    L'ECUYER & PNL_RNG_L_ECUYER & pseudo\\
    TAUSWORTHE & PNL_RNG_TAUSWORTHE & pseudo\\
    MERSENNE & PNL_RNG_MERSENNE & pseudo\\
    SQRT & PNL_RNG_SQRT & quasi\\
    HALTON & PNL_RNG_HALTON & quasi\\
    FAURE & PNL_RNG_FAURE & quasi\\
    SOBOL_I4 & PNL_RNG_SOBOL_I4 & quasi & uses 32 bit intergers\\
    SOBOL_I8 & PNL_RNG_SOBOL2_I8 & quasi & uses 64 bit intergers\\
    NIEDERREITER & PNL_RNG_NIEDERREITER & quasi
  \end{tabular}
  \caption{Indices of the random generators}
  \label{rng-indices}
\end{table}


\subsection{The rng interface}
\label{rng-int}

It is possible to create random number generators each with its own state
variable so that they can evolve independently in a shared memory environment.
These generators are suitable for use in multi-threaded programs. 

\describestruct{PnlRng}
\begin{verbatim}
typedef struct _PnlRng PnlRng;
struct _PnlRng
{
  PnlObject object;
  int type; /*!< generator type *
  void (*Compute)(PnlRng *g, double *sample); /*!< the function to compute the
                                                next number in the sequence */
  int rand_or_quasi; /*!< can be PNL_MC or PNL_QMC */
  int dimension; /*!< dimension of the space in which we draw the samples */
  int counter; /*!< counter = number of samples already drawn */
  int has_gauss; /*!< Is a gaussian deviate available? */
  double gauss; /*!< If has_gauss==1, gauss a gaussian sample */
  int size_state; /*!< size in bytes of the state variable */
  void *state; /*!< state of the random generator */
};
\end{verbatim}


\begin{itemize}
\item \describefun{void}{pnl_rng_free}{\PnlRng \ptr \ptr }
  \sshortdescribe Free a \PnlRng.
\item \describefun{\PnlRng\ptr }{pnl_rng_create}{int type}
  \sshortdescribe Create a \PnlRng corresponding to \var{type}
  which can be any of the values \var{PNL_RNG_XXX} listed in
  Table~\ref{rng-indices} which correspond to {\bf pseudo} random number generators.
  Once a generator has been created, you {\bf must} call
  \reffun{pnl_rng_sseed} before using it. 
\item \describefun{void}{pnl_rng_sseed}{\PnlRng \ptr rng, unsigned
    long int s}
  \sshortdescribe Set the seed of the genrator \var{rng} using \var{s}. If
  \var{s=0}, then a default seed (depending on the generator) is used.
\item \describefun{int}{pnl_rng_sdim}{\PnlRng \ptr rng, int dim}
  \sshortdescribe Set the dimension of the state space for a QMC generator and
  initializes it accordingly.  Returns OK if the generator has been initialized
  properly and FAIL otherwise.
\item \describefun{\PnlRng\ptr }{pnl_rng_copy}{const
  \PnlRng \ptr rng}
  \sshortdescribe Create a copy of \var{rng}.
\item \describefun{void}{pnl_rng_clone}{\PnlRng \ptr dest, const
  \PnlRng \ptr src}
  \sshortdescribe Copy the content of \var{src} into the already existing
  basis \var{dest}. On exit, \var{src} and \var{dest} are identical but
  independent. 
\item \describefun{\PnlRng\ptr}{pnl_rng_dcmt_create_id}{int id, ulong
    seed}
  \sshortdescribe Create a generator with type \var{PNL_RNG_DCMT} and identifier
  \var{id}. Two generators with different \var{id}s are independent. Note that
  the returned generator must be initialized with \reffun{pnl_rng_sseed} before
  usage. The identifier \var{id} can for instance correspond to the thread
  number or the processor rank in parallel computing.
\item \describefun{\PnlRng\ptr \ptr
  }{pnl_rng_dcmt_create_array_id}{int start_id, int max_id, ulong seed, int \ptr
    count}
  \sshortdescribe Create an array of generators with types \var{PNL_RNG_DCMT}
  and identifiers linearly varying between \var{start_id} and \var{max_id}. The
  number of generators created is \var{max_id - start_id + 1}. All the
  generators are independent. Note that each generator of the returned array
  must be initialized with \reffun{pnl_rng_sseed} before usage.
\item \describefun{\PnlRng \ptr \ptr}{pnl_rng_dcmt_create_array}
  {int n, ulong seed, int \ptr count}
  \sshortdescribe Create an array of \var{n} independent DCMT. \var{seed} is
  the seed used to initialize the Mersenne Twister generator internally used to
  find new DCMT. On exit, \var{count} contains the number of generators actually
  created. Same function as \reffun{pnl_dcmt_create_array} instead that it
  directly returns an array of \PnlRng. Before using the generators, you
  must initialize each of them by calling the function \reffun{pnl_rng_sseed}
  \var{count} times.
\end{itemize}

Some auxiliary functions internally used (to use with caution)
\begin{itemize}
\item \describefun{\PnlRng\ptr }{pnl_rng_new}{}
  \sshortdescribe Create an empty \PnlRng.
\item \describefun{void}{pnl_rng_init}{\PnlRng \ptr rng, int type}
  \sshortdescribe Initialize an empty \PnlRng as returned by
  \reffun{pnl_rng_new} to become a generator of type \var{type} which can be
  any of the values \var{PNL_RNG_XXX} listed in Table~\ref{rng-indices} which
  correspond to {\bf pseudo} random number generators.
  Calling \reffun{pnl_rng_create} is equivalent to calling first
  \reffun{pnl_rng_new} and then \reffun{pnl_rng_init}. 
\item \describefun{\PnlRng\ptr }{pnl_rng_get_from_id} {int id}
  \sshortdescribe Return the global generator described by its macro name.
  The variable \var{id} can be any of the values \var{PNL_RNG_XXX} listed in
  Table~\ref{rng-indices}.
\end{itemize}


The following functions return one sample from a specified law.
\begin{itemize}
\item \describefun{int}{pnl_rng_bernoulli}{double p, \PnlRng \ptr rng}
  \sshortdescribe Generate a sample from the Bernouilli law on $\{0, 1\}$ with
  parameter \var{p}.

\item \describefun{long}{pnl_rng_poisson}{double lambda, \PnlRng \ptr rng}
  \sshortdescribe Generate a sample from the Poisson law with
  parameter \var{lambda}.

\item \describefun{double}{pnl_rng_exp}{double lambda, \PnlRng \ptr rng}
  \sshortdescribe Generate a sample from the Exponential law with
  parameter \var{lambda}.

\item \describefun{double}{pnl_rng_dblexp}{double lambda_p, double lambda_m, double p, \PnlRng \ptr rng}
  \sshortdescribe Generate a sample from the asymmetric exponential distribution
  with density 
  \begin{equation*}
    p \lambda_p \expp{-\lambda_p y} \ind{y > 0} + (1-p) \lambda_m
    \expp{-\lambda_m |y|} \ind{y < 0}
  \end{equation*}
  where $\lambda_p >0, \lambda_m >0$ and $p \in [0, 1]$.

\item \describefun{double}{pnl_rng_uni} {\PnlRng \ptr rng}
  \sshortdescribe Generate a sample from the Uniform law on $]0, 1]$.

\item \describefun{double}{pnl_rng_uni_ab} {double a, double b,
    \PnlRng \ptr rng}
  \sshortdescribe Generate a sample from the Uniform law on $[a, b]$.

\item \describefun{double}{pnl_rng_normal} {\PnlRng \ptr rng}
  \sshortdescribe Generate a sample from the standard normal distribution.

\item \describefun{double}{pnl_rng_lognormal}{double m, double sigma2, \PnlRng \ptr rng}
  \sshortdescribe Generate a sample from the log-normal distribution. The
  underlying normal distribution has mean \var{m} and variance \var{sigma2}.

\item \describefun{double}{pnl_rng_invgauss}{double mu, double lambda, \PnlRng \ptr rng}
  \sshortdescribe Generate a sample from the inverse Gaussian distribution with
  mean \var{mu} and shape parameter \var{lambda}.

\item \describefun{long}{pnl_rng_poisson1}{double lambda, double
    t,\PnlRng \ptr rng}
  \sshortdescribe Generate a sample from a Poisson process with intensity
  \var{lambda} at time \var{t}.

\item \describefun{double}{pnl_rng_gamma} {double a, double b, \PnlRng \ptr rng}
  \sshortdescribe Generate a sample from the $\Gamma(a, b)$ distribution.

\item \describefun{double}{pnl_rng_chi2} {double n, \PnlRng \ptr rng}
  \sshortdescribe Generate a sample from the centered $\chi^2(n)$ distribution.
\item \describefun{double}{pnl_rng_bessel} {double v, double a,\PnlRng \ptr rng}
  \sshortdescribe Generate a sample from the Bessel distribution with parameters
  \var{v > -1} and \var{a > 0}.
\item \describefun{double}{pnl_rng_gauss}{int d, int
    create_or_retrieve, int index, PnlRng \ptr rng}
  \sshortdescribe The second argument can be either \var{CREATE} (to actually
  draw the sample) or \var{RETRIEVE} (to retrieve that element of index
  \var{index}). With \var{CREATE}, it draws \var{d} random normal variables
  and stores them for future usage. They can be withdrawn using \var{RETRIEVE}
  with the index of the number to be retrieved.
\end{itemize}

The following functions take an already existing \PnlVect\ptr  as
its first argument and fill each entry of the vector with a sample from the
specified law. All the entries are independent. The difference between
$n-$samples from a distribution in dimension $1$, and one sample from the same
distribution in dimension $n$ only matters when using a {\bf Quasi} random
number generator.
\begin{itemize}
\item \describefun{void}{pnl_vect_rng_bernoulli}{\PnlVect \ptr V, int samples,
  double a, double b, double p, \PnlRng \ptr rng}
  \sshortdescribe Simulate an i.i.d. sample from the Bernoulli
  distribution with values in \var{a,b} and parameter \var{p}. The result is
  stored in \var{V}.

\item \describefun{void}{pnl_vect_rng_bernoulli_d}{\PnlVect \ptr V, int
  dimension, const \PnlVect \ptr a, const \PnlVect \ptr b, const \PnlVect \ptr p, \PnlRng \ptr rng}
  \sshortdescribe Simulate a random vector according to the Bernoulli
  distribution with values in \var{\{a,b\}} and parameter \var{p}. The result is
  stored in \var{V}, ie. \var{V(i)} follows a Bernoulli distribution on
  \var{\{a(i), b(i)\}} with parameter \var{p(i)}.

\item \describefun{void}{pnl_mat_bernoulli}{\PnlMat \ptr M, int samples, int dimension,
  const \PnlVect \ptr a, const \PnlVect \ptr b, const \PnlVect \ptr p, \PnlRng \ptr rng}
  \sshortdescribe Compute a random matrix with independent rows, each of them
  having a vector Bernoulli distribution, ie. \var{M(i, j)} follows a
  Bernoulli distribution on \var{\{a(j), b(j)\}} with parameter \var{p(j)}.


\item \describefun{void}{pnl_vect_rng_poisson}{\PnlVect \ptr V, int samples,
  double lambda,  \PnlRng \ptr rng}
  \sshortdescribe Simulate an i.i.d. sample from the Poisson distribution with
  parameter \var{lambda}. The result is stored in \var{V}. Note that, we are
  using double based vectors and not integer based vectors.

\item \describefun{void}{pnl_vect_rng_poisson_d}{\PnlVect \ptr V, int
  dimension, const \PnlVect \ptr lambda, \PnlRng \ptr rng}
  \sshortdescribe Simulate a random vector according to the Poisson distribution
  with \textbf{vector} parameter \var{lambda}. The result is stored in \var{V},
  ie. \var{V(i)} follows a Poisson distribution with parameter \var{lambda(i)}.
  Note that, we are using double based vectors and not integer based vectors.
  
\item \describefun{void}{pnl_mat_poisson}{\PnlMat \ptr M, int samples, int dimension,
  const \PnlVect \ptr lambda, \PnlRng \ptr rng}
  \sshortdescribe Compute a random matrix with independent rows, each of them
  having a vector Poisson distribution, ie. \var{M(i, j)} follows a
  Poisson distribution with parameter \var{p(j)}.

\item \describefun{void}{pnl_vect_rng_uni}{\PnlVect \ptr G, int
    samples, double a, double b, \PnlRng \ptr rng}
  \sshortdescribe \var{G} is a vector of independent and identically distributed
  samples from the uniform distribution on $[a, b]$.

\item \describefun{void}{pnl_vect_rng_normal}{\PnlVect \ptr G,
    int samples, \PnlRng \ptr rng}
  \sshortdescribe \var{G} is a vector of independent and identically distributed
  samples from the standard normal distribution.

\item \describefun{void}{pnl_vect_rng_uni_d}{\PnlVect \ptr G, int
    d, double a, double b, \PnlRng \ptr rng}
  \sshortdescribe \var{G} is a sample from the uniform distribution on $[a,
  b]^{\text{d}}$.

\item \describefun{void}{pnl_vect_rng_normal_d}{\PnlVect \ptr G,
    int d, \PnlRng \ptr rng}
  \sshortdescribe \var{G} is a sample from the \var{d}-dimensional
  standard normal distribution.

\end{itemize}

The following functions take an already existing \PnlMat\ptr  as
first argument and fill each entry of the vector with a sample from the
specified law. All the entries are in-dependant. On return, the matrix \var{M}
is of size \verb!samples x dimension!. The rows of \var{M} are independently
and identically distributed. Each row is a sample from the given law in
dimension \var{dimension}.
\begin{itemize}
\item \describefun{void}{pnl_mat_rng_uni}{\PnlMat \ptr M, int
    samples, int d, const PnlVect \ptr a, const PnlVect \ptr b,
    \PnlRng \ptr rng}
  \sshortdescribe \var{M} contains \var{samples} samples from the uniform
  distribution on $\prod_{i=1}^d [a_i, b_i]$.

\item \describefun{void}{pnl_mat_rng_uni2}{\PnlMat \ptr M, int
    samples, int d, double a, double b, \PnlRng \ptr rng}
  \sshortdescribe \var{M} contains \var{samples} samples from the uniform
  distribution on $[a, b]^{\text{d}}$.

\item \describefun{void}{pnl_mat_rng_normal}{\PnlMat \ptr M, int
    samples, int d, \PnlRng \ptr rng}
  \sshortdescribe \var{M} contains \var{samples} samples from the
  \var{d}-dimensional standard normal distribution.
\end{itemize}

Some examples
\begin{verbatim}
#include <stdlib.h>
#include "pnl/pnl_random.h"

int main ()
{
  int i, M;
  PnlRng *rng = pnl_rng_create (PNL_RNG_MERSENNE);
  PnlVect *v = pnl_vect_new ();
  M = 10000;

  /* rng must be initialized. When sseed=0, a default 
     value depending on the generator is used */
     pnl_rng_sseed (rng, 0);

  for ( i=0 ; i<M ; i++ )
  {
    /* Simulates a normal random vector in R^{10} */
    pnl_vect_rng_normal (v, 10, rng);
    /* Do something with v */
  }

  pnl_vect_free (&v);
  pnl_rng_free (&rng); /* Frees the generator */
  exit (0);
}
\end{verbatim}

\begin{verbatim}
#include <stdlib.h>
#include <time.h>
#include "pnl/pnl_random.h"

int main ()
{
  int i, M;
  double E;
  PnlRng *rng = pnl_rng_create (PNL_RNG_MERSENNE);
  M = 10000;

  /* rng must be initialized. */
  rng = pnl_rng_sseed (time (NULL));

  for ( i=0 ; i<M ; i++ )
  {
    /* Simulates an exponential random variable */
    E = pnl_rng_exp (1, rng);
    /* Do something with E */
  }

  pnl_rng_free (&rng); /* Frees the generator */
  exit (0);
}
\end{verbatim}





\subsection{The {\em rand} interface (deprecated)}
\label{rand-int}

{\itshape 
\textbf{Note}:
For backward compatibility with older versions of the PNL, we still provide the old
{\em rand} interface to random number generation although we strongly encourage users
to use the new {\em rng} interface (see section~\ref{rng-int}).
}

Every generator is identified by an integer valued macro. One must {\bf NOT} refer
to a generator using directly the value of the macro \var{PNL_RNG_XXX} because there
is no warranty that the order used to store the generators will remain the same in
future releases.  Instead, one should call generators directly using their macro
names.

The initial seeds of all the generators are fixed by the function
\reffun{pnl_rand_init} but you can change it by calling \reffun{pnl_rand_sseed}.

Before starting to use random number generators, you {\bf must} initialize them by
calling
\begin{itemize}
\item \describefun{int}{pnl_rand_init}{int type_generator, int
    simulation_dim, long samples}
  \sshortdescribe It resets the sample counter to $0$ and checks that the
  generator described by \var{type_generator} can actually generate
  \var{samples} in dimension \var{simulation_dim} and fixes the seed.
\end{itemize}

\begin{itemize}

\item \describefun{int}{pnl_rand_or_quasi}{int type_generator}
  \sshortdescribe Return the type the generator of index \var{type_generator},
  \var{PNL_MC} or \var{PNL_QMC}
\item \describefun{void}{pnl_rand_sseed}{(int type_generator, unsigned long int
  seed)}
  \sshortdescribe It sets the seed of the generator \var{type_generator} with
  \var{seed}.
\item \describefun{const char \ptr }{pnl_rand_name}{int type_generator}
  \sshortdescribe Return the name of the generator of index \var{type_generator}
\end{itemize}

Once a generator is chosen, there are several functions available in the
library to draw samples according to a given law.

The following functions return one sample from a specified law.
\begin{itemize}
\item \describefun{int}{pnl_rand_bernoulli}{double p, int type_generator}
  \sshortdescribe Generate a sample from the Bernouilli law on $\{0, 1\}$ with
  parameter \var{p}.

\item \describefun{long}{pnl_rand_poisson}{double lambda, int type_generator}
  \sshortdescribe Generate a sample from the Poisson law with
  parameter \var{lambda}.

\item \describefun{double}{pnl_rand_exp}{double lambda, int type_generator}
  \sshortdescribe Generate a sample from the Exponential law with
  parameter \var{lambda}.

\item \describefun{double}{pnl_rand_uni} {int type_generator}
  \sshortdescribe Generate a sample from the Uniform law on $[0, 1]$.

\item \describefun{double}{pnl_rand_uni_ab} {double a, double b, int
    type_generator}
  \sshortdescribe Generate a sample from the Uniform law on $[a, b]$.

\item \describefun{double}{pnl_rand_normal} {int type_generator}
  \sshortdescribe Generate a sample from the standard normal distribution.

\item \describefun{long}{pnl_rand_poisson1}{double lambda, double t, int
    type_generator}
  \sshortdescribe Generate a sample from a Poisson process with intensity
  \var{lambda} at time \var{t}.

\item \describefun{double}{pnl_rand_gamma} {double a, double b, int type_generator}
  \sshortdescribe Generate a sample from the $\Gamma(a, b)$ distribution.

\item \describefun{double}{pnl_rand_chi2} {double n, int type_generator}
  \sshortdescribe Generate a sample from the centered $\chi^2(n)$ distribution.
\item \describefun{double}{pnl_rand_bessel} {double v, double a, int generator}
  \sshortdescribe Generate a sample from the Bessel distribution with parameters
  \var{v > -1} and \var{a > 0}.
\end{itemize}

The following functions take an already existing \PnlVect\ptr\  as
its first argument and fill each entry of the vector with a sample from the
specified law. All the entries are independent. The difference between
$n-$samples from a distribution in dimension $1$, and one sample from the same
distribution in dimension $n$ only matters when using a {\bf Quasi} random
number generator.
\begin{itemize}
\item \describefun{void}{pnl_vect_rand_uni}{\PnlVect \ptr G, int
    samples, double a, double b, int type_generator}
  \sshortdescribe \var{G} is a vector of independent and identically distributed
  samples from the uniform distribution on $[a, b]$.

\item \describefun{void}{pnl_vect_rand_normal}{\PnlVect \ptr G,
    int samples, int generator}
  \sshortdescribe \var{G} is a vector of independent and identically distributed
  samples from the standard normal distribution.

\item \describefun{void}{pnl_vect_rand_uni_d}{\PnlVect \ptr G, int
    d, double a, double b, int type_generator}
  \sshortdescribe \var{G} is a sample from the uniform distribution on $[a,
  b]^{\text{d}}$.

\item \describefun{void}{pnl_vect_rand_normal_d}{\PnlVect \ptr G,
    int d, int generator}
  \sshortdescribe \var{G} is a sample from the \var{d}-dimensional
  standard normal distribution.

\end{itemize}

The following functions take an already existing \PnlMat\ptr\  as
first argument and fill each entry of the vector with a sample from the
specified law. All the entries are in-dependant. On return, the matrix \var{M}
is of size \verb!samples x dimension!. The rows of \var{M} are independently
and identically distributed. Each row is a sample from the given law in
dimension \var{dimension}.
\begin{itemize}
\item \describefun{void}{pnl_mat_rand_uni}{\PnlMat \ptr M, int
    samples, int d, const PnlVect \ptr a, const PnlVect \ptr b, int
    type_generator}
  \sshortdescribe \var{M} contains \var{samples} samples from the uniform
  distribution on $\prod_{i=1}^d [a_i, b_i]$.

\item \describefun{void}{pnl_mat_rand_uni2}{\PnlMat \ptr M, int
    samples, int d, double a, double b, int type_generator}
  \sshortdescribe \var{M} contains \var{samples} samples from the uniform
  distribution on $[a, b]^{\text{d}}$.

\item \describefun{void}{pnl_mat_rand_normal}{\PnlMat \ptr M, int
    samples, int d, int type_generator}
  \sshortdescribe \var{M} contains \var{samples} samples from the
  \var{d}-dimensional standard normal distribution.
\end{itemize}

Because of the use of {\bf Quasi} random number generators, you may need to
draw a set of samples at once because they represent one sample from a
multi-dimensional distribution. The following function enables to draw one
sample from the \var{dimension}-dimensional standard normal distribution and
store it so that you can access the elements individually afterwards.
\begin{itemize}
\item \describefun{double}{pnl_rand_gauss}{int d, int
    create_or_retrieve, int index, int type_generator}
  \sshortdescribe The second argument can be either \var{CREATE} (to actually
  draw the sample) or \var{RETRIEVE} (to retrieve that element of index
  \var{index}). With \var{CREATE}, it draws \var{d} random normal variables
  and stores them for future usage. They can be withdrawn using \var{RETRIEVE}
  with the index of the number to be retrieved.
\end{itemize}


\subsubsection{Advanced usage}

We also provide functions for directly manipulating Mersenne Twister and
``Dynamically created Mersenne Twister'' random number generators, although we
believe one should rather use the new {\em rng} interface.

\paragraph{Mersenne Twister}

It is possible to create Mersenne Twister random number generators each with
its state variable.
\begin{verbatim}
typedef struct
{
  unsigned long mt[624];
  int mti;
} mt_state;
typedef unsigned long ulong;
\end{verbatim}

\begin{itemize}
\item \describefun{void}{pnl_mt_sseed}{mt_state \ptr state, unsigned long int
    s}
  \sshortdescribe Set the initial value of variable \var{state} using \var{s}
\item \describefun{ulong}{pnl_mt_genrand}{mt_state \ptr state}
  \sshortdescribe Return the following number in the sequence as an unsigned
  long variable. A mask is applied so that only the lowest 32-bits are used.
\item \describefun{double}{pnl_mt_genrand_double}{mt_state \ptr state}
  \sshortdescribe Return the following number in the sequence as a double.
\end{itemize}


\paragraph{Dynamically created Mersenne Twister}


These are Mersenne Twister type generators with Mersenne exponent fixed to
\var{p=521} and word length \var{w=32} bits. These choices are hard coded and
cannot be changed without altering the code directly.

\begin{verbatim}
typedef struct
{
  ulong aaa;
  int mm,nn,rr,ww;
  ulong wmask,umask,lmask;
  int shift0, shift1, shiftB, shiftC;
  ulong maskB, maskC;
  int i;
  ulong state[17];
} dcmt_state;
\end{verbatim}

Some functions to use ``Dynamically Created Mersenne Twister'' random number
generators (DCMT).
\begin{itemize}
\item \describefun{dcmt_state\ptr}{pnl_dcmt_get_parameter}{ulong seed}
  \sshortdescribe Create a DCMT. \var{seed} is the seed used to initialize
  the Mersenne Twister generator internally used to find new DCMT.
\item \describefun{dcmt_state \ptr \ptr}{pnl_dcmt_create_array}{int n, ulong seed, int \ptr count}
  \sshortdescribe Create an array of \var{n} independent DCMT. \var{seed} is
  the seed used to initialize the Mersenne Twister generator internally used to
  find new DCMT. On exit, \var{count} contains the number of generators actually
  created.
\item \describefun{double}{pnl_dcmt_genrand_double}{dcmt_state \ptr mts}
  \sshortdescribe Generate a uniformly distributed random variable on \var{[0,1]}.
\item \describefun{void}{pnl_dcmt_free}{dcmt_state \ptr \ptr mts}
  \sshortdescribe Free a dcmt.
\item \describefun{void}{pnl_dcmt_free_array}{dcmt_state \ptr \ptr mts, int count}
  \sshortdescribe Free an array of dcmt as returned by \reffun{pnl_dcmt_create_array}
\end{itemize}

% vim:spelllang=en:spell:



\section{Function bases and regression}
\subsection{Overview}

To use these functionalities, you should include \verb!pnl/pnl_basis.h!.

\describestruct{PnlBasis}
\begin{verbatim}
struct PnlBasis_t {
  int          id;
  const char  *label; /*!< string to label the basis *
  int          nb_variates;  /*!< number of variates *
  int          nb_func; /*!< number of elements in the basis *
  PnlMatInt   *T; /*!< Tensor matrix *
  PnlSpMatInt *SpT; /*!< Sparse Tensor matrix *
  double     (*f)(double   x, int i); /*!< Computes the i-th element
                                      of the one dimensional basis */
  double     (*Df)(double  x, int i); /*!< Computes the first derivative
                   of the i-th element of the one dimensional basis */
  double     (*D2f)(double x, int i); /*!< Computes the second derivative
                       of the i-th element of the one dimensional basis */
  int          isreduced; /* TRUE if the basis is reduced */
  double      *center; /*!< center of the domain */
  double      *scale; /*<! inverse of the scaling factor to map the 
                          domain to [-1, 1]^nb_variates */
};
\end{verbatim}

A multi-dimensional basis is built as a tensor product of one dimensional
elements. Hence, we only need a tensor matrix to describe a multi-dimensional
basis in terms of the one dimensional one.
The two tensors \var{T} and \var{SpT} do actually store the same information ---
\var{T(i,j)} is the degree w.r.t the \var{j}-th variable in the \var{i}-th
function. Originally, we were only using the dense representation \var{T}, which
is far more convenient to use when building the basis but it slows down the
evaluation of the basis by a great deal. To overcome this lack of efficiency, a
sparse storage has been added. 

\begin{table}[h!]
  \begin{describeconst}
    \constentry{PNL_BASIS_CANONICAL}{for the Canonical polynomials}
    \constentry{PNL_BASIS_HERMITE}{for the Hermite polynomials}
    \constentry{PNL_BASIS_TCHEBYCHEV}{for the Tchebychev polynomials}
  \end{describeconst}
  \caption{Names of the bases. See also function
  \reffun{pnl_basis_type_register} to register more basis types.}
  \label{basis_index}
\end{table}

The Hermite polynomials are defined by
\begin{equation*}
  H_n(x) = (-1)^n \expp{\frac{x^2}{2}} \frac{d^n}{dx^n} \expp{-\frac{x^2}{2}}.
\end{equation*}
If $G$ is a real valued standard normal random variable, ${\mathbb E}[H_n(G) H_m(G)] = n!
\ind{n = m}$. \\

In this section, we provide functions to solve regression problems on
finite dimensional bases. Let $(x_i, i=1 \dots n)$ be $n$ points in $\R^d$ and a
function $g$ defined by the data $(y_i = g(x_i), i=1 \dots n)$. Assume you
want to approximate the function $g$ by its decomposition on a family of $N$
functions $(f_j, j=1\dots N)$. Then, we want to compute the vector
$\alpha^\star \in \R^N$ solving
\begin{equation*} \alpha^\star = \arg\min_\alpha \sum_{i=1}^{n}
  \left(\sum_{j=0}^N \alpha_j f_j(x_i) - y_i\right)^2
\end{equation*}

\subsection{Functions}

\begin{itemize}
\item \describefun{int}{pnl_basis_type_register}{const char *name, double (*f)(double, int), double (*Df)(double, int), double (*D2f)(double, int)}
\sshortdescribe Register a new basis type and return the index to be passed to
\reffun{pnl_basis_create} . The variable \var{name} is a unique
string identifier of the family. The variables \var{f}, \var{Df}, \var{D2f} are
the one dimensional basis functions, its first and second order derivatives.
Each of these functions must return a \var{double} and take two arguments : the
first one is the point at which evaluating the basis functions, the second one
is the index of function. Here is a toy example to show how the canonical basis
is registered (this family is actually already available with the id
PNL_BASIS_CANONICAL, so the following example may look a little fake)
\begin{verbatim}
  double f(double x, int n) { return pnl_pow_i(x, n); }
  double Df(double x, int n) { return n * pnl_pow_i(x, n-1); }
  double f(double x, int n) { return n * (n-1) * pnl_pow_i(x, n-2); }

  int id = pnl_basis_register ("Canonic", f, Df, D2f);
  /*
   * B is the Canonical basis of polynomials with degree less or equal than 2 in
   * dimension 5.
   */
  PnlBasis *B = pnl_basis_create_from_degree (id, 2, 5);
  \end{verbatim}

\item \describefun{\PnlBasis *}{pnl_basis_new}{}
  \sshortdescribe Create an empty \PnlBasis.

\item \describefun{void}{pnl_basis_print}{const \PnlBasis \ptr B}
  \sshortdescribe Print the characteristics of a basis.
  
  
\item \describefun{\PnlBasis *}{pnl_basis_create}{int index, int
    nb_func, int nb_variates}
  \sshortdescribe Create a \PnlBasis for the family
  defined by \var{index} (see Table~\ref{basis_index} and
  \reffun{pnl_basis_type_register}) with \var{nb_variates}
  variates. The basis will contain \var{nb_func}.

\item \describefun{\PnlBasis *}{pnl_basis_create_from_degree}{int
    index, int degree, int nb_variates}
  \sshortdescribe Create a \PnlBasis for the family
  defined by \var{index} (see Table~\ref{basis_index} and \reffun{pnl_basis_type_register}) with total degree less
  or equal than \var{degree} and \var{nb_variates} variates. The total degree is
  the sum of the partial degrees.\\
  For instance, calling \verb!pnl_basis_create_from_degree (index, 2, 4)! is
  equivalent to calling \verb!pnl_basis_create_from_tensor (index, T)! where
  \var{T} is given by
  \[ \left(
    \begin{array}{cccc}
      0 & 0 & 0 & 0\\
      1 & 0 & 0 & 0\\
      0 & 1 & 0 & 0\\
      0 & 0 & 1 & 0\\
      0 & 0 & 0 & 1\\
      1 & 1 & 0 & 0\\
      1 & 0 & 1 & 0\\
      1 & 0 & 0 & 1\\
      0 & 1 & 1 & 0\\
      0 & 1 & 0 & 1\\
      0 & 0 & 1 & 1\\
      2 & 0 & 0 & 0\\
      0 & 2 & 0 & 0\\
      0 & 0 & 2 & 0\\
      0 & 0 & 0 & 2\\
    \end{array}
  \right) \]
\item \describefun{\PnlBasis *}{pnl_basis_create_from_prod_degree}{int
    index, int degree, int nb_variates}
  \sshortdescribe Create a \PnlBasis for the family
  defined by \var{index} (see Table~\ref{basis_index} and \reffun{pnl_basis_type_register}) with total degree less
  or equal than \var{degree} and \var{nb_variates} variates. The total degree is
  the product of \var{MAX(1, d_i)} where the \var{d_i} are the partial degrees.


\item \describefun{\PnlBasis *}{pnl_basis_create_from_tensor}{int
    index, PnlMatInt \ptr T}
  \sshortdescribe Create a \PnlBasis for the polynomial family
  defined by \var{index} (see Table~\ref{basis_index}) using the basis
  described by the tensor matrix \var{T}. The number of lines of \var{T} is
  the number of functions of the basis whereas the numbers of columns of
  \var{T} is the number of variates of the functions.
  Note that \var{T} is not copied inside this function but only its address is
  stored, so {\bf never} free \var{T}. It will be freed when calling
  \reffun{pnl_basis_free} on the returned object. i\\
  Here is an example of a tensor matrix. Assume you are working with three
  variate functions, the basis \verb!{ 1, x, y, z, x^2, xy, yz, z^3}! is
  decomposed in the one dimensional canonical basis using the following tensor
  matrix
  \[ \left(
    \begin{array}{ccc}
      0 & 0 & 0 \\
      1 & 0 & 0 \\
      0 & 1 & 0 \\
      0 & 0 & 1 \\
      2 & 0 & 0 \\
      1 & 1 & 0 \\
      0 & 1 & 1\\
      0 & 0 & 3
    \end{array}
  \right) \]

\item \describefun{void}{pnl_basis_clone}{\PnlBasis \ptr dest, const \PnlBasis \ptr src}
  \sshortdescribe Clone \var{src} into \var{dest}. The basis \var{dest} must
  already exist before calling this function. On exit, \var{dest} and \var{src}
  are identical and independent.
\item \describefun{\PnlBasis\ptr }{pnl_basis_copy}{const \PnlBasis \ptr B}
  \sshortdescribe Create a copy of \var{B}.
\item \describefun{void }{pnl_basis_set_from_tensor}{\PnlBasis \ptr
    b, int index, const \PnlMatInt \ptr T}
  \sshortdescribe Set an alredy existing basis \var{b} to a polynomial family
  defined by \var{index} (see Table~\ref{basis_index}) using the basis
  described by the tensor matrix \var{T}. The number of lines of \var{T} is
  the number of functions of the basis whereas the numbers of columns of
  \var{T} is the number of variates of the functions. \\
  Same function as \reffun{pnl_basis_create_from_tensor} except that it
  operates on an already existing basis.

\item  \describefun{\PnlBasis\ptr}{pnl_basis_create_from_hyperbolic_degree}
  {int index, double degree, double q, int n}
  \sshortdescribe Create a sparse basis of polynomial with \var{n}
  variates. We give the example of the Canonical basis. A canonical polynomial
  with \var{n} variates writes $X_1^{\alpha_1} X_2^{\alpha_2} \dots
  X_n^{\alpha_n}$. To be a member of the basis, it must satisfy $\left(\sum_{i=1}^n
    {\alpha_i}^q \right)^{1/q} \leq degree$. This kind of basis based on an
  hyperbolic set of indices gives priority to polynomials associated to low
  order interaction.

\item  \describefun{void}{pnl_basis_free}{\PnlBasis \ptr\ptr basis}
  \sshortdescribe Free a \PnlBasis created by
  \reffun{pnl_basis_create}. Beware that \var{basis} is the address of a
  \PnlBasis\ptr.


\item \describefun{void}{pnl_basis_del_elt}{\PnlBasis \ptr B, const \PnlVectInt \ptr d}
  \sshortdescribe Remove the function defined by the tensor product \var{d} from
  an existing basis \var{B}.

\item \describefun{void}{pnl_basis_del_elt_i}{\PnlBasis \ptr B, int i}
  \sshortdescribe Remove the \var{i-th} element of basis \var{B}.

\item \describefun{void}{pnl_basis_add_elt}{\PnlBasis \ptr B, const \PnlVectInt \ptr d}
  \sshortdescribe Add the function defined by the tensor \var{d} to the Basis \var{B}.


\end{itemize}


Functional regression based on a least square approach often leads to ill
conditioned linear systems. One way of improving the stability of the system is to
use centered and renormalised polynomials so that the original domain of interest
$\cD$ (a subset of $\R^d$) is mapped to $[-1,1]^d$. If the domain $\cD$ is
rectangular and writes $[a, b]$ where $a,b \in \R^d$, the mapping is done by 
\begin{equation}
  \label{basis_reduced}
  x \in \cD \longmapsto \left(\frac{x_i - (b_i+a_i)/2}{(b_i - a_i)/2}
  \right)_{i=1,\cdots,d}
\end{equation}
\begin{itemize}
\item \describefun{void}{pnl_basis_set_domain}{\PnlBasis \ptr B, 
  const \PnlVect \ptr a, const \PnlVect \ptr b}
  \sshortdescribe This function declares \var{B} as a centered and normalised basis
  as defined by Equation~\ref{basis_reduced}. Calling this function is equivalent to
  calling \reffun{pnl_basis_set_reduced} with \var{center=(b+a)/2} and
  \var{scale=(b-a)/2}.
\item \describefun{void}{pnl_basis_set_reduced}{\PnlBasis \ptr B,
  const \PnlVect \ptr center, const \PnlVect \ptr scale}
  \sshortdescribe This function declares \var{B} as a centered and normalised basis
  using the mapping
  \begin{equation*}
    x \in \cD \longmapsto \left(\frac{x_i - \var{center}_i }{\var{scale}_i}
    \right)_{i=1,\cdots,d}
  \end{equation*}
\end{itemize}


\begin{itemize}
\item \describefun{int}{pnl_basis_fit_ls}{\PnlBasis \ptr
    P, \PnlVect \ptr  coef, \PnlMat \ptr  x,
    \PnlVect \ptr  y}
  \sshortdescribe Compute the coefficients \var{coef} defined by
  \begin{equation*}
    \var{coef} = \arg\min_\alpha \sum_{i=1}^n
    \left( \sum_{j=0}^{\var{N}} \alpha_j  P_j(x_i) - y_i\right)^2
  \end{equation*}
  where \var{N} is the number of functions to regress upon and $n$ is the number
  of points at which the values of the original function are known. $P_j$ is the
  $j-th$ basis function. Each row of the matrix \var{x} defines the coordinates
  of one point $x_i$. The function to be approximated is defined by the vector
  \var{y} of the values of the function at the points \var{x}.

\item \describefun{double}{pnl_basis_ik_vect}{const \PnlBasis \ptr b, const \PnlVect \ptr x, int i, int k}
  \sshortdescribe An element of a basis writes $\prod_{l=0}^{\var{nb_variates}}
  \phi_l(x_l)$ where the $\phi$'s are one dimensional polynomials. This
  functions computes the therm $\phi_k$ of the \var{i-th} basis function at the
  point \var{x}.
\item \describefun{double}{pnl_basis_i_vect}{const \PnlBasis \ptr b, const \PnlVect \ptr x, int i}
  \sshortdescribe If \var{b} is composed of $f_0, \dots, f_{\var{nb\_func}-1}$,
  then this function returns $f_i(x)$. 

\item \describefun{double}{pnl_basis_i_D_vect}{const \PnlBasis \ptr b, const \PnlVect \ptr x, int i, int j}
  \sshortdescribe If \var{b} is composed of $f_0, \dots, f_{\var{nb\_func}-1}$,
  then this function returns $\partial_{x_{\var{j}}} f_i(x)$.

  
\item \describefun{double}{pnl_basis_i_D2_vect}{const \PnlBasis \ptr b, const \PnlVect \ptr x, int i, int j1, int j2}
  \sshortdescribe If \var{b} is composed of $f_0, \dots, f_{\var{nb\_func}-1}$,
  then this function returns $\partial^2_{x_{\var{j1}}, x_{\var{j2}}}
  f_i(x)$.


\item \describefun{void}{pnl_basis_eval_derivs_vect}{const \PnlBasis \ptr b, const \PnlVect \ptr coef, const \PnlVect \ptr x, double \ptr fx, \PnlVect \ptr Dfx, \PnlMat \ptr D2fx}
  \sshortdescribe Compute the function, the gradient and the Hessian matrix
  of $\sum_{k=0}^n \var{coef}_k  P_k(\cdot)$ at the point \var{x}.
  On output, \var{fx} contains the value of the function, \var{Dfx} its
  gradient and \var{D2fx} its Hessian matrix. This function is optimized and
  performs much better than calling \reffun{pnl_basis_eval},
  \reffun{pnl_basis_eval_D} and \reffun{pnl_basis_eval_D2} sequentially.

\item \describefun{double}{pnl_basis_eval_vect}{const \PnlBasis \ptr basis, const \PnlVect \ptr coef, const \PnlVect \ptr x}
  \sshortdescribe Compute the linear combination of \var{P_k(x)} defined by
  \var{coef}. Given the coefficients computed by the function
  \reffun{pnl_basis_fit_ls}, this function returns $\sum_{k=0}^n
  \var{coef}_k  P_k(\var{x})$.

\item \describefun{double}{pnl_basis_eval_D_vect}{const \PnlBasis \ptr basis, const \PnlVect \ptr coef, const \PnlVect \ptr x, int i}
  \sshortdescribe Compute the derivative with respect to \var{x_i} of the
  linear combination of \var{P_k(x)} defined by \var{coef}. Given the
  coefficients computed by the function \reffun{pnl_basis_fit_ls}, this function
  returns $\partial_{x_i} \sum_{k=0}^n \var{coef}_k  P_k(\var{x})$ The index
  \var{i} may vary between \var{0} and \var{P->nb_variates - 1}.


\item \describefun{double}{pnl_basis_eval_D2_vect}{const \PnlBasis \ptr basis, const \PnlVect \ptr coef, const \PnlVect \ptr x, int i, int j}
  \sshortdescribe Compute the derivative with respect to \var{x_i} of the
  linear combination of \var{P_k(x)} defined by \var{coef}. Given the
  coefficients computed by the function \reffun{pnl_basis_fit_ls}, this function
  returns $\partial_{x_i} \partial_{x_j} \sum_{k=0}^n \var{coef}_k
  P_k(\var{x})$.  The indices \var{i} and \var{j} may vary between \var{0} and
  \var{P->nb_variates - 1}.

\end{itemize}
The following functions are provided for compatibility purposes but are marked as
deprecated. Use the functions with the \verb!_vect! extension.
\begin{itemize}
\item \describefun{double}{pnl_basis_ik}{const \PnlBasis \ptr b, const double \ptr x, int i, int k}
  \sshortdescribe Same as function \reffun{pnl_basis_ik_vect} but takes a
  C array as the point of evaluation.
\item  \describefun{double}{pnl_basis_i}{\PnlBasis \ptr b, double \ptr x, int i}
  \sshortdescribe Same as function \reffun{pnl_basis_i_vect} but takes a
  C array as the point of evaluation.
\item \describefun{double}{pnl_basis_i_D}{ const \PnlBasis \ptr b, const double \ptr x, int i, int j }
  \sshortdescribe Same as function \reffun{pnl_basis_i_D_vect} but takes a
  C array as the point of evaluation.
\item \describefun{double}{pnl_basis_i_D2}{const \PnlBasis \ptr b, const double \ptr x, int i, int j1, int j2}
  \sshortdescribe Same as function \reffun{pnl_basis_i_D2_vect} but takes a
  C array as the point of evaluation.
\item \describefun{double}{pnl_basis_eval}{\PnlBasis \ptr P, \PnlVect\ptr  coef, double \ptr x}
  \sshortdescribe Same as function \reffun{pnl_basis_eval_vect} but takes a
  C array as the point of evaluation.
\item \describefun{double}{pnl_basis_eval_D}{\PnlBasis \ptr P, \PnlVect \ptr  coef, double \ptr x, int i}
  \sshortdescribe Same as function \reffun{pnl_basis_eval_D_vect} but takes a
  C array as the point of evaluation.
\item \describefun{double}{pnl_basis_eval_D2}{\PnlBasis \ptr  P, \PnlVect \ptr  coef, double \ptr x,  int i, int j}
  \sshortdescribe Same as function \reffun{pnl_basis_eval_D2_vect} but takes a 
  C array as the point of evaluation.
\item \describefun{void}{pnl_basis_eval_derivs}{\PnlBasis \ptr P, \PnlVect\ptr coef, double \ptr x, double \ptr fx, \PnlVect \ptr Dfx, \PnlMat \ptr D2fx}
  \sshortdescribe Same as function \reffun{pnl_basis_eval_derivs_vect} but takes a
  C array as the point of evaluation.
\end{itemize}


% vim:spelllang=en:spell:


\section{Numerical integration}
\subsection{Overview}

To use these functionalities, you should include \verb!pnl/pnl_integration.h!.

Numerical integration methods are designed to numerically evaluate the integral
over a finite or non finite interval (resp. over a square) of real valued
functions defined on $\R$ (resp. on $\R^2$).

\begin{verbatim}
typedef struct {
  double (*function) (double x, void *params);
  void *params;
} PnlFunc;

typedef struct {
  double (*function) (double x, double y, void *params);
  void *params;
} PnlFunc2D;
\end{verbatim}

We provide the following two macros to evaluate a \refstruct{PnlFunc} or
\refstruct{PnlFunc2D} at a given point
\begin{verbatim}
#define PNL_EVAL_FUNC(F, x) (*((F)->function))(x, (F)->params)
#define PNL_EVAL_FUNC2D(F, x, y) (*((F)->function))(x, y, (F)->params)
\end{verbatim}



\subsection{Functions}

\begin{itemize}
\item \describefun{double}{pnl_integration}{\refstruct{PnlFunc} \ptr F,
    double x0, double x1, int n, char \ptr meth}
  \sshortdescribe Evaluates $\int_{x_0}^{x_1} F$ using \var{n} discretization
  steps. The method used to discretize the integral is defined by \var{meth}
  which can be \var{"rect"} (rectangle rule), \var{"trap"} (trapezoidal rule),
  \var{"simpson"} (Simpson's rule).

\item \describefun{double}{pnl_integration_2d}{\refstruct{PnlFunc2D} \ptr F,
    double x0, double x1, double y0, double y1, int nx, int ny, char \ptr meth}
  \sshortdescribe Evaluates $\int_{[x_0, x_1] \times [y_0, y_1]} F$ using
  \var{nx} (resp. \var{ny}) discretization steps for \var{[x0, x1]}
  (resp. \var{[y0, y1]}). The method used to discretize the integral is
  defined by \var{meth} which can be \var{"rect"} (rectangle rule),
  \var{"trap"} (trapezoidal rule),   \var{"simpson"} (Simpson's rule).


\item \describefun{int}{pnl_integration_qng}{\refstruct{PnlFunc} \ptr F,
    double x0, double x1, double epsabs, double epsrel, double \ptr result,
    double \ptr abserr,  int \ptr neval}
  \sshortdescribe Evaluates $\int_{x_0}^{x_1} F$ with an absolute error less
  than \var{espabs} and a relative error less than \var{esprel}. The value of
  the integral is stored in \var{result}, while the variables \var{abserr} and
  \var{neval} respectively contain the absolute error and the number of function
  evaluations. This function returns \var{OK} if the required accuracy has been
  reached and \var{FAIL} otherwise. This function uses a non-adaptive Gauss
  Konrod procedure (qng routine from {\it QuadPack}).

\item \describefun{int}{pnl_integration_GK}{\refstruct{PnlFunc} \ptr F,
    double x0, double x1, double epsabs, double epsrel, double \ptr result,
    double \ptr abserr,  int \ptr neval}
  \sshortdescribe This function is a synonymous of
  \reffun{pnl_integration_qng} and is only available for backward
  compatibility. It is deprecated, please use \reffun{pnl_integration_qng}
  instead.

\item \describefun{int}{pnl_integration_qng_2d}{\refstruct{PnlFunc2D} \ptr F,
    double x0, double x1, double y0, double y1, double epsabs, double epsrel,
    double \ptr result, double \ptr abserr, int \ptr neval}
  \sshortdescribe Evaluates $\int_{[x_0, x_1] \times [y_0, y_1]} F$ with an
  absolute error less than \var{espabs} and a relative error less than
  \var{esprel}. The value of the integral is stored in \var{result}, while the
  variables \var{abserr} and \var{neval} respectively contain the absolute error
  and the number of function evaluations. This function returns \var{OK} if the
  required accuracy has been reached and \var{FAIL} otherwise.

\item \describefun{int}{pnl_integration_GK2D}{\refstruct{PnlFunc} \ptr F,
    double x0, double x1, double epsabs, double epsrel, double \ptr result,
    double \ptr abserr,  int \ptr neval}
  \sshortdescribe This function is a synonymous of
  \reffun{pnl_integration_qng_2d} and is only available for backward
  compatibility. It is deprecated, please use \reffun{pnl_integration_qng_2d}
  instead.

\item \describefun{int}{pnl_integration_qag}{\refstruct{PnlFunc} \ptr F,
    double x0, double x1, double epsabs, int limit, double epsrel, double \ptr
    result, double \ptr abserr,  int \ptr neval}
  \sshortdescribe Evaluates $\int_{x_0}^{x_1} F$ with an absolute error less
  than \var{espabs} and a relative error less than \var{esprel}. \var{x0} and
  \var{x1} can be non finite (i.e. \var{PNL_NEGINF} or \var{PNL_POSINF}). The
  value of the integral is stored in \var{result}, while the variables
  \var{abserr} and \var{neval} respectively contain the absolute error and the
  number of iterations. \var{limit} is the maximum number of subdivisions of the
  interval \var{(x0,x1)} used during the integration. If on input, \var{limit
    0}, then 750 subdivisions are used.  This function returns \var{OK} if the
  required accuracy has been reached and \var{FAIL} otherwise. This function
  uses some adaptive procedures (qags and qagi routines from {\it QuadPack}).
  This function is able to handle functions \var{F} with integrable
  singularities on the interval \var{[x0,x1]}.

\item \describefun{int}{pnl_integration_qagp}{\refstruct{PnlFunc} \ptr F,
    double x0, double x1, const PnlVect \ptr singularities, double epsabs,
    int limit, double epsrel, double \ptr result, double \ptr abserr,  int \ptr neval}
  \sshortdescribe Evaluates $\int_{x_0}^{x_1} F$  for a function
  \var{F} with known singularities listed in \var{singularities}.
  \var{singularities} must be a sorted vector which does not contain \var{x0}
  nor \var{x1}.  \var{x0} and \var{x1} must be  finite. The value of the
  integral is stored in \var{result}, while the variables \var{abserr} and
  \var{neval} respectively contain the absolute error and the number of
  iterations. \var{limit} is the maximum number of subdivisions of the interval
  \var{(x0,x1)} used during the integration. If on input, \var{limit = 0}, then
  750 subdivisions are used.  This function returns \var{OK} if the required
  accuracy has been reached and \var{FAIL} otherwise. This function uses some
  adaptive procedures (qagp routine from {\it QuadPack}).  This function is
  able to handle functions \var{F} with integrable singularities on the interval
  \var{[x0,x1]}.
\end{itemize}

% vim:spelllang=en:spell:


\section{Fast Fourier Transform}
\subsection{Overview}

The coefficients of the Fourier transform of a rela function satisfy the
following relation
\begin{equation}
  \label{eq:fft-sym}
  z_k = \overline{z_{N-k}},
\end{equation}
where $N$ is the number of discretization points.

A few remarks on the FFT of real functions and its inverse transform:
\begin{itemize}
\item We only need half of the coefficients.
\item When a value is known to be real, its imaginary part is not stored.
  So the imaginary part of the zero-frequency component is never stored as it is
  known to be zero.
\item For a sequence of even length the imaginary part of the frequency
  $n/2$ is not stored either, since the symmetry (\ref{eq:fft-sym}) implies
  that this is purely real too.
\end{itemize}


\paragraph{FFTPack storage}
\label{sec:fftpack-storage}

The functions use the fftpack storage convention for half-complex sequences.
In this convention, the half-complex transform of a real sequence is stored
with frequencies in increasing order, starting from zero, with the real and
imaginary parts of each frequency in neighboring locations.

The storage scheme is best shown by some examples. The table below shows the
output for an odd-length sequence, $n=5$.  The two columns give the
correspondence between the $5$ values in the half-complex sequence (stored in
a PnlVect $V$) and the values (PnlVectComplex $C$) that would be returned if
the same real input sequence were passed to pnl_dft_complex as a complex
sequence (with imaginary parts set to 0),
\begin{equation}
  \begin{array}{l}
    C(0) =  V(0) + \imath 0, \\
    C(1) =  V(1) + \imath V(2), \\
    C(2) =  V(3) + \imath V(4), \\
    C(3) = V(3) - \imath V(4)=  \overline{C(2)} , \\
    C(4) = V(1) + \imath V(2)=  \overline{C(1)}
  \end{array}
\end{equation}

The elements of index greater than $N/2$ of the complex array, as $C(3)$
$C(4)$, are filled in using the symmetry condition.

The next table shows the output for an even-length sequence, $n=6$.
In the even case there are two values which are purely real,
\begin{equation}
  \begin{array}{l}
    C(0) =  V(0) + \imath 0, \\
    C(1) =  V(1) + \imath V(2), \\
    C(2) =  V(3) + \imath V(4), \\
    C(3) = V(5) - \imath 0    =  \overline{C(0)} , \\
    C(4) = V(3) - \imath V(4) =  \overline{C(2)} , \\
    C(5) = V(1) + \imath V(2) =  \overline{C(1)}
  \end{array}
\end{equation}


\subsection{Functions}

To use the following functions, you should include \verb!pnl/pnl_fft.h!.

The following functions comes from a C version of the Fortran FFTPack library
available on \url{http://www.netlib.org/fftpack}.
\begin{itemize}
\item \describefun{int}{pnl_fft_inplace}{\refstruct{PnlVectComplex} \ptr data}
  \sshortdescribe Computes the FFT of \var{data} in place. The original content
  of \var{data} is lost.

\item \describefun{int}{pnl_ifft_inplace}{\refstruct{PnlVectComplex} \ptr data}
  \sshortdescribe Computes the inverse FFT of \var{data} in place. The
  original content of \var{data} is lost.

\item \describefun{int}{pnl_fft}{const \refstruct{PnlVectComplex} \ptr in,
    \refstruct{PnlVectComplex} \ptr out}
  \sshortdescribe Computes the FFT of \var{in} and stores it into \var{out}.

\item \describefun{int}{pnl_ifft}{const \refstruct{PnlVectComplex} \ptr in,
    \refstruct{PnlVectComplex} \ptr out}
  \sshortdescribe Computes the inverse FFT of \var{in} and stores it into \var{out}.

\item \describefun{int}{pnl_fft2}{double \ptr re, double \ptr im, int n}
  \sshortdescribe Computes the FFT of the vector of length \var{n} whose real
  (resp. imaginary) parts are given by the arrays \var{re}
  (resp. \var{im}). The real and imaginary parts of the FFT are respectively
  stored in \var{re} and \var{im} on output.

\item \describefun{int}{pnl_ifft2}{double \ptr re, double \ptr im, int n}
  \sshortdescribe Computes the inverse FFT of the vector of length \var{n}
  whose real (resp. imaginary) parts are given by the arrays \var{re}
  (resp. \var{im}). The real and imaginary parts of the inverse FFT are
  respectively stored in \var{re} and \var{im} on output.

\item \describefun{int}{pnl_real_fft}{const \refstruct{PnlVect} \ptr in,
    \refstruct{PnlVectComplex} \ptr out}
  \sshortdescribe Computes the FFT of the real valued sequence \var{in} and
  stores it into \var{out}.

\item \describefun{int}{pnl_real_ifft}{const \refstruct{PnlVect} \ptr in,
    \refstruct{PnlVectComplex} \ptr out}
  \sshortdescribe Computes the inverse FFT of \var{in} and stores it into \var{out}.

\item \describefun{int}{pnl_real_fft_inplace}{double \ptr data, int n}
  \sshortdescribe Computes the FFT of the real valued vector \var{data} of
  length \var{n}. The result is stored in \var{data} using the FFTPack storage
  described above, see~\ref{sec:fftpack-storage}.

\item \describefun{int}{pnl_real_ifft_inplace}{double \ptr data, int n}
  \sshortdescribe Computes the inverse FFT of the vector \var{data} of length
  \var{n}. \var{data} is supposed to be the FFT coefficients a real valued
  sequence stored using the FFTPack storage. On output, \var{data} contains
  the inverse FFT.

\item \describefun{int}{pnl_real_fft2}{double \ptr re, double \ptr im, int n}
  \sshortdescribe Computes the FFT of the real vector \var{re} of length \var{n}.
  \var{im} is only used on output to store the imaginary part the FFT. The
  real part is stored into \var{re}

\item \describefun{int}{pnl_real_ifft2}{double \ptr re, double \ptr im, int n}
  \sshortdescribe Computes the inverse FFT of the vector \var{re + i * im} of
  length \var{n}, which is supposed to be the FFT of a real valued
  sequence. On exit, \var{im} is unused.
\end{itemize}

% vim:spelllang=en:spell:



\section{Inverse Laplace Transform}

For a real valued function $f$ such that $t \longmapsto f(t) \expp{- \sigma_c
  t}$ is integrable over $\R^+$, we can define its Laplace transform
\begin{equation*}
  \hat{f}(\lambda) = \int_0^\infty f(t) \expp{- \lambda t} dt \qquad
  \mbox{for $\lambda \in \C$ with $\real{\lambda} \ge \sigma_c$}.
\end{equation*}

To use the following functions, you should include \verb!pnl/pnl_laplace.h!.

\describestruct{PnlCmplxFunc}
\begin{verbatim}
typedef struct
{
  dcomplex (*F) (dcomplex x, void *params);
  void *params;
} PnlCmplxFunc;
 \end{verbatim}

\begin{itemize}
\item \describefun{double}{pnl_ilap_euler}{\PnlCmplxFunc
    \ptr fhat, double t, int N, int M}
  \sshortdescribe Compute $f(\var{t})$ where $f$ is given by its Laplace
  transform \var{fhat} by numerically inverting the Laplace transform using
  Euler's summation. The values \var{N = M = 15} usually give a very good
  accuracy. For more details on the accuracy of the method.

\item \describefun{double}{pnl_ilap_cdf_euler}{\PnlCmplxFunc
    \ptr fhat, double t, double h, int N, int M}
  \sshortdescribe Compute the cumulative distribution function $F(\var{t})$
  where $F(x) = \int_0^x f(t) dt$ and $f$ is a density function with values on
  the positive real linegiven by its Laplace transform \var{fhat}. The
  computation is carried out by numerical inversion of the Laplace transform
  using Euler's summation. The values \var{N = M = 15} usually give a very
  good accuracy. The parameter \var{h} is the discretization step, the
  algorithm is very sensitive to the choice of \var{h}.

\item \describefun{double}{pnl_ilap_fft}{\PnlVect \ptr res,
    \PnlCmplxFunc \ptr fhat, double T, double eps}
  \sshortdescribe Compute $f(t)$ for $t \in [h, \var{T}]$ on a regular grid
  and stores the values in \var{res}, where $h = T / {\mathrm size}(res)$. The
  function $f$ is defined by its Laplace transform \var{fhat}, which is
  numerically inverted using a Fast Fourier Transform algorithm. The size of
  \var{res} is related to the choice of the relative precision \var{eps}
  required on the value of $f(t)$ for all $t \le T$.

\item \describefun{double}{pnl_ilap_gs}{\refstruct{PnlFunc} \ptr fhat, double
    t, int n}
  \sshortdescribe Compute $f(\var{t})$ where $f$ is given by its Laplace
  transform \var{fhat} by numerically inverting the Laplace transform using a
  weighted combination of different Gaver Stehfest's algorithms. Note that
  this function does not need the complex valued Laplace transform but only the
  real valued one. \var{n} is the number of terms used in the weighted combination.

\item \describefun{double}{pnl_ilap_gs_basic}{\refstruct{PnlFunc}
    \ptr fhat, double t, int n}
  \sshortdescribe Compute $f(\var{t})$ where $f$ is given by its Laplace
  transform \var{fhat} by numerically inverting the Laplace transform using
  Gaver Stehfest's method. Note that this function does not
  need the complex valued Laplace transform but only the real valued
  one. \var{n} is the number of iterations of the algorithm.
  {\bf Note : }~This function is provided for test purposes only. The 
  function \reffun{pnl_ilap_gs} gives far more accurate results.
\end{itemize}

% vim:spelllang=en:spell:

\section{Ordinary differential equations}
\subsection{Overview}

To use these functionalities, you should include \verb!pnl/pnl_integration.h!.

These functions are designed for numerically solving $n-$dimensional first order
ordinary differential equation of the general form
\begin{equation*}
  \frac{dy_i}{dt}(t) = f_i(t, y_1(t), \cdots, y_n(t)) 
\end{equation*}
The system of equations is defined by the following structure
\begin{verbatim}
typedef struct
{
  void (*function) (int neqn, double t, const double *y, double *yp, void *params);
  int neqn; 
  void *params;
} PnlODEFunc ;
\end{verbatim}

\begin{itemize}
\item \describevar{int}{neqn} 
  \sshortdescribe Number of equations
\item \describevar{void \ptr}{params} 
  \sshortdescribe An untyped structure used to pass extra
  arguments to the function \var{f} defining the system
\item \describefun*{void}{(\ptr\ function)}{int neqn, double t, const double \ptr
    y, double \ptr yp, void \ptr params}
  \sshortdescribe After calling the fuction, \var{yp} should be defined as
  follows \var{yp_i = f_i(neqn, t, y, params)}. \var{y} and \var{yp} should be
  both of size \var{neqn}
\end{itemize}
We provide the following macro to evaluate a \refstruct{PnlODEFunc}
at a given point
\begin{verbatim}
#define PNL_EVAL_ODEFUNC(F, t, y, yp) \
        (*((F)->function))((F)->neqn, t, y, yp, (F)->params)
\end{verbatim}

\subsection{Functions}

\begin{itemize}
\item \describefun{int}{pnl_ode_rkf45}{\refstruct{PnlODEFunc} \ptr f, double
    \ptr y, double t, double t_out, double relerr, double abserr, int \ptr flag}
  \sshortdescribe This function computes the solution of the system defined by
  the \refstruct{PnlODEFunc} \var{f} at the point \var{t_out}. On input,
  $\var{(t,y)}$ should be the initial condition, \var{abserr,relerr} are the
  maximum absolute and relative errors for local error tests (at each step,
    \var{abs(local error)} should be less that \var{relerr * abs(y) + abserr}).
  Note that if \var{abserr = 0} or \var{relerr = 0}  on input, an optimal value
  for these variables is computed inside the function The function returns an
  error \var{OK} or \var{FAIL}. In case of an \var{OK} code, the \var{y}
  contains the solution computed at \var{t_out}, in case of a \var{FAIL} code,
  \var{flag} should be examined to determine the reason of the error. Here are
  the different possible values for \var{flag}
  \begin{itemize}
  \item \var{flag = 2} : integration reached \var{t_out}, it indicates
    successful return and is the normal mode for continuing integration.
 \item \var{flag = 3} : integration was not completed because relative error
   tolerance was too small. relerr has been increased appropriately for
   continuing.
 \item \var{flag = 4} : integration was not completed because more than 3000
   derivative evaluations were needed. this is approximately 500 steps.
 \item \var{flag = 5} : integration was not completed because solution vanished
   making a pure relative error test impossible. must use non-zero abserr to
   continue.  using the one-step integration mode for one step is a good way to
   proceed.
 \item \var{flag = 6} : integration was not completed because requested accuracy
   could not be achieved using smallest allowable stepsize. user must increase
   the error tolerance before continued integration can be attempted.
 \item \var{flag = 7} : it is likely that rkf45 is inefficient for solving this
   problem. too much output is restricting the natural stepsize choice. use the
   one-step integrator mode. see \reffun{pnl_ode_rkf45_step}.
 \item \var{flag = 8} : invalid input parameters this indicator occurs if any of
   the following is satisfied -   neqn <= 0, t=tout,  relerr or abserr <= 0.
  \end{itemize}
\item \describefun{int}{pnl_ode_rkf45_step}{\refstruct{PnlODEFunc} \ptr f,
    double \ptr y, double \ptr t, double t_out, double \ptr relerr, double
    abserr, double \ptr work, int \ptr iwork, int \ptr flag} 
  \sshortdescribe Same as \reffun{pnl_ode_rkf45} but it only computes one step
  of integration in the direction of \var{t_out}. \var{work} and \var{iwork} are
  working arrays of size \var{3 + 6 * neqn} and \var{5} respectively and should
  remain untouched between successive calls to the function. 
  On output \var{t} holds the point at which integration stopped and \var{y} the
  value of the solution at that point.
\end{itemize}

% vim:spelllang=en:spell:


\section{Nonlinear Constrained Optimization}
\subsection{Overview}

A standard Constrained Nonlinear Optimization problem can be written as:

\begin{equation*}
  (O)\;
  \left\{
    \begin{tabular}{l}
      $\displaystyle   \min \; f(x)$ \\
      $\displaystyle c^I(x) \geq 0$ \\
      $\displaystyle c^E(x) = 0$
    \end{tabular}
  \right.
\end{equation*}

where the function $f : \mathbb{R}^n \rightarrow  \mathbb{R}$ is the objective function, $c^I : \mathbb{R}^n \rightarrow  \mathbb{R}^{m_I} $ are the inequality constraints and $c^E : \mathbb{R}^n \rightarrow  \mathbb{R}^{m_E} $ are the equality constraints. These functions are supposed to be smooth.

In general, the inequality constraints are of the form $c^I(x) = \left (g(x), \; x-l, \; u-x \right )$. The vector $l$ and $u$ are the lower and upper bounds on the variables $x$ and $g(x)$ and the non linear inequality constraints.

Under some conditions, if $x \in \mathbb{R}^n$ is a solution of problem ($O$), then there exist a vector $\lambda=(\lambda^I,\lambda^E) \in \mathbb{R}^{m_I} \times \mathbb{R}^{m_E}$, such that the well known Karush-Kuhn-Tucker (KKT) optimality conditions are satisfied:

\begin{equation*}
  (P)\;
  \left\{
    \begin{tabular}{c}
      $\nabla \ell(x,\lambda^I, \lambda^E) = \nabla f(x) - \nabla c^I(x) \lambda^I - \nabla c^E(x) \lambda^E= 0$ \\
      $c^E(x) = 0 $ \\
      $c^I(x) \geq 0 $ \\
      $\lambda^I \geq 0 $\\
      $c^I_i(x) \lambda^I_i =0, \; i=1...m_I$ \\
    \end{tabular}
  \right.
\end{equation*}

$l$ is known as the lagrangian of the problem $(O)$, $\lambda^I$ and $\lambda^E$ as the dual variables while $x$ is the primal variable.

\subsection{Functions}

To use the following functions, you should include \verb!pnl/pnl_optim.h!.

To solve an inequality constrained optimization problem, ie $m_E=0$, we provide the following function.
\begin{itemize}
\item \describefun{int}{pnl_optim_intpoints_bfgs_solve}{\refstruct{PnlRnFuncR}\ptr func, \refstruct{PnlRnFuncRm}\ptr grad_func, \refstruct{PnlRnFuncRm}\ptr nl_constraints, \refstruct{PnlVect}\ptr lower_bounds, \refstruct{PnlVect}\ptr upper_bounds, \refstruct{PnlVect}\ptr x_input, double tolerance, int iter_max, int print_inner_steps, \refstruct{PnlVect}\ptr output}
  \sshortdescribe This function has the following arguments:

  \begin{itemize}
  \item \var{func} is the function to minimize $f$.
  \item \var{grad} is the gradient of $f$. If this gradient is not available, then enter \var{grad}=NULL. In this case, finite difference will be used to estimate the gradient.
  \item \var{nl_constraints} is the function $g(x)$, ie the non linear inequality constraints.
  \item \var{lower_bounds} are the lower bounds on $x$. Can be NULL if there is no
    lower bound.
  \item \var{upper_bounds} are the upper bounds on $x$. Can be NULL if there is no
    upper bound.
  \item \var{x_input} is the initial point where the algorithm starts.
  \item \var{tolerance} is the precision required in solving (P).
  \item \var{iter_max} is the maximum number of iterations in the algorithm.
  \item \var{print_algo_steps} is a flag to decide to print information.
  \item \var{x_output} is the point where the algorithm stops.
  \end{itemize}

  The algorithm returns an $int$, its value depends on the output status of the algorithm. We have 4 cases:

  \begin{itemize}
  \item 0: Failure: Initial point is not strictly feasible.
  \item 1: Step too small, we stop the algorithm.
  \item 2: Maximum iteration reached.
  \item 3: A solution has been found up to the required accuracy.
  \end{itemize}

  The last case is equivalent to the two inequalities:

  $$ || \nabla \ell(x,\lambda^I)||_{\infty} = ||\nabla f(x) - \nabla c^I(x) \lambda^I ||_{\infty} < \var{tolerance} $$
  $$ || c^I(x) \lambda^I ||_{\infty} < \var{tolerance} $$

  where $c^I(x) \lambda^I$ is a vector of term by term multiplication of $c^I(x)$ and $\lambda^I$.

  The first inequality is known as the optimality condition, the second one as the complementarity condition.\\

  \textbf{Important Remark 1}: The algorithm we implement requires that the initial point $x_0$, given as an input to the algorithm, to be strictly feasible, ie: $c(x_0)>0$.\\
  \textbf{Important Remark 2}: The algorithm try to find a pair ($x$, $\lambda$) that solves the equations ($P$), but this does not guarantee that $x$ is a global minimum of $f$ on the set $\{c(x)\geq0\}$.

\end{itemize}


%%% Local Variables: 
%%% mode: latex
%%% TeX-master: "pnl-manual"
%%% End: 

\section{Root finding}
\subsection{Overview}

To provide a uniformed framework to root finding functions, we use several
structures for storing different kind of functions. The pointer
\var{params} is used to store the extra parameters. These new types come
with dedicated macros starting in \verb!PNL_EVAL!  to evaluate the function
and their Jacobian.
\describestruct{PnlFunc}
\begin{verbatim}
/*
 * f: R --> R
 * The function  pointer returns f(x)
 *
typedef struct {
  double (*function) (double x, void *params);
  void *params;
} PnlFunc ;
#define PNL_EVAL_FUNC(F, x) (*((F)->function))(x, (F)->params)
\end{verbatim}

\describestruct{PnlFunc2D}
\begin{verbatim}
/*
 * f: R^2 --> R
 * The function pointer returns f(x)
 *
typedef struct {
  double (*function) (double x, double y, void *params);
  void *params;
} PnlFunc2D ;
#define PNL_EVAL_FUNC2D(F, x, y) (*((F)->function))(x, y, (F)->params)
\end{verbatim}

\describestruct{PnlFuncDFunc}
\begin{verbatim}
/*
 * f: R --> R
 * The function pointer computes f(x) and Df(x) and stores them in fx
 * and dfx respectively
 *
typedef struct {
  void (*function) (double x, double *fx, double *dfx, void *params);
  void *params;
} PnlFuncDFunc ;
#define PNL_EVAL_FUNC_DFUNC(F, x, fx, dfx) (*((F)->function))(x, fx, dfx, (F)->params)
\end{verbatim}

\describestruct{PnlRnFuncR}
\begin{verbatim}
/*
 * f: R^n --> R
 * The function pointer returns f(x)
 *
typedef struct {
  double (*function) (const PnlVect *x, void *params);
  void *params;
} PnlRnFuncR ;
#define PNL_EVAL_RNFUNCR(F, x) (*((F)->function))(x, (F)->params)
\end{verbatim}

\describestruct{PnlRnFuncRm}
\describestruct{PnlRnFuncRn}
\begin{verbatim}
/*
 * f: R^n --> R^m
 * The function pointer computes the vector f(x) and stores it in
 * fx (vector of size m)
 *
typedef struct {
  void (*function) (const PnlVect *x, PnlVect *fx, void *params);
  void *params;
} PnlRnFuncRm ;
#define PNL_EVAL_RNFUNCRM(F, x, fx) (*((F)->function))(x, fx, (F)->params)

/*
 * Synonymous of PnlRnFuncRm for f:R^n --> R^n
 *
typedef PnlRnFuncRm PnlRnFuncRn;
#define PNL_EVAL_RNFUNCRN  PNL_EVAL_RNFUNCRM
\end{verbatim}

\describestruct{PnlRnFuncRmDFunc}
\describestruct{PnlRnFuncRnDFunc}
\begin{verbatim}
/*
 * f: R^n --> R^m
 * The function pointer computes the vector f(x) and stores it in fx
 * (vector of size m)
 * The Dfunction pointer computes the matrix Df(x) and stores it in dfx
 * (matrix of size m x n)
 *
typedef struct {
  void (*function) (const PnlVect *x, PnlVect *fx, void *params);
  void (*Dfunction) (const PnlVect *x, PnlMat *dfx, void *params);
  void *params;
} PnlRnFuncRmDFunc ;
#define PNL_EVAL_RNFUNCRM_DFUNC(F, x, dfx) (*((F)->Dfunction))(x, dfx, (F)->params)

/*
 * Synonymous of PnlRnFuncRmDFunc for f:R^n --> R^m
 *
typedef PnlRnFuncRmDFunc PnlRnFuncRnDFunc;
#define PNL_EVAL_RNFUNCRN_DFUNC PNL_EVAL_RNFUNCRM_DFUNC
\end{verbatim}

\subsection{Functions}

To use the following functions, you should include \verb!pnl/pnl_root.h!.

For finding the zero of a real valued function we provide the following
functions.
\begin{itemize}
\item \describefun{double}{pnl_root_brent}{\refstruct{PnlFunc}\ptr  F, double
    x1, double  x2, double \ptr tol}
  \sshortdescribe Find the root of \var{F} between \var{x1} and \var{x2} with
  an accuracy of order \var{tol}. On exit \var{tol} is an upper bound of the
  error.

\item \describefun{int}{pnl_find_root}{\refstruct{PnlFuncDFunc}\ptr  Func,
    double x_min, double x_max, double tol, int N_Max, double\ptr  res}
  \sshortdescribe Find the root of \var{F} between \var{x1} and \var{x2} with
  an accuracy of order \var{tol} and a maximum of \var{N_max} iterations. On
  exit, the root is stored in \var{res}. Note that the function \var{F} must
  also compute the first derivative of the function.


\item \describefun{int}{pnl_root_newton}{\refstruct{PnlFuncDFunc} \ptr Func,
    double x0, double epsrel, double epsabs, int N_max, double \ptr res}
  \sshortdescribe Find the root of \var{F} starting from \var{x0} with an
  accuracy given both by \var{epsrel} and \var{epsabs} and a maximum number of
  iterations \var{N_max}. On exit, the root is stored in \var{res}.Note that
  the function \var{F} must also compute the first derivative of the function.

\item \describefun{int}{pnl_root_bisection}{\refstruct{PnlFunc} \ptr Func,
    double xmin, double xmax, double epsrel, double espabs, int N_max, double
    \ptr res}
  \sshortdescribe Find the root of \var{F} between \var{x1} and \var{x2} with
  an accuracy given both by \var{epsrel} and \var{epsabs} and a maximum number
  of iterations \var{N_max}. On exit, the root is stored in \var{res}
\end{itemize}

Searching for the zero of a multivariate and vector valued function is a
complicated problem and we rely on minpack for doing this. Here, we provide
two wrappers for calling minpack routines.
\begin{itemize}
\item \describefun
  {int}{pnl_root_fsolve}{\refstruct{PnlRnFuncRnDFunc} \ptr f,
    \refstruct{PnlVect} \ptr x, \refstruct{PnlVect} \ptr fx, double xtol,
    int maxfev, int \ptr nfev, \refstruct{PnlVect} \ptr scale, int
    error_msg}
  \sshortdescribe Compute the root of a function $f:\R^n \longmapsto
  \R^n$. Note that the number of components of \var{f} must be equal to the
  number of variates of \var{f}. This function returns \var{OK} or
  \var{FAIL} if something went wrong.
  \parameters
  \begin{itemize}
  \item \var{f} is a pointer to a \refstruct{PnlRnFuncRnDFunc} used to
    store the function whose root is to be found. \var{f} can also
    store the Jacobian of the function, if not it is computed using
    finite differences (see the file \url{examples/minpack_test.c} for
    a usage example),
  \item  \var{x} contains on input the starting point of the search and
    an approximation of the root of \var{f} on output,
  \item \var{xtol} is the precision required on \var{x}, if set to 0 a
    default value is used.
  \item \var{maxfev} is the maximum number of evaluations of the function
    \var{f} before the algorithm returns, if set to 0, a coherent
    number is determined internally.
  \item \var{nfev} contains on output the number of evaluations of
    \var{f} during the algorithm,
  \item \var{scale} is a vector used to rescale \var{x} in a way that
    each coordinate of the solution is approximately of order 1 after
    rescaling. If on input \var{scale=NULL}, a scaling vector is
    computed internally by the algorithm.
  \item \var{error_msg} is a boolean
    (\var{TRUE} or \var{FALSE}) to specify if an error message should be
    printed when the algorithm stops before having converged.
  \item On output, \var{fx} contains \var{f(x)}.
  \end{itemize}

\item \describefun {int}{pnl_root_fsolve_lsq}{\refstruct{PnlRnFuncRmDFunc}
    \ptr f, \refstruct{PnlVect} \ptr x, int m, \refstruct{PnlVect} \ptr fx,
    double xtol, double ftol, double gtol, int maxfev, int \ptr nfev,
    \refstruct{PnlVect} \ptr scale, int error_msg}
  \sshortdescribe Compute the root of $x \in \R^n \longmapsto
  \sum_{i=1}^m f_i(x)^2$, note that there is no reason why \var{m} should
  be equal to \var{n}.
  \parameters
  \begin{itemize}
  \item \var{f} is a pointer to a \refstruct{PnlRnFuncRmDFunc} used to
    store the function whose root is to be found. \var{f} can also
    store the Jacobian of the function, if not it is computed using
    finite differences (see the file \url{examples/minpack_test.c} for
    a usage example),
  \item  \var{x} contains on input the starting
    point of the search and an approximation of the root of \var{f} on
    output,
  \item \var{m} is the number of components of \var{f},
  \item \var{xtol} is the precision required on \var{x}, if set to 0 a
    default value is used.
  \item \var{ftol} is the precision required on \var{f}, if set to 0 a
    default value is used.
  \item \var{gtol} is the precision required on the Jacobian of
    \var{f}, if set to 0 a default value is used.
  \item \var{maxfev} is the maximum number of evaluations of the function
    \var{f} before the algorithm returns, if set to 0, a coherent
    number is determined internally.
  \item \var{nfev} contains on output the number of evaluations of
    \var{f} during the algorithm,
  \item \var{scale} is a vector used to rescale \var{x} in a way that
    each coordinate of the solution is approximately of order 1 after
    rescaling.  If on input \var{scale=NULL}, a scaling vector is
    computed internally by the algorithm.
  \item \var{error_msg} is a boolean (\var{TRUE} or \var{FALSE}) to
    specify if an error message should be printed when the algorithm
    stops before having converged.
  \item On output, \var{fx} contains \var{f(x)}.
  \end{itemize}
\end{itemize}

% vim:spelllang=en:spell:



\section{Special functions}

The special function approximations are defined in the header \verb!pnl/pnl_specfun.h!.\\

Most of these functions rely on the {\it Cephes} library which uses its own
error mechanism which can be activated or deactivated using the two following
functions
\begin{itemize}
  \item \describefun{void}{pnl_deactivate_mtherr}{}
    \sshortdescribe Deactivate Cephes error mechanism
  \item \describefun{void}{pnl_activate_mtherr}{}
    \sshortdescribe Activate Cephes error mechanism
\end{itemize}


\subsection{Real Bessel functions}

\begin{itemize}
\item \describefun{double}{pnl_bessel_i}{double v, double x}
  \sshortdescribe   Modified Bessel function of the first
  kind of order \var{v}.
\item \describefun{double}{pnl_bessel_i_scaled}{double v, double x}
  \sshortdescribe   Modified Bessel function of the first
  kind of order \var{v} divided by $e^{|x|}$.
\item \describefun{double}{pnl_bessel_rati}{double v, double x}
  \sshortdescribe Ratio of modified Bessel functions of the first kind : $I_{v+1}(x) /
  I_v (x)$.
\item \describefun{double}{pnl_bessel_j}{double v, double x}
  \sshortdescribe    Bessel function of the first
  kind of order \var{v}.
\item \describefun{double}{pnl_bessel_j_scaled}{double v, double x}
  \sshortdescribe    Bessel function of the first
  kind of order \var{v}. Same function as \reffun{pnl_bessel_j}.
\item \describefun{double}{pnl_bessel_y}{double v, double x}
  \sshortdescribe   Modified Bessel function of the second
  kind of order \var{v}.
\item \describefun{double}{pnl_bessel_y_scaled}{double v, double x}
  \sshortdescribe   Modified Bessel function of the second
  kind of order \var{v}. Same function as \reffun{pnl_bessel_y}.
\item \describefun{double}{pnl_bessel_k}{double v, double x}
  \sshortdescribe   Bessel function of the third
  kind of order \var{v}.
\item \describefun{double}{pnl_bessel_k_scaled}{double v, double x}
  \sshortdescribe   Bessel function of the third
  kind of order \var{v} multiplied by $e^{x}$.
\item \describefun{dcomplex}{pnl_bessel_h1}{double v, double x}
  \sshortdescribe   Hankel function of the first kind of
  order \var{v}. 
\item \describefun{dcomplex}{pnl_bessel_h1_scaled}{double v, double x}
  \sshortdescribe  Hankel function of the first kind of order
  \var{v}  and divided by $e^{I x}$.
\item \describefun{dcomplex}{pnl_bessel_h2}{double v, double x}
  \sshortdescribe  Hankel function of the second kind of
  order \var{v}. 
\item \describefun{dcomplex}{pnl_bessel_h2_scaled}{double v, double x}
  \sshortdescribe  Hankel function of the second kind of
  order \var{v}  and multiplied by $e^{I x}$.
\end{itemize}

\subsection{Complex Bessel functions}

\begin{itemize}
\item \describefun{dcomplex}{pnl_complex_bessel_i}{double v, dcomplex z}
  \sshortdescribe  Complex Modified Bessel function of the first
  kind of order \var{v}.
\item \describefun{dcomplex}{pnl_complex_bessel_i_scaled}{double v, dcomplex z}
  \sshortdescribe  Complex Modified Bessel function of the first
  kind of order \var{v} divided by $e^{|Creal(z)|}$.
\item \describefun{dcomplex}{pnl_complex_bessel_rati}{double v, dcomplex x}
  \sshortdescribe Ratio of complex modified Bessel functions of the first kind : $I_{v+1}(x) /
  I_v (x)$.
\item \describefun{dcomplex}{pnl_complex_bessel_j}{double v, dcomplex z}
  \sshortdescribe  Complex  Bessel function of the first
  kind of order \var{v}.
\item \describefun{dcomplex}{pnl_complex_bessel_j_scaled}{double v, dcomplex z}
  \sshortdescribe  Complex  Bessel function of the first
  kind of order \var{v} divided by $e^{|Cimag(z)|}$.
\item \describefun{dcomplex}{pnl_complex_bessel_y}{double v, dcomplex z}
  \sshortdescribe  Complex Modified Bessel function of the second
  kind of order \var{v}.
\item \describefun{dcomplex}{pnl_complex_bessel_y_scaled}{double v, dcomplex z}
  \sshortdescribe  Complex Modified Bessel function of the second
  kind of order \var{v} divided by $e^{|Cimag(z)|}$.
\item \describefun{dcomplex}{pnl_complex_bessel_k}{double v, dcomplex z}
  \sshortdescribe  Complex Bessel function of the third
  kind of order \var{v}.
\item \describefun{dcomplex}{pnl_complex_bessel_k_scaled}{double v, dcomplex z}
  \sshortdescribe  Complex Bessel function of the third
  kind of order \var{v} multiplied by $e^{z}$.
\item \describefun{dcomplex}{pnl_complex_bessel_h1}{double v, dcomplex z}
  \sshortdescribe  Complex Hankel function of the first kind of
  order \var{v}. 
\item \describefun{dcomplex}{pnl_complex_bessel_h1_scaled}{double v, dcomplex z}
  \sshortdescribe  Complex  Hankel function of the first kind of order
  \var{v}  and divided by $e^{I z}$.
\item \describefun{dcomplex}{pnl_complex_bessel_h2}{double v, dcomplex z}
  \sshortdescribe  Complex  Hankel function of the second kind of
  order \var{v}. 
\item \describefun{dcomplex}{pnl_complex_bessel_h2_scaled}{double v, dcomplex z}
  \sshortdescribe  Complex  Hankel function of the second kind of
  order \var{v}  and multiplied by $e^{I z}$.
\end{itemize}

\subsection{Error functions}

\begin{itemize}
\item \describefun{double}{pnl_sf_erf}{double x}
  \sshortdescribe Compute the error function $\frac{2}{\pi}\int_0^\infty \expp{-t^2} dt$.
\item \describefun{dcomplex}{pnl_sf_complex_erf}{dcomplex z}
  \sshortdescribe Same as \reffun{pnl_sf_erf} for complex arguments.
\item \describefun{double}{pnl_sf_erfc}{double x}
  \sshortdescribe Compute the complementary error function \var{1. - erf(x)}.
\item \describefun{dcomplex}{pnl_sf_complex_erfc}{dcomplex x}
  \sshortdescribe Same as \reffun{pnl_sf_erfc} for complex arguments.
\item \describefun{double}{pnl_sf_erfcx}{double x}
  \sshortdescribe Compute the scaled complementary error function of x, defined by
  $\expp{x^2} \text{erfc}(x)$.
\item \describefun{dcomplex}{pnl_sf_complex_erfcx}{dcomplex z}
  \sshortdescribe Same as \reffun{pnl_sf_erfcx} for complex arguments. Note that
  \var{erfcx(-i x) = w(x)}.
\item \describefun{dcomplex}{pnl_sf_w}{dcomplex z}
  \sshortdescribe Compute $\expp{-z^2} \, \text{erfc}(-i \, z)$.
\item \describefun{double}{pnl_sf_w_im}{double x}
  \sshortdescribe Compute $2 \, \text{Dawson}(x)/\sqrt{\pi}$
\item \describefun{double}{pnl_sf_erfi}{double x}
  \sshortdescribe Compute \var{-i erf(i z)}
\item \describefun{dcomplex}{pnl_sf_complex_erfi}{dcomplex z}
  \sshortdescribe Same as \reffun{pnl_sf_erfi} for complex arguments.
\item \describefun{double}{pnl_sf_dawson}{double x}
  \sshortdescribe Compute $\sqrt{\pi}/2 \, \expp{-x^2} \, \text{erfi}(x)$.
\item \describefun{dcomplex}{pnl_sf_complex_dawson}{dcomplex z}
  \sshortdescribe Same as \reffun{pnl_sf_dawson} for complex arguments.
\item \describefun{double}{pnl_sf_log_erf}{double x}
  \sshortdescribe Compute $\log$ \reffun{pnl_sf_erf}$(x)$
\item \describefun{double}{pnl_sf_log_erfc}{double x}
  \sshortdescribe Compute $\log$ \reffun{pnl_sf_erfc}$(x)$
\end{itemize}

\subsection{Gamma functions}

For $x>0$, the Gamma Function is defined by
\begin{equation*}
  \Gamma(x)=\int_0^{\infty} \expp{-u} u^{x-1} du.
\end{equation*}

\begin{itemize}
\item \describefun{double}{pnl_sf_fact}{int n}
  \sshortdescribe   Computes factorial of \var{n} $ \Gamma (n+1)$.
\item \describefun{double}{pnl_sf_gamma}{double x}
  \sshortdescribe   Computes $\Gamma(x), x \geq 0$
\item \describefun{double}{pnl_sf_log_gamma}{double x}
  \sshortdescribe   Computes $\log(\Gamma(x)), x \geq 0$
\item \describefun{int}{pnl_sf_log_gamma_sgn}{double x, double \ptr y, int \ptr sgn}
  \sshortdescribe   Computes $y = \log(|\Gamma(x)|)$ for \var{x > 0} \var{sgn}
  contains the sign of $\Gamma(x)$ (-1 or +1).
\item \describefun{double}{pnl_sf_choose}{int n, int k}
  \sshortdescribe Computes the binomial coefficient $\binom{n}{k} = 
  \frac{n!}{k! (n-k)!}$ for $0 \le k \le n$ in double precision.
\end{itemize}

\subsection{Digamma function}

For $x>0$, the digamma function $\psi$ is defined as the logarithmic derivative of the
Gamma function $\Gamma$
\begin{equation*}
  \psi(x) = \frac{d}{dx} \log \Gamma (x) = \frac{\Gamma'(x)}{\Gamma(x)}.
\end{equation*}
The function $\psi$ admits the following integral representation
\begin{equation*}
  \psi (x) = \int_{0}^{\infty} \left( \frac{\expp{-u}}{u} - \frac{\expp{-xu}}{1
  - \expp{-u}} \right).
\end{equation*}

\begin{itemize}
  \item \describefun{double}{pnl_sf_psi}{double x}
    \sshortdescribe Return $\psi(x)$.
\end{itemize}

\subsection{Incomplete Gamma functions}

For $a \in \R$ and $x>0$, the Incomplete Gamma Function is defined by
\begin{equation*}
  \Gamma(a, x)=\int_x^{\infty} \expp{-u} u^{a-1} du.
\end{equation*}
A relation similar to the one existing for the standard Gamma function holds
\begin{equation*}
  \Gamma\paren{a, x}= \frac{- x^{a} \expp{-x} + \Gamma (a+1, x)}{a}.
\end{equation*}
\begin{align*}
  \Gamma(a)&=\int_0^{\infty} u^{a-1} \expp{-u}du\\ 
  P(a, x) &= \frac{\Gamma(a) - \Gamma(a, x)}{\Gamma(a)} =
  \frac{1}{\Gamma(a)} \int_0^x u^{a-1} \expp{-u}  du\\ 
  Q(a, x) &= 1-P(a, x) =\frac{\Gamma(a, x)}{\Gamma(a)} =
  \frac{1}{\Gamma(a)} \int_x^{\infty} \expp{-u} u^{a-1} du. 
\end{align*}

\begin{itemize}
\item \describefun{double}{pnl_sf_gamma_inc}{double a, double x}
  \sshortdescribe   Computes $\Gamma(a, x), \quad a \in \R , x \geq 0$
\item \describefun{void}{pnl_sf_gamma_inc_P}{double a, double x}
  \sshortdescribe  Computes $P(a, x), \quad a > 0 , x \geq 0$
\item \describefun{void}{pnl_sf_gamma_inc_Q}{double a, double x}
  \sshortdescribe  Computes $Q(a, x), \quad a > 0 , x \geq 0$
\end{itemize}

\subsection{Exponential integrals}
For $x>0$ and $n \in \N$, the  function $E_n$ is defined by
\begin{equation*}
  E_n\paren{x}=\int_{1}^{\infty} \expp{-x u} u^{-n} du
\end{equation*}

This function is linked to the Incomplete Gamma function by 
\begin{equation*}
  E_n\paren{x}=\int_{x}^{\infty}
  \expp{-xu} (xu)^{-n} x^{n-1} d(xu)=x^{n-1} \int_{x}^{\infty}
  \expp{-t} t^{-n} dt =  x^{n-1}  \Gamma\paren{1-n, x}, 
\end{equation*}
from which we can deduce
\begin{equation*}
  n E_{n+1}(x)  =   \expp{-x} - x E_n(x).
\end{equation*}
For $n>1$, the series expansion is given by
\begin{equation*}
  E_n(x)=x^{n-1}
  \Gamma(1-n)+\recaco{-\frac{1}{1-n}+\frac{x}{2-n}-\frac{x^2}{2(3-n)}
    +\frac{x^3}{6(4-n)}-\dots}.     
\end{equation*}
The asymptotic behaviour is given by
\begin{equation*}
  E_n(x)=\frac{\expp{-x}}{x}\recaco{1-\frac{n}{x}+\frac{n(n+1)}{x^2}+\dots}. 
\end{equation*}
The special case $n=1$ gives 
\begin{equation*}
  E_1(x) = \int_x^\infty \frac{e^{-u}}{u}\, du, \quad |\mathrm{Arg}(x)| \ge \pi. 
\end{equation*}
For any complex number $x$ with positive real part, this can be written
\begin{equation*}
  E_1(x) = \int_1^\infty \frac{e^{-ux}}{u}\, du, \quad \Re(x) \ge 0. 
\end{equation*}
By integrating the Taylor expansion of $\expp{-t}/t$, and extracting the
logarithmic singularity, we can derive the following series representation for
$E_1(x)$, 
\begin{equation*}
  {E_1}(x) =-\gamma-\ln x-\sum_{k=1}^{\infty}\frac{(-1)^k x^k}{k\; k!}
  \qquad |\mathrm{Arg}(x)| < \pi. 
\end{equation*}
The function $E_1$ is linked to the exponential integral $Ei$
\begin{equation*}
  Ei(x)=\int_{-\infty}^x\frac{e^u}u\, du=-\int_{-x}^{\infty}
  \frac{e^{-u}}{u}\, du \quad \forall x \neq 0. 
\end{equation*}
The above definition  can be used for positive values of $x$, but the integral
has to be understood in terms of its Cauchy principal value, due to the
singularity of the integrand at zero.
\begin{equation*}
  {Ei}(-x) = -{E}_1(x) , \quad \Re(x) \ge 0.
\end{equation*}
We deduce, 
\begin{equation*}
  Ei(x) = \gamma + \ln x+ \sum_{k=1}^{\infty} \frac{x^k}{k\; k!}, \quad x>0.
\end{equation*}
For $x \in \R$ 
\begin{equation*}
  \Gamma(0, x)=\left\{
    \begin{array}{l}
      -Ei(-x)-\imath \pi  \quad  x<0, \\
      -Ei(-x) \quad x>0. 
    \end{array}\right.
\end{equation*}

\begin{itemize}
\item \describefun{double}{pnl_sf_expint_En}{int n, double x}
  \sshortdescribe   Computes \var{E_n(x)} for  $ n\geq 0, x \geq 0$, or $x>0$
  when $n=0$ or $1$.
\end{itemize}

\subsection{Hypergeometric functions}

\begin{itemize}
\item \describefun{double}{pnl_sf_hyperg_2F1}{double a, double b, double c,
    double x}
  \sshortdescribe Compute the Gauss hypergeometric function \var{2F1(a,b,c,x)}
  for \var{|x| < 1} and for \var{x < -1} when \var{b,a,c,(b-a),(c-a),(c-b)} are
  not integers
\item \describefun{double}{pnl_sf_hyperg_1F1}{double a, double b, double x}
  \sshortdescribe Compute the hypergeometric function \var{1F1(a,b,x)}
\item \describefun{double}{pnl_sf_hyperg_2F0}{double a, double b, double x}
  \sshortdescribe Compute the hypergeometric function \var{2F0(a,b,x)}  for
  \var{x<0} using the relation $2F0 (a,b,x) = (-x)^{-a} U(a,1+a-b,-\frac{1}{x})$.
\item \describefun{double}{pnl_sf_hyperg_0F1}{double c, double x}
  \sshortdescribe Compute the hypergeometric function \var{0F1(c,x)}
\item \describefun{double}{pnl_sf_hyperg_U}{double a, double b, double x}
  \sshortdescribe Compute the confluent hypergeometric function \var{U(a,b,x)}
  with \var{x > 0}
\end{itemize}


% vim:spelllang=en:spell:


\section{Some bindings}

% --------------------------------------------------------------------- %%
% MPI
\subsection{MPI bindings}
\subsubsection{Short Description}

We provide some bindings for the MPI library to natively handle {\it PnlObjects}.

The functionnalities described in this chapter are declared in \verb!pnl/pnl_mpi.h!.

\subsubsection{Functions}

All the following functions return an error code as an integer value. This
returned value should be tested against \var{MPI_SUCCESS} to check that no
error occurred.

\begin{itemize}
\item \describefun{int}{pnl_object_mpi_pack_size}{const
    \refstruct{PnlObject} \ptr Obj, MPI_Comm comm, int \ptr size}
  \sshortdescribe Computes in \var{size} the amount of space needed to pack \var{Obj}.
\item \describefun{int}{pnl_object_mpi_pack}{const \refstruct{PnlObject}
    \ptr Obj, void \ptr buf, int bufsize, int \ptr pos, MPI_Comm comm}
  \sshortdescribe Packs \var{Obj} into \var{buf} which must be at least of
  length \var{size}. \var{size} must be at least equal to the value returned
  by \reffun{pnl_object_mpi_pack_size}. 
\item \describefun{int}{pnl_object_mpi_unpack}{\refstruct{PnlObject} \ptr
    Obj, void \ptr buf, int bufsize, int \ptr pos, MPI_Comm comm}
  \sshortdescribe Unpacks the content of \var{buf} starting at position
  \var{pos} (unless several objects have been packed contiguously, \var{\ptr
    pos} should be equal to \var{0}). \var{buf} is a contigous memery area
  of length \var{bufsize} (note that the size is counted in bytes).
  \var{pos} is incremented and is on output the first location in the input
  buffer after the locations occupied by the message that was
  unpacked. \var{pos} is properly set for a future call to {\it MPI_Unpack}
  if any.
  
\item \describefun{int}{pnl_object_mpi_send}{const \refstruct{PnlObject}
    \ptr Obj, int dest, int tag, MPI_Comm comm}
  \sshortdescribe Performs a standard-mode blocking send of \var{Obj}. The
  object is sent to the process with rank \var{dest}.

\item \describefun{int}{pnl_object_mpi_ssend}{const \refstruct{PnlObject} \ptr
    Obj, int dest, int tag, MPI_Comm comm}
  \sshortdescribe Performs a standard-mode synchronous blocking send of
  \var{Obj}. The object is sent to the process with rank \var{dest}.
  
\item \describefun{int}{pnl_object_mpi_recv}{\refstruct{PnlObject} \ptr Obj,
    int src, int tag, MPI_Comm comm, MPI_Status \ptr status} 
  \sshortdescribe Performs a standard-mode blocking receive of \var{Obj}. The
  object is sent to the process with rank \var{dest}. Note that \var{Obj}
  should be an already allocated object and that its type should match the
  true type of the object to be received. \var{src} is the rank of the
  process who sent the object.

  
\item \describefun{int}{pnl_object_mpi_bcast}{\refstruct{PnlObject} \ptr
    Obj, int root, MPI_Comm comm}
  \sshortdescribe Broadcasts the object \var{Obj} from the process with rank
  \var{root} to all other processes of the group \var{comm}.
\end{itemize}

For more expect users, we provide the following nonblocking functions.
\begin{itemize}
\item \describefun{int}{pnl_object_mpi_isend}{const \refstruct{PnlObject}
    \ptr Obj, int dest, int tag, MPI_Comm comm, MPI_Request \ptr request}
  \sshortdescribe Starts a standard-mode, nonblocking send of object
  \var{Obj} to the process with rank \var{dest}.
  
  
\item \describefun{int}{pnl_object_mpi_irecv}{void \ptr \ptr buf, int \ptr
    size, int src, int tag, MPI_Comm comm, int \ptr flag, MPI_Request \ptr
    request}
  \sshortdescribe Starts a standard-mode, nonblocking receive of object
  \var{Obj} from the process with rank \var{root}. On output \var{flag} equals
  to \var{TRUE} if the object can be received and \var{FALSE} otherwise (this
  is the same as for {\it MPI_Iprobe}).
\end{itemize}

\subsection{The save/load interface}

The interface is only accessible when the MPI bindings are compiled since it
is based on the Packing/Unpacking facilities of MPI.

The functionnalities described in this chapter are declared in \verb!pnl/pnl_mpi.h!.
\begin{itemize}
\item \describefun{\refstruct{PnlRng}\ptr \ptr}
  {pnl_rng_create_from_file}{char \ptr str, int n}
  \sshortdescribe Loads \var{n} rng from the file of name \var{str} and
  returns an array of \var{n} \refstruct{PnlRng}.
\item \describefun{int}{pnl_rng_save_to_file}{\refstruct{PnlRng} \ptr \ptr
    rngtab, int n, char \ptr str}
    \sshortdescribe Saves \var{n} rng stored in \var{rngtab} into the file of
  name \var{str}.
\item \describefun{int}{pnl_object_save}{PnlObject \ptr O, FILE *stream}
  \sshortdescribe Saves the object \var{O} into the stream \var{stream}. \var{stream}
  is typically created by calling fopen with \var{mode="wb"}. This function can be
  called several times to save several objects in the same stream.
\item \describefun{PnlObject\ptr}{pnl_object_load}{FILE *stream}
  \sshortdescribe Loads an object from the stream \var{stream}. \var{stream}
  is typically created by calling fopen with \var{mode="rb"}.  This function can be
  called several times to load several objects from the same stream. If \var{stream}
  was empty or it did not contain any PnlObjects, the function returns \var{NULL}.
\item \describefun{PnlList\ptr}{pnl_object_load_into_list}{FILE *stream}
  \sshortdescribe Loads as many objects as possible from the stream \var{stream} and
  stores them into a \refstruct{PnlList}. \var{stream} is typically created by
  calling fopen with \var{mode="rb"}. If \var{stream} was empty or it did not contain
  any PnlObjects, the function returns \var{NULL}.
\end{itemize}


%%% Local Variables: 
%%% mode: latex
%%% TeX-master: "pnl-manual"
%%% End: 



\section{Financial functions}

The financial functions are defined in the header \verb!pnl/pnl_finance.h!.\\

\begin{itemize}
\item
  \describefun{int}{pnl_cf_call_bs}{double s, double k, double T, double r, 
    double divid, double sigma, double \ptr ptprice, double \ptr ptdelta}
  \sshortdescribe Compute the price and delta of a call option $(\var{s} -
  \var{k})_+$ in the Black-Scholes model with volatility \var{sigma},
  instantaneous interest rate \var{r}, maturity \var{T} and dividend rate
  \var{divid}. The parameters \var{ptprice} and \var{ptdelta} are respectively
  set to the price and delta on output.

\item
  \describefun{int}{pnl_cf_put_bs}{double s, double k, double T, double r, 
    double divid, double sigma, double \ptr ptprice, double \ptr ptdelta}
  \sshortdescribe Compute the price and delta of a put option $(\var{k} - 
  \var{s})_+$ in the Black-Scholes model with volatility \var{sigma},
  instantaneous interest rate \var{r}, maturity \var{T} and dividend rate
  \var{divid}.  The parameters \var{ptprice} and \var{ptdelta} are respectively
  set to the price and delta on output.

\end{itemize}


\begin{itemize}
\item 
  \describefun{double}{pnl_bs_call}{double s, double k, double T,
    double r, double divid, double sigma}
  \sshortdescribe Compute the price of a call option with spot \var{s}
  and strike \var{k} in the Black-Scholes model with volatility \var{sigma},
  instantaneous interest rate \var{r}, maturity \var{T} and dividend rate
  \var{divid}.

\item 
  \describefun{double}{pnl_bs_put}{double s, double k, double T,
    double r, double divid, double sigma}
  \sshortdescribe Compute the price a put option with spot \var{s}
  and strike \var{k} in the Black-Scholes model with volatility \var{sigma},
  instantaneous interest rate \var{r}, maturity \var{T} and dividend rate
  \var{divid}.

\item 
  \describefun{double}{pnl_bs_call_put}{int iscall, double s, double k,
    double T, double r, double divid, double sigma}
  \sshortdescribe Compute the price of a put option if \var{iscall=0} or a
  call option if \var{iscall=1} with spot \var{s} and strike \var{k} in the
  Black-Scholes model with volatility \var{sigma}, instantaneous interest rate
  \var{r}, maturity \var{T} and dividend rate \var{divid}.

\item 
  \describefun{double}{pnl_bs_vega}{double s, double k, double T,
    double r, double divid, double sigma}
  \sshortdescribe Compute the vega of a put or call option with spot \var{s}
  and strike \var{k} in the Black-Scholes model with volatility \var{sigma},
  instantaneous interest rate \var{r}, maturity \var{T} and dividend rate
  \var{divid}.

\item 
  \describefun{double}{pnl_bs_gamma}{double s, double k, double T, double r,
    double divid, double sigma}
  \sshortdescribe Compute the gamma of a put or call option with spot \var{s}
  and strike \var{k} in the Black-Scholes model with volatility \var{sigma},
  instantaneous interest rate \var{r}, maturity \var{T} and dividend rate
  \var{divid}.
\end{itemize}

Practitioners do not speak in terms of option prices, but rather compare
prices in terms of their implied Black \& Scholes volatilities. So this
parameter is very useful in practice. Here, we propose two functions to
compute $\sigma_{impl}$ : the first one is for one up-let, maturity,
strike, option price.  the second function is for a list of strikes and
maturities, a matrix of prices (with strikes varying row-wise).

\begin{itemize}
\item 
  \describefun{double}{pnl_bs_implicit_vol}{int is_call, double Price, double s,
    double K, double T, double r, double divid, int *error}
  \sshortdescribe Compute the implied volatility of a put option if
  \var{iscall=0} or a call option if \var{iscall=1} with spot \var{s} and
  strike \var{K} in the Black-Scholes model with instantaneous interest rate
  \var{r}, maturity \var{T} and dividend rate \var{divid}. On output
  \var{error} is \var{OK} if the computation of the implied volatility succeeded
  or \var{FAIL} if it failed.

\item 
  \describefun{int}{pnl_bs_matrix_implicit_vol}{const \PnlMatInt
    \ptr iscall, const \PnlMat \ptr Price, double s, 
    double r, double divid, const \PnlVect \ptr K, 
    const \PnlVect \ptr T, \PnlMat \ptr Vol}
  \sshortdescribe Compute the matrix of implied volatilities \var{Vol(i,j)}
  of a put option if \var{iscall(i,j)=0} or a call option if
  \var{iscall(i,j)=1} with spot \var{s} and strike \var{K(j)} in the
  Black-Scholes model with instantaneous interest rate \var{r}, maturity
  \var{T(j)} and dividend rate \var{divid}. This function returns the number
  of failures, when everything succeeded it returns $0$.
\end{itemize}

% vim:spelllang=en:spell:


\clearpage
\addcontentsline{toc}{section}{Index}
\printindex


\end{document}
