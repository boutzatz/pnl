\section{Cumulative distribution Functions}

The functions related to this chapter are declared in \verb!pnl/pnl_cdf.h!.

For various distribution functions, we provide functions named
\var{pnl_cdf_xxx} where \var{xxx} is the abbreviation of the distribution
name. All these functions are based on the same prototype
\begin{equation*}
  p = 1-q; \quad p = \int^x density(u) du
\end{equation*}

\begin{itemize}
\item \var{which} If \var{which=1}, it computes \var{p} and \var{q}. If
  \var{which=2}, it computes \var{x}. For higher values of \var{which} it
  computes one the parameters characterizing the distribution using all the
  others, \var{p, q, x}.
\item \var{p} the probability $\int^x density(u) du $
\item \var{q} $= 1 - p$
\item \var{x} the upper bound of the integral
\item \var{status} an integer which indicates on exit the success of the
  computation. (0) if calculation completed correctly. (-I) if the input
  parameter number I was out of range. (1) if the answer appears to be lower
  than the lowest search bound.  (2) if the answer appears to be higher than
  the greatest search bound.  (3) if $p + q \ne 1$.
\item \var{bound} is undefined if STATUS is 0.  Bound exceeded by parameter
  number I if STATUS is negative. Lower search bound if STATUS is 1.  Upper
  search bound if STATUS is 2.
\end{itemize}

\begin{itemize}
\item \describefun{void}{pnl_cdf_bet}{int \ptr which, double \ptr p, double
    \ptr q, double \ptr x, double \ptr y, double \ptr a, double \ptr b,
    int \ptr status, double \ptr bound}
  \sshortdescribe Cumulative Distribution Function BETA distribution.

\item \describefun{void}{pnl_cdf_bin}{int \ptr which, double \ptr p, double
    \ptr q, double \ptr x, double \ptr xn, double \ptr pr, double
    \ptr ompr, int \ptr status, double \ptr bound}
  \sshortdescribe Cumulative Distribution Function BINa distribution.

\item \describefun{void}{pnl_cdf_chi}{int \ptr which, double \ptr p, double
    \ptr q, double \ptr x, double \ptr df, int \ptr status, double
    \ptr bound}
  \sshortdescribe Cumulative Distribution Function CHI-Square distribution.

\item \describefun{void}{pnl_cdf_chn}{int \ptr which, double \ptr p, double
    \ptr q, double \ptr x, double \ptr df, double \ptr pnonc, int
    \ptr status, double \ptr bound}
  \sshortdescribe Cumulative Distribution Function Non-central Chi-Square distribution.

\item \describefun{void}{pnl_cdf_f}{int \ptr which, double \ptr p, double
    \ptr q, double \ptr x, double \ptr dfn, double \ptr dfd, int
    \ptr status, double \ptr bound}
  \sshortdescribe Cumulative Distribution Function F distribution.

\item \describefun{void}{pnl_cdf_fnc}{int \ptr which, double \ptr p, double
    \ptr q, double \ptr x, double \ptr dfn, double \ptr dfd, double
    \ptr pnonc, int \ptr status, double \ptr bound}
  \sshortdescribe Cumulative Distribution Function Non-central F distribution.

\item \describefun{void}{pnl_cdf_gam}{int \ptr which, double \ptr p, double
    \ptr q, double \ptr x, double \ptr shape, double \ptr scale, int
    \ptr status, double \ptr bound}
  \sshortdescribe Cumulative Distribution Function GAMma distribution.

\item \describefun{void}{pnl_cdf_nbn}{int \ptr which, double \ptr p, double
    \ptr q, double \ptr x, double \ptr xn, double \ptr pr, double
    \ptr ompr, int \ptr status, double \ptr bound}
  \sshortdescribe Cumulative Distribution Function Negative BiNomial distribution.

\item \describefun{void}{pnl_cdf_nor}{int \ptr which, double \ptr p, double
    \ptr q, double \ptr x, double \ptr mean, double \ptr sd, int
    \ptr status, double \ptr bound}
  \sshortdescribe Cumulative Distribution Function NORmal distribution.

\item \describefun{void}{pnl_cdf_poi}{int \ptr which, double \ptr p, double
    \ptr q, double \ptr x, double \ptr xlam, int \ptr status, double
    \ptr bound}
  \sshortdescribe Cumulative Distribution Function POIsson distribution.

\item \describefun{void}{pnl_cdf_t}{int \ptr which, double \ptr p, double
    \ptr q, double \ptr x, double \ptr df, int \ptr status, double
    \ptr bound}
  \sshortdescribe Cumulative Distribution Function T distribution.

\item \describefun{double}{pnl_cdfchi2n}{double x, double df, double ncparam}
  \sshortdescribe Computes the cumulative density function at \var{x} of the
  non central $\chi^2$ distribution with \var{df} degrees of freedom and non
  centrality parameter \var{ncparam}.

\item \describefun{void}{pnl_cdfbchi2n}{double x, double df, double ncparam, double
    beta, double \ptr P}
  \sshortdescribe Stores in \var{P} the cumulative density function at \var{x}
  of the random variable \var{beta \ptr X} where \var{X} is non central $\chi^2$
  random variable with \var{df} degrees of freedom and non
  centrality parameter \var{ncparam}.


\item \describefun{double}{pnl_normal_density}{double x}
  \sshortdescribe Normal density function.

\item \describefun{double}{pnl_cdfnor}{double x}
  \sshortdescribe Cumulative normal distribution function.

\item \describefun{double}{pnl_cdf2nor}{double a, double b, double r}
  \sshortdescribe Cumulative bivariate normal distribution function, returns
  $\frac{1} {2\pi \sqrt{1-r^2}} \int_{- \infty}^a\int_{- \infty}^b e^{-
    \frac{x^2 - 2 r xy+y^2} {2(1-r^2)} } dxdy.$

\item \describefun{double}{pnl_inv_cdfnor}{double x}
  \sshortdescribe Inverse of the cumulative normal distribution function.
\end{itemize}


\section{Random Number Generators}

The functionnalities described in this chapter are declared in
\verb!pnl/pnl_random.h!.

Random number generators can be accessed using two different interfaces : the
new {\em rng} interface based on the \refstruct{PnlRng} object and the older
{\em rand} interface. We will keep the {\em rand} interface for compatibility
issues but we highly encourage to use the new {\rm rng} interface which is far
more flexible and uses reentrant functions suitable for multi-threaded
applications.

\begin{table}[h!]
  \begin{tabular}{l|l|l}
    Random generator & index & Type \\
    \hline
    KNUTH & PNL_RNG_KNUTH & pseudo\\
    MRGK3 & PNL_RNG_MRGK3 & pseudo\\
    MRGK5 & PNL_RNG_MRGK5 & pseudo\\
    SHUFL & PNL_RNG_SHUFL & pseudo\\
    L'ECUYER & PNL_RNG_L_ECUYER & pseudo\\
    TAUSWORTHE & PNL_RNG_TAUSWORTHE & pseudo\\
    MERSENNE & PNL_RNG_MERSENNE & pseudo\\
    MERSENNE (Random Seed) & PNL_RNG_MERSENNE_RANDOM_SEED & pseudo\\
    SQRT & PNL_RNG_SQRT & quasi\\
    HALTON & PNL_RNG_HALTON & quasi\\
    FAURE & PNL_RNG_FAURE & quasi\\
    SOBOL & PNL_RNG_SOBOL & quasi\\
    SOBOL2 & PNL_RNG_SOBOL2 & quasi\\
    NIEDERREITER & PNL_RNG_NIEDERREITER & quasi
  \end{tabular}
  \caption{Indices of the random generators}
  \label{rng-indices}
\end{table}


\subsection{The rng interface}
\label{rng-int}

It is possible to create random number generators each with its own state
variable so that they can evolve independently in a shared memory
environment. These generators are suitable for use in multi-threaded programms. For the
moment, only pseudo random number generators can be created. The quasi random
number generators can only be accessed using the older {\em rand} interface
(see Section~\ref{rand-int})

\begin{verbatim}
typedef struct _PnlRng PnlRng;
struct _PnlRng
{
  PnlObject object;
  int type; /*!< generator type *
  void (*Compute)(PnlRng *g, double *sample); /*!< the function to compute the
                                                next number in the sequence *
  int rand_or_quasi; /*!< can be MC or QMC *
  int dimension; /*!< dimension of the space in which we draw the samples *
  int counter; /*!< counter = number of samples already drawn *
  int has_gauss; /*!< Is a gaussian deviate available? *
  double gauss; /*!< If has_gauss==1, gauss a gaussian sample *
  int size_state; /*!< size in bytes of the state variable *
  void *state; /*!< state of the random generator *
};
\end{verbatim}


\begin{itemize}
\item \describefun{void}{pnl_rng_free}{\refstruct{PnlRng} \ptr \ptr }
  \sshortdescribe Frees a \refstruct{PnlRng}.
\item \describefun{\refstruct{PnlRng}\ptr }{pnl_rng_create}{int type}
  \sshortdescribe Creates a \refstruct{PnlRng} corresponding to \var{type}
  which can be any of the values \var{PNL_RNG_XXX} listed in
  Table~\ref{rng-indices} which correspond to {\bf pseudo} random number generators.
  Once a generator has been created, you {\bf must} call
  \reffun{pnl_rng_sseed} before using it. To create several PNL_RNG_DCMT type
  generators it is better to use \reffun{pnl_rng_dcmt_create_array} because the
  function \reffun{pnl_rng_create} produces a memory leak.
\item \describefun{\refstruct{PnlRng} \ptr \ptr}{pnl_rng_dcmt_create_array}
  {int n, ulong seed, int \ptr count}
  \sshortdescribe Creates an array of \var{n} independent DCMT. \var{seed} is
  the seed used to initialize the Mersenne Twister generator internally used to
  find new DCMT. On exit, \var{count} contains the number of generators actually
  created. Same function as \reffun{pnl_dcmt_create_array} instead that it
  directly returns an array of \refstruct{PnlRng}. Before using the generators, you
  must initialize each of them by calling the function \reffun{pnl_rng_sseed}
  \var{count} times.
\item \describefun{void}{pnl_rng_sseed}{\refstruct{PnlRng} \ptr rng, unsigned
    long int s}
  \sshortdescribe Sets the seed of the genrator \var{rng} using \var{s}. If
  \var{s=0}, then a default seed (depending on the generator) is used.
\item \describefun{void}{pnl_rng_sdim}{\refstruct{PnlRng} \ptr rng, int dim}
  \sshortdescribe Sets the dimension of the state space for a QMC generator.
\end{itemize}

Some auxiliary functions internally used but we should only be used with high
caution.
\begin{itemize}
\item \describefun{\refstruct{PnlRng}\ptr }{pnl_rng_new}{}
  \sshortdescribe Creates an empty \refstruct{PnlRng}.
\item \describefun{void}{pnl_rng_init}{\refstruct{PnlRng} \ptr rng, int type}
  \sshortdescribe Initializes an empty \refstruct{PnlRng} as returned by
  \reffun{pnl_rng_new} to become a generator of type \var{type} which can be
  any of the values \var{PNL_RNG_XXX} listed in Table~\ref{rng-indices} which
  correspond to {\bf pseudo} random number generators.
  Calling \reffun{pnl_rng_create} is equivalent to calling first
  \reffun{pnl_rng_new} and then \reffun{pnl_rng_init}. 
\item \describefun{\refstruct{PnlRng}\ptr }{pnl_rng_get_from_id}
  {\refstruct{PnlRngType} id}
  \sshortdescribe Retuns the global generator described by its macro name.
\end{itemize}


The following functions return one sample from a specified law.
\begin{itemize}
\item \describefun{int}{pnl_rng_bernoulli}{double p, \refstruct{PnlRng} \ptr rng}
  \sshortdescribe Generates a sample from the Bernouilli law on $\{0, 1\}$ with
  parameter \var{p}.

\item \describefun{long}{pnl_rng_poisson}{double lambda, \refstruct{PnlRng} \ptr rng}
  \sshortdescribe Generates a sample from the Poisson law with
  parameter \var{lambda}.

\item \describefun{double}{pnl_rng_exp}{double lambda, \refstruct{PnlRng} \ptr rng}
  \sshortdescribe Generates a sample from the Exponential law with
  parameter \var{lambda}.

\item \describefun{double}{pnl_rng_uni} {\refstruct{PnlRng} \ptr rng}
  \sshortdescribe Generates a sample from the Uniform law on $[0, 1]$.

\item \describefun{double}{pnl_rng_uni_ab} {double a, double b,
    \refstruct{PnlRng} \ptr rng}
  \sshortdescribe Generates a sample from the Uniform law on $[a, b]$.

\item \describefun{double}{pnl_rng_normal} {\refstruct{PnlRng} \ptr rng}
  \sshortdescribe Generates a sample from the standard normal distribution.

\item \describefun{long}{pnl_rng_poisson1}{double lambda, double
    t,\refstruct{PnlRng} \ptr rng}
  \sshortdescribe Generates a sample from a Poisson process with intensity
  \var{lambda} at time \var{t}.

\item \describefun{double}{pnl_rng_gamma} {double a, double b, \refstruct{PnlRng} \ptr rng}
  \sshortdescribe Generates a sample from the $\Gamma(a, b)$ distribution.

\item \describefun{double}{pnl_rng_chi2} {double n, \refstruct{PnlRng} \ptr rng}
  \sshortdescribe Generates a sample from the centered $\chi^2(n)$ distribution.
\item \describefun{double}{pnl_rng_bessel} {double v, double a,\refstruct{PnlRng} \ptr rng}
  \sshortdescribe Generates a sample from the Bessel distribution with parameters
  \var{v > -1} and \var{a > 0}.
\item \describefun{double}{pnl_rng_gauss}{int d, int
    create_or_retrieve, int index, PnlRng \ptr rng}
  \sshortdescribe The second argument can be either \var{CREATE} (to actually
  draw the sample) or \var{RETRIEVE} (to retrieve that element of index
  \var{index}). With \var{CREATE}, it draws \var{d} random normal variables
  and stores them for future usage. They can be withdrawn using \var{RETRIEVE}
  with the index of the number to be retrieved.
\end{itemize}

The following functions take an already existing \refstruct{PnlVect}\ptr  as
its first argument and fill each entry of the vector with a sample from the
specified law. All the entries are independent. The difference between
$n-$samples from a distribution in dimension $1$, and one sample from the same
distribution in dimension $n$ only matters when using a {\bf Quasi} random
number generator.
\begin{itemize}
\item \describefun{void}{pnl_vect_rng_uni}{\refstruct{PnlVect} \ptr G, int
    samples, double a, double b, \refstruct{PnlRng} \ptr rng}
  \sshortdescribe \var{G} is a vector of independent and identically distributed
  samples from the uniform distribution on $[a, b]$.

\item \describefun{void}{pnl_vect_rng_normal}{\refstruct{PnlVect} \ptr G,
    int samples, \refstruct{PnlRng} \ptr rng}
  \sshortdescribe \var{G} is a vector of independent and identically distributed
  samples from the standard normal distribution.

\item \describefun{void}{pnl_vect_rng_uni_d}{\refstruct{PnlVect} \ptr G, int
    d, double a, double b, \refstruct{PnlRng} \ptr rng}
  \sshortdescribe \var{G} is a sample from the uniform distribution on $[a,
  b]^{\text{d}}$.

\item \describefun{void}{pnl_vect_rng_normal_d}{\refstruct{PnlVect} \ptr G,
    int d, \refstruct{PnlRng} \ptr rng}
  \sshortdescribe \var{G} is a sample from the \var{d}-dimensional
  standard normal distribution.

\end{itemize}

The following functions take an already existing \refstruct{PnlMat}\ptr  as
first argument and fill each entry of the vector with a sample from the
specified law. All the entries are in-dependant. On return, the matrix \var{M}
is of size \verb!samples x dimension!. The rows of \var{M} are independently
and identically distributed. Each row is a sample from the given law in
dimension \var{dimension}.
\begin{itemize}
\item \describefun{void}{pnl_mat_rng_uni}{\refstruct{PnlMat} \ptr M, int
    samples, int d, const PnlVect \ptr a, const PnlVect \ptr b,
    \refstruct{PnlRng} \ptr rng}
  \sshortdescribe \var{M} contains \var{samples} samples from the uniform
  distribution on $\prod_{i=1}^d [a_i, b_i]$.

\item \describefun{void}{pnl_mat_rng_uni2}{\refstruct{PnlMat} \ptr M, int
    samples, int d, double a, double b, \refstruct{PnlRng} \ptr rng}
  \sshortdescribe \var{M} contains \var{samples} samples from the uniform
  distribution on $[a, b]^{\text{d}}$.

\item \describefun{void}{pnl_mat_rng_normal}{\refstruct{PnlMat} \ptr M, int
    samples, int d, \refstruct{PnlRng} \ptr rng}
  \sshortdescribe \var{M} contains \var{samples} samples from the
  \var{d}-dimensional standard normal distribution.
\end{itemize}

Some examples
\example
\begin{verbatim}
PnlRng *rng = pnl_rng_create (PNL_RNG_MERSENNE);
PnlVect *v = pnl_vect_new ();

/* rng must be initialized. When sseed=0, a default value depending on the
generator is used */
rng = pnl_rng_sseed (0);
/* simulates a normal random vector in R^{10} */
pnl_vect_rng_normal (v, 10, rng);

pnl_vect_free (&v);
pnl_rng_free (&rng);
\end{verbatim}



\subsection{The {\em rand} interface}
\label{rand-int}

For backward compatibility with older versions of the PNL, we still provide the old
{\em rand} interface to random number generation although we strongly encourage users
to use the new {\em rng} interface (see section~\ref{rng-int}).

Every generator is identified by an integer valued macro. One must {\bf NOT} refer
to a generator using directly the value of the macro \var{PNL_RNG_XXX} because there
is no warranty that the order used to store the generators will remain the same in
future releases.  Instead, one should call generators directly using their macro
names.

The initial seeds of all the generators are fixed by the function
\reffun{pnl_rand_init} but you can change it by calling \reffun{pnl_rand_sseed}.

Before starting to use random number generators, you {\bf must} initialize them by
calling
\begin{itemize}
\item \describefun{int}{pnl_rand_init}{int type_generator, int
    simulation_dim, long samples}
  \sshortdescribe It resets the sample counter to $0$ and checks that the
  generator described by \var{type_generator} can actually generate
  \var{samples} in dimension \var{simulation_dim} and fixes the seed.
\end{itemize}

\begin{itemize}

\item \describefun{int}{pnl_rand_or_quasi}{int type_generator}
  \sshortdescribe Returns the type the generator of index \var{type_generator},
  \var{MC} or \var{QMC}
\item \describefun{void}{pnl_rand_sseed}{(int type_generator, unsigned long int
  seed)}
  \sshortdescribe It sets the seed of the generator \var{type_generator} with
  \var{seed}.
\item \describefun{const char \ptr }{pnl_rand_name}{int type_generator}
  \sshortdescribe Returns the name of the generator of index \var{type_generator}
\end{itemize}

Once a generator is chosen, there are several functions available in the
library to draw samples according to a given law.

The following functions return one sample from a specified law.
\begin{itemize}
\item \describefun{int}{pnl_rand_bernoulli}{double p, int type_generator}
  \sshortdescribe Generates a sample from the Bernouilli law on $\{0, 1\}$ with
  parameter \var{p}.

\item \describefun{long}{pnl_rand_poisson}{double lambda, int type_generator}
  \sshortdescribe Generates a sample from the Poisson law with
  parameter \var{lambda}.

\item \describefun{double}{pnl_rand_exp}{double lambda, int type_generator}
  \sshortdescribe Generates a sample from the Exponential law with
  parameter \var{lambda}.

\item \describefun{double}{pnl_rand_uni} {int type_generator}
  \sshortdescribe Generates a sample from the Uniform law on $[0, 1]$.

\item \describefun{double}{pnl_rand_uni_ab} {double a, double b, int
    type_generator}
  \sshortdescribe Generates a sample from the Uniform law on $[a, b]$.

\item \describefun{double}{pnl_rand_normal} {int type_generator}
  \sshortdescribe Generates a sample from the standard normal distribution.

\item \describefun{long}{pnl_rand_poisson1}{double lambda, double t, int
    type_generator}
  \sshortdescribe Generates a sample from a Poisson process with intensity
  \var{lambda} at time \var{t}.

\item \describefun{double}{pnl_rand_gamma} {double a, double b, int type_generator}
  \sshortdescribe Generates a sample from the $\Gamma(a, b)$ distribution.

\item \describefun{double}{pnl_rand_chi2} {double n, int type_generator}
  \sshortdescribe Generates a sample from the centered $\chi^2(n)$ distribution.
\item \describefun{double}{pnl_rand_bessel} {double v, double a, int generator}
  \sshortdescribe Generates a sample from the Bessel distribution with parameters
  \var{v > -1} and \var{a > 0}.
\end{itemize}

The following functions take an already existing \refstruct{PnlVect}\ptr\  as
its first argument and fill each entry of the vector with a sample from the
specified law. All the entries are independent. The difference between
$n-$samples from a distribution in dimension $1$, and one sample from the same
distribution in dimension $n$ only matters when using a {\bf Quasi} random
number generator.
\begin{itemize}
\item \describefun{void}{pnl_vect_rand_uni}{\refstruct{PnlVect} \ptr G, int
    samples, double a, double b, int type_generator}
  \sshortdescribe \var{G} is a vector of independent and identically distributed
  samples from the uniform distribution on $[a, b]$.

\item \describefun{void}{pnl_vect_rand_normal}{\refstruct{PnlVect} \ptr G,
    int samples, int generator}
  \sshortdescribe \var{G} is a vector of independent and identically distributed
  samples from the standard normal distribution.

\item \describefun{void}{pnl_vect_rand_uni_d}{\refstruct{PnlVect} \ptr G, int
    d, double a, double b, int type_generator}
  \sshortdescribe \var{G} is a sample from the uniform distribution on $[a,
  b]^{\text{d}}$.

\item \describefun{void}{pnl_vect_rand_normal_d}{\refstruct{PnlVect} \ptr G,
    int d, int generator}
  \sshortdescribe \var{G} is a sample from the \var{d}-dimensional
  standard normal distribution.

\end{itemize}

The following functions take an already existing \refstruct{PnlMat}\ptr\  as
first argument and fill each entry of the vector with a sample from the
specified law. All the entries are in-dependant. On return, the matrix \var{M}
is of size \verb!samples x dimension!. The rows of \var{M} are independently
and identically distributed. Each row is a sample from the given law in
dimension \var{dimension}.
\begin{itemize}
\item \describefun{void}{pnl_mat_rand_uni}{\refstruct{PnlMat} \ptr M, int
    samples, int d, const PnlVect \ptr a, const PnlVect \ptr b, int
    type_generator}
  \sshortdescribe \var{M} contains \var{samples} samples from the uniform
  distribution on $\prod_{i=1}^d [a_i, b_i]$.

\item \describefun{void}{pnl_mat_rand_uni2}{\refstruct{PnlMat} \ptr M, int
    samples, int d, double a, double b, int type_generator}
  \sshortdescribe \var{M} contains \var{samples} samples from the uniform
  distribution on $[a, b]^{\text{d}}$.

\item \describefun{void}{pnl_mat_rand_normal}{\refstruct{PnlMat} \ptr M, int
    samples, int d, int type_generator}
  \sshortdescribe \var{M} contains \var{samples} samples from the
  \var{d}-dimensional standard normal distribution.
\end{itemize}

Because of the use of {\bf Quasi} random number generators, you may need to
draw a set of samples at once because they represent one sample from a
multi-dimensional distribution. The following function enables to draw one
sample from the \var{dimension}-dimensional standard normal distribution and
store it so that you can access the elements individually afterwards.
\begin{itemize}
\item \describefun{double}{pnl_rand_gauss}{int d, int
    create_or_retrieve, int index, int type_generator}
  \sshortdescribe The second argument can be either \var{CREATE} (to actually
  draw the sample) or \var{RETRIEVE} (to retrieve that element of index
  \var{index}). With \var{CREATE}, it draws \var{d} random normal variables
  and stores them for future usage. They can be withdrawn using \var{RETRIEVE}
  with the index of the number to be retrieved.
\end{itemize}


\subsubsection{Advanced usage}

We also provide functions for directly manipulating Mersenne Twister and
``Dynamically created Mersenne Twister'' random number generators, although we
believe one should rather use the new {\em rng} interface.

\paragraph{Mersenne Twister}

It is possible to create Mersenne Twister random number generators each with
its state variable.
\begin{verbatim}
typedef struct
{
  unsigned long mt[624];
  int mti;
} mt_state;
typedef unsigned long ulong;
\end{verbatim}

\begin{itemize}
\item \describefun{void}{pnl_mt_sseed}{mt_state \ptr state, unsigned long int
    s}
  \sshortdescribe Sets the initial value of variable \var{state} using \var{s}
\item \describefun{ulong}{pnl_mt_genrand}{mt_state \ptr state}
  \sshortdescribe Returns the following number in the sequence as an unsigned
  long variable. A mask is applied so that only the lowest 32-bits are used.
\item \describefun{double}{pnl_mt_genrand_double}{mt_state \ptr state}
  \sshortdescribe Returns the following number in the sequence as a double.
\end{itemize}


\paragraph{Dynamically created Mersenne Twister}

\begin{verbatim}
typedef struct
{
  ulong aaa;
  int mm,nn,rr,ww;
  ulong wmask,umask,lmask;
  int shift0, shift1, shiftB, shiftC;
  ulong maskB, maskC;
  int i;
  ulong state[17];
} dcmt_state;
\end{verbatim}

Some functions to use ``Dynamically Created Mersenne Twister'' random number
generators (DCMT).
\begin{itemize}
\item \describefun{dcmt_state\ptr}{pnl_dcmt_get_parameter}{ulong seed}
  \sshortdescribe Creates a DCMT. \var{seed} is the seed used to initialize
  the Mersenne Twister generator internally used to find new DCMT.
\item \describefun{dcmt_state \ptr \ptr}{pnl_dcmt_create_array}{int n, ulong seed, int \ptr count}
  \sshortdescribe Creates an array of \var{n} independent DCMT. \var{seed} is
  the seed used to initialize the Mersenne Twister generator internally used to
  find new DCMT. On exit, \var{count} contains the number of generators actually
  created.
\item \describefun{void}{pnl_dcmt_sseed}{dcmt_state \ptr mts, ulong s}
  \sshortdescribe Generates a uniformly distributed unsigned integer
\item \describefun{double}{pnl_dcmt_genrand_double}{dcmt_state \ptr mts}
  \sshortdescribe Generates a uniformly distributed random variable on \var{[0,1]}.
\item \describefun{void}{pnl_dcmt_free}{dcmt_state \ptr \ptr mts}
  \sshortdescribe Frees a dcmt.
\item \describefun{void}{pnl_dcmt_free_array}{dcmt_state \ptr \ptr mts, int count}
  \sshortdescribe Frees an array of dcmt as returned by \reffun{pnl_dcmt_create_array}
\end{itemize}

\clearpage

%%% Local Variables: 
%%% mode: latex
%%% TeX-master: "pnl-manual"
%%% End: 
