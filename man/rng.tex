\section{Random Number Generators}

The functionalities described in this chapter are declared in
\verb!pnl/pnl_random.h!.

Random number generators should be called through the new {\em rng} interface
based on the \PnlRng object. This interface uses reentrant functions
and is suitable for multi-threaded applications.

The older {\em rand} interface is kept for compatibility
issues only and hsould not be used in newly written code.
\begin{table}[h!]
  \begin{tabular}{l|l|l|l}
    Random generator & index & Type & Info\\
    \hline
    KNUTH & PNL_RNG_KNUTH & pseudo\\
    MRGK3 & PNL_RNG_MRGK3 & pseudo\\
    MRGK5 & PNL_RNG_MRGK5 & pseudo\\
    SHUFL & PNL_RNG_SHUFL & pseudo\\
    L'ECUYER & PNL_RNG_L_ECUYER & pseudo\\
    TAUSWORTHE & PNL_RNG_TAUSWORTHE & pseudo\\
    MERSENNE & PNL_RNG_MERSENNE & pseudo\\
    SQRT & PNL_RNG_SQRT & quasi\\
    HALTON & PNL_RNG_HALTON & quasi\\
    FAURE & PNL_RNG_FAURE & quasi\\
    SOBOL_I4 & PNL_RNG_SOBOL_I4 & quasi & uses 32 bit intergers\\
    SOBOL_I8 & PNL_RNG_SOBOL2_I8 & quasi & uses 64 bit intergers\\
    NIEDERREITER & PNL_RNG_NIEDERREITER & quasi
  \end{tabular}
  \caption{Indices of the random generators}
  \label{rng-indices}
\end{table}


\subsection{The rng interface}
\label{rng-int}

It is possible to create random number generators each with its own state
variable so that they can evolve independently in a shared memory environment.
These generators are suitable for use in multi-threaded programs. 

\describestruct{PnlRng}
\begin{verbatim}
typedef struct _PnlRng PnlRng;
struct _PnlRng
{
  PnlObject object;
  int type; /*!< generator type *
  void (*Compute)(PnlRng *g, double *sample); /*!< the function to compute the
                                                next number in the sequence */
  int rand_or_quasi; /*!< can be PNL_MC or PNL_QMC */
  int dimension; /*!< dimension of the space in which we draw the samples */
  int counter; /*!< counter = number of samples already drawn */
  int has_gauss; /*!< Is a gaussian deviate available? */
  double gauss; /*!< If has_gauss==1, gauss a gaussian sample */
  int size_state; /*!< size in bytes of the state variable */
  void *state; /*!< state of the random generator */
};
\end{verbatim}


\begin{itemize}
\item \describefun{void}{pnl_rng_free}{\PnlRng \ptr \ptr }
  \sshortdescribe Free a \PnlRng.
\item \describefun{\PnlRng\ptr }{pnl_rng_create}{int type}
  \sshortdescribe Create a \PnlRng corresponding to \var{type}
  which can be any of the values \var{PNL_RNG_XXX} listed in
  Table~\ref{rng-indices} which correspond to {\bf pseudo} random number generators.
  Once a generator has been created, you {\bf must} call
  \reffun{pnl_rng_sseed} before using it. 
\item \describefun{void}{pnl_rng_sseed}{\PnlRng \ptr rng, unsigned
    long int s}
  \sshortdescribe Set the seed of the genrator \var{rng} using \var{s}. If
  \var{s=0}, then a default seed (depending on the generator) is used.
\item \describefun{int}{pnl_rng_sdim}{\PnlRng \ptr rng, int dim}
  \sshortdescribe Set the dimension of the state space for a QMC generator and
  initializes it accordingly.  Returns OK if the generator has been initialized
  properly and FAIL otherwise.
\item \describefun{\PnlRng\ptr }{pnl_rng_copy}{const
  \PnlRng \ptr rng}
  \sshortdescribe Create a copy of \var{rng}.
\item \describefun{void}{pnl_rng_clone}{\PnlRng \ptr dest, const
  \PnlRng \ptr src}
  \sshortdescribe Copy the content of \var{src} into the already existing
  basis \var{dest}. On exit, \var{src} and \var{dest} are identical but
  independent. 
\item \describefun{\PnlRng\ptr}{pnl_rng_dcmt_create_id}{int id, ulong
    seed}
  \sshortdescribe Create a generator with type \var{PNL_RNG_DCMT} and identifier
  \var{id}. Two generators with different \var{id}s are independent. Note that
  the returned generator must be initialized with \reffun{pnl_rng_sseed} before
  usage. The identifier \var{id} can for instance correspond to the thread
  number or the processor rank in parallel computing.
\item \describefun{\PnlRng\ptr \ptr
  }{pnl_rng_dcmt_create_array_id}{int start_id, int max_id, ulong seed, int \ptr
    count}
  \sshortdescribe Create an array of generators with types \var{PNL_RNG_DCMT}
  and identifiers linearly varying between \var{start_id} and \var{max_id}. The
  number of generators created is \var{max_id - start_id + 1}. All the
  generators are independent. Note that each generator of the returned array
  must be initialized with \reffun{pnl_rng_sseed} before usage.
\item \describefun{\PnlRng \ptr \ptr}{pnl_rng_dcmt_create_array}
  {int n, ulong seed, int \ptr count}
  \sshortdescribe Create an array of \var{n} independent DCMT. \var{seed} is
  the seed used to initialize the Mersenne Twister generator internally used to
  find new DCMT. On exit, \var{count} contains the number of generators actually
  created. Same function as \reffun{pnl_dcmt_create_array} instead that it
  directly returns an array of \PnlRng. Before using the generators, you
  must initialize each of them by calling the function \reffun{pnl_rng_sseed}
  \var{count} times.
\end{itemize}

Some auxiliary functions internally used (to use with caution)
\begin{itemize}
\item \describefun{\PnlRng\ptr }{pnl_rng_new}{}
  \sshortdescribe Create an empty \PnlRng.
\item \describefun{void}{pnl_rng_init}{\PnlRng \ptr rng, int type}
  \sshortdescribe Initialize an empty \PnlRng as returned by
  \reffun{pnl_rng_new} to become a generator of type \var{type} which can be
  any of the values \var{PNL_RNG_XXX} listed in Table~\ref{rng-indices} which
  correspond to {\bf pseudo} random number generators.
  Calling \reffun{pnl_rng_create} is equivalent to calling first
  \reffun{pnl_rng_new} and then \reffun{pnl_rng_init}. 
\item \describefun{\PnlRng\ptr }{pnl_rng_get_from_id} {int id}
  \sshortdescribe Return the global generator described by its macro name.
  The variable \var{id} can be any of the values \var{PNL_RNG_XXX} listed in
  Table~\ref{rng-indices}.
\end{itemize}


The following functions return one sample from a specified law.
\begin{itemize}
\item \describefun{int}{pnl_rng_bernoulli}{double p, \PnlRng \ptr rng}
  \sshortdescribe Generate a sample from the Bernouilli law on $\{0, 1\}$ with
  parameter \var{p}.

\item \describefun{long}{pnl_rng_poisson}{double lambda, \PnlRng \ptr rng}
  \sshortdescribe Generate a sample from the Poisson law with
  parameter \var{lambda}.

\item \describefun{double}{pnl_rng_exp}{double lambda, \PnlRng \ptr rng}
  \sshortdescribe Generate a sample from the Exponential law with
  parameter \var{lambda}.

\item \describefun{double}{pnl_rng_dblexp}{double lambda_p, double lambda_m, double p, \PnlRng \ptr rng}
  \sshortdescribe Generate a sample from the asymmetric exponential distribution
  with density 
  \begin{equation*}
    p \lambda_p \expp{-\lambda_p y} \ind{y > 0} + (1-p) \lambda_m
    \expp{-\lambda_m |y|} \ind{y < 0}
  \end{equation*}
  where $\lambda_p >0, \lambda_m >0$ and $p \in [0, 1]$.

\item \describefun{double}{pnl_rng_uni} {\PnlRng \ptr rng}
  \sshortdescribe Generate a sample from the Uniform law on $]0, 1]$.

\item \describefun{double}{pnl_rng_uni_ab} {double a, double b,
    \PnlRng \ptr rng}
  \sshortdescribe Generate a sample from the Uniform law on $[a, b]$.

\item \describefun{double}{pnl_rng_normal} {\PnlRng \ptr rng}
  \sshortdescribe Generate a sample from the standard normal distribution.

\item \describefun{double}{pnl_rng_lognormal}{double m, double sigma2, \PnlRng \ptr rng}
  \sshortdescribe Generate a sample from the log-normal distribution. The
  underlying normal distribution has mean \var{m} and variance \var{sigma2}.

\item \describefun{double}{pnl_rng_invgauss}{double mu, double lambda, \PnlRng \ptr rng}
  \sshortdescribe Generate a sample from the inverse Gaussian distribution with
  mean \var{mu} and shape parameter \var{lambda}.

\item \describefun{long}{pnl_rng_poisson1}{double lambda, double
    t,\PnlRng \ptr rng}
  \sshortdescribe Generate a sample from a Poisson process with intensity
  \var{lambda} at time \var{t}.

\item \describefun{double}{pnl_rng_gamma} {double a, double b, \PnlRng \ptr rng}
  \sshortdescribe Generate a sample from the $\Gamma(a, b)$ distribution.

\item \describefun{double}{pnl_rng_chi2} {double n, \PnlRng \ptr rng}
  \sshortdescribe Generate a sample from the centered $\chi^2(n)$ distribution.
\item \describefun{double}{pnl_rng_bessel} {double v, double a,\PnlRng \ptr rng}
  \sshortdescribe Generate a sample from the Bessel distribution with parameters
  \var{v > -1} and \var{a > 0}.
\item \describefun{double}{pnl_rng_gauss}{int d, int
    create_or_retrieve, int index, PnlRng \ptr rng}
  \sshortdescribe The second argument can be either \var{CREATE} (to actually
  draw the sample) or \var{RETRIEVE} (to retrieve that element of index
  \var{index}). With \var{CREATE}, it draws \var{d} random normal variables
  and stores them for future usage. They can be withdrawn using \var{RETRIEVE}
  with the index of the number to be retrieved.
\end{itemize}

The following functions take an already existing \PnlVect\ptr  as
its first argument and fill each entry of the vector with a sample from the
specified law. All the entries are independent. The difference between
$n-$samples from a distribution in dimension $1$, and one sample from the same
distribution in dimension $n$ only matters when using a {\bf Quasi} random
number generator.
\begin{itemize}
  \item \describefun{void}{pnl_vect_rng_bernoulli}{\PnlVect \ptr V, int samples,
    double a, double b, double p, \PnlRng \ptr rng}
    \sshortdescribe Simulate a random vector according to the Bernoulli
    distribution with values in \var{a,b} and parameter \var{p}. The result is
    stored in \var{V}.
     
\item \describefun{void}{pnl_vect_rng_uni}{\PnlVect \ptr G, int
    samples, double a, double b, \PnlRng \ptr rng}
  \sshortdescribe \var{G} is a vector of independent and identically distributed
  samples from the uniform distribution on $[a, b]$.

\item \describefun{void}{pnl_vect_rng_normal}{\PnlVect \ptr G,
    int samples, \PnlRng \ptr rng}
  \sshortdescribe \var{G} is a vector of independent and identically distributed
  samples from the standard normal distribution.

\item \describefun{void}{pnl_vect_rng_uni_d}{\PnlVect \ptr G, int
    d, double a, double b, \PnlRng \ptr rng}
  \sshortdescribe \var{G} is a sample from the uniform distribution on $[a,
  b]^{\text{d}}$.

\item \describefun{void}{pnl_vect_rng_normal_d}{\PnlVect \ptr G,
    int d, \PnlRng \ptr rng}
  \sshortdescribe \var{G} is a sample from the \var{d}-dimensional
  standard normal distribution.

\end{itemize}

The following functions take an already existing \PnlMat\ptr  as
first argument and fill each entry of the vector with a sample from the
specified law. All the entries are in-dependant. On return, the matrix \var{M}
is of size \verb!samples x dimension!. The rows of \var{M} are independently
and identically distributed. Each row is a sample from the given law in
dimension \var{dimension}.
\begin{itemize}
\item \describefun{void}{pnl_mat_rng_uni}{\PnlMat \ptr M, int
    samples, int d, const PnlVect \ptr a, const PnlVect \ptr b,
    \PnlRng \ptr rng}
  \sshortdescribe \var{M} contains \var{samples} samples from the uniform
  distribution on $\prod_{i=1}^d [a_i, b_i]$.

\item \describefun{void}{pnl_mat_rng_uni2}{\PnlMat \ptr M, int
    samples, int d, double a, double b, \PnlRng \ptr rng}
  \sshortdescribe \var{M} contains \var{samples} samples from the uniform
  distribution on $[a, b]^{\text{d}}$.

\item \describefun{void}{pnl_mat_rng_normal}{\PnlMat \ptr M, int
    samples, int d, \PnlRng \ptr rng}
  \sshortdescribe \var{M} contains \var{samples} samples from the
  \var{d}-dimensional standard normal distribution.
\end{itemize}

Some examples
\begin{verbatim}
#include <stdlib.h>
#include "pnl/pnl_random.h"

int main ()
{
  int i, M;
  PnlRng *rng = pnl_rng_create (PNL_RNG_MERSENNE);
  PnlVect *v = pnl_vect_new ();
  M = 10000;

  /* rng must be initialized. When sseed=0, a default 
     value depending on the generator is used */
     pnl_rng_sseed (rng, 0);

  for ( i=0 ; i<M ; i++ )
  {
    /* Simulates a normal random vector in R^{10} */
    pnl_vect_rng_normal (v, 10, rng);
    /* Do something with v */
  }

  pnl_vect_free (&v);
  pnl_rng_free (&rng); /* Frees the generator */
  exit (0);
}
\end{verbatim}

\begin{verbatim}
#include <stdlib.h>
#include <time.h>
#include "pnl/pnl_random.h"

int main ()
{
  int i, M;
  double E;
  PnlRng *rng = pnl_rng_create (PNL_RNG_MERSENNE);
  M = 10000;

  /* rng must be initialized. */
  rng = pnl_rng_sseed (time (NULL));

  for ( i=0 ; i<M ; i++ )
  {
    /* Simulates an exponential random variable */
    E = pnl_rng_exp (1, rng);
    /* Do something with E */
  }

  pnl_rng_free (&rng); /* Frees the generator */
  exit (0);
}
\end{verbatim}





\subsection{The {\em rand} interface (deprecated)}
\label{rand-int}

{\itshape 
\textbf{Note}:
For backward compatibility with older versions of the PNL, we still provide the old
{\em rand} interface to random number generation although we strongly encourage users
to use the new {\em rng} interface (see section~\ref{rng-int}).
}

Every generator is identified by an integer valued macro. One must {\bf NOT} refer
to a generator using directly the value of the macro \var{PNL_RNG_XXX} because there
is no warranty that the order used to store the generators will remain the same in
future releases.  Instead, one should call generators directly using their macro
names.

The initial seeds of all the generators are fixed by the function
\reffun{pnl_rand_init} but you can change it by calling \reffun{pnl_rand_sseed}.

Before starting to use random number generators, you {\bf must} initialize them by
calling
\begin{itemize}
\item \describefun{int}{pnl_rand_init}{int type_generator, int
    simulation_dim, long samples}
  \sshortdescribe It resets the sample counter to $0$ and checks that the
  generator described by \var{type_generator} can actually generate
  \var{samples} in dimension \var{simulation_dim} and fixes the seed.
\end{itemize}

\begin{itemize}

\item \describefun{int}{pnl_rand_or_quasi}{int type_generator}
  \sshortdescribe Return the type the generator of index \var{type_generator},
  \var{PNL_MC} or \var{PNL_QMC}
\item \describefun{void}{pnl_rand_sseed}{(int type_generator, unsigned long int
  seed)}
  \sshortdescribe It sets the seed of the generator \var{type_generator} with
  \var{seed}.
\item \describefun{const char \ptr }{pnl_rand_name}{int type_generator}
  \sshortdescribe Return the name of the generator of index \var{type_generator}
\end{itemize}

Once a generator is chosen, there are several functions available in the
library to draw samples according to a given law.

The following functions return one sample from a specified law.
\begin{itemize}
\item \describefun{int}{pnl_rand_bernoulli}{double p, int type_generator}
  \sshortdescribe Generate a sample from the Bernouilli law on $\{0, 1\}$ with
  parameter \var{p}.

\item \describefun{long}{pnl_rand_poisson}{double lambda, int type_generator}
  \sshortdescribe Generate a sample from the Poisson law with
  parameter \var{lambda}.

\item \describefun{double}{pnl_rand_exp}{double lambda, int type_generator}
  \sshortdescribe Generate a sample from the Exponential law with
  parameter \var{lambda}.

\item \describefun{double}{pnl_rand_uni} {int type_generator}
  \sshortdescribe Generate a sample from the Uniform law on $[0, 1]$.

\item \describefun{double}{pnl_rand_uni_ab} {double a, double b, int
    type_generator}
  \sshortdescribe Generate a sample from the Uniform law on $[a, b]$.

\item \describefun{double}{pnl_rand_normal} {int type_generator}
  \sshortdescribe Generate a sample from the standard normal distribution.

\item \describefun{long}{pnl_rand_poisson1}{double lambda, double t, int
    type_generator}
  \sshortdescribe Generate a sample from a Poisson process with intensity
  \var{lambda} at time \var{t}.

\item \describefun{double}{pnl_rand_gamma} {double a, double b, int type_generator}
  \sshortdescribe Generate a sample from the $\Gamma(a, b)$ distribution.

\item \describefun{double}{pnl_rand_chi2} {double n, int type_generator}
  \sshortdescribe Generate a sample from the centered $\chi^2(n)$ distribution.
\item \describefun{double}{pnl_rand_bessel} {double v, double a, int generator}
  \sshortdescribe Generate a sample from the Bessel distribution with parameters
  \var{v > -1} and \var{a > 0}.
\end{itemize}

The following functions take an already existing \PnlVect\ptr\  as
its first argument and fill each entry of the vector with a sample from the
specified law. All the entries are independent. The difference between
$n-$samples from a distribution in dimension $1$, and one sample from the same
distribution in dimension $n$ only matters when using a {\bf Quasi} random
number generator.
\begin{itemize}
\item \describefun{void}{pnl_vect_rand_uni}{\PnlVect \ptr G, int
    samples, double a, double b, int type_generator}
  \sshortdescribe \var{G} is a vector of independent and identically distributed
  samples from the uniform distribution on $[a, b]$.

\item \describefun{void}{pnl_vect_rand_normal}{\PnlVect \ptr G,
    int samples, int generator}
  \sshortdescribe \var{G} is a vector of independent and identically distributed
  samples from the standard normal distribution.

\item \describefun{void}{pnl_vect_rand_uni_d}{\PnlVect \ptr G, int
    d, double a, double b, int type_generator}
  \sshortdescribe \var{G} is a sample from the uniform distribution on $[a,
  b]^{\text{d}}$.

\item \describefun{void}{pnl_vect_rand_normal_d}{\PnlVect \ptr G,
    int d, int generator}
  \sshortdescribe \var{G} is a sample from the \var{d}-dimensional
  standard normal distribution.

\end{itemize}

The following functions take an already existing \PnlMat\ptr\  as
first argument and fill each entry of the vector with a sample from the
specified law. All the entries are in-dependant. On return, the matrix \var{M}
is of size \verb!samples x dimension!. The rows of \var{M} are independently
and identically distributed. Each row is a sample from the given law in
dimension \var{dimension}.
\begin{itemize}
\item \describefun{void}{pnl_mat_rand_uni}{\PnlMat \ptr M, int
    samples, int d, const PnlVect \ptr a, const PnlVect \ptr b, int
    type_generator}
  \sshortdescribe \var{M} contains \var{samples} samples from the uniform
  distribution on $\prod_{i=1}^d [a_i, b_i]$.

\item \describefun{void}{pnl_mat_rand_uni2}{\PnlMat \ptr M, int
    samples, int d, double a, double b, int type_generator}
  \sshortdescribe \var{M} contains \var{samples} samples from the uniform
  distribution on $[a, b]^{\text{d}}$.

\item \describefun{void}{pnl_mat_rand_normal}{\PnlMat \ptr M, int
    samples, int d, int type_generator}
  \sshortdescribe \var{M} contains \var{samples} samples from the
  \var{d}-dimensional standard normal distribution.
\end{itemize}

Because of the use of {\bf Quasi} random number generators, you may need to
draw a set of samples at once because they represent one sample from a
multi-dimensional distribution. The following function enables to draw one
sample from the \var{dimension}-dimensional standard normal distribution and
store it so that you can access the elements individually afterwards.
\begin{itemize}
\item \describefun{double}{pnl_rand_gauss}{int d, int
    create_or_retrieve, int index, int type_generator}
  \sshortdescribe The second argument can be either \var{CREATE} (to actually
  draw the sample) or \var{RETRIEVE} (to retrieve that element of index
  \var{index}). With \var{CREATE}, it draws \var{d} random normal variables
  and stores them for future usage. They can be withdrawn using \var{RETRIEVE}
  with the index of the number to be retrieved.
\end{itemize}


\subsubsection{Advanced usage}

We also provide functions for directly manipulating Mersenne Twister and
``Dynamically created Mersenne Twister'' random number generators, although we
believe one should rather use the new {\em rng} interface.

\paragraph{Mersenne Twister}

It is possible to create Mersenne Twister random number generators each with
its state variable.
\begin{verbatim}
typedef struct
{
  unsigned long mt[624];
  int mti;
} mt_state;
typedef unsigned long ulong;
\end{verbatim}

\begin{itemize}
\item \describefun{void}{pnl_mt_sseed}{mt_state \ptr state, unsigned long int
    s}
  \sshortdescribe Set the initial value of variable \var{state} using \var{s}
\item \describefun{ulong}{pnl_mt_genrand}{mt_state \ptr state}
  \sshortdescribe Return the following number in the sequence as an unsigned
  long variable. A mask is applied so that only the lowest 32-bits are used.
\item \describefun{double}{pnl_mt_genrand_double}{mt_state \ptr state}
  \sshortdescribe Return the following number in the sequence as a double.
\end{itemize}


\paragraph{Dynamically created Mersenne Twister}


These are Mersenne Twister type generators with Mersenne exponent fixed to
\var{p=521} and word length \var{w=32} bits. These choices are hard coded and
cannot be changed without altering the code directly.

\begin{verbatim}
typedef struct
{
  ulong aaa;
  int mm,nn,rr,ww;
  ulong wmask,umask,lmask;
  int shift0, shift1, shiftB, shiftC;
  ulong maskB, maskC;
  int i;
  ulong state[17];
} dcmt_state;
\end{verbatim}

Some functions to use ``Dynamically Created Mersenne Twister'' random number
generators (DCMT).
\begin{itemize}
\item \describefun{dcmt_state\ptr}{pnl_dcmt_get_parameter}{ulong seed}
  \sshortdescribe Create a DCMT. \var{seed} is the seed used to initialize
  the Mersenne Twister generator internally used to find new DCMT.
\item \describefun{dcmt_state \ptr \ptr}{pnl_dcmt_create_array}{int n, ulong seed, int \ptr count}
  \sshortdescribe Create an array of \var{n} independent DCMT. \var{seed} is
  the seed used to initialize the Mersenne Twister generator internally used to
  find new DCMT. On exit, \var{count} contains the number of generators actually
  created.
\item \describefun{double}{pnl_dcmt_genrand_double}{dcmt_state \ptr mts}
  \sshortdescribe Generate a uniformly distributed random variable on \var{[0,1]}.
\item \describefun{void}{pnl_dcmt_free}{dcmt_state \ptr \ptr mts}
  \sshortdescribe Free a dcmt.
\item \describefun{void}{pnl_dcmt_free_array}{dcmt_state \ptr \ptr mts, int count}
  \sshortdescribe Free an array of dcmt as returned by \reffun{pnl_dcmt_create_array}
\end{itemize}

% vim:spelllang=en:spell:

