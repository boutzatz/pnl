\section{Some bindings}

% --------------------------------------------------------------------- %%
% MPI
\subsection{MPI bindings}
\subsubsection{Short Description}

We provide some bindings for the MPI library to natively handle {\it PnlObjects}.

\subsubsection{Functions}

All the following functions return an error code as an integer value. This
returned value should be tested against \var{MPI_SUCCESS} to check that no
error occurred.

\begin{itemize}
\item \describefun{int}{pnl_object_mpi_pack_size}{const
    \refstruct{PnlObject} \ptr Obj, MPI_Comm comm, int \ptr size}
  \sshortdescribe Computes in \var{size} the amount of space needed to pack \var{Obj}.
\item \describefun{int}{pnl_object_mpi_pack}{const \refstruct{PnlObject}
    \ptr Obj, void \ptr buf, int bufsize, int \ptr pos, MPI_Comm comm}
  \sshortdescribe Packs \var{Obj} into \var{buf} which must be at least of
  length \var{size}. \var{size} must be at least equal to the value returned
  by \reffun{pnl_object_mpi_pack_size}. 
\item \describefun{int}{pnl_object_mpi_unpack}{\refstruct{PnlObject} \ptr
    Obj, void \ptr buf, int bufsize, int \ptr pos, MPI_Comm comm}
  \sshortdescribe Unpacks the content of \var{buf} starting at position
  \var{pos} (unless several objects have been packed contiguously, \var{\ptr
    pos} should be equal to \var{0}). \var{buf} is a contigous memery area
  of length \var{bufsize} (note that the size is counted in bytes).
  \var{pos} is incremented and is on output the first location in the input
  buffer after the locations occupied by the message that was
  unpacked. \var{pos} is properly set for a future call to {\it MPI_Unpack}
  if any.
  
\item \describefun{int}{pnl_object_mpi_send}{const \refstruct{PnlObject}
    \ptr Obj, int dest, int tag, MPI_Comm comm}
  \sshortdescribe Performs a standard-mode blocking send of \var{Obj}. The
  object is sent to the process with rank \var{dest}.

\item \describefun{int}{pnl_object_mpi_ssend}{const \refstruct{PnlObject} \ptr
    Obj, int dest, int tag, MPI_Comm comm}
  \sshortdescribe Performs a standard-mode synchronous blocking send of
  \var{Obj}. The object is sent to the process with rank \var{dest}.
  
\item \describefun{int}{pnl_object_mpi_recv}{\refstruct{PnlObject} \ptr Obj,
    int src, int tag, MPI_Comm comm, MPI_Status \ptr status} 
  \sshortdescribe Performs a standard-mode blocking receive of \var{Obj}. The
  object is sent to the process with rank \var{dest}. Note that \var{Obj}
  should be an already allocated object and that its type should match the
  true type of the object to be received. \var{src} is the rank of the
  process who sent the object.

  
\item \describefun{int}{pnl_object_mpi_bcast}{\refstruct{PnlObject} \ptr
    Obj, int root, MPI_Comm comm}
  \sshortdescribe Broadcasts the object \var{Obj} from the process with rank
  \var{root} to all other processes of the group \var{comm}.
\end{itemize}

For more expect users, we provide the following nonblocking functions.
\begin{itemize}
\item \describefun{int}{pnl_object_mpi_isend}{const \refstruct{PnlObject}
    \ptr Obj, int dest, int tag, MPI_Comm comm, MPI_Request \ptr request}
  \sshortdescribe Starts a standard-mode, nonblocking send of object
  \var{Obj} to the process with rank \var{dest}.
  
  
\item \describefun{int}{pnl_object_mpi_irecv}{void \ptr \ptr buf, int \ptr
    size, int src, int tag, MPI_Comm comm, int \ptr flag, MPI_Request \ptr
    request}
  \sshortdescribe Starts a standard-mode, nonblocking receive of object
  \var{Obj} from the process with rank \var{root}. On output \var{flag} equals
  to \var{TRUE} if the object can be received and \var{FALSE} otherwise (this
  is the same as for {\it MPI_Iprobe}).
\end{itemize}

\subsection{The save/load interface}

The interface is only accessible when the MPI bindings are compiled since it
is based on the Packing/Unpacking facilities of MPI.

For the moment, we can only save load random number generators
\begin{itemize}
\item \describefun{\refstruct{PnlRng}\ptr \ptr}
  {pnl_rng_create_from_file}{char \ptr str, int n}
  \sshortdescribe Loads \var{n} rng from the file of name \var{str} and
  returns an array of \var{n} \refstruct{PnlRng}.
\item \describefun{int}{pnl_rng_save_to_file}{\refstruct{PnlRng} \ptr \ptr
    rngtab, int n, char \ptr str}
    \sshortdescribe Saves \var{n} rng stored in \var{rngtab} into the file of
  name \var{str}.
\item \describefun{int}{pnl_object_save}{PnlObject \ptr O, FILE *stream}
  \sshortdescribe Saves the object \var{O} into the stream \var{stream}. \var{stream}
  is typically created by calling fopen with \var{mode="wb"}. This function can be
  called several times to save several objects in the same stream.
\item \describefun{PnlObject\ptr}{pnl_object_load}{FILE *stream}
  \sshortdescribe Loads an object from the stream \var{stream}. \var{stream}
  is typically created by calling fopen with \var{mode="rb"}.  This function can be
  called several times to load several objects from the same stream. If \var{stream}
  was empty or it did not contain any PnlObjects, the function returns \var{NULL}.
\item \describefun{PnlList\ptr}{pnl_object_load_into_list}{FILE *stream}
  \sshortdescribe Loads as many objects as possible from the stream \var{stream} and
  stores them into a \refstruct{PnlList}. \var{stream} is typically created by
  calling fopen with \var{mode="rb"}. If \var{stream} was empty or it did not contain
  any PnlObjects, the function returns \var{NULL}.
\end{itemize}


%%% Local Variables: 
%%% mode: latex
%%% TeX-master: "pnl-manual"
%%% End: 
