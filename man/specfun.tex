
\section{Special functions}

The special function approximations are defined in the header \verb!pnl/pnl_specfun.h!.\\

Most of these functions rely on the {\it Cephes} library which uses its own
error mechanism which can be activated or deactivated using the two following
functions
\begin{itemize}
  \item \describefun{void}{pnl_deactivate_mtherr}{}
    \sshortdescribe Deactivate Cephes error mechanism
  \item \describefun{void}{pnl_activate_mtherr}{}
    \sshortdescribe Activate Cephes error mechanism
\end{itemize}


\subsection{Real Bessel functions}

\begin{itemize}
\item \describefun{double}{pnl_bessel_i}{double v, double x}
  \sshortdescribe   Modified Bessel function of the first
  kind of order \var{v}.
\item \describefun{double}{pnl_bessel_i_scaled}{double v, double x}
  \sshortdescribe   Modified Bessel function of the first
  kind of order \var{v} divided by $e^{|x|}$.
\item \describefun{double}{pnl_bessel_rati}{double v, double x}
  \sshortdescribe Ratio of modified Bessel functions of the first kind : $I_{v+1}(x) /
  I_v (x)$.
\item \describefun{double}{pnl_bessel_j}{double v, double x}
  \sshortdescribe    Bessel function of the first
  kind of order \var{v}.
\item \describefun{double}{pnl_bessel_j_scaled}{double v, double x}
  \sshortdescribe    Bessel function of the first
  kind of order \var{v}. Same function as \reffun{pnl_bessel_j}.
\item \describefun{double}{pnl_bessel_y}{double v, double x}
  \sshortdescribe   Modified Bessel function of the second
  kind of order \var{v}.
\item \describefun{double}{pnl_bessel_y_scaled}{double v, double x}
  \sshortdescribe   Modified Bessel function of the second
  kind of order \var{v}. Same function as \reffun{pnl_bessel_y}.
\item \describefun{double}{pnl_bessel_k}{double v, double x}
  \sshortdescribe   Bessel function of the third
  kind of order \var{v}.
\item \describefun{double}{pnl_bessel_k_scaled}{double v, double x}
  \sshortdescribe   Bessel function of the third
  kind of order \var{v} multiplied by $e^{x}$.
\item \describefun{dcomplex}{pnl_bessel_h1}{double v, double x}
  \sshortdescribe   Hankel function of the first kind of
  order \var{v}. 
\item \describefun{dcomplex}{pnl_bessel_h1_scaled}{double v, double x}
  \sshortdescribe  Hankel function of the first kind of order
  \var{v}  and divided by $e^{I x}$.
\item \describefun{dcomplex}{pnl_bessel_h2}{double v, double x}
  \sshortdescribe  Hankel function of the second kind of
  order \var{v}. 
\item \describefun{dcomplex}{pnl_bessel_h2_scaled}{double v, double x}
  \sshortdescribe  Hankel function of the second kind of
  order \var{v}  and multiplied by $e^{I x}$.
\end{itemize}

\subsection{Complex Bessel functions}

\begin{itemize}
\item \describefun{dcomplex}{pnl_complex_bessel_i}{double v, dcomplex z}
  \sshortdescribe  Complex Modified Bessel function of the first
  kind of order \var{v}.
\item \describefun{dcomplex}{pnl_complex_bessel_i_scaled}{double v, dcomplex z}
  \sshortdescribe  Complex Modified Bessel function of the first
  kind of order \var{v} divided by $e^{|Creal(z)|}$.
\item \describefun{dcomplex}{pnl_complex_bessel_rati}{double v, dcomplex x}
  \sshortdescribe Ratio of complex modified Bessel functions of the first kind : $I_{v+1}(x) /
  I_v (x)$.
\item \describefun{dcomplex}{pnl_complex_bessel_j}{double v, dcomplex z}
  \sshortdescribe  Complex  Bessel function of the first
  kind of order \var{v}.
\item \describefun{dcomplex}{pnl_complex_bessel_j_scaled}{double v, dcomplex z}
  \sshortdescribe  Complex  Bessel function of the first
  kind of order \var{v} divided by $e^{|Cimag(z)|}$.
\item \describefun{dcomplex}{pnl_complex_bessel_y}{double v, dcomplex z}
  \sshortdescribe  Complex Modified Bessel function of the second
  kind of order \var{v}.
\item \describefun{dcomplex}{pnl_complex_bessel_y_scaled}{double v, dcomplex z}
  \sshortdescribe  Complex Modified Bessel function of the second
  kind of order \var{v} divided by $e^{|Cimag(z)|}$.
\item \describefun{dcomplex}{pnl_complex_bessel_k}{double v, dcomplex z}
  \sshortdescribe  Complex Bessel function of the third
  kind of order \var{v}.
\item \describefun{dcomplex}{pnl_complex_bessel_k_scaled}{double v, dcomplex z}
  \sshortdescribe  Complex Bessel function of the third
  kind of order \var{v} multiplied by $e^{z}$.
\item \describefun{dcomplex}{pnl_complex_bessel_h1}{double v, dcomplex z}
  \sshortdescribe  Complex Hankel function of the first kind of
  order \var{v}. 
\item \describefun{dcomplex}{pnl_complex_bessel_h1_scaled}{double v, dcomplex z}
  \sshortdescribe  Complex  Hankel function of the first kind of order
  \var{v}  and divided by $e^{I z}$.
\item \describefun{dcomplex}{pnl_complex_bessel_h2}{double v, dcomplex z}
  \sshortdescribe  Complex  Hankel function of the second kind of
  order \var{v}. 
\item \describefun{dcomplex}{pnl_complex_bessel_h2_scaled}{double v, dcomplex z}
  \sshortdescribe  Complex  Hankel function of the second kind of
  order \var{v}  and multiplied by $e^{I z}$.
\end{itemize}

\subsection{Error functions}

\begin{itemize}
\item \describefun{double}{pnl_sf_erf}{double x}
  \sshortdescribe Compute the error function $\frac{2}{\pi}\int_0^\infty \expp{-t^2} dt$.
\item \describefun{dcomplex}{pnl_sf_complex_erf}{dcomplex z}
  \sshortdescribe Same as \reffun{pnl_sf_erf} for complex arguments.
\item \describefun{double}{pnl_sf_erfc}{double x}
  \sshortdescribe Compute the complementary error function \var{1. - erf(x)}.
\item \describefun{dcomplex}{pnl_sf_complex_erfc}{dcomplex x}
  \sshortdescribe Same as \reffun{pnl_sf_erfc} for complex arguments.
\item \describefun{double}{pnl_sf_erfcx}{double x}
  \sshortdescribe Compute the scaled complementary error function of x, defined by
  $\expp{x^2} \text{erfc}(x)$.
\item \describefun{dcomplex}{pnl_sf_complex_erfcx}{dcomplex z}
  \sshortdescribe Same as \reffun{pnl_sf_erfcx} for complex arguments. Note that
  \var{erfcx(-i x) = w(x)}.
\item \describefun{dcomplex}{pnl_sf_w}{dcomplex z}
  \sshortdescribe Compute $\expp{-z^2} \, \text{erfc}(-i \, z)$.
\item \describefun{double}{pnl_sf_w_im}{double x}
  \sshortdescribe Compute $2 \, \text{Dawson}(x)/\sqrt{\pi}$
\item \describefun{double}{pnl_sf_erfi}{double x}
  \sshortdescribe Compute \var{-i erf(i z)}
\item \describefun{dcomplex}{pnl_sf_complex_erfi}{dcomplex z}
  \sshortdescribe Same as \reffun{pnl_sf_erfi} for complex arguments.
\item \describefun{double}{pnl_sf_dawson}{double x}
  \sshortdescribe Compute $\sqrt{\pi}/2 \, \expp{-x^2} \, \text{erfi}(x)$.
\item \describefun{dcomplex}{pnl_sf_complex_dawson}{dcomplex z}
  \sshortdescribe Same as \reffun{pnl_sf_dawson} for complex arguments.
\item \describefun{double}{pnl_sf_log_erf}{double x}
  \sshortdescribe Compute $\log$ \reffun{pnl_sf_erf}$(x)$
\item \describefun{double}{pnl_sf_log_erfc}{double x}
  \sshortdescribe Compute $\log$ \reffun{pnl_sf_erfc}$(x)$
\end{itemize}

\subsection{Gamma functions}

For $x>0$, the Gamma Function is defined by
\begin{equation*}
  \Gamma(x)=\int_0^{\infty} \expp{-u} u^{x-1} du.
\end{equation*}

\begin{itemize}
\item \describefun{double}{pnl_sf_fact}{int n}
  \sshortdescribe   Computes factorial of \var{n} $ \Gamma (n+1)$.
\item \describefun{double}{pnl_sf_gamma}{double x}
  \sshortdescribe   Computes $\Gamma(x), x \geq 0$
\item \describefun{double}{pnl_sf_log_gamma}{double x}
  \sshortdescribe   Computes $\log(\Gamma(x)), x \geq 0$
\item \describefun{int}{pnl_sf_log_gamma_sgn}{double x, double \ptr y, int \ptr sgn}
  \sshortdescribe   Computes $y = \log(|\Gamma(x)|)$ for \var{x > 0} \var{sgn}
  contains the sign of $\Gamma(x)$ (-1 or +1).
\item \describefun{double}{pnl_sf_choose}{int n, int k}
  \sshortdescribe Computes the binomial coefficient $\binom{n}{k} = 
  \frac{n!}{k! (n-k)!}$ for $0 \le k \le n$ in double precision.
\end{itemize}

\subsection{Digamma function}

For $x>0$, the digamma function $\psi$ is defined as the logarithmic derivative of the
Gamma function $\Gamma$
\begin{equation*}
  \psi(x) = \frac{d}{dx} \log \Gamma (x) = \frac{\Gamma'(x)}{\Gamma(x)}.
\end{equation*}
The function $\psi$ admits the following integral representation
\begin{equation*}
  \psi (x) = \int_{0}^{\infty} \left( \frac{\expp{-u}}{u} - \frac{\expp{-xu}}{1
  - \expp{-u}} \right).
\end{equation*}

\begin{itemize}
  \item \describefun{double}{pnl_sf_psi}{double x}
    \sshortdescribe Return $\psi(x)$.
\end{itemize}

\subsection{Incomplete Gamma functions}

For $a \in \R$ and $x>0$, the Incomplete Gamma Function is defined by
\begin{equation*}
  \Gamma(a, x)=\int_x^{\infty} \expp{-u} u^{a-1} du.
\end{equation*}
A relation similar to the one existing for the standard Gamma function holds
\begin{equation*}
  \Gamma\paren{a, x}= \frac{- x^{a} \expp{-x} + \Gamma (a+1, x)}{a}.
\end{equation*}
\begin{align*}
  \Gamma(a)&=\int_0^{\infty} u^{a-1} \expp{-u}du\\ 
  P(a, x) &= \frac{\Gamma(a) - \Gamma(a, x)}{\Gamma(a)} =
  \frac{1}{\Gamma(a)} \int_0^x u^{a-1} \expp{-u}  du\\ 
  Q(a, x) &= 1-P(a, x) =\frac{\Gamma(a, x)}{\Gamma(a)} =
  \frac{1}{\Gamma(a)} \int_x^{\infty} \expp{-u} u^{a-1} du. 
\end{align*}

\begin{itemize}
\item \describefun{double}{pnl_sf_gamma_inc}{double a, double x}
  \sshortdescribe   Computes $\Gamma(a, x), \quad a \in \R , x \geq 0$
\item \describefun{void}{pnl_sf_gamma_inc_P}{double a, double x}
  \sshortdescribe  Computes $P(a, x), \quad a > 0 , x \geq 0$
\item \describefun{void}{pnl_sf_gamma_inc_Q}{double a, double x}
  \sshortdescribe  Computes $Q(a, x), \quad a > 0 , x \geq 0$
\end{itemize}

\subsection{Exponential integrals}
For $x>0$ and $n \in \N$, the  function $E_n$ is defined by
\begin{equation*}
  E_n\paren{x}=\int_{1}^{\infty} \expp{-x u} u^{-n} du
\end{equation*}

This function is linked to the Incomplete Gamma function by 
\begin{equation*}
  E_n\paren{x}=\int_{x}^{\infty}
  \expp{-xu} (xu)^{-n} x^{n-1} d(xu)=x^{n-1} \int_{x}^{\infty}
  \expp{-t} t^{-n} dt =  x^{n-1}  \Gamma\paren{1-n, x}, 
\end{equation*}
from which we can deduce
\begin{equation*}
  n E_{n+1}(x)  =   \expp{-x} - x E_n(x).
\end{equation*}
For $n>1$, the series expansion is given by
\begin{equation*}
  E_n(x)=x^{n-1}
  \Gamma(1-n)+\recaco{-\frac{1}{1-n}+\frac{x}{2-n}-\frac{x^2}{2(3-n)}
    +\frac{x^3}{6(4-n)}-\dots}.     
\end{equation*}
The asymptotic behaviour is given by
\begin{equation*}
  E_n(x)=\frac{\expp{-x}}{x}\recaco{1-\frac{n}{x}+\frac{n(n+1)}{x^2}+\dots}. 
\end{equation*}
The special case $n=1$ gives 
\begin{equation*}
  E_1(x) = \int_x^\infty \frac{e^{-u}}{u}\, du, \quad |\mathrm{Arg}(x)| \ge \pi. 
\end{equation*}
For any complex number $x$ with positive real part, this can be written
\begin{equation*}
  E_1(x) = \int_1^\infty \frac{e^{-ux}}{u}\, du, \quad \Re(x) \ge 0. 
\end{equation*}
By integrating the Taylor expansion of $\expp{-t}/t$, and extracting the
logarithmic singularity, we can derive the following series representation for
$E_1(x)$, 
\begin{equation*}
  {E_1}(x) =-\gamma-\ln x-\sum_{k=1}^{\infty}\frac{(-1)^k x^k}{k\; k!}
  \qquad |\mathrm{Arg}(x)| < \pi. 
\end{equation*}
The function $E_1$ is linked to the exponential integral $Ei$
\begin{equation*}
  Ei(x)=\int_{-\infty}^x\frac{e^u}u\, du=-\int_{-x}^{\infty}
  \frac{e^{-u}}{u}\, du \quad \forall x \neq 0. 
\end{equation*}
The above definition  can be used for positive values of $x$, but the integral
has to be understood in terms of its Cauchy principal value, due to the
singularity of the integrand at zero.
\begin{equation*}
  {Ei}(-x) = -{E}_1(x) , \quad \Re(x) \ge 0.
\end{equation*}
We deduce, 
\begin{equation*}
  Ei(x) = \gamma + \ln x+ \sum_{k=1}^{\infty} \frac{x^k}{k\; k!}, \quad x>0.
\end{equation*}
For $x \in \R$ 
\begin{equation*}
  \Gamma(0, x)=\left\{
    \begin{array}{l}
      -Ei(-x)-\imath \pi  \quad  x<0, \\
      -Ei(-x) \quad x>0. 
    \end{array}\right.
\end{equation*}

\begin{itemize}
\item \describefun{double}{pnl_sf_expint_En}{int n, double x}
  \sshortdescribe   Computes \var{E_n(x)} for  $ n\geq 0, x \geq 0$, or $x>0$
  when $n=0$ or $1$.
\end{itemize}

\subsection{Hypergeometric functions}

\begin{itemize}
\item \describefun{double}{pnl_sf_hyperg_2F1}{double a, double b, double c,
    double x}
  \sshortdescribe Compute the Gauss hypergeometric function \var{2F1(a,b,c,x)}
  for \var{|x| < 1} and for \var{x < -1} when \var{b,a,c,(b-a),(c-a),(c-b)} are
  not integers
\item \describefun{double}{pnl_sf_hyperg_1F1}{double a, double b, double x}
  \sshortdescribe Compute the hypergeometric function \var{1F1(a,b,x)}
\item \describefun{double}{pnl_sf_hyperg_2F0}{double a, double b, double x}
  \sshortdescribe Compute the hypergeometric function \var{2F0(a,b,x)}  for
  \var{x<0} using the relation $2F0 (a,b,x) = (-x)^{-a} U(a,1+a-b,-\frac{1}{x})$.
\item \describefun{double}{pnl_sf_hyperg_0F1}{double c, double x}
  \sshortdescribe Compute the hypergeometric function \var{0F1(c,x)}
\item \describefun{double}{pnl_sf_hyperg_U}{double a, double b, double x}
  \sshortdescribe Compute the confluent hypergeometric function \var{U(a,b,x)}
  with \var{x > 0}
\end{itemize}


% vim:spelllang=en:spell:
