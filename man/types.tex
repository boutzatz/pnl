\section{Objects}

\subsection{The top-level object}

The PnlObject structure is used to simulate some inheritance between the
ojbects of Pnl.  It must be the first element of all the objects existing in
Pnl so that casting any object to a PnlObject is legal

\begin{verbatim}
typedef unsigned int PnlType; 

typedef void (destroy_func) (void **);
struct _PnlObject
{
  PnlType type; /*!< a unique integer id */
  const char *label; /*!< a string identifier (for the moment not useful) */
  PnlType parent_type; /*!< the identifier of the parent object is any,
                          otherwise parent_type=id */
  destroy_func *destroy; /*!< frees an object */
};
\end{verbatim}

Here is the list of all the types actually defined
\begin{table}
  \centering
  \begin{tabular}{l|l}
    \hline
    PnlType & Description \\
    \hline
    PNL_TYPE_VECTOR & general vectors  \\
    PNL_TYPE_VECTOR_DOUBLE & real vectors \\
    PNL_TYPE_VECTOR_INT & integer vectors \\
    PNL_TYPE_VECTOR_COMPLEX & complex vectors \\
    PNL_TYPE_MATRIX & general matrices  \\
    PNL_TYPE_MATRIX_DOUBLE & real matrices \\
    PNL_TYPE_MATRIX_INT & integer matrices \\
    PNL_TYPE_MATRIX_COMPLEX & complex matrices \\
    PNL_TYPE_TRIDIAG_MATRIX & general tridiagonal matrices \\
    PNL_TYPE_TRIDIAG_MATRIX_DOUBLE & real  tridiagonal matrices \\
    PNL_TYPE_BAND_MATRIX & general band matrices \\
    PNL_TYPE_BAND_MATRIX_DOUBLE & real band matrices \\
    PNL_TYPE_HMATRIX & general hyper matrices \\
    PNL_TYPE_HMATRIX_DOUBLE & real hyper matrices \\
    PNL_TYPE_HMATRIX_INT & integer hyper matrices \\
    PNL_TYPE_HMATRIX_COMPLEX & complex hyper matrices \\
    PNL_TYPE_BASIS & bases \\
    PNL_TYPE_RNG & random number generators \\
    PNL_TYPE_LIST & doubly linked list
  \end{tabular}
  \caption{PnlTypes}
  \label{types}
\end{table}

We provide several macros for manipulating PnlObejcts.
\begin{itemize}
\item \describemacro{PNL_OBJECT}{o}
  \sshortdescribe Casts any object into a PnlObject

\item \describemacro{PNL_VECT_OBJECT}{o}
  \sshortdescribe Casts any object into a PnlVectObject

\item \describemacro{PNL_MAT_OBJECT}{o}
  \sshortdescribe Casts any object into a PnlMatObject

\item \describemacro{PNL_HMAT_OBJECT}{o}
  \sshortdescribe Casts any object into a PnlHmatObject

\item \describemacro{PNL_BAND_MAT_OBJECT}{o}
  \sshortdescribe Casts any object into a PnlBandMatObject

\item \describemacro{PNL_TRIDIAGMAT_OBJECT}{o}
  \sshortdescribe Casts any object into a PnlTridiagMatObject

\item \describemacro{PNL_BASIS_OBJECT}{o}
  \sshortdescribe Casts any object into a PnlBasis

\item \describemacro{PNL_RNG_OBJECT}{o}
  \sshortdescribe Casts any object into a PnlRng

\item \describemacro{PNL_LIST_OBJECT}{o}
  \sshortdescribe Casts any object into a PnlList

\item \describemacro{PNL_GET_TYPENAME}{o}
  \sshortdescribe Returns the name of the type of any object inheriting from PnlObject

\item \describemacro{PNL_GET_TYPE}{o}
  \sshortdescribe Returns the type of any object inheriting from PnlObject
  
\item \describemacro{PNL_GET_PARENT_TYPE}{o}
  \sshortdescribe Returns the parent type of any object inheriting from PnlObject
\end{itemize}

\subsubsection{Functions}

\begin{itemize}
\item \describefun{\refstruct{PnlObject}\ptr }{pnl_object_create}{PnlType t}
  \sshortdescribe Creates an empty PnlObject of type \var{t} which can any of
  the registered types, see Table~\ref{types}.
\end{itemize}

\subsection{List object}

This section describes functions for creating an manipulating lists. Lists are
internally stored as doubly linked lists.

\subsubsection{Short Description}

The structures and functions related to lists are declared in
\verb!pnl_list.h!.

\begin{verbatim}
typedef struct _PnlCell PnlCell;
struct _PnlCell
{
  struct _PnlCell *prev;  /*!< previous cell or 0 */
  struct _PnlCell *next;  /*!< next cell or 0 */
  PnlObject *self;       /*!< stored object */
};


typedef struct _PnlList PnlList;
struct _PnlList
{
  /**
   * Must be the first element in order for the object mechanism to work
   * properly. This allows any PnlList pointer to be cast to a PnlObject
   */
  PnlObject object; 
  PnlCell *first; /*!< first element of the list */
  PnlCell *last; /*!< last element of the list */
  PnlCell *curcell; /*!< last accessed element,
                         if never accessed is NULL */
  int icurcell; /*!< index of the last accessed element,
                     if never accessed is NULLINT */
  int len; /*!< length of the list */
};
\end{verbatim}

\subsubsection{Functions}

\begin{itemize}
\item \describefun{\refstruct{PnlList} \ptr }{pnl_list_new}{}
  \sshortdescribe Creates an empty list
\item \describefun{\refstruct{PnlCell} \ptr }{pnl_cell_new}{}
  \sshortdescribe Creates an cell list
\item \describefun{void}{pnl_list_free}{\refstruct{PnlList}  \ptr \ptr L}
  \sshortdescribe Frees a list
\item \describefun{void}{pnl_cell_free}{\refstruct{PnlCell}  \ptr \ptr c}
  \sshortdescribe Frees a list
\item \describefun{\refstruct{PnlObject}\ptr}{pnl_list_get}{
    \refstruct{PnlList} \ptr L, int i}
  \sshortdescribe This function returns the content of the \var{i}--th cell of
  the list \var{L}. This function is optimized for linearly accessing all the
  elements, so it can be used inside a for loop for instance.
\item \describefun{void}{pnl_list_insert_first}{\refstruct{PnlList}  \ptr L,
    \refstruct{PnlObject}  \ptr o}
  \sshortdescribe Insert the object \var{o} on top of the list \var{L}. Note that
  \var{o} is not copied in \var{L}, so do  {\bf not} free \var{o} yourself, it
  will be done automatically when calling \reffun{pnl_list_free}
\item \describefun{void}{pnl_list_insert_last}{\refstruct{PnlList}  \ptr L,
    \refstruct{PnlObject}  \ptr o}
  \sshortdescribe Insert the object \var{o} at the bottom of the list \var{L}. Note that
  \var{o} is not copied in \var{L}, so do  {\bf not} free \var{o} yourself, it
  will be done automatically when calling \reffun{pnl_list_free}
\item \describefun{void}{pnl_list_remove_last}{\refstruct{PnlList}  \ptr L}
  \sshortdescribe Removes the last element of the list \var{L} and frees it.
\item \describefun{void}{pnl_list_remove_first}{\refstruct{PnlList}  \ptr L}
  \sshortdescribe Removes the first element of the list \var{L} and frees it.
\item \describefun{void}{pnl_list_remove_i}{\refstruct{PnlList}  \ptr L, int i}
  \sshortdescribe Removes the \var{i-th} element of the list \var{L} and frees it.
\item \describefun{void}{pnl_list_concat}{\refstruct{PnlList}  \ptr L1,
    \refstruct{PnlList}  \ptr L2}
  \sshortdescribe Concatenates the two lists \var{L1} and \var{L2}. The
  resulting list is store in \var{L1} on exit. Do {\bf not} free \var{L2}
  since concatenation does not actually copy objects but only manipulates
  addresses.
\item \describefun{void}{pnl_list_print}{\refstruct{PnlList}  \ptr L}
  \sshortdescribe Not yet implemented because it would require that the
  structure \refstruct{PnlObject} has a field copy.
\end{itemize}
